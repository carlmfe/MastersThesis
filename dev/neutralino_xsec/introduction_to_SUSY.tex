\documentclass[english,notitlepage]{article}

\usepackage{overhead}
\usepackage[style=verbose]{biblatex}

\addbibresource{susy.bib}


\title{Introduction to Supersymmetry}
\author{Carl Martin Fevang}

\begin{document}
\maketitle

\section{Supersymmetry}
\textit{Introduce the notion of supersymmetry and some terms that will be necessary.}

\section{The Super-Poincaré group}
\textit{Introduce the Poincaré group and outline its supersymmetric extension algebraically.}

\section{Superspace}
In order to make fields that transform appropriately under super-Poincaré transformations, it is useful to embed them in what is called \emph{superspace}.
This is an extension of the space-time on which the coordinate transformations of the Poincaré group work.
The additional operators (supercharges) of the Super-Poincaré group will be analogous to these coordinate transformations, but on superspace.

We extend spacetime with four \emph{Grassmann number}-valued coordinates, \(\theta_A, \bar{\theta}^{\dot{A}}\) where \(A, \dot{A} \in 1, 2\).
Grassmann numbers are similar to ordinary numbers, but are \emph{anti}-commutative under multiplication, i.e. \(\theta_1 \theta_2 = -\theta_2 \theta_1\).
As a consequence of this, any repeated Grassmann number is zero, since \(\theta_1\theta_1 = -\theta_1\theta_1 = 0\).\footnote{I note that 0 is itself a (trivial) Grassmann number, and it constitutes the intersection between the real number line and the Grassmann number line.}
These numbers arise naturally in structures that have anti-commuting properties, such as fermionic spinors, and the Grassmann dimensions of superspace mirror the anti-commuting properties of the supercharges \(Q_A\).


\end{document}

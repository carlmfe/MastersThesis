\documentclass[english, notitlepage]{article}

\usepackage{overhead}
\usepackage[style=verbose]{biblatex}

\addbibresource{neutralino_interactions.bib}

\title{Neutralino Interactions in the MSSM}
\author{Carl Martin Fevang}

\begin{document}
\maketitle

\begin{abstract}
    \noindent
    In this document, I derive how to construct the fermion interaction
    Lagrangian from kinetic and Yukawa terms of the superlagrangian.
    This is then applied to the MSSM superlagrangian to get the Feynman rules
    for neutralino interaction with the SM \(Z\)-boson and quark/squark pairs.
\end{abstract}

\section{Superfields}
Here I list some general expansions of fields over superspace,
\textit{superfield}. The fields are expanded in the superspace coordinates
\(\theta_{A=1,2}, {\theta^\dagger}^{\dot{A}=1,2}\) that are four Grassmann
coordinates imposed in a spinor structure with one left-handed Weyl spinor and
a right-handed Weyl spinor.

A left-handed scalar superfield \(\Phi\) can be written out in terms of
component
fields as\footnote{Parentheses are used to clarify Weyl spinor contraction.}
\begin{align}
    \Phi = A + i(\theta\sigma^\mu\theta^\dagger) \partial_\mu A - \frac{1}{4}
    (\theta\theta)(\theta\theta)^\dagger \dA A +
    \sqrt{2} (\theta\psi) - \frac{i}{\sqrt{2}} (\theta\theta) (\partial_\mu
    \psi \sigma^\mu \theta^\dagger) + (\theta\theta) F,
\end{align}
where \(A, F\) are complex scalar fields and \(\psi\) is a left-handed Weyl
spinor
field.
\(\Phi\) has a right-handed scalar superfield compliment found by conjugating
it:
\begin{align}
    \Phi^\dag = A^\ast - i(\theta\sigma^\mu\theta^\dagger) \partial_\mu A^\ast
    - \frac{1}{4} (\theta\theta)(\theta\theta)^\dagger \dA A^\ast +
    \sqrt{2} (\theta\psi)^\dagger + \frac{i}{\sqrt{2}} (\theta\theta)^\dagger
    (\theta \sigma^\mu \partial_\mu \psi^\dagger) + (\theta\theta)^\dagger
    F^\ast,
\end{align}
where \(\psi^\dagger\) is the right-handed compliment of \(\psi\) such that
\({\psi^\dagger}^{\dot A} = \delta^{\dot A A} {(\psi_A)}^\ast.\)

A vector superfield \(V\) can be written in Wess-Zumino gauge as
\begin{align}
    \msub{V}{WZ} = (\theta\sigma^\mu\theta^\dagger) \bclosed{V_\mu +
        i\partial_\mu (A - A^\ast)} + (\theta\theta)
    (\theta\lambda)^\dagger +
    (\theta\theta)^\dagger(\theta\lambda) + \frac{1}{2}
    (\theta\theta)(\theta\theta)^\dagger D,
\end{align}
where \(V_\mu\) is a real vector field, \(\lambda\) is a left-handed Weyl
spinor
field and \(D\) is a (auxiliary) complex scalar field.
The \(\partial_\mu (A-A^\ast)\)-term represents the gauge freedom remaining in
the choice of
supergauge after choosing Wess-Zumino gauge, and can be ignored when working
out the interaction terms.
I note that this gauge implies that no powers of the vector superfield above 2
are non-zero because of the Grassmann content.
For the remainder of this document, vector superfields will be assumed to be in
Wess-Zumino gauge.

\section{MSSM fields}
For completeness, I list here the relevant superfields containing the
neutralinos and the superfields that couple directly to them.
\subsection{The superfields}
The neutralino fields are found in the scalar superfield \(SU(2)_L\) doublets
\(H_u = \begin{pmatrix} H_u^+ \\ H_u^0 \end{pmatrix}\), \(H_d = \begin{pmatrix}
    H_d^0 \\ H_d^- \end{pmatrix}\), and the vector superfields \(B^0\) for
the
\(U(1)_Y\) gauge group, and \(W^0\) for the \(SU(2)_L\) gauge group.
The fields that couple directly to them are found in these same superfields,
the remaining \(SU(2)_L\) vector superfields \(W^\pm\), the scalar superfield
\(SU(2)_L\) doublets \(L_i = \begin{pmatrix} l_i \\ \nu_i \end{pmatrix}\) and
\(Q_i = \begin{pmatrix} u_i \\ d_i \end{pmatrix}\), and the \(SU(2)_L\) singlet
superfields \(\wbar{E}_i\), \(\wbar{U}_i\) and \(\wbar{D}_i\), where
\(i=1,2,3\)
enumerates the three generations of leptons/quarks.
\begin{table}[h!]
    \label[tab]{tab:MSSM-fields}
    \centering
    \begin{tabular}{|l|ccc|}
        \hline
        Superfield     & Boson field              & Fermion field          &
        Auxiliary field
        \\
        \hline
        \(H_{u/d}^0\)  & \(H_{u/d}^0\)            & \(\wtilde{H}_{u/d}^0\) &
        \(F_{H^0_{u/d}}\)
        \\
        \(H_{u}^{+}\)  & \(H_{u}^{+}\)            & \(\wtilde{H}_{u}^{+}\) &
        \(F_{H^+_{u}}\)
        \\
        \(H_{d}^{-}\)  & \(H_{d}^{-}\)            & \(\wtilde{H}_{d}^{-}\) &
        \(F_{H^-_{d}}\)
        \\
        \hline
        \(l_i\)        & \(\wtilde{l}_{iL}\)      & \(l_i\)                &
        \(F_{l_i}\)
        \\
        \(\wbar{E}_i\) & \(\wtilde{l}_{iR}^\ast\) & \(\wbar{e}_{i}\)       &
        \(F_{\wbar{E}_i}^\ast\)
        \\
        \(\nu_i\)      & \(\wtilde{\nu}_{iL}\)    & \(\nu_i\)              &
        \(F_{\nu_i}\)
        \\
        \hline
        \(u_i\)        & \(\wtilde{u}_{iL}\)      & \(u_i\)                &
        \(F_{u_i}\)
        \\
        \(\wbar{U}_i\) & \(\wtilde{u}_{iR}^\ast\) & \(\wbar{u}_i\)         &
        \(F_{\wbar{U}_i}^\ast\)
        \\
        \(d_i\)        & \(\wtilde{d}_{iL}\)      & \(d_i\)                &
        \(F_{d_i}\)
        \\
        \(\wbar{D}_i\) & \(\wtilde{d}_{iR}^\ast\) & \(\wbar{d}_i\)         &
        \(F_{\wbar{D}_i}^\ast\)
        \\
        \hline
        \(B^0\)        & \(B^0_\mu\)              & \(\bino\)              &
        \(D_{B^0}\)
        \\
        \(W^0\)        & \(W^0_\mu\)              & \(\wino\)              &
        \(D_{W^0}\)
        \\
        \(W^\pm\)      & \(W^\pm_\mu\)            & \(\wtilde{W}^\pm\)     &
        \(D_{W^\pm}\)
        \\
        \hline
    \end{tabular}
    \caption{Table of the MSSM superfields and their component field
        names.Note
        that the fermion fields are left-handed Weyl spinors, in spite of
        any
        \(L\) or
        \(R\) in the subscript.
        The conjugate superfields changes these to right-handed Weyl
        spinors.}
\end{table}

\subsection{The neutralino component fields}
Letting \(\Psi_{\nino} = \pclosed{\bino, \wino, \hinod, \hinou}\) denote a
vector\footnote{I use row vector notation here for convenience. In
    equations
    this is understood to be a column vector.} of the fermion component
fields
in
the \(B^0\) and \(W^0\) vector superfields and the \(H^0_{u/d}\) scalar
superfields
--- the neutralino mass eigenstates are given by
\begin{align}
    \nino = \pclosed{\nino[1], \nino[2], \nino[3], \nino[4]} = N
    \Psi_{\nino},
\end{align}
where \(N\) is a unitary matrix diagonalising the neutralino mass matrix
\(M_{\nino\text{-mass}}\).
\(N\) can be chosen such that the neutralino masses are real and positive.
The neutralino interactions terms are then found in terms in the
superlagrangian that include the vector superfields \(B^0\) and \(W^0\),
and
the
scalar superfields \(H^0_{u/d}\).
For later use, I note that translating from the \(\Psi_{\nino}\)-basis to
the
\(\nino\)-basis, we have that
\begin{align}
    \label[eq]{eq:nino_basis}
    \bino = N_{i1}^\ast \nino[i],\quad \wino = N_{i2}^\ast \nino[i],\quad
    \hinod = N_{i3}^\ast \nino[i],\quad \hinou = N_{i4}^\ast \nino[i],
\end{align}
where a sum over \(i\) is implied.

\section{MSSM Superlagrangian}
Here, I summarise the relevant MSSM superlagrangian terms in which the
interactions are found.
% Idea here is to possibly expand document to include other interactions (like charginos). Right now, this section is bare.
\subsection{Neutralino interactions}
The neutralino interactions appear in the kinetic terms of all the
superfields
that couple to the \(U(1)_Y\) and \(SU(2)_L\) gauge groups, and the Yukawa
terms
that include the Higgs fields in the superpotential. The exception is
neutralino interaction with the charginos, which is found in the
supersymmetric
field strength term.

\section{Interaction Lagrangian}
In the following, I go through how ordinary Lagrangian interaction terms
are
found from superfield terms in the superlagrangian.
\subsection{General superlagrangian terms}
The ordinary interaction Lagrangian is found from integrating over the
Grassmann variables of the superlagrangian. Only terms containing all four
Grassmann variables survive this, so we only need to look for the
superlagrangian terms that include all of
\((\theta\theta)(\theta\theta)^\dagger\).\footnote{Terms with an
    insufficient
    amount of \(\theta\)s are ignored in the following.} Looking first at a
general
kinetic term of a left-handed scalar superfield \(\Phi\) coupled to a
\(U(1)\)
vector superfield \(V\), it has the form\footcite{Binetruy:2006ad}
\begin{align}
    \msub{\L}{kin} = \Phi^\dag \exp{2qV} \Phi.
\end{align}
Using Weyl identities from \cref{app:sec:weyl}, the interactions that
include
the \(\lambda\)-fields of the vector superfield can be found as
\begin{align}
    \label[eq]{eq:lambda-terms}
    \nonumber
    \msub{\L}{kin} & \stackrel{\lambda, \lambda^\dagger}{\supset}
    2q\cclosed{
        A^\ast (\theta\theta)^\dagger(\theta\lambda) \sqrt{2} (\theta\psi)
        +
        \sqrt{2}
        (\theta\psi)^\dagger (\theta\theta)(\theta\lambda)^\dagger A }
    \\
                   & \stackrel{\text{\cref{app:eq:weyl1}}}{=} -\sqrt{2}q
    (\theta\theta)(\theta\theta)^\dagger \cclosed{ (\lambda\psi)A^\ast +
        (\lambda\psi)^\dagger A }.
\end{align}

Ignoring ordinary kinetic terms, the interactions that include the
\(\psi\)-fields of the scalar superfields include
\begin{align}
    \label[eq]{eq:psi-terms}
    \nonumber
    \msub{\L}{kin} & \stackrel{\psi, \psi^\dagger}{\supset} 2q \cclosed{
        A^\ast
        (\theta\theta)^\dagger (\theta\lambda) \sqrt{2} (\theta\psi) +
        \sqrt{2}
        (\theta\psi)^\dagger (\theta\sigma^\mu\theta^\dagger) V_\mu
        \sqrt{2}
        (\theta\psi) + \sqrt{2} (\theta\psi)^\dagger (\theta\theta)
        (\theta\lambda)^\dagger A }
    \\
                   & \stackrel{\text{\cref{app:eq:weyl1,app:eq:weyl2}}}{=}
    q(\theta\theta)(\theta\theta)^\dagger \cclosed{ -\sqrt{2} (\lambda\psi)
        A^\ast
        + (\psi\sigma^\mu\psi^\dagger)V_\mu - \sqrt{2}
        (\psi\lambda)^\dagger A
    },
\end{align}
where we notice that the \(\lambda\psi A\)-interactions are the same as the
ones
we found in \cref{eq:lambda-terms}.
With a Yukawa superpotential on the form
\begin{align}
    W = y_{ij} \Phi_i \Phi \Phi_j,
\end{align}
the superlagrangian looks like
\begin{align}
    \msub{\L}{Yukawa} = y_{ij} (\theta\theta)^\dagger \Phi_i \Phi \Phi_j +
    \cc
\end{align}
Extracting the \(\psi\) fermion interactions from the \(\Phi\) superfield,
we
have
\begin{align}
    \label[eq]{psi-yukawa-terms}
    \nonumber
    \msub{\L}{Yukawa} & \stackrel{\psi, \psi^\dagger}{\supset} y_{ij}
    (\theta\theta)^\dagger \sqrt{2} (\theta\psi) \cclosed{ A_i
    \sqrt{2}(\theta\psi_i) + \sqrt{2}(\theta\psi_j)A_j } + \cc           \\
                      & \stackrel{\text{\cref{app:eq:weyl1}}}{=} -y_{ij}
    (\theta\theta)(\theta\theta)^\dagger \cclosed{ A_i(\psi\psi_j) +
        (\psi_i\psi)A_j + \cc }
\end{align}

\subsection{Neutralino interaction terms}
Now to relate this to the neutralino fields in the MSSM.
\subsubsection{Gaugino parts}
First, I will look at the bino and wino interactions. From electroweak
unification, we have that the coupling \(g\) and \(g'\) of the \(W^a\) and
\(B^0\)
superfields respectively are related by
\begin{align}
    g' = g t_W,
\end{align}
where \(t_W = s_W/c_W \equiv \sin\theta_W/\cos\theta_W\) where \(\theta_W\)
is the
Weinberg mixing angle.
Rewriting using \(\sigma_\pm = \frac{1}{2}\pclosed{\sigma_1 \pm i\sigma_2}\),
and
defining \(W^\pm = W^1 \mp iW^2\) and \(W^0 \equiv W^3\), we have
\begin{align}
    Yg'B^0 + \frac{1}{2}g\sigma^a W^a = g\cclosed{ Yt_W B^0 +
        \frac{1}{2}\sigma_3 W^0 + \frac{1}{2}\sigma_+ W^+ + \frac{1}{2}\sigma_-
        W^- }
\end{align}
So an MSSM superfield doublet \(\Phi = \begin{pmatrix} \Phi_+ \\ \Phi_-
\end{pmatrix}\) charged under \({U(1)}_Y\) with charge \(Y\) and \({SU(2)}_L\)
will
have a kinetic term
\begin{align}
    \L_{\Phi\text{-kin}} = \Phi^\dag \exp{2g\bclosed{ Yt_W B^0 +
            \frac{1}{2}\sigma_3 W^0 + \frac{1}{2}\sigma_+ W^+ +
            \frac{1}{2}\sigma_- W^- }}
    \Phi.
\end{align}
We can extract the fermion interactions from the vector superfields \(B^0\)
and
\(W^0\) using \cref{eq:lambda-terms} to be
\begin{align}
    \nonumber
    \L_{\Phi\text{-kin}} \stackrel{\bino, \wino}{\supset} -\sqrt{2}g
    (\theta\theta)(\theta\theta)^\dagger & \Big\{ Yt_W (\bino \psi_+)
    A_+^\ast
    +
    \frac{1}{2} (\wino \psi_+) A_+^\ast
    \\
    \label{eq:gaugino_doublet_interaction}
                                         & + Yt_W (\bino\psi^-) A_-^\ast
    -\frac{1}{2} (\wino
    \psi_-) A_-^\ast + \cc \Big\}.
\end{align}
In the MSSM, the Dirac fermions are made up from two scalar superfields,
supplying the left- and right-handed components separately.
Both superfields couple to the \({U(1)}_Y\) gauge group with charge
\(Q-I^3\),\footnote{The field supplying the right-handed part has the opposite
    sign charge such that \(\Phi^\dag_R\) and \(\Phi_L\) have the same sign.}
where \(Q\) is the electric charge of the fermion, and \(I^3\) is the weak
isospin;
either \(\pm \frac{1}{2}\) for the superfields supplying left-handed
fermions
or
0 for the superfields supplying the right-handed ones.
Only the left-handed field couples to the \({SU(2)}_L\) gauge group.

\(\wtilde{f}_{L}, f \in f\) and \(\wtilde{f}_R^\ast, \wbar{f} \in \wbar{F}\)

Thus, using \cref{eq:gaugino_doublet_interaction} and \cref{eq:lambda-terms},
the bino and wino interaction with a pair of MSSM fermions formed from a
superfield doublet \(F = \begin{pmatrix} f_+ \\ f_- \end{pmatrix}\) and the
superfields \(\wbar{F}_\pm\) are
\begin{align}
    \label[eq]{F-nino-terms}
    \nonumber
    \L_{\text{EW-kin}} \stackrel{\bino, \wino}{\supset} -\sqrt{2} g
    (\theta\theta)(\theta\theta)^\dagger & \Big\{ \pclosed{Q_+-\frac{1}{2}}t_W
    (\bino f_+) \wtilde{f}_{L+}^\ast + \frac{1}{2} (\wino f_+)
    \wtilde{f}_{L+}^\ast
    \\
    \nonumber
                                         & + \pclosed{Q_-+\frac{1}{2}}t_W
    (\bino
    f_-)\wtilde{f}_{L-}^\ast - \frac{1}{2} (\wino f_-)\wtilde{f}_{L-}^\ast
    \\
                                         & - Q_+ t_W
    (\wtilde{B}^0\wbar{f}_+)^\dagger
    \wtilde{f}_{R-}^\ast - Q_- t_W (\wtilde{B}^0 \wbar{f}_-)^\dagger
    \wtilde{f}_{R-}^\ast + \cc \Big\}.
\end{align}

To get the Lagrangian on a familiar form in terms of Dirac spinors, I will
define the following fields in a familiar way. For clarity, I suppress the
\(\pm\) in the fields, and rather write the final Lagrangian on a form which
generalises to both of them using \(I^3_f = \pm \frac{1}{2}\) where
appropriate.
\begin{align}
    f_D = \begin{pmatrix}
              f \\ \wbar{f}^\dagger
          \end{pmatrix},
    \quad
    \wtilde{B}^0_D = \begin{pmatrix}
                         \wtilde{B}^0 \\ \wtilde{B}^{0 \dagger}
                     \end{pmatrix},
    \quad
    \wtilde{W}^0_D = \begin{pmatrix}
                         \wtilde{W}^0 \\ \wtilde{W}^{0 \dagger}
                     \end{pmatrix},
\end{align}
with conjugates
\begin{align}
    \wbar{f}_D = \begin{pmatrix}
                     \wbar{f} \\ f^\dagger
                 \end{pmatrix}^T,
    \quad
    \wbar{\wtilde{B}}^0_D = \begin{pmatrix}
                                \wtilde{B}^0 \\ \wtilde{B}^{0 \dagger}
                            \end{pmatrix}^T,
    \quad
    \wbar{\wtilde{W}}^0_D = \begin{pmatrix}
                                \wtilde{W}^0 \\ \wtilde{W}^{0 \dagger}
                            \end{pmatrix}^T.
\end{align}
Using \cref{app:sec:weyl2dirac} and integrating (trivially) over the Grassmann
coordinates using \(\int \! \mathrm{d}^4 \theta
\,(\theta\theta)(\theta\theta)^\dagger = 1\), we are left with the ordinary
Lagrangian term
\begin{align}
    \label[eq]{f-bino-wino-terms}
    \nonumber
    \L_{f\bino\wino} = -\sqrt{2} g & \Big\{ \wbar{\wtilde{B}}^0_D \bclosed{
        \pclosed{Q_f - I^3_f} t_W \wtilde{f}_L^\ast P_L - Q_f t_W
        \wtilde{f}_R^\ast P_R
    } f_D                                                                   \\
                                   & + \wbar{\wtilde{W}}^0_D \pclosed{I^3_f
        \wtilde{f}_L^\ast P_L} f_D + \cc \Big\}.
\end{align}
Changing to the \(\nino[]\)-basis we have
\begin{align}
    \label[eq]{nino-sf-f-terms}
    \L_{\nino[]\wtilde{f}f} = -\sqrt{2} g \sum_{i} \wbar{\wtilde{\chi}}^0_i
     &
    \Big\{ \big[ \underbrace{\pclosed{Q_f - I^3_f}t_W N_{i1}^\ast  + I^3_f
            N_{i2}^\ast}_{\equiv C_{\nino[i] \wtilde{f} f}^{L\ast}} \big]
    \wtilde{f}_L^\ast
    P_L \underbrace{- Q_f t_W N_{i1}}_{\equiv C_{\nino[i] \wtilde{f}
                f}^{R\ast}}
    \wtilde{f}_R^\ast P_R \Big\} f_D + \cc
\end{align}

We can generalise this to include squark mixing between the left- and
right-handed squarks into mass eigenstates \(\wtilde{f}_{A=1,2}\), where
\begin{align}
    \begin{pmatrix}
        \wtilde{f}_1 \\
        \wtilde{f}_2
    \end{pmatrix}
    =
    V_{\wtilde{f}} \begin{pmatrix}
                       \wtilde{f}_L \\
                       \wtilde{f}_R
                   \end{pmatrix}
    =
    \begin{bmatrix}
        V_{11} & V_{12} \\
        V_{21} & V_{22}
    \end{bmatrix}
    \begin{pmatrix}
        \wtilde{f}_L \\
        \wtilde{f}_R
    \end{pmatrix}.
\end{align}
This leaves us with the Lagrangian terms
\begin{align}
    \label[eq]{nino-sfA-f-terms}
    \L_{\nino[] \wtilde{f} f} = -\sqrt{2} g \sum_{i} \sum_{A}
    \wbar{\wtilde{\chi}}^0_i & \Big\{ \underbrace{R_{A1} C_{\nino[i]
                \wtilde{f}
                f}^{L}}_{\equiv C_{\nino[i] \wtilde{f}_A f}^{L\ast}} P_L +
    \underbrace{R_{A2}
        C_{\nino[i] \wtilde{f} f}^{R}}_{\equiv C_{\nino[i] \wtilde{f}_A
                f}^{R\ast}} P_R
    \Big\} \wtilde{f}_A^\ast f_D + \cc
    \\
\end{align}

\subsubsection{Higgsino parts}
The Higgs superfield doublets \(H_u = \begin{pmatrix} H_u^+ \\ H_u^0
\end{pmatrix}\) and \(H_d = \begin{pmatrix} H_d^0 \\ H_d^- \end{pmatrix}\)
have
kinetic terms
\begin{align}
    \L_{H\text{-kin}} = H_u^\dag \exp{g\bclosed{\frac{1}{2}\sigma^a W^a +
            \frac{1}{2}t_W B^0}} H_u + H_d^\dag
    \exp{g\bclosed{\frac{1}{2}\sigma^a W^a -
            \frac{1}{2}t_WB^0}} H_d.
\end{align}
These give rise to multiple neutralino interaction terms from the neutral
Higgs
superfields
\begin{align}
    \L_{H^0\text{-kin}} = {H_u^0}^\dag \exp{g\bclosed{-\frac{1}{2}W^0 +
            \frac{1}{2}t_W B^0}} H_u^0 + {H_d^0}^\dag
    \exp{g\bclosed{\frac{1}{2}W^0 -
            \frac{1}{2}t_W B^0}} H_d^0.
\end{align}
Using \cref{eq:psi-terms} we have the higgsino interaction terms (upper
signs
correspond to \(u\), and lower signs to \(d\))
\begin{align}
    \nonumber
    \L_{\wtilde{H}^0\mathrm{-int}} = & \mp \frac{g}{2} (\wtilde{H}^0_{u/d}
    \sigma^\mu \wbar{\wtilde{H}}^0_{u/d})\pclosed{W^0_\mu - t_W B^0_\mu}
    \\
    \label[eq]{eq:L-higgsino_int}
                                     & \pm \frac{g}{\sqrt{2}}
    \bclosed{(\wtilde{W}^0 \wtilde{H}^0_{u/d}) H^{0\,*}_{u/d} - t_W
    (\wtilde{B}^0
    \wtilde{H}^0_{u/d}) H^{0\,*}_{u/d} + \cc}.
\end{align}
From electroweak symmetry breaking, we have that the \(Z\)-boson vector
field
is given by \(Z_\mu = c_W W^0_\mu - s_W B^0_\mu\), meaning we can extract a
\(Z\)-boson interaction from \cref{eq:L-higgsino_int} as
\begin{align}
    \nonumber
    \L_{\wtilde{H}^0Z} & = -\frac{g}{2c_W} Z_\mu
    \bclosed{(\hinou\sigma^\mu\barhinou) - (u \leftrightarrow d)}
    \\
    \label[eq]{eq:L-higgsino_Z}
                       & \stackrel{\mathrm{\cref{app:eq:weyl3}}}{=}
    \frac{g}{2c_W} Z_\mu \bclosed{(\barhinou\wbar{\sigma}^\mu\hinou) - (u
    \leftrightarrow d)}                                             \\
\end{align}
Converting the Weyl spinors into Dirac spinors in the usual way (see
\cref{app:sec:weyl2dirac}), and changing to the \(\nino\)-basis according
to
\cref{eq:nino_basis}, we get
\begin{align}
    \nonumber
    \L_{\wtilde{H}^0Z} & = \frac{g}{2c_W} Z_\mu \sum_{ij}
    \bclosed{N_{i4}N_{j4}^\ast \barnino[i]\gamma^\mu P_L \nino[j] - (4
    \leftrightarrow 3)}
    \\
    \nonumber
                       &
    \stackrel{\mathrm{\cref{app:eq:majorana_chiral_identity}}}{=}
    \frac{g}{4c_W}
    Z_\mu \sum_{ij} \bclosed{N_{i4}N_{j4}^\ast \barnino[i]\gamma^\mu P_L
    \nino[j] -
    N_{i4}N_{j4}^\ast \barnino[j]\gamma^\mu P_R \nino[i] - (4
    \leftrightarrow
    3)}
    \\
    \nonumber
                       & = \frac{g}{4c_W} Z_\mu \sum_{ij}
    \bclosed{N_{i4}N_{j4}^\ast \barnino[i]\gamma^\mu P_L \nino[j] -
    N_{i4}^\ast
    N_{j4} \barnino[i]\gamma^\mu P_R \nino[j] - (4 \leftrightarrow 3)}
    \\
    \label[eq]{eq:L-nino_Z}
                       & = \frac{g}{2} Z_\mu \sum_{ij} \barnino[i]
    \gamma^\mu
    \Big[ \underbrace{\frac{1}{2c_W}\pclosed{N_{i4}N_{j4}^\ast -
    N_{i3}N_{j3}^\ast}}_{\equiv O^{\prime\prime L}_{ij}}P_L
    \underbrace{-\frac{1}{2c_W}\pclosed{N_{i4}^\ast N_{j4} - N_{i3}^\ast
            N_{j3}}}_{\equiv O^{\prime\prime R}_{ij}}P_R \Big] \nino[j],
\end{align}
where in the third line I relabelled the indices of the second term.
\medskip

The Yakawa terms of the superpotential on the form
\begin{equation}
    \L = \pclosed{\theta\theta}^\dagger \cclosed{-y^u_{ij} \bar{U}_i H_u^0
    U_j
    + y^d_{ij} \bar{D}_i H_d^0 D_j} + \cc,
\end{equation}
also contribute Higgsino parts.
Let us just look at the \(u\)-term, as the \(d\)-term can be found by
adequate
replacements detailed later.
Using \cref{psi-yukawa-terms} these are
\begin{align}
    \L = y^u_{ij} \bclosed{\tilde{u}_{Ri}^\ast \pclosed{\tilde{H}_u^0 u_j}
        +
        \tilde{u}_{Lj} \pclosed{\tilde{H}_u^0 \bar{u}_i}} + \cc
\end{align}
We can rewrite this using that \(y^u_{ji} = y^u_{ij}\) and
\((y^u_{ij})^\ast =
y^u_{ij}\) to get
\begin{equation}
    \L = y^u_{ij} \bclosed{\tilde{u}_{Ri}^\ast (\hinou u_j) +
        \tilde{u}_{Li}^\ast (\hinou \bar{u}_j)^\dagger} + \cc
\end{equation}
Converting to the \(\nino[]\)-basis and \(\tilde{q}_A\)-bases as earlier
and
using \cref{app:eq:w2d1,app:eq:w2d2}
\begin{align}
    \L = y^u_{ij} \sum_{k, A} \tilde{u}_A^\ast \barnino[k]
    \bclosed{R^{\tilde{u}}_{A(i+3)} N_{k4}^\ast P_L + R^{\tilde{u}}_{Ai}
        N_{k4}
        P_R} u_{j D} + \cc
\end{align}

% \begin{align}
%     \nonumber
%     \L_{\wtilde{H}^0 \wino \bino H^0} & = \frac{g}{\sqrt{2}} \bclosed{ \pclosed{(\wino\hinou) - t_W (\bino\hinou)} {H^0_u}^\ast + \cc - (u \leftrightarrow d)  }                                         \\
%     \nonumber
%                                       & = \frac{g}{2\sqrt{2}} \sum_{ij} \bclosed{ \pclosed{N_{i2} - t_W N_{i1}}N_{j4}^\ast(\barnino[i] P_L \nino[j]) {H^0_u}^\ast + \cc - (u,4 \leftrightarrow d,3)  }   \\
%                                       & = \frac{g}{2\sqrt{2}} \sum_{ij} \bclosed{ \pclosed{N_{i2} - t_W N_{i1}}(N_{j4}^\ast {H^0_u}^\ast - N_{j3}^\ast {H^0_d}^\ast )(\barnino[i] P_L \nino[j]) + \cc  }
% \end{align}

\subsubsection{Neutralino Feynman rules}
The neutralino interactions with a quark/squark pair, and with a
\(Z\)-boson,
can be summarised in the following Feynman rules. Due to the symmetry of
Majorana particles, we can match either index \(i, j\) with the external
neutralinos, such that the corresponding Feynman rule is multiplied by two.
In
the case of identical neutralino interaction, symmetry dictates that the
factor
of two be removed. The external lines drawn indicate that the Feynman rule
is
identical either way you choose to read it.\footnote{I go more into detail
    on
    how to read Feynman diagrams with neutralinos elsewhere.}
\begin{subequations}
    \begin{align}
        % \vcenter{\hbox{
        %         \inputtikz{qqZ_vertex}
        %     }}
        %  & = -\frac{ig}{c_W} \gamma^\mu \bclosed{C_{Zqq}^L P_L + C_{Zqq}^R P_R}                                \\
        \vcenter{\hbox{
                \inputtikz{chii-chij-Z_vertex}
            }}
         & = ig \gamma^\mu \bclosed{O_{ij}^{\prime\prime L} P_L +
            O_{ij}^{\prime\prime R} P_R}
        \\
        \vcenter{\hbox{
                \inputtikz{q-chi-sq_vertex}
            }}
         & = -i\sqrt{2}g \bclosed{ C_{\nino[i] \wtilde{q}_A q}^{L\ast} P_L
            +
            C_{\nino[i] \wtilde{q}_A q}^{R\ast} P_R }
    \end{align}
\end{subequations}

{\renewcommand{\arraystretch}{2}
\begin{table}
    \centering
    \begin{tabular}{|c|c|}
        \hline
        Variable                            & Value
        \\
        \hline
        \(C_{\nino[i] \wtilde{f}_A f}^{L}\) &
        \(\pclosed{R^{\tilde{f}}_{A1}}^\ast \bclosed{\pclosed{Q_f-I^3_f}
            t_W
            N_{i1} +
            I^3_f N_{i2}}\)
        \\
        \(C_{\nino[i] \wtilde{f}_A f}^{R}\) &
        \(-\pclosed{R^{\tilde{f}}_{A2}}^\ast Q_f t_W N_{i1}^\ast\)
        \\
        \hline
        \(O^{\prime\prime L}_{ij}\)         & \(\frac{1}{2c_W}
        \pclosed{N_{i4}
            N_{j4}^\ast - N_{i3} N_{j3}^\ast}\)
        \\
        \(O^{\prime\prime R}_{ij}\)         & \(-\frac{1}{2c_W}
        \pclosed{N_{i4}^\ast N_{j4} - N_{i3}^\ast N_{j3}}\)
        \\
        \hline
    \end{tabular}
    \caption{A summary of the variables used in the derived Feynman rules
        and
        their definitions.}
    \label[tab]{tab:variable_definitions}
\end{table}
}

\appendix
\section{Weyl identities}\label[app]{app:sec:weyl}
A brief summary of the Weyl identities that have been used in the
derivations
made in this document.
Given two arbitrary left-handed Weyl spinors \(\psi, \chi\), with
right-handed
compliments \(\wbar{\psi}, \wbar{\chi}\), the following identities hold.
\begin{subequations}
    \begin{align}
        \label[eq]{app:eq:weyl1}
        (\theta\psi)(\theta\chi)
         & =
        -\frac{1}{2} (\theta\theta) (\psi\chi)
        \\
        \label[eq]{app:eq:weyl3}
        (\psi \sigma^\mu \wbar\chi)
         & =
        -(\wbar\chi \wbar\sigma^\mu \psi)
        \\
        \label[eq]{app:eq:weyl2}
        (\theta\psi)^\dagger (\theta\sigma^\mu\theta^\dagger) (\theta\psi)
         & =
        \frac{1}{4} (\theta\theta) (\theta\theta)^\dagger
        (\psi\sigma^\mu\psi^\dagger)
    \end{align}
\end{subequations}

\section{Weyl spinors to Dirac spinors}\label[app]{app:sec:weyl2dirac}
We can build a Dirac spinor \(\Psi\) using a left-handed Weyl spinor
\(\psi\)
and a
right-handed Weyl spinor \(\wbar{\psi}^\dagger\) such that
\begin{align}
    \Psi = \begin{pmatrix} \psi \\ \wbar{\psi}^\dagger \end{pmatrix}.
\end{align}
The \(\gamma\)-matrices in the Weyl representation are
\begin{align}
    \gamma^\mu = \begin{bmatrix} 0 & \sigma^\mu \\ \wbar{\sigma}^\mu & 0
                 \end{bmatrix},
\end{align}
so we define the conjugate Dirac spinor
\begin{align}
    \wbar{\Psi} \equiv \Psi^\dag \gamma^0 = \begin{pmatrix} \wbar{\psi} \\
                                                \psi^\dagger\end{pmatrix}^T.
\end{align}
A Majorana fermion is constructed from just one Weyl spinor, such that
\(\psi_L
= \psi_R \equiv \psi\).
The projection operators \(P_{L/R}\) project out the left-handed or
right-handed
Weyl spinors from the Dirac spinor.
The following Weyl spinor products can then be rewritten in terms of Dirac
spinors:
\begin{subequations}
    \begin{align}
        \label{app:eq:w2d1}
        (\wbar{\psi} \phi_{L})                       & = \wbar{\Psi} P_L \Phi
        \\
        \label{app:eq:w2d2}
        (\psi \wbar{\phi})^\dagger                   & = \wbar{\Psi} P_R \Phi
        \\
        \label{app:eq:w2d3}
        (\psi^\dagger \wbar{\sigma}^\mu \phi)        & = \wbar{\Psi} \gamma^\mu
        P_L \Phi
        \\
        \label{app:eq:w2d4}
        (\wbar{\psi} \sigma^\mu \wbar{\phi}^\dagger) & = \wbar{\Psi} \gamma^\mu
        P_R \Phi
    \end{align}
\end{subequations}
Using equation \cref{app:eq:weyl3}, we get the Dirac spinor relation between
two Majorana spinors \(\Psi_M, \Phi_M\)
\begin{align}
    \label[eq]{app:eq:majorana_chiral_identity}
    \wbar{\Psi}_M \gamma^\mu P_{L/R} \Phi_M = -\wbar{\Phi}_M \gamma^\mu P_{R/L}
    \Psi_M
\end{align}

% \section{Electroweak theory}
%     \begin{subequations}
%         \begin{align}
%             B^0_\mu & = c_W A^0_\mu - s_W Z^0_\mu \\
%             W^0_\mu & = s_W A^0_\mu + c_W Z^0_\mu
%         \end{align}
%     \end{subequations}

\end{document}
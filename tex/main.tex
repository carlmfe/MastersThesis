\documentclass[UKenglish]{uiomaster/uiomasterthesis}
\usepackage[UKenglish]{uiomaster/uiomasterfp}

\usepackage{style/style}
\usepackage{style/commands}
\usepackage{style/writingtools}

\graphicspath{{../code/plotting/plots/}} % Compiling main
% \graphicspath{{code/plotting/plots/}} % Compiling subfiles

\begin{document}
\uiomasterfp[author=Carl Martin Fevang,
  colour=green,
  date=\today,
  dept=Department of Physics,
  fac=Faculty of Mathematics and Natural Sciences,
  long,
  subtitle=Cross-Sections for Electroweakino Pair Production in the Complex Minimal Supersymmetric Standard Model,
  supervisor=Are Raklev,
  title=Supersymmetry in Proton--Proton Collisions]

\frontmatter\

\begin{abstract}
  To meaningfully scan the large parameter spaces beyond the Standard Model theories such as the Minimal Supersymmetric Standard Model (MSSM), accurate predictions are required for comparison to experimental data.
  Particularly, in particle collider experiments, fast and accurate cross-section calculations are necessary.
  In this thesis, I compute the leading order cross-sections for pair production of electroweakinos in proton--proton collisions at the Large Hadron Collider at CERN\@.
  Furthermore, next-to-leading order corrections are computed for higgsino-like neutralinos, and all neutralino pair production processes are implemented numerically for arbitrary complex-valued MSSM parameters.
  I explore physical scenarios derived from MSSM parameters, and present cross-section dependence with errors as the parameters are varied.
  Particular emphasis is placed on cross-sections with charge-parity invariance violating complex phases in the MSSM parameters, for which no implementation is currently available at next-to-leading order in the literature.
  I show that in at least one MSSM scenario, the effects of a complex phase is larger on leading order cross-sections than the addition of leading log resummation to next-to-leading order results.
\end{abstract}

\chapter*{Acknowledgements}
As I finish my Master's thesis, there are many who deserve a mention.
First of all, I would like to thank my supervisor Are Raklev, for being a tremendous guide and dedicating so much time in my supervision.
You have been an excellent teacher and a great source of inspiration, academically and beyond.
Next, I must thank my mother, who has always been my greatest supporter.
Your faith in me has never wavered, and I am eternally grateful for the understanding and support you have shown in my dedication to this academic pursuit.
I would also like to take a moment to recognise my roommates, Håkon Olav, Ida and Johan.
During my time writing this thesis, you have not only shown me patience, but you have always provided a consoling and entertaining atmosphere making every day better.

Finally, I could not look back at my time here in the theory section without thinking of the people I have met along the way.
From the people I only met in passing, to those I have seen every day -- you have enriched every day, providing so many thoughtful, fun and interesting discussions and debates, in lunch breaks, next to the coffee machine, or in the numerous social events we have had.
Being surrounded by such smart, insightful and enjoyable people every day has made this experience unforgettable.
In particular, I would like to thank my dear friends whom I have worked with.
Your contributions are evident in this thesis.



\tableofcontents

\listoffigures

\listoftables

\mainmatter\

%%%%%%%%%%%%%%%%%%%%
%%% Introduction %%%
%%%%%%%%%%%%%%%%%%%%
\chapter*{Introduction} \addcontentsline{toc}{chapter}{Introduction}
Since its inception, quantum field theory (QFT) has pushed the boundaries of what we know, and has fostered undoubtedly one of the greatest triumphs of modern physics -- the Standard Model (SM). From a few principles, the SM has been able to make agreement between theory and experiment up to 12 significant digits, or one part in a trillion~\cite{g-2}.
The predictions of the SM come from a set of matter-particles, the fermions, subjected to three internal gauge symmetries~\cite{Yang:1954ek} providing three fundamental forces, and a Higgs mechanism~\cite{Higgs:1964pj} providing masses to the particles.
% The latter being a great triumph of theoretical and experimental physics -- it shows the change in the discovery process of new physics where the search was dictated by the proposal of a mechanism solving theoretical problems, rather than developing the theory to explain anomalous experimental data.

However, there still remains theoretical, as well as experimental, discrepancies in the SM -- it does not explain gravity or cosmological observations such as dark matter or matter-antimatter imbalance, and its structure is somewhat \emph{ad hoc}, without a unifying principle.
Particularly, there is a large discrepancy between the energy scale of the electroweak sector at \(\sim 100\) GeV, and the Planck scale around \(\sim 10^{18}\) GeV, where we know gravity to become relevant.
Without a mechanism for cancellation, the Higgs boson should be sensitive to any physics between these scales, including the Planck scale itself.
The SM provides no solution to how the Higgs mass is of the same order as the physics we have already observed, known as the hierarchy problem~\cite{hierchy1,hierchy2}.
Furthermore, the matter-antimatter imbalance we see in the Universe today can only be explained by a violation of combined charge and parity (CP) invariance and baryon number~\cite{Sakharov:1967dj}, for which there is a mechanism in the SM~\cite{CKM1,CKM2}, insufficient~\cite{SMBaryogenesis}.
In short, the Standard Model is not a complete theory, and the search for beyond the Standard Model (BSM) theories persists.
\medskip

Motivated by solving many of these discrepancies, supersymmetry remains one of the most intensely studied and developed theories addressing them.
Of all the extensions to the Standard Model, its minimal supersymmetric extension, the Minimal Supersymmetric Standard Model (MSSM), provides one of the most explored models for BSM physics.
In this framework, every SM particle receives a \emph{superpartner}, a particle with the same properties but opposite spin-statistics.
That is, every fermion receives a bosonic counterpart and every boson receives a fermionic counterpart.
These superpartners provide possible explanations for dark matter~\cite{susy_dark_matter} and suggest a `grand unification' of the gauge symmetries at some higher energy scale~\cite{susy_grand_unification}.
Particularly, the MSSM, where the lightest supersymmetric particle is absolutely stable, provides a stable, weakly interacting massive particle (WIMP), which is a good candidate for cold dark matter~\cite{WIMP}.
In many MSSM scenarios, the lightest supersymmetric particle will be the lightest \emph{neutralino}, a partner to the Higgs boson, photon and \(Z\)-boson of the SM\@.
More broadly, the Higgs and electroweak boson superpartners are called electroweakinos, and will be the focus of this thesis.
\medskip

As supersymmetry must be broken at low energy scales, agnosticism to a breaking mechanism means that the parameter space of the MSSM becomes vast.
In exploring this parameter space, the most unambiguous mode of discovery is direct production of the superpartners, as in particle collider experiments.
Seeing as the extra particles of the MSSM are predicted to be in a mass range of \(\sim 10\) GeV to a few TeVs -- 10 to 1000 times that of the proton mass -- direct production will require great energies.
This makes the proton--proton collisions at the Large Hadron Collider (LHC) at CERN a natural avenue for discovery.
In this thesis, I will focus on the calculation of the production cross-sections for electroweakino pairs in the setting of proton--proton collisions at the LHC\@.

Particularly, I will focus on the derivation of the electroweakino interactions from the most general MSSM model, and computing the cross-sections with general scenarios in mind.
This means keeping complex and off-diagonal parameter values that allow for CP violating interactions and effects from quark flavour violation in the electroweakino sector.
With general parameter dependence, MSSM scenarios explaining multiple issues such as insufficient CP violation and providing dark matter candidates simultaneously can be explored.
To increase the sensitivity when comparing the theoretical results to experiment, higher order corrections from perturbative QFT are needed, which in the context of proton collisions means we must add corrections to the electroweakino pair production from quantum chromodynamics (QCD).
However, higher order corrections lead to many problems in the perturbative framework we have.
Specifically, divergences appear in the theory, leading to the necessity of reassessing the parameters and fields.
This leads to the procedure of renormalisation, which ultimately renders finite observable values for the cross-sections.

Another point of subtlety coming from the QCD nature of protons is also addressed.
Due to the phenomena of colour confinement and asymptotic freedom~\cite{Gross:1973id}, the fundamental particles of QCD are confined to structures such as the proton, whereas interactions with the proton is described by interaction with the fundamental particles of QCD\@.
This leads to the formulation of the parton model for proton scattering, where the structure of the constituents of the proton is captured by \emph{parton distribution functions}, allowing for the translation of scattering between fundamental QCD particles to the full proton scattering.



\section*{Outline}
This Master's thesis is roughly divided into three parts:
The first two chapters deal with the theoretical framework used in the calculations later on.
The next two chapters tackle the calculation of cross-sections for electroweakino pairs in proton--proton collisions.
Finally, the last chapter deals with numerical implementation of the cross-sections, getting concrete results from different combination of parameter values of the MSSM\@.
The general outline of the thesis follows:
% First, I will outline the theoretical framework for this thesis in two chapters.
% This includes the perturbative quantum field theory framework which I will use for the calculations, and building the supersymmetric model I will use in this framework.
% Next, I go through the theoretical calculation of electroweakino pair production cross-sections in high-energy proton-proton collisions.
% This includes perturbative leading order (LO) results for all possible electroweakino pairs, and next-to-leading order (NLO) results for neutralino pair production specifically.
% Finally, I implement the theoretical calculation numerically, and present a collection of results for a selection of MSSM scenarios.

\begin{itemize}
  \item [\cref{chap:qft}:] To start, I outline the theoretical framework used for the calculations in this thesis. Quantum field theory is introduced, and methods for computing finite observables are presented.
  \item [\cref{chap:susy}:] Next, I introduce supersymmetry, including tools for building manifestly supersymmetric quantum field theories. These are then put to use building the Minimal Supersymmetric Standard Model, and the Feynman rules for electroweakinos are derived from this.
  \item [\cref{chap:part:parton_calculation}:] In this chapter, the calculations for electroweakino pair production from quarks and gluons are computed, up to next-to-leading order.
  \item [\cref{had:chap:hadron_calculation}:] The calculations from the previous chapter are here sewn together to create finite, observable cross-sections for proton--proton collisions using the parton model. General forms for such cross-sections are presented to next-to-leading order.
  \item [\cref{chap:results}:] Finally, numerical implementation of the cross-sections of the previous chapter is presented. Deriving a selection of physical scenarios from the MSSM parameters using external packages, general parameter dependence with error estimates are derived.
\end{itemize}


%%%%%%%%%%%%%%%%%%%%%%%%%%%%
%%% Quantum Field Theory %%%
%%%%%%%%%%%%%%%%%%%%%%%%%%%%
\subfile{chapters/QFT}

%%%%%%%%%%%%%%%%%%%%%
%%% Supersymmetry %%%
%%%%%%%%%%%%%%%%%%%%%
\subfile{chapters/SUSY}

%%%%%%%%%%%%%%%%%%%%%%%%%%%%%%%%%%%%%%%%%%%%%%%%%%%%%%
%%% Electroweakino Pair Production at Parton Level %%%
%%%%%%%%%%%%%%%%%%%%%%%%%%%%%%%%%%%%%%%%%%%%%%%%%%%%%%
\subfile{chapters/parton_calculation}

%%%%%%%%%%%%%%%%%%%%%%%%%%%%%%%%%%%%%%%%%%%%%%%%%%%%%%
%%% Electroweakino Pair Production at Hadron Level %%%
%%%%%%%%%%%%%%%%%%%%%%%%%%%%%%%%%%%%%%%%%%%%%%%%%%%%%%
\subfile{chapters/hadron_calculation}

%%%%%%%%%%%%%%%%%%%%%%%%%
%%% Numerical Results %%%
%%%%%%%%%%%%%%%%%%%%%%%%%
\subfile{chapters/numerical_results}


\backmatter\
%%%%%%%%%%%%%%%%%%
%%% Conclusion %%%
%%%%%%%%%%%%%%%%%%
\chapter*{Conclusion} \addcontentsline{toc}{chapter}{Conclusion}
Finally, some concluding remarks on the concepts and results of the previous chapter.
In this thesis, I have performed the calculation of pair production of electroweakinos in proton--proton collisions in the context of Large Hadron Collider experiments.
Furthermore, next-to-leading order contributions from quantum chromodynamics to higgsino-like neutralino production has been calculated, and implemented numerically, with complete generality in the Minimal Supersymmetric Standard Model (MSSM) parameters of the theory, including charge-parity invariance (CP) violating and quark flavour violating effects from complex and off-diagonal values.
This generality extends the current results found in the literature.
\medskip

A brief summary and outlook on the results are in order.
To start, I went over some of the fundaments of quantum field theory as relevant to the calculations I performed.
This was done largely by using simple examples, referencing the most important results along the way.
Some time was spent developing the notion of the perturbation series and Feynman rules for deriving scattering interactions between particles.
Particular detail was given to the interaction of fermions, including Majorana fermions.
The foundation of the Standard Model interactions was introduced through Yang-Mills gauge theories, before reviewing the regularisation techniques and renormalisation procedures to be used in this thesis.

Next, I reviewed the building of supersymmetric Quantum Field Theories, specifically detailing the elements used to construct the MSSM\@.
This included introduction of the super-Poincaré group, developing superspace as a vehicle for Lagrangian formulation of manifestly supersymmetric theories.
I then developed the Yang-Mills gauge theory in this context, formulating the MSSM as a manifestly supersymmetric mirror to the Standard Model.
From there, I derived the Feynman interaction rules for the neutralinos, and generalised them to the chargino interactions.
Lastly, I outlined the procedure of deriving the electroweakino particle spectrum from the MSSM Lagrangian parameters, presenting an algorithm for the numerical diagonalisation of the neutralino mass matrix that works for complex Lagrangian parameters.
\medskip

The parton-level calculation of the cross-sections were then performed symbolically, using self-written \verb|Mathematica| scripts.
I started by going through the leading order calculation for neutralino pair production, before generalising the result to any electroweakino pair production process.
Next, I showed how the \(Z\)-boson mediated higgsino-like contribution to the neutralino pair production can be factorised into two separate processes, simplifying the computation of next-to-leading order contributions from quantum chromodynamics.
The calculation of these contributions to the inclusive cross-section of neutral higgsino pairs was then performed, including the effects of real emission of gluons and quarks.
These analytic expressions were symbolically compared and verified using \verb|Mathematica| scripts with existing results in the literature.
An assessment was then made of the remaining next-to-leading order contributions to the general neutralino pair production process, outlining the procedure used in Catani-Seymour dipole formalism.
The generalisation of the next-to-leading order contributions to the other electroweakino processes was commented on.

The hadron-level cross-section for proton--proton collisions was then calculated.
I briefly outlined the parton model and factorisation of the cross-section into a soft part between the partons of a single hadron, and a hard part between the partons the colliding hadrons.
The hadron level cross-section in the form of an integral over the parton distribution functions and parton-level cross-sections was then presented.
Furthermore, I outlined the procedure of renormalising the parton distribution functions, handling the last remaining divergences in the parton-level cross-sections.
The complete result for next-to-leading order neutralino pair production at hadron-level was then shown explicitly.
\medskip

Finally, I implemented the general hadron-level cross-section for neutralino pair production to next-to-leading order numerically.
Generating particle spectra from MSSM parameters using \verb|FlexibleSUSY|, I presented two classes of scenarios, one as a complex, CP-violating extension of the SPS1a benchmark point, and another where we have a hierarchical split between light higgsino-like neutralinos and heavy gaugino-like neutralinos.

Correct calculation and numerical implementation was verified by comparing the results with results computed with \verb|Resummino|.
The results from this thesis improves upon \verb|Resummino| by allowing for more general parameter values, specifically complex parameters and more general mixing in the squark sector.
Theoretical errors from scale dependence, and PDF errors from uncertainty in parton distribution functions were also explored and presented, creating accurate uncertainties for a potential scan of the parameter space.
This included investigating the dependence on both the renormalisation and factorisation scales, showing that the two dependencies can cancel when varied together.
In detail, I presented parameter dependence on the complex phase of the Higgs mass parameter \(\mu\), neutralino masses, squark masses and gluon mass, showing the potential for the calculations of this thesis to be used in a broader context of parameter scans of the MSSM\@.
This complex phase is not implemented anywhere else in the literature, and I show that it has greater relative effect on leading order cross-sections than higher order corrections such as those from resummation of large logarithms in a Higgsino scenario.


\section*{Outlook}
Looking forward, there are many avenues are for future exploration possible.
An obvious place to start would be to implement the calculations for other electroweakino pairs than the neutralinos numerically, generalising the code for neutralino pair production that is already implemented.
Seeing as the next-to-leading order contributions were quite significant in the case of higgsino-like neutralino production, another point of interest would be to do the complete calculation of next-to-leading order contributions, for instance by using the Catani-Seymour formalism briefly presented in this thesis.
This would allow for parameter scans where the higgsino-like states and gaugino-like states mix generally, and allow for the computation of the next-to-leading order cross-sections for light gaugino-like states, common in mSUGRA models for instance.
\medskip

A proper scan of the parameter space is perhaps the most enticing possibility from this thesis, making a statistical treatment of collision events simulated using the cross-sections from this thesis and comparing to event data from the ATLAS experiments.
In doing so, cross-section values for many areas of the parameter space would be useful, and so using machine learning to perform regression on the parameter dependence of the cross-sections, as this can significantly improve the computational time necessary to get cross-section results for analysis.



%%%%%%%%%%%%%%%%%%
%%% Appendices %%%
%%%%%%%%%%%%%%%%%%
\appendix
\appendixpage\
\subfile{chapters/appendix}

%%%%%%%%%%%%%%%%%%
%%% References %%%
%%%%%%%%%%%%%%%%%%
\bibliography{references}{}
\bibliographystyle{style/JHEP}


\end{document}
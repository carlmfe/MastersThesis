
%%%%%%%%%%%%%%%%%%%%%%%%%%%%%%%%%%%%%%%%%%%%%%%%%%%%%%%%%%%%%%%%%%%%%%
% writeLaTeX Example: A quick guide to LaTeX
%
% Source: Dave Richeson (divisbyzero.com), Dickinson College
% 
% A one-size-fits-all LaTeX cheat sheet. Kept to two pages, so it 
% can be printed (double-sided) on one piece of paper
% 
% Feel free to distribute this example, but please keep the referral
% to divisbyzero.com
% 
%%%%%%%%%%%%%%%%%%%%%%%%%%%%%%%%%%%%%%%%%%%%%%%%%%%%%%%%%%%%%%%%%%%%%%
% How to use writeLaTeX: 
%
% You edit the source code here on the left, and the preview on the
% right shows you the result within a few seconds.
%
% Bookmark this page and share the URL with your co-authors. They can
% edit at the same time!
%
% You can upload figures, bibliographies, custom classes and
% styles using the files menu.
%
% If you're new to LaTeX, the wikibook is a great place to start:
% http://en.wikibooks.org/wiki/LaTeX
%
%%%%%%%%%%%%%%%%%%%%%%%%%%%%%%%%%%%%%%%%%%%%%%%%%%%%%%%%%%%%%%%%%%%%%%

\documentclass[a4paper]{article}
\usepackage{amssymb,amsmath,amsthm,amsfonts}
\usepackage{dsfont}
\usepackage{multicol,multirow}
\usepackage{calc}
\usepackage{ifthen}
\usepackage[landscape]{geometry}
\usepackage[colorlinks=true,citecolor=blue,linkcolor=blue]{hyperref}

\newcommand{\pclosed}[1]{\left(#1\right)}
\newcommand{\bclosed}[1]{\left[#1\right]}
\newcommand{\cclosed}[1]{\left\{#1\right\}}
\newcommand{\eye}{\mathds{1}}
\newcommand{\trace}[1]{\operatorname{Tr}\cclosed{#1}}


\ifthenelse{\lengthtest { \paperwidth = 11in}}
    { \geometry{top=.5in,left=.5in,right=.5in,bottom=.5in} }
	{\ifthenelse{ \lengthtest{ \paperwidth = 297mm}}
		{\geometry{top=1cm,left=1cm,right=1cm,bottom=1cm} }
		{\geometry{top=1cm,left=1cm,right=1cm,bottom=1cm} }
	}

\pagestyle{empty}
\makeatletter
\renewcommand{\section}{\@startsection{section}{1}{0mm}%
                                {-1ex plus -.5ex minus -.2ex}%
                                {0.5ex plus .2ex}%x
                                {\normalfont\large\bfseries}}
\renewcommand{\subsection}{\@startsection{subsection}{2}{0mm}%
                                {-1explus -.5ex minus -.2ex}%
                                {0.5ex plus .2ex}%
                                {\normalfont\normalsize\bfseries}}
\renewcommand{\subsubsection}{\@startsection{subsubsection}{3}{0mm}%
                                {-1ex plus -.5ex minus -.2ex}%
                                {1ex plus .2ex}%
                                {\normalfont\small\bfseries}}
\makeatother
\setcounter{secnumdepth}{0}
\setlength{\parindent}{0pt}
\setlength{\parskip}{0pt plus 0.5ex}
% -----------------------------------------------------------------------

\title{Weyl to go}

\begin{document}

\raggedright
\footnotesize

\begin{center}
    \Large{\textbf{Weyl to go}} \\
\end{center}
\begin{multicols}{3}
    \setlength{\premulticols}{1pt}
    \setlength{\postmulticols}{1pt}
    \setlength{\multicolsep}{1pt}
    \setlength{\columnsep}{2pt}

    \section{Weyl spinor definitions}
        \begin{align*}
            \psi_A = \begin{pmatrix} \psi_1 \\ \psi_2 \end{pmatrix},\quad \bar{\psi}_A = \begin{pmatrix} \bar{\psi}_1 \\ \bar{\psi}_2 \end{pmatrix}
        \end{align*}
        \begin{align*}
            \psi^A               & = \epsilon^{AB}\psi_B = \begin{pmatrix} \psi_2 \\ -\psi_1 \end{pmatrix}^T                               \\
            \bar{\psi}^{\dot{A}} & = \epsilon^{\dot{A}\dot{B}}\bar{\psi}_B = \begin{pmatrix} \bar{\psi}_2 \\ -\bar{\psi}_1 \end{pmatrix}^T
        \end{align*}
        \begin{align*}
            \epsilon_{AB} & = \epsilon_{\dot{A}\dot{B}} = \begin{pmatrix}
                                                              0 & -1 \\ 1 & 0
                                                          \end{pmatrix} \\
            \epsilon^{AB} & = \epsilon^{\dot{A}\dot{B}} = \begin{pmatrix}
                                                              0 & 1 \\ -1 & 0
                                                          \end{pmatrix}
        \end{align*}
        \begin{align*}
            \pclosed{\psi_A}^\dagger = \bar{\psi}_{\dot{A}}
        \end{align*}
        \begin{align*}
            \psi \chi             & \equiv \psi^A \chi_A
            = \psi_2\chi_1 - \psi_1\chi_2
            = \psi^2\chi^1 - \psi^1\chi^2                                            \\
            \bar{\psi} \bar{\chi} & \equiv \bar{\psi}_{\dot{A}} \bar{\chi}^{\dot{A}}
            = -\bar{\psi}_2\bar{\chi}_1 + \bar{\psi}_1\bar{\chi}_2
            = -\bar{\psi}^2\bar{\chi}^1 + \bar{\psi}^1\bar{\chi}^2
        \end{align*}
        \begin{align*}
            \psi^2       & = -2\psi_1\psi_2            \\
            \bar{\psi}^2 & = 2\bar{\psi}_1\bar{\psi}_2
        \end{align*}
        \begin{align*}
            \sigma^\mu       & = \pclosed{\eye, \vec{\sigma}}                                          \\
            \bar{\sigma}^\mu & = \pclosed{\eye, -\vec{\sigma}}                                         \\
            \sigma^{\mu\nu}  & = \frac{i}{4}(\sigma^\mu \bar{\sigma}^\nu - \sigma^\nu\bar{\sigma}^\mu)
        \end{align*}
        \begin{align*}
            \sigma_1 = \begin{pmatrix} 0 & 1 \\ 1 & 0 \end{pmatrix},\quad
            \sigma_2 = \begin{pmatrix} 0 & -i \\ i & 0 \end{pmatrix},\quad
            \sigma_3 = \begin{pmatrix} 1 & 0 \\ 0 & -1 \end{pmatrix}
        \end{align*}

    \section{Weyl spinor contractions}
        \begin{align*}
            \eta\psi                                                    & = \psi\eta                                                                      \\
            \bar{\eta}\bar{\psi}                                        & = \bar{\psi}\bar{\eta}                                                          \\
            \pclosed{\eta\psi}^\dagger                                  & = \bar{\psi}\bar{\eta}                                                          \\
            \pclosed{\eta\psi}\pclosed{\eta\phi}                        & = -\frac{1}{2} \pclosed{\eta\eta}\pclosed{\psi\phi}                             \\
            \eta\sigma^\mu\bar{\psi}                                    & = -\bar{\psi} \bar{\sigma}^\mu \eta                                             \\
            (\sigma^\mu \bar{\eta})_A (\eta\sigma^\nu\bar{\eta})        & = \frac{1}{2} g^{\mu\nu} \eta_A (\bar{\eta}\bar{\eta})                          \\
            (\eta\sigma^\mu\bar{\eta})(\eta\sigma^\nu\bar{\eta})        & = \frac{1}{2} g^{\mu\nu} (\eta\eta)(\bar{\eta}\bar{\eta})                       \\
            (\eta\sigma^\mu \partial_\mu \bar{\psi}) (\eta\psi)         & = -\frac{1}{2} (\psi\sigma^\mu \partial_\mu \bar{\psi})(\eta\eta)               \\
            (\partial_\mu \sigma^\mu \bar{\eta}) (\bar{\eta}\bar{\psi}) & = -\frac{1}{2} (\partial_\mu \psi \sigma^\mu \bar{\psi}) (\bar{\eta}\bar{\eta}) \\
            (\bar\eta\bar\psi) (\eta\sigma^\mu\bar\eta) (\eta\psi)      & = \frac{1}{4} (\eta\eta) (\bar\eta\bar\eta) (\psi\sigma^\mu\bar\psi)            \\
            \eta \sigma^{\mu\nu} \psi                                   & = -\psi \sigma^{\mu\nu} \eta
        \end{align*}

    \section{Charge conjugation}
        \begin{align*}
            \Gamma^r = \cclosed{\eye, \gamma^5, \gamma^\mu, \gamma^5\gamma^\mu, \gamma^{\mu\nu}}
        \end{align*}
        \begin{align*}
            C^\dagger & = C^{-1} \\
            C^T       & = -C     \\
        \end{align*}
        \begin{align*}
            C^{-1} \Gamma^r C & = \eta^r {\Gamma^r}^T   \\
            \eta^r            & = \begin{cases}
                                      1  & 1 \leq r \leq 6  \\
                                      -1 & 7 \leq r \leq 16
                                  \end{cases}
        \end{align*}
        For \(w = u, v\) and \(w^\prime = v ,u\)
        \begin{align*}
            w = C{\bar{w}}^{\prime T}
        \end{align*}
        \begin{align*}
            \bar{w}_1 \Gamma^r w_2 = -\eta^r \bar{w}_2^\prime \Gamma^r w_1^\prime
        \end{align*}

    \section{Weyl-Dirac translation}
        \begin{align*}
            \Psi_{D} = \begin{pmatrix} \psi_L \\ \bar{\psi}_R \end{pmatrix},\quad \chi_{M} = \begin{pmatrix} \chi \\ \bar{\chi} \end{pmatrix}
        \end{align*}
        \begin{align*}
            \gamma^\mu & = \begin{bmatrix} 0 & \sigma^\mu \\ \bar{\sigma}^\mu & 0 \end{bmatrix}                                                         \\
            \gamma^5   & = i\gamma^0\gamma^1\gamma^2\gamma^3 = \frac{i}{4!} \epsilon^{\mu\nu\rho\sigma} \gamma_\mu \gamma_\nu \gamma_\rho \gamma_\sigma
        \end{align*}
        \begin{align*}
            \bar{\Psi} \equiv \Psi^\dag \gamma^0 = \begin{pmatrix} \psi_R \\ \bar{\psi}_L \end{pmatrix}^T.
        \end{align*}
        \begin{align*}
            \bar{\Psi}_M \gamma^\mu P_{L/R} \Phi_M = -\bar{\Phi}_M \gamma^\mu P_{R/L} \Psi_M
        \end{align*}

    \section{Chiral identities}
        \begin{align*}
            \Psi_{L/R} = \frac{1}{2} \pclosed{\eye \mp \gamma^5}
        \end{align*}
        \begin{align*}
            (\psi_{R} \phi_{L})                    & = \bar{\Psi} P_L \Phi            \\
            (\bar{\psi}_L \bar{\phi}_R)            & = \bar{\Psi} P_R \Phi            \\
            (\bar{\psi}_L \bar{\sigma}^\mu \phi_L) & = \bar{\Psi} \gamma^\mu P_L \Phi \\
            (\psi_R \sigma^\mu \bar{\phi}_R)       & = \bar{\Psi} \gamma^\mu P_R \Phi
        \end{align*}

    \section{Dirac algebra}
            (Both in 4 and \(d = 4-\epsilon\) dimensions)
        \begin{align*}
            \cclosed{\gamma^\mu, \gamma^\nu} & = 2 g^{\mu\nu} \eye_{4\times 4} \\
            \cclosed{\gamma^\mu, \gamma^5}   & = 0
        \end{align*}
        \begin{align*}
            \pclosed{\gamma^\mu}^\dagger & = \gamma^0 \gamma^\mu \gamma^0 \\
            \pclosed{\gamma^5}^\dagger   & = \gamma^5
        \end{align*}
        \begin{align*}
            \gamma^{\mu\nu} = \frac{i}{2} \pclosed{\gamma^\mu \gamma^\nu - \gamma^\nu \gamma^\mu}
        \end{align*}
        % \begin{align*}
        %     \gamma^\mu\gamma_\mu                                        & = d,                                                                                     \\
        %     \gamma^\mu \gamma^\nu \gamma_\mu                            & = (2-d) \gamma^\nu,                                                                      \\
        %     \gamma^\mu \gamma^\nu \gamma^\lambda \gamma_\mu             & = 4g^{\nu\lambda} + (d-4)\gamma^\nu \gamma^\lambda,                                      \\
        %     \gamma^\mu \gamma^\nu \gamma^\lambda \gamma^\rho \gamma_\mu & = (4-d) \gamma^\nu \gamma^\lambda \gamma^\rho - 2 \gamma^\rho \gamma^\lambda \gamma^\nu.
        % \end{align*}
        \begin{align*}
            M_n^{\mu_1\dots \mu_n}         & \equiv \gamma^\nu \gamma^{\mu_1} \dots  \gamma^{\mu_n} \gamma_\nu                                    \\
            M_{n+1}^{\mu_1\dots \mu_{n+1}} & = 2\gamma^{\mu_{n+1}} \gamma^{\mu_1}\dots \gamma^{\mu_n} - M_n^{\mu_1\dots \mu_n} \gamma^{\mu_{n+1}} \\
            M_0                            & = d                                                                                                  \\
            M_1^\mu                        & = (2-d)\gamma^\mu                                                                                    \\
            M_2^{\mu\nu}                   & = 4g^{\mu\nu} \eye_{4\times 4} + (d-4) \gamma^\mu\gamma^\nu                                          \\
            M_3^{\mu\nu\rho}               & = (4-d) \gamma^\mu\gamma^\nu\gamma^\rho - 2\gamma^\rho\gamma^\nu\gamma^\mu
        \end{align*}
        \subsection{Traces}
            (Both in 4 and \(d = 4-\epsilon\) dimensions.)
            \begin{align*}
                T_n^{\mu_1\ldots\mu_n}           & = \trace{\gamma^{\mu_1} \cdots \gamma^{\mu_n}}                                                                        \\
                T_{2n+1}^{\mu_1\ldots\mu_{2n+1}} & = 0                                                                                                                   \\
                T_{2n}^{\mu_1\ldots\mu_{2n}}     & = \sum_{i=2}^{2n} {(-1)}^{i} g^{\mu_1 \mu_i} T_{2n-2}^{\mu_2\ldots\mu_{i-1}\mu_{i+1}\ldots\mu_{2n}}   \qquad n \geq 1 \\
                T_0                              & = 4                                                                                                                   \\
                T_2^{\mu\nu}                     & = 4g^{\mu\nu}                                                                                                         \\
                T_4^{\mu\nu\rho\sigma}           & = 4\pclosed{g^{\mu\nu}g^{\rho\sigma} - g^{\mu\rho}g^{\nu\sigma} + g^{\mu\sigma}g^{\nu\rho}}                           \\
            \end{align*}
            \begin{align*}
                \tilde{T}_n^{\mu_1\ldots\mu_n}           = & \trace{\gamma^5 \gamma^{\mu_1} \cdots \gamma^{\mu_n}}                                                                                   \\
                \tilde{T}_{2n+1}^{\mu_1\ldots\mu_{2n+1}} = & 0                                                                                                                                       \\
                \tilde{T}_{2n}^{\mu_1\ldots\mu_{2n}}     = & \sum_{i=2}^{2n} {(-1)}^{i} g^{\mu_1 \mu_i} \tilde{T}_{2n-2}^{\mu_2\ldots\mu_{i-1}\mu_{i+1}\ldots\mu_{2n}} \qquad n \geq 3               \\
                \tilde{T}_0                              = & 0                                                                                                                                       \\
                \tilde{T}_2^{\mu\nu}                     = & 0                                                                                                                                       \\
                \tilde{T}_4^{\mu\nu\rho\sigma}           = & -4i\epsilon^{\mu\nu\rho\sigma}                                                                                                          \\
                \tilde{T}_6^{\mu\nu\rho\sigma\tau\phi}   = & -4i\Big( g^{\mu\nu} \epsilon^{\rho\sigma\tau\phi} - g^{\mu\rho} \epsilon^{\nu\sigma\tau\phi} + g^{\mu\sigma} \epsilon^{\nu\rho\tau\phi} \\
                                                           & \hspace{4mm} - g^{\mu\tau} \epsilon^{\nu\rho\sigma\phi} + g^{\mu\phi} \epsilon^{\nu\rho\sigma\tau} \Big)
            \end{align*}
            % \begin{align*}
            %     \trace{\gamma^5}      = & \trace{\text{odd \# of }\gamma\text{-matrices}} = 0           \\
            %     \trace{ab}            = & 4g^{ab},                                                      \\
            %     \trace{abcd}          = & 4 \pclosed{g^{ab}\,g^{cd} - g^{ac}\,g^{cd} + g^{ad}\,g^{bc}}, \\
            %     \trace{\gamma^5 abcd} = & -4i \epsilon^{abcd},                                          \\
            % \end{align*}
            % \begin{align*}
            %     \trace{\gamma^{\mu_1} \dots \gamma^{\mu_{2n}}} = \sum_{i=2}^{2n} (-1)^{i} g^{\mu_1 \mu_i} \trace{\gamma^{\mu_2} \dots \gamma^{\mu_{i-1}} \gamma^{\mu_{i+1}} \dots \gamma^{2n}}
            % \end{align*}
            % \begin{align*}
            %     \trace{P_{\scriptstyle{L/R}} ab}   & = \frac{1}{2} \trace{ab},                        \\
            %     \trace{P_{\scriptstyle{L/R}} abcd} & = \frac{1}{2}\trace{abcd} \pm 2i\epsilon^{abcd}, \\
            % \end{align*}


\end{multicols}

\end{document}
\documentclass[../main.tex]{subfiles}

\begin{document}
\chapter{Supersymmetry}


\section{Introduction to Supersymmetry}
\label{susy:sec:introduction}
\phpar[Introduce supersymmetry, what the symmetry is and how it transforms fermionic and bosonic fields through each other.
  Introduce superspace, Grassmann calculus, superfields and superlagrangians.]

Introductory passage\ldots\\
In this section, I will make extensive use of Weyl spinor notation and Grassmann calculcus for which I go in depth in Appendix \needcite.

\subsection*{A Simple Supersymmetric Theory}
To illustrate what supersymmetry looks like in practice, it can be helpful to
look at a simple example. Take a Lagrangian for a massive complex scalar field
\(\phi\) and a massive Weyl spinor field \(\psi\),
\begin{equation}
  \label{susy:eq:simple_thoery}
  \L = (\partial\_{\mu} \phi)(\partial\^{\mu}\phi^\ast) + i \psi \sigma\^{\mu} \partial\_{\mu} \psi^\dagger
  - \abs{m_\phi}^2 \phi \phi^\ast - \frac{1}{2}m_\psi (\psi\psi) - \frac{1}{2} m_\psi^\ast (\psi\psi)^\dagger.
\end{equation}
To impose some symmetry between the bosonic and fermionic degrees of freedom, we want to examine a transformation of the scalar field through the spinor field and vice versa.
Imposing Lorentz invariance a general, infinitesimal, such transformation can be parametrised by
\begin{subequations}
  \begin{align}
    \delta\phi                         & = \epsilon a (\theta\psi),                                                                                                                                   \\
    \delta\phi^\ast                    & = \epsilon a^\ast (\theta\psi)^\dagger,                                                                                                                      \\
    \delta\psi\_{\alpha}               & = \epsilon \pclosed{c (\sigma\^{\mu} \theta^\dagger)\_{\alpha} \partial\_{\mu} \phi + F(\phi, \phi^\ast) \theta\_{\alpha}},                                  \\
    \delta{\psi^\dagger}\_{\dot\alpha} & = \epsilon \pclosed{c^\ast (\theta \sigma\^{\mu})\_{\dot\alpha} \partial\_{\mu} \phi^\ast + F^\ast(\phi, \phi^\ast) \theta\indices*{^\dagger_{\dot\alpha}}},
  \end{align}
\end{subequations}
where \(\epsilon\) is some infinitesimal parameter for the transformation, \(\theta\) is some Grassmann-valued Weyl spinor, \(a, c\) are complex coefficients of the transformation and \(F(\phi, \phi^\ast)\) is some linear function of \(\phi\) and \(\phi^\ast\).
The resulting change in the scalar field part of the Lagrangian is
\begin{equation}
  \label{susy:eq:dLphi}
  \delta\L_\phi / \epsilon = a (\theta\partial\_{\mu}\psi) \pclosed{\partial\^{\mu} \phi^\ast} - a\abs{m_\phi}^2 (\theta\psi) \phi^\ast + \cc,
\end{equation}
and likewise for the spinor field part
\begin{equation}
  \label{susy:eq:dLpsi}
  \delta\L_\psi/\epsilon = -i c^\ast (\psi \sigma\^{\mu}\bar{\sigma}\^{\nu}\theta) \partial\_{\mu}\partial\_{\nu} \phi^\ast + i (\psi \sigma\^{\mu} \theta^\dagger) \partial\_{\mu} F^\ast + m_\psi \bclosed{c (\psi\sigma\^{\mu}\theta^\dagger) \partial\_{\mu} \phi + (\psi\theta) F} + \cc
\end{equation}
The first term in \cref{susy:eq:dLpsi} can be rewritten using the commutativity of partial derivatives and the identity \(\tensor{\pclosed{\sigma\^{\mu} \bar{\sigma}\^{\nu} + \sigma\^{\nu} \bar{\sigma}\^{\mu}}}{_\alpha^\beta} = -2\tensor{g}{^\mu^\nu} \tensor*{\delta}{_\alpha^\beta}\) to get \(i c^\ast \pclosed{\theta \psi} \partial\_{\mu}\partial\^{\mu} \phi^\ast\).
Up to a total derivative, we can then write the change in the spinor part as
\begin{equation}
  \delta\L_\psi/\epsilon = -i c^\ast (\theta\partial\_{\mu}\psi)\partial\^{\mu} \phi^\ast + (\psi \sigma\^{\mu} \theta^\dagger) \partial\_{\mu} \pclosed{i F^\ast + m_\psi c \phi} + m_\psi (\theta\psi) F + \cc.
\end{equation}
The change of the total Lagrangian (again up to a total derivate) can then be grouped as
\begin{align}
  \nonumber
  \delta\L/\epsilon = & \pclosed{a-ic^\ast} (\theta\partial\_{\mu}\psi) \pclosed{\partial\^{\mu} \phi^\ast}
  + (\psi \sigma\^{\mu} \theta^\dagger) \partial\_{\mu} \pclosed{i F^\ast + m_\psi c \phi}                  \\
                      & + (\theta\psi) \pclosed{a\abs{m_\phi}^2\phi^\ast + m_\psi F} + \cc,
\end{align}
giving us three different conditions for the action to be invariant:
\begin{subequations}
  \begin{align}
    a - ic^\ast                          & = 0, \\
    i F^\ast + m_\psi c \phi             & = 0, \\
    a\abs{m_\phi}^2 \phi^\ast + m_\psi F & = 0.
  \end{align}
\end{subequations}
This is fulfilled if
\begin{subequations}
  \label{susy:eq:susy_transformation_requirements}
  \begin{align}
    c               & = i a^\ast,                 \\
    F               & = -a m_\psi^\ast \phi^\ast, \\
    a\abs{m_\phi}^2 & = a^\ast \abs{m_\psi}^2.
  \end{align}
\end{subequations}
What is interesting is the last condition, because it requires \(a\) to be real, as both \(\abs{m_\phi}^2\) and \(\abs{m_\psi}^2\) are real, but also requires \(\abs{m_\phi}^2 = \abs{m_\psi}^2\).
For the theory to be supersymmetric in this sense, the masses of the boson and fermion must be the same!
Since the phase of \(m_\phi\) does not appear in the Lagrangian, we are free to set \(m_\phi = m_\psi \equiv m\), suppressing any mass subscripts henceforth.

Revisiting \(F\), it can be introduced as an auxiliary field to bookkeep the
supersymmetry transformation. By including the non-dynamical term to the
Lagrangian \(\L_F = F^\ast F + m F \phi + m^\ast F^\ast \phi^\ast\), we make
sure \(F\) takes the correct value in the transformation from its equation of
motion \(\pd[F]{\L} = F^\ast + m \phi \mbeq 0\). Inserting \(F\) back into the
Lagrangian reproduces the mass term of the scalar field, allowing us to write
the original Lagrangian as
\begin{align}
  \L = & (\partial\_{\mu} \phi)(\partial\^{\mu}\phi^\ast) + i \psi \sigma\^{\mu} \partial\_{\mu} \psi^\dagger + F^\ast F + m F \phi + m^\ast F^\ast \phi^\ast - \frac{1}{2}m (\psi\psi) - \frac{1}{2}m^\ast (\psi\psi)^\dagger,
\end{align}
with the \emph{supersymmetry transformation} rules
\begin{subequations}
  \begin{align}
     & \delta\phi       = \epsilon (\theta \psi),                                                                                                                                       &
     & \delta\phi^\ast = \epsilon (\theta\psi)^\dagger,                                                                                                                                   \\
     & \delta\psi\_{\alpha} = \epsilon \pclosed{-i(\sigma\^{\mu} \theta^\dagger)\_{\alpha} \partial\_{\mu} \phi  + F \theta\_{\alpha}},                                                 &
     & \delta\psi\indices*{^\dagger_{\dot\alpha}} = \epsilon \pclosed{i(\theta \sigma\^{\mu})\_{\dot\alpha} \partial\_{\mu} \phi^\ast + F^\ast \theta\indices*{^\dagger_{\dot\alpha}}},   \\
     & \delta F         = i \epsilon \pclosed{\partial\_{\mu} \psi \sigma\^{\mu} \theta^\dagger},                                                                                       &
     & \delta F^\ast = -i \epsilon \pclosed{\theta \sigma\^{\mu} \partial\_{\mu} \psi^\dagger},
  \end{align}
\end{subequations}
where I have set \(a=1\) without loss of generality,\footnote{It can be absorbed by a redefinition of the parameter \(\epsilon\) for instance.} and found the appropriate transformation law for \(F\) such that the Lagrangian is invariant up to total derivatives.
The dynamics of this Lagrangian are the same as before, but the supersymmetry is now made manifest, i.e.~the transformation is free of any dependence on the contents of the Lagrangian as we had in \cref{susy:eq:susy_transformation_requirements}.

In fact, one can show that a general supersymmetric Lagrangian consisting of a
scalar field and a fermion field can be written
\begin{equation}
  \L = (\partial\_{\mu} \phi)(\partial\^{\mu}\phi^\ast) + i \psi \sigma\^{\mu} \partial\_{\mu} \psi^\dagger + F^\ast F + \cclosed{m F \phi - \frac{1}{2}m (\psi\psi) - \lambda \phi (\psi\psi) + \cc}
\end{equation}
up to renormalisable interactions.





\section{The Super-Poincaré Group}
\phpar[Introduce the super-Poincaré algebra, and superspace as a vessel for manifestly supersymmetric theories. Lead into superfields, and general supersymmetric superlagrangians.]

To introduce more involved supersymmetric QFTs than our simple example from \cref{susy:sec:introduction}, it will be useful to first introduce a framework that will manifestly carry the supersymmetry.
This will alleviate the need to figure out the correct transformation laws, and the constraints they may carry to the parameters of the theory.
To this end, I will outline a common way of introducing supersymmetric theories -- extending our fields from representations of the Poincaré group of cooridinate transformations to the super-Poincaré group.
This will hopefully give an algebraic geometrical understanding to \emph{superfields} as the building blocks of a supersymmetric field theory.

\subsection{The Poincaré and Super-Poincaré Algebras}
As we have already seen in \cref{qft:sec:yang-mills}, sets of transformations for a symmetry can be described by a group.
To introduce supersymmetry in this context, it will be clearer to study the \emph{generators} of the algebra of the group, so I would like to take a moment to motivate this change of perspective, before describing the fundamental symmetries we will be using.

The group describing the basic set of \emph{coordinate transformations} under which the fields theories we will consider are symmetric is called the \emph{Poincaré group}, denoted \(P\).
Theories that are symmetric under this group will be manifestly relativistic, and will exhibit the ordinary freedom in choice of coordinate system.
The Poincaré group consists of any transformation of space-time coordinates \(x^\mu\) such that
\begin{equation}
  {x^\prime}\^\mu = \Lambda\indices{^\mu_\nu} x\^\nu + a\^\mu,
\end{equation}
for a real, orthogonal \(4\times 4\) matrix \(\Lambda\) and real numbers \(a\^\mu\).
As a group it is the semi-direct product of Lorentz group \(O(1,3)\) and group of 4D space-time translations \(T(1,3)\)
\begin{equation}
  P \equiv O(1,3) \rtimes T(1,3).
\end{equation}
For completeness, the semi-direct product is defined such that the product of two group elements \((\Lambda_1, p_1), (\Lambda_2, p_2) \in P\) where \(\Lambda_1, \Lambda_2 \in O(1,3)\) and \(p_1, p_2 \in T(1,3)\) is
\begin{equation}
  (\Lambda_1, p_1) \circ (\Lambda_2, p_2) \equiv \pclosed{\Lambda_1 \circ_O \Lambda_2, p_1 \circ_T \Lambda_1(p_2)},
\end{equation}
where we understand \(\circ_{O/T}\) as the group multiplication operations of \(O(1,3)\) and \(T(1,3)\) respectively.\footnote{We see also that \(O(1,3)\) must also be a map \(T(1,3) \to T(1,3)\).
  We will later see that this means that the generators of translations are in a representation of the Lorentz group.}

For our purposes, it will suffice to work simply with the local structure of the Poincaré group, \needcite[and being Lie groups], this can be reproduced with the exponential map we have used already in \cref{qft:eq:exponential_map} \(\text{exp} : \mathfrak{g} \to G\), where \(\mathfrak{g}\) is the \emph{Lie algebra} of the Lie group \(G\).
In this way, the algebra is said to \emph{generate} the group, and a basis set \(\cclosed{T^a}\) of the algebra \(\mathfrak{g}\) is said to be the \emph{generators} of the group.\footnote{The algebra of a Lie group can be shown to be a vector space, and as such there exists a basis set spanning the algebra.}
Accordingly, the local behaviour of the group can be inferred simply from the properties of the generators \(T^a\).
The generators of the Poincaré group can be structured by an antisymmetric Lorentz tensor \(M^{\mu\nu}\), and a four-vector \(P\^\mu\).
The properties of the algebra these generators span can be inferred from their commutation relations
\begin{subequations}
  \begin{align}
    \label{susy:eq:poincare_commutation1}
    \commutator{P\^\mu}{P\^\nu} =             & 0,                                                                                                                                                \\
    \label{susy:eq:poincare_commutation2}
    \commutator{M^{\mu\nu}}{P^\rho} =         & i\pclosed{\metric{\mu}{\sigma} P^\nu - \metric{\nu}{\sigma} P^\mu},                                                                               \\
    \label{susy:eq:poincare_commutation3}
    \commutator{M^{\mu\nu}}{M^{\rho\sigma}} = & i\pclosed{\metric{\mu}{\rho}M^{\nu\sigma} - \metric{\mu}{\sigma}M^{\nu\rho} - \metric{\nu}{\rho}M^{\mu\sigma} + \metric{\nu}{\sigma}M^{\nu\rho}}.
  \end{align}
\end{subequations}

To construct the super-Poincaré group, we can then just extend the algebra, and the rest of the group will follow.
This is done by extending the Lie algebra to a \emph{graded Lie superalgebra} by adding new generators.
A \emph{graded Lie superalgebra} is constructed from two vector spaces \(\mathfrak{l}_0, \mathfrak{l}_1\) and is denoted \(\mathfrak{l}_0 \oplus \mathfrak{l}_1\).
It is itself a vector space with a bilinear operation such that for any elements \(x_i \in \mathfrak{l}_i\) we have
\begin{align}
   & x_j \circ x_j \in \mathfrak{l}_{i+j \text{ mod } 2}, \tag{grading}                                                                     \\
  \label{susy:eq:supersymmetrisation}
   & x_i \circ x_j = -(-1)^{i\cdot j} x_j \circ x_i, \tag{supersymmetrisation}                                                              \\
  \nonumber
   & x_i \circ (x_j \circ x_k) (-1)^{i\cdot k} + x_j \circ (x_k \circ x_i) (-1)^{j \cdot i} + x_k\circ (x_i \circ x_j) (-1)^{k\cdot j} = 0. \\
  \tag{generalised Jacobi identity}
\end{align}
I note that in this case, \(\mathfrak{l}_0\) acts as an ordinary Lie algebra where \(\circ\) is the ordinary commutator, and \(\mathfrak{l}_1\) gets anti-commutator relations rather than commutator relations.\footnote{This can be seen from supersymmetrisation as for any \(x_1, x_1^\prime \in \mathfrak{l}_1\) we have that \(x_1 \circ x_1^\prime = x_1^\prime \circ x_1\).}

The \emph{super-Poincaré algebra}, denoted \(\mathfrak{sp}\), is the graded Lie superalgebra resulting from the Poincaré algebra \(\mathfrak{p}\) and the vector space \(\mathfrak{q}\).
Here \(\mathfrak{p}\) is the Lie algebra of the Poincaré group \(P\) and \(\mathfrak{q}\) is the vector space spanned by the generators \(Q_\alpha, Q^\dagger_{\dot\alpha}\) that form two Weyl spinors.
In addition to the commutation relations \cref{susy:eq:poincare_commutation1,susy:eq:poincare_commutation2,susy:eq:poincare_commutation3}, the Poincaré superalgebra is specified by the (anti-)commutator relations
\begin{subequations}
  \begin{align}
    \commutator{Q\_\alpha}{P\^\mu} = \commutator{Q\indices*{^\dagger_{\dot\alpha}}}{P\_\mu} =               & 0                                                          \\
    \commutator{Q\_\alpha}{M\^{\mu\nu}} =                                                                   & \pclosed{\sigma\^{\mu\nu}}\indices{_\alpha^\beta} Q\_\beta \\
    \anticommutator{Q\_\alpha}{Q\_\beta} = \anticommutator{Q^\dagger_{\dot\alpha}}{Q^\dagger_{\dot\beta}} = & 0,                                                         \\
    \anticommutator{Q\_\alpha}{Q^\dagger_{\dot\beta}} =                                                     & 2(\sigma^\mu)\_{\alpha \dot\beta} P\_\mu
  \end{align}
\end{subequations}
where \(\sigma\^{\mu\nu} = \frac{i}{4}\pclosed{\sigma\^\mu \bar{\sigma}\^\nu - \sigma\^\nu \bar{\sigma}\^\mu}\), \(\sigma^\mu = \pclosed{\eye, \sigma^i}\), \(\bar\sigma^\mu = \pclosed{\eye, -\sigma^i}\) and \(\sigma^i\) are the Pauli matrices.


\subsection{Superspace}
The idea behind \emph{superspace} is to create a coordinate system for which supersymmetry transformation manifest as coordinate transformations similarly to the way Poincaré transformations work on ordinary space-time coordinates.
To this end, we can start by considering a general element of the super-Poincaré group \(g \in SP\); it can be parametrised through the exponential map like this.
\begin{equation}
  g = \exptext{ix\^{\mu} P\_{\mu} + i(\theta Q) + i (\theta Q)^\dagger + \frac{i}{2}\tensor{\omega}{_\mu_\nu}\tensor{M}{^\mu^\nu}},
\end{equation}
where \(x^\mu, \theta\^{\alpha}, \theta\indices*{^\dagger_{\dot\alpha}}, \tensor{\omega}{_\mu_\nu}\) parametrise the group, and \(P_\mu, Q\_{\alpha}, Q\indices*{^\dagger^{\dot\alpha}}, \tensor{M}{^\mu^\nu}\) are the generators of the group as we have already seen.
Since the parameters \(x\^{\mu}, \theta\^{\alpha}, \theta\indices*{^\dagger_{\dot\alpha}}\) live in irreps of the Lorentz algebra (four-vector and Weyl spinor representations respectively) generated by \(\tensor{M}{^\mu^\nu}\), the effect of the Lorentz part of the super-Poincaré group on the parameters can be determined easily.
Likewise, the parameters \(\tensor{\omega}{_\mu_\nu}\) are in a trivial representation of the algebra generated by \(P\_{\mu}, Q\_{\alpha}, Q\indices*{^\dagger^{\dot\alpha}}\), and need not then be considered.
It is therefore expedient to create a space with \(x\^{\mu}, \theta\^{\alpha}, \theta\indices*{^\dagger_{\dot\alpha}}\) as the coordinates, modding out the Lorentz algebra part.


We create superspace as a coordinate system with coordinates \(z\^{\pi} = (x\^{\mu}, \theta\^{\alpha}, \theta\indices*{^\dagger_{\dot\alpha}})\), and look at how they transform under super-Poincaré group transformations.
A function \(F(z)\) on superspace can then be written using the generators \(K\_\pi = (P\_\mu, Q\_\alpha, Q\indices*{^\dagger^{\dot\alpha}})\) as \(F(z) = \exptext{iz\^\pi K\_\pi} F(0)\).
Applying a super-Poincaré group element without the Lorentz generators \(\bar{g}(a, \eta) = \exptext{ia\^\mu P\_\mu + i (\eta Q) + i (\eta Q)^\dagger}\) we have
\begin{equation}
  F(z^\prime) = \exptext{i{z^\prime}\^\pi K\_\pi} F(0) = \exptext{ia\^\mu P\_\mu + i (\eta Q) + i (\eta Q)^\dagger} \exptext{iz\^\pi K\_\pi} F(0),
\end{equation}
which by the \needcite[Baker-Campbell-Hausdorff] formula (BCH) gives to first order in the commutators
\begin{equation}
  {z^\prime}\^\pi K\_\pi = (x\^\mu + a\^\mu) P\_\mu + (\theta\^\alpha + \eta\^\alpha)Q\_\alpha + (\theta\indices*{^\dagger_{\dot\alpha}} + \eta\indices*{^\dagger_{\dot\alpha}})Q\indices*{^\dagger^{\dot\alpha}} + \frac{i}{2}\bclosed{a\^\mu P\_\mu + (\eta Q) + (\eta Q)^\dagger, z\^\pi K\_\pi} + \ldots
\end{equation}
Now, \(P\_\mu\) commutes with all of \(K\_\pi\), and \(Q\_\alpha\) (\(Q\indices*{^\dagger^{\dot\alpha}}\)) anti-commute with themselves, for every combination of different \(\alpha\) (\(\dot\alpha\)), so the only relevant part of the commutator is
\begin{equation}
  \commutator{(\eta Q)}{(\theta Q)^\dagger} + \commutator{(\eta Q)^\dagger}{(\theta Q)} = -\eta\^\alpha \anticommutator{Q\_\alpha}{Q\indices*{^\dagger_{\dot\alpha}}} \theta\indices*{^\dagger^{\dot\alpha}} + (\eta \leftrightarrow \theta) = -2(\eta \sigma\^\mu \theta^\dagger)P\_\mu + (\eta \leftrightarrow \theta).
\end{equation}
Since this commutator is proportional to \(P\_\mu\) which in turn commutes with everything, all higher order commutators of BCH vanish, and we can conclude that the transformed coordinates \({z^\prime}\^\pi\) are given by
\begin{equation}
  \label{susy:eq:zprime}
  {z^\prime}\^\pi = \pclosed{x\^\mu + a\^\mu + i(\theta \sigma\^\mu \eta^\dagger) - i(\eta \sigma\^\mu \theta^\dagger), \theta\^\alpha + \eta\^\alpha, \theta\indices*{^\dagger_{\dot\alpha}} + \eta\indices*{^\dagger_{\dot\alpha}}}.
\end{equation}
This gives us a differential representation of the \(K\_\pi\) generators as
\begin{subequations}
  \begin{align}
    P\_\mu                            & = -i \partial\_\mu,                                                                   \\
    Q\_\alpha                         & = - (\sigma\^\mu \theta^\dagger)\_\alpha \partial\_\mu - i \partial\_\alpha,          \\
    Q\indices*{^\dagger_{\dot\alpha}} & = -(\theta \bar{\sigma}\^\mu)\_{\dot\alpha} \partial\_\mu - i \partial\_{\dot\alpha}.
  \end{align}
\end{subequations}

Now, to see what the these functions of superspace look like, we can expand \(F(z)\) in terms of the coordinates \(\theta\^\alpha, \theta\indices*{^\dagger_{\dot\alpha}}\), as these expansions are finite due to the fact that none of these coordinates can appear more than once per term.
Demanding that the function \(F(z)\) be invariant under Lorentz transformations, the \(x\^\mu\)-dependendt coefficients of the expansion must transform such that each term is a scalar (or fully contracted Lorentz structure).
This limits a general such function of superspace to be written as
\begin{align}
  \label{susy:eq:superspace_function}
  \nonumber
  F(z) = & f(x) + \theta\^\alpha \phi\_\alpha(x) + \theta\indices*{^\dagger_{\dot\alpha}} \chi\indices*{^\dagger^{\dot\alpha}}(x) + (\theta\theta) m(x) + (\theta\theta)^\dagger n(x)                                                                              \\
         & + (\theta \sigma\^\mu \theta^\dagger) V\_\mu(x) + (\theta\theta) \theta\indices*{^\dagger_{\dot\alpha}} \lambda\indices*{^\dagger^{\dot\alpha}}(x) + (\theta\theta)^\dagger \theta\^\alpha \psi\_\alpha(x) + (\theta\theta)(\theta\theta)^\dagger d(x).
\end{align}

\subsection{Superfields}
To construct a manifestly supersymmetric theory, it will be useful to start with finding representations of the super-Poincaré group.
This is exactly what we have already done; the functions on superspace find themselves in the representation space of a differential representation of the \(K\_\pi\) generators of the super-Poincaré group, and a scalar representation of the remaining Lorentz generators (i.e. the Lorentz generators leave the superspace functions unchanged).
Inside the general function on superspace \cref{susy:eq:superspace_function}, we find many component functions in different representation spaces of the Lorentz group.
Furthermore, supersymmetry transformations transform these fields into one another.
This seems like an ideal vessel for constructing supersymmetric fields theories.

We define the \emph{superfield} \(\Phi\) as an operator-valued function on
superspace.\footnote{For our purposes, it suffices to look at them simply as complex valued functions, but strictly speaking, they are operator-valued in a quantised field theory.} The general one \cref{susy:eq:superspace_function} is in a
reducible representation space of the super-Poincaré group, so we define three
\emph{irreducible} representations that will be useful going forward:\footnote{I will not prove that these in fact are irreducible representations.}
\begin{align}
  \textit{Left-handed scalar superfield:}  &  & \bar{D}\_{\dot\alpha}\Phi & = 0,    \\
  \textit{Right-handed scalar superfield:} &  & D\_{\alpha}\Phi^\dagger   & = 0,    \\
  \label{susy:eq:vector_superfield}
  \textit{Vector superfield:}              &  & \Phi^\dagger              & = \Phi.
\end{align}
Here the dagger operation refers to complex conjugation, and the differential operators \(D\_\alpha, \bar{D}\_{\dot\alpha}\) are defined as
\begin{subequations}
  \begin{align}
    D\_\alpha             & = \partial\_\alpha + i(\sigma\^\mu \theta^\dagger)\_\alpha \partial\_\mu,      \\
    \bar{D}\_{\dot\alpha} & = -\partial\_{\dot\alpha} - i (\theta\sigma\^\mu)\_{\dot\alpha} \partial\_\mu.
  \end{align}
\end{subequations}
These differential operators are covariant differentials in the sense that the commute with supersymmetry transformations, i.e. \(D\_\alpha F(z) \to {D^\prime}\_\alpha (\bar{g} F(z)) = \bar{g} \pclosed{D\_\alpha F(z)}\)
Collectively, the left- and right-handed scalar superfields are referred to as \emph{chiral superfields}.

\subsection{Superlagrangian}
We are now ready to define the action of a quantum field theory on superspace.
Letting the superlagrangian \(\L\) be a function superfields \(\cclosed{\Phi}\), their derivatives and of superspace coordinates \(z^\pi\) to the reals, the action becomes the functional
\begin{equation}
  S\bclosed{\cclosed{\Phi_i}} = \integral{^4x \mathrm{d}^4\theta} \L\pclosed{\cclosed{\Phi}, \cclosed{\pd[z^\pi]{\Phi}}, z}.
\end{equation}
The ordinary Lagrangian density as a function of the component fields \(\phi\) of the superfields is recovered simply by integrating over the Grassmann coordinates:
\begin{equation}
  \L_\text{ordinary}\pclosed{\cclosed{\phi}, \cclosed{\pd[x^\mu]{\phi}}, x} = \integral{^4\theta} \L\pclosed{\cclosed{\Phi}, \cclosed{\pd[z^\pi]{\Phi}}, x, \theta, \theta^\dagger}
\end{equation}

Effectively what the integration over the Grassmann coordinates does is that it picks out only the terms saturated with Grassmann variables, i.e. terms proportional to \((\theta\theta)(\theta\theta)^\dagger\).
These are the terms that are invariant under supersymmetry transformations -- looking back at \cref{susy:eq:zprime}, we can see that apart from the translation parametrised by \(a^\mu\), the rest of the transformation is proportional to some \(\theta\).
If the term to be transformed is already saturated with \(\theta\)'s, this part of the transformation will not result in any change.

For details on how the calculus of Grassmann coordinates is defined, I refer to Appendix \needcite.
\\\comment{Perhaps this is the place for a superspace calculus interlude?}



\subsection{Revisiting our Simple Supersymmetric Theory}
Now that we have developed a structure for creating manifestly supersymmetric theories, we can take a moment to revisit our simple theory from \cref{susy:eq:simple_thoery} to see what it would look like within the superspace framework.
We can use a left-handed scalar superfield \(\Phi\) as the vessel for our scalar field \(\phi\), fermionic field \(\psi\) and auxiliary field \(F\):
\begin{align}
  \nonumber
  \Phi(\theta, \theta^\dagger, x) = & \phi(x) + i(\theta\sigma^\mu \theta^\dagger) \partial_\mu \phi(x) - \frac{1}{4} (\theta\theta)(\theta\theta)^\dagger \dA \phi(x)      \\
                                    & + \sqrt{2}(\theta \psi(x)) - \frac{i}{\sqrt{2}} (\theta\theta) (\partial_\mu \psi(x) \sigma^\mu \theta^\dagger) + (\theta\theta)F(x).
\end{align}
\todo{Derive or cite the field content of chiral superfields (and perhaps vector superfields too).}
The kinetic terms are reproduced through
\begin{align}
  \nonumber
  \L_\text{kin} = & \integral{^4\theta} \Phi^\dagger \Phi = \integral{^4\theta} \Bigg\{-\frac{1}{4}\pclosed{\phi^\ast \dA \phi + \phi \dA \phi^\ast} + (\theta \sigma^\mu \theta^\dagger)(\theta \sigma^\nu \theta^\dagger) \partial_\mu\phi^\ast \partial_\nu\phi \\
  \nonumber
                  & -i \bclosed{(\theta \psi)^\dagger (\theta\theta) (\partial_\mu \psi \sigma^\mu \theta^\dagger) - (\theta\theta)^\dagger (\theta \sigma^\mu \partial_\mu \psi^\dagger) (\theta \psi)} + (\theta\theta)(\theta\theta)^\dagger F^\ast F   \Bigg\} \\
  =               & (\partial_\mu \phi) (\partial^\mu\phi^\ast) + i(\psi \sigma^\mu \partial_\mu \psi^\dagger) + F^\ast F.
\end{align}
The remaining mass term can be recreated by the superlagrangian term \(\frac{m}{2} (\theta\theta)^\dagger \Phi\Phi + \frac{m^\ast}{2} (\theta\theta) \Phi^\dagger\Phi^\dagger\), which yields
\begin{align}
  \nonumber
  \L_\text{mass} = & \integral{^4\theta} \cclosed{\frac{m}{2} (\theta\theta)^\dagger \Phi\Phi + \cc} = \integral{^4\theta} \cclosed{\frac{m}{2} (\theta\theta)^\dagger \pclosed{2\phi (\theta\theta) F + 2(\theta\psi)(\theta\psi)} + \cc} \\
  =                & m \phi F + m^\ast \phi^\ast F^\ast + \frac{m}{2}(\psi\psi) + \frac{m^\ast}{2} (\psi\psi)^\dagger.
\end{align}




\section{Minimal Supersymmetric Standard Model}
\phpar[Introduce the field content of the MSSM and explain conventions and names.
  Talk about the relevant parts of the superlagrangian.]

\begin{TODO}
  \item Talk about Wess-Zumino gauge.
\end{TODO}

\subsection{Superymmetric Yang-Mills Theory}
\label{susy:ssec:susy_yang_mills}
Before getting into the MSSM content, we must introduce what Yang-Mills theory
looks like at a superlagangian level. We define a \emph{supergauge
  transformation} of a left-handed scalar superfield multiplet \(\Phi\)
analogously to the ordinary case \cref{qft:eq:gauge_transformation}
\begin{equation}
  \label{susy:eq:supergauge_transformation}
  \Phi \to \exptext{i \Lambda} \Phi,
\end{equation}
where \(\Lambda \equiv \Lambda^a T^a\), \(\Lambda^a\) are the parameters of the transformation and \(T^a\) are again the generators of the gauge group.
To get a sense of what these parameters are, we can require the transformed superfield to be left-handed
\begin{align*}
  D_{\dot{\alpha}}^\dagger \exptext{i\Lambda} \Phi = & i\pclosed{D_{\dot{\alpha}}^\dagger \Lambda^a} T^a \exptext{i\Lambda^a T^a} \Phi + \exptext{i\Lambda^a T^a} D_{\dot{\alpha}}^\dagger \Phi \\
  =                                                  & i\pclosed{D_{\dot{\alpha}}^\dagger \Lambda^a} T^a \exptext{i\Lambda^a T^a} \Phi \mbeq 0,
\end{align*}
which means that we must require \(D_{\dot{\alpha}}^\dagger \Lambda^a = 0\), meaning that the parameters are themselves left-handed scalar superfields.
Examining how the kinetic term \(\Phi^\dagger \Phi\) does under this transformation we can see that\footnote{Using the \needcite[Baker-Campell-Hausdorff formula] (BCH) to combine the exponentials.}
\begin{equation}
  \Phi^\dagger \Phi \to \Phi^\dagger \exp{-i\Lambda^\dagger} \exp{i\Lambda} \Phi = \Phi^\dagger \exp{i\pclosed{\Lambda - \Lambda^\dagger} - \frac{1}{2} \commutator{\Lambda}{\Lambda^\dagger} + \ldots} \Phi,
\end{equation}
which is not invariant.
To remedy this, we will introduce a term to compensate for this change, like before.
For this we define a   \emph{supergauge field} \(\mathcal{V} \equiv V^a T^a\) which transforms according to\footnote{The factor of 2 in the exponential here seems arbitrary at first, and is just a matter of choice. It is chosen to be 2 here such that the transformation of law for \(\mathcal{V}\) is proportional to \(\Lambda\) without any numerical prefactors.}
\begin{equation}
  \exp{2q\mathcal{V}} \to \exp{i\Lambda^\dagger} \exp{2q\mathcal{V}} \exp{-i\Lambda}
\end{equation}
or infinitesimally
\begin{equation}
  \mathcal{V} \to \mathcal{V} - \frac{i}{2q} \pclosed{\Lambda - \Lambda^\dagger} + \frac{i}{2} \commutator{\Lambda + \Lambda^\dagger}{\mathcal{V}}.
\end{equation}
Changing the kinetic term to \(\Phi^\dagger \exp{2q\mathcal{V}} \Phi\) will then yield it invariant under supergauge transformations.
Since we require the superlagrangian term to be real, we must require \(\mathcal{V}^\dagger = \mathcal{V}\), meaning it must be a vector superfield according to \cref{susy:eq:vector_superfield}.

As before, we would also like to add dynamics to the (super)gauge field \(\mathcal{V}\).
To this end, we introduce the supersymmetric field strength \(\mathcal{W}_\alpha \equiv W\indices*{^a_\alpha} T^a\) for which we require the transformation law
\begin{equation}
  \mathcal{W}\_\alpha \to \exp{i\Lambda} \mathcal{W}\_\alpha \exp{-i\Lambda}.
\end{equation}
It can be shown that the left-handed chiral superfield construction
\begin{equation}
  \label{susy:eq:superfield_strength}
  \mathcal{W}\_\alpha = -\frac{1}{4} (\widebar{D}\widebar{D}) \pclosed{\exp{-2\mathcal{V}} D\_\alpha \exp{2\mathcal{V}}}
\end{equation}
transforms this way, and recreates field-strength tensor earlier in \cref{qft:sec:yang-mills}.\cite{Martin:1997ns}
The gauge invariant superlagrangian kinetic term for the supergauge field becomes
\begin{equation}
  \label{susy:eq:Vkin}
  \L_{\mathcal{V}\text{-kin}} = \frac{1}{4 T(R)} \tr\cclosed{ \mathcal{W}\^\alpha \mathcal{W}\_\alpha }
\end{equation}
analogously to \cref{qft:eq:Akin}.
\\\comment{Why is it that the coupling \(q\) sometimes is included in the exponentials \cref{susy:eq:superfield_strength} to be cancelled \cref{susy:eq:Vkin}?}



\subsection{Field Content}
Here I give a very brief overview of the field content and naming scheme of the MSSM -- for a more comprehensive introduction I will refer to \needcite.
The basic idea is to embed every SM fermion into a chiral superfield, and the vector bosons into the vector superfields arising from local gauge invariance.
Since the SM fermions are Dirac fermions, they require two different Weyl spinors, which means that two superfields are required to provide each fermion.

Consider an SM Dirac fermion
\begin{equation}
  f_D = \begin{pmatrix}
    f \\ \bar{f}^\dagger
  \end{pmatrix},
\end{equation}
where \(f\) and \(\bar{f}^\dagger\) are two \emph{different} left-handed and right-handed Weyl spinors respectively.
The left-handed Weyl spinor part \(f\) is embedded into a superfield \(f\) wherein it receives a scalar \emph{superpartner} \(\tilde{f}_L\).\footnote{The subscript \(L\) on the scalar fields carries no indication of any chirality, but rather to the origin of the field as a superpartner to the left-handed chiral part of the fermion field \(f_D\).}
The superfield and Weyl spinor have the exact same name, which might seem needlessly confusing. However, it does lead to less cluttered notation, and context should clarify which is meant.
The right-handed Weyl spinor part \(\bar{f}^\dagger\) is likewise embedded into a right-handed scalar superfield \(\bar{F}^\dagger\), with a scalar superfield partner \(\tilde{f}_R\).
Furthermore, the left-handed scalar superfield \(f\) is part of an \(SU(2)_L\) doublet of superfields \(F\), matching the uppercase naming of the right-handed superfield \(\bar{F}^\dagger\).
The bar on superfields and right-handed Weyl spinors and signify that they are \(SU(2)_L\) singlets, i.e.\ they do not transform under such symmetry transformations, and to clarify that they are separate variables belonging to the same SM fermion field.
Collectively, the scalar superpartners to the SM fermions are referred to as \emph{sfermions}.

The gauge groups of the MSSM are the same as in the SM, but the gauge fields are replaced by vector superfield gauge fields as detailed in \cref{susy:ssec:susy_yang_mills}.
This way, an SM vector boson \(V\^\mu\) is embedded in a vector superfield \(V\) where it receives a left-handed Weyl spinor superpartner \(\tilde{V}\) with its right-handed compliment \(\tilde{V}^\dagger\).
Lastly, and perhaps the most intricate, is the extension of the Higgs sector in the MSSM.
As it turns out\needcite, the MSSM requires two Higgs doublets for anomaly cancellation within the \(U(1)_Y\) gauge group sector.
This means that there are two scalar Higgs doublets \(H_u, H_d\) before electroweak symmetry breaking (EWSB), giving mass to fermions in the upper/lower part of \(SU(2)_L\) fermion doublets respectively.
For the cancellation to work out, we must require hypercharge \(+1\) for \(H_u\) and \(-1\) for \(H_d\).
These scalar Higgs field doublets are embedded in left-handed chiral superfields \(H_{u/d}\) together with fermion superpartners.
The superfield doublet components are named according to \(H_u = (H_u^+, H_u^0)^T\) and \(H_d = (H_d^0, H_d^-)^T\), where the superscript indicates the QED charge of the superfields after EWSB.
The fermion partners to both the vector bosons and the Higgs bosons are called \emph{bosinos} collectively.
For reference all the superfields, their names and their component fields are tabulated in \cref{susy:tab:MSSM-fields}.

{
\renewcommand{\arraystretch}{1.5}
\begin{table}[ht!]
  \centering
  \begin{tabular}{|l|l|ccc|}
    \hline
                                             & Superfield     & Boson field              & Fermion field          &
    Auxiliary field
    \\
    \hline
    \multirow{3}{*}{\rotatebox{90}{Higgs}}   & \(H_{u/d}^0\)  & \(H_{u/d}^0\)            & \(\wtilde{H}_{u/d}^0\) &
    \(F_{H^0_{u/d}}\)
    \\
                                             & \(H_{u}^{+}\)  & \(H_{u}^{+}\)            & \(\wtilde{H}_{u}^{+}\) &
    \(F_{H^+_{u}}\)
    \\
                                             & \(H_{d}^{-}\)  & \(H_{d}^{-}\)            & \(\wtilde{H}_{d}^{-}\) &
    \(F_{H^-_{d}}\)
    \\
    \hline
    \multirow{3}{*}{\rotatebox{90}{Leptons}} & \(l_i\)        & \(\wtilde{l}_{iL}\)      & \(l_i\)                &
    \(F_{l_i}\)
    \\
                                             & \(\wbar{E}_i\) & \(\wtilde{l}_{iR}^\ast\) & \(\wbar{e}_{i}\)       &
    \(F_{\wbar{E}_i}^\ast\)
    \\
                                             & \(\nu_i\)      & \(\wtilde{\nu}_{iL}\)    & \(\nu_i\)              &
    \(F_{\nu_i}\)
    \\
    \hline
    \multirow{3}{*}{\rotatebox{90}{Quarks}}  & \(u_i\)        & \(\wtilde{u}_{iL}\)      & \(u_i\)                &
    \(F_{u_i}\)
    \\
                                             & \(\wbar{U}_i\) & \(\wtilde{u}_{iR}^\ast\) & \(\wbar{u}_i\)         &
    \(F_{\wbar{U}_i}^\ast\)
    \\
                                             & \(d_i\)        & \(\wtilde{d}_{iL}\)      & \(d_i\)                &
    \(F_{d_i}\)
    \\
                                             & \(\wbar{D}_i\) & \(\wtilde{d}_{iR}^\ast\) & \(\wbar{d}_i\)         &
    \(F_{\wbar{D}_i}^\ast\)
    \\
    \hline
    \multirow{3}{*}{\rotatebox{90}{Bosons}}
                                             & \(B^0\)        & \(B^0_\mu\)              & \(\bino\)              &
    \(D_{B^0}\)
    \\
                                             & \(W^0\)        & \(W^0_\mu\)              & \(\wino\)              &
    \(D_{W^0}\)
    \\
                                             & \(W^\pm\)      & \(W^\pm_\mu\)            & \(\wtilde{W}^\pm\)     &
    \(D_{W^\pm}\)
    \\
    \hline
  \end{tabular}
  \caption{Table of superfields of the electroweak part of the MSSM, and their component field names.
    Note that the fermion fields are left-handed Weyl spinors, in spite of any \(L\) or \(R\) in the subscript.
    The conjugate superfields changes these to right-handed Weyl spinors.
    The indices \(i\) enumerate the three generations of leptons/quarks.}
  \label{susy:tab:MSSM-fields}
\end{table}
}




\section{Electroweakinos}
\phpar[Introduce the electroweakinos and how they are derived from the various fermions partners of electroweak bosons.
  Talk about mass matrices and mass eigenstates]

Up to this point, the building blocks for the MSSM have been introduced, and I will now shift focus how these are put together to create the minimal supersymmetric extension of the SM.
In particular, I will focus on the electroweak sector, which will be the focus of this thesis.

\subsection{Mass mixing}





\section{Feynman Rules of Neutralinos}
\phpar[Derive the ordinary Lagrangian for neutralinos from the superlagrangian.]

\todo{Make sure the component form of the superfields is introduced somewhere}
\begin{temporary}
  As a reminder of the form of the superfields we will use, I list them on component form here:
  \begin{subequations}
    \begin{align}
      \Phi =         & A + i(\theta\sigma^\mu\theta^\dagger) \partial_\mu A - \frac{1}{4} (\theta\theta)(\theta\theta)^\dagger \dA A +
      \sqrt{2} (\theta\psi) - \frac{i}{\sqrt{2}} (\theta\theta) (\partial_\mu \psi \sigma^\mu \theta^\dagger) + (\theta\theta) F,                     \\
      \Phi^\dag =    & A^\ast - i(\theta\sigma^\mu\theta^\dagger) \partial_\mu A^\ast - \frac{1}{4} (\theta\theta)(\theta\theta)^\dagger \dA A^\ast +
      \sqrt{2} (\theta\psi)^\dagger + \frac{i}{\sqrt{2}} (\theta\theta)^\dagger
      (\theta \sigma^\mu \partial_\mu \psi^\dagger) + (\theta\theta)^\dagger F^\ast,                                                                  \\
      \msub{V}{WZ} = & (\theta\sigma^\mu\theta^\dagger) V_\mu + (\theta\theta) (\theta\lambda)^\dagger +
      (\theta\theta)^\dagger(\theta\lambda) + \frac{1}{2}
      (\theta\theta)(\theta\theta)^\dagger D,
    \end{align}
  \end{subequations}
\end{temporary}

\subsection{Fermion Interactions from Supersymmetric Yang-Mills theory}
\comment{This subsection might be best suited for an appendix?}

Considering a superlagrangian kinetic term \(\L = \sum_{ij} \Phi^\dagger_i
\pclosed{\exp{2q \mathcal{V}}}_{ij} \Phi_j\), I will extract the interaction
terms containing either the fermion field multiplets \(\psi\) (\(\psi^\dagger\)) from the left-handed (right-handed) scalar superfield multiplets \(\Phi\) (\(\Phi^\dagger\)),
and the fermion fields \(\lambda \equiv \lambda^a T^a\) from vector superfields \(\mathcal{V} \equiv V^a T^a\).
Up to terms with the appropriate amount of \(\theta\)s, we have
\begin{align}
  \label{susy:eq:psi_kin_int}
  \nonumber
  \L \stackrel{\psi, \psi^\dagger, \lambda}{\supset} & 2q \sum_{ij} \bigg\{ A^\ast_i (\theta\theta)^\dagger (\theta\lambda_{ij}) \sqrt{2} (\theta\psi_j) + \sqrt{2} (\theta\psi_i)^\dagger (\theta\sigma^\mu\theta^\dagger) \pclosed{\mathcal{V}_\mu}_{ij} \sqrt{2} (\theta\psi_j) \\
  \nonumber
                                                     & + \sqrt{2} (\theta\psi_i)^\dagger (\theta\theta) (\theta\lambda_{ij})^\dagger A_j \bigg\}                                                                                                                                   \\
  =                                                  & q(\theta\theta)(\theta\theta)^\dagger \sum_{ij} \cclosed{ -\sqrt{2} A^\ast_i (\lambda_{ij}\psi_j) + (\psi_i \sigma^\mu \pclosed{\mathcal{V}_\mu}_{ij} \psi^\dagger_j) - \sqrt{2} (\psi_i\lambda_{ij})^\dagger A_j },
\end{align}
where I have used \needcite[Weyl spinor relations]. \comment{Perhaps it should be clarified that these are all the \(\psi\)- \emph{and} \(\lambda\)-interactions.}
\medskip

There are also Yukawa terms coming from the superpotential of the form \(\L =
y_{ij} (\theta\theta)^\dagger \Phi_i \Phi \Phi_j + \cc\) Extracting the
interaction terms of fermion field \(\psi\) from \(\Phi\), we find
\begin{align}
  \label{susy:eq:psi_yuk_int}
  \nonumber
  \L & \stackrel{\psi, \psi^\dagger}{\supset} y_{ij} (\theta\theta)^\dagger \sqrt{2} (\theta\psi) \cclosed{ A_i \sqrt{2}(\theta\psi_i) + \sqrt{2}(\theta\psi_j)A_j } + \cc \\
     & = -y_{ij} (\theta\theta)(\theta\theta)^\dagger \cclosed{ A_i(\psi\psi_j) + (\psi_i\psi)A_j + \cc }
\end{align}

\subsection{Wino and Bino Interactions}
First, I will look at the bino and wino interactions. Writing out the \(W^a\)
vector superfields in the basis \(W^\pm, W^0\), we are only interested in the
electrically neutral \(W^0\) bit. The interactions will come from kinetic terms
of scalar superfields \(\Phi\), whose relevant part can be written as
\begin{equation}
  \L = \Phi^\dagger \exp{2g\cclosed{Y t_W B^0 \pclosed{+ \frac{1}{2}\sigma_3 W^0}}} \Phi,
\end{equation}
where \(t_W \equiv \tan\theta_W\) is the tangent of the Weinberg angle and \(Y\) is the hypercharge of \(\Phi\).
To generalise this, I will use the isospin \(I^3\), which is \(+\frac{1}{2}\) for fields in the upper part of an SU(2) doublet, \(-\frac{1}{2}\) for fields in the lower part and 0 for SU(2) singlet fields.
Then the kinetic term can compactly be written as
\begin{equation}
  \L = \Phi^\dagger \exp{2g\cclosed{(Q_e - I^3) t_W B^0 + I^3 W^0}} \Phi,
\end{equation}
where \(Q_e\) is the electric charge of \(\Phi\).
\medskip

Extracting the interactions of the fermion fields \(\bino, \wino\) in \(B^0,
W^0\) using \cref{susy:eq:psi_kin_int}, we are left with (up to approprate
\(\theta\)s)
\begin{equation}
  \L \stackrel{\bino, \wino}{\supset} -\sqrt{2}g (\theta\theta)(\theta\theta)^\dagger \Big\{ (Q_e - I^3)t_W (\bino \psi) A^\ast + I^3 (\wino \psi) A^\ast + \cc \Big\}.
\end{equation}

\begin{temporary}
  \(q, \tilde{q}_L \in Q\), \(\bar{q}, \tilde{q}_R^\ast \in \bar{Q}\)
\end{temporary}

Considering an SM quark \(q_g\) of generation \(g\), derived from the superfields \(Q_g\) and \(\bar{Q}_g\), with
electric charge \(Q_e\) and isospins \(I^3\) and 0 respectively, we can write out the interaction as
\begin{equation}
  \L = -\sqrt{2}g \Big\{ (Q_e - I^3)t_W (\bino q_g) \tilde{q}_{gL}^\ast + I^3 (\wino q_g) \tilde{q}_{gL}^\ast + Q_e t_W (\bino \bar{q}_i) \tilde{q}_{gR}^\ast + \cc \Big\}.
\end{equation}
Changing to the \(\nino\)-basis, we have that \(\bino = \sum_i N_{i1}^\ast \nino[i]\), \(\wino = \sum_i N_{i2}^\ast \nino[i]\), which together with writing out the Weyl products on Dirac spinor form yields
\begin{equation}
  \label{susy:eq:nino_sf_f_int}
  \L_{\tilde{\chi}^0 \tilde{q}q} = -\sqrt{2} g \sum_{i} \barnino[i] \Big\{ \big[ \underbrace{\pclosed{Q_e - I^3}t_W N_{i1}^\ast  + I^3 N_{i2}^\ast}_{\equiv C_{\tilde{\chi}^0_i \tilde{q} q_g}^{L\ast}} \big] \wtilde{q}_{gL}^\ast P_L \underbrace{- Q_f t_W N_{i1}}_{\equiv C_{\tilde{\chi}^0_i \tilde{q} q_g}^{R\ast}} \wtilde{q}_{gR}^\ast P_R \Big\} q_{gD} + \cc
\end{equation}

Generalising this further to include squark mixing between the left- and
right-handed squarks in a generation \(g\), we have
\begin{equation}
  \tilde{q}_{gA} = R_{A1}^{\tilde{q}_g} \tilde{q}_{gL} + R_{A2}^{\tilde{q}_g} \tilde{q}_{gR},
\end{equation}
where \(R^{\tilde{q}_g}\) is a \(2\times 2\) unitary matrix transforming the quarks of type \(q = u, d\) in generation \(g\) to their mass eigenstates.
As such, we can write \(\tilde{q}_{gL} = \sum_A \pclosed{R^{\tilde{q}_g}_{A1}}^\ast \tilde{q}_{gA}\), \(\tilde{q}_{gR} = \sum_A \pclosed{R^{\tilde{q}_g}_{A2}}^\ast \tilde{q}_{gA}\) to get
\begin{equation}
  \label{susy:eq:nino_sqA_f_int}
  \L_{\tilde{\chi}^0 \tilde{q} q_g} = -\sqrt{2} g \sum_{i} \sum_A \barnino[i] \Big\{ \underbrace{\pclosed{R^{\tilde{q}_g}_{A1}}^\ast C_{\tilde{\chi}^0_i \tilde{q} q_g}^{L\ast}}_{\equiv C_{\tilde{\chi}^0_i \tilde{q}_A q_g}^{L\ast}} P_L + \underbrace{\pclosed{R^{\tilde{q}_g}_{A2}}^\ast C_{\tilde{\chi}^0_i \tilde{q} q_g}^{R\ast}}_{\equiv C_{\tilde{\chi}^0_i \tilde{q}_A q_g}^{R\ast}} P_R \Big\} \wtilde{q}_{gA}^\ast q_{gD} + \cc
\end{equation}
\todo{Maybe comment on the extension to flavour violation.}



\subsubsection*{Flavour-Violating Squark Sector}
This last part was done under the assumption that squarks do not mix between fermion generations.
However, this can happen if supersymmetry-breaking parameters coupling squarks between generations or loop corrections are added to the squark sector.
The generalisation is fairly straight forward: Instead of three \(2 \by 2\) mixing matrices \(R^{\tilde{q}_i}\) for each quark type (up or down), there is one \(6 \by 6\) mixing matrix \(R^{\tilde{q}}\).
This mixing matrix can be defined using different conventions, but in this thesis, I will use that it fulfils
\begin{equation}
  \begin{pmatrix}
    \tilde{q}_1 \\ \tilde{q}_2 \\ \tilde{q}_3 \\ \tilde{q}_4 \\ \tilde{q}_5 \\ \tilde{q}_6
  \end{pmatrix}
  = R^{\tilde{q}} \begin{pmatrix}
    \tilde{q}_{1L} \\ \tilde{q}_{2L} \\ \tilde{q}_{3L} \\ \tilde{q}_{1R} \\ \tilde{q}_{2R} \\ \tilde{q}_{3R}.
  \end{pmatrix}
\end{equation}
This means that the chiral squarks in generation \(g = 1,2,3\) will rather be given by
\begin{subequations}
  \begin{align}
    \tilde{q}_{gL} = \sum_A R^{\tilde{q}}_{A,g} \tilde{q}_A, \\
    \tilde{q}_{gR} = \sum_A R^{\tilde{q}}_{A,g+3} \tilde{q}_A.
  \end{align}
\end{subequations}
What this means for the interaction Lagrangian in \cref{susy:eq:nino_sqA_f_int} is that the sum over \(A\) changes to go from 1 to 6 and the definition of the coupling parameter changes slightly to
\begin{subequations}
  \begin{align}
    C^{L}_{\tilde\chi_i^0 \tilde{q}_A q_g} = & R^{\tilde{q}}_{A,g} \bclosed{\pclosed{Q_e-I^3}t_W N_{i1} + I^3 N_{i2}} \\
    C^{R}_{\tilde\chi_i^0 \tilde{q}_A q_g} = & -R^{\tilde{q}}_{A,g+3} Q_e t_W N_{i1}^\ast
  \end{align}
\end{subequations}



\subsection{Higgsino Interactions}
The Higgsino interaction with the (s)quarks comes from the Yukawa terms of the
superpotential, but seeing that this interaction is proportional to the quark
mass, it will be ignored at the centre-of-mass energies we are interested in.

The relevant interaction that remains is that with the \(Z\)-boson. This
interaction again comes from the kinetic term, but this time of the neutral
Higgs superfields in the superfield multiplets \(H_u = \pclosed{H_u^+,
  H_u^0}^T\), \(H_d = \pclosed{H_d^0, H_d^-}^T\). The Lagrangian is of the form
\begin{equation}
  \L = \pclosed{H_{u/d}^0}^\dagger \exp{\mp g\pclosed{W^0 - t_W B^0}} H_{u/d}^0.
\end{equation}
Integrating over the Grassman variables and using equation \cref{susy:eq:psi_kin_int} we get
\begin{equation}
  \integral{^4\theta} \L \stackrel{\tilde{H}_{u/d}^0, W^0_\mu, B^0_\mu}{=} \mp \frac{g}{2} \pclosed{\tilde{H}_{u/d}^0 \sigma^\mu \pclosed{\tilde{H}_{u/d}^0}^\dagger}\pclosed{W^0_\mu - t_W B_\mu^0}.
\end{equation}
Switching to Dirac spinors, the mass eigenbasis for the neutralinos and the \(Z\) boson \(Z_\mu = c_W W^0_\mu - s_W B^0_\mu\), we end up with\margincomment{Mention Weyl/Dirac identities necessary for this.}
\begin{align}
  \nonumber
  \L_{Z \tilde{\chi}^0} = \frac{g}{2c_W} Z_\mu \sum_{ij}\pclosed{N_{i4}N_{j4}^\ast - N_{i3}N_{j3}^\ast} \barnino[i] \gamma^\mu P_L \nino[j] \\
  = \frac{g}{2} Z_\mu \sum_{ij}\barnino[i] \gamma^\mu \bigg[ \underbrace{\frac{1}{2c_W}\pclosed{N_{i4}N_{j4}^\ast - N_{i3}N_{j3}^\ast}}_{\equiv O^{\prime\prime L}_{ij}} P_L \underbrace{-\frac{1}{2c_W}\pclosed{N_{i4}^\ast N_{j4} - N_{i3}^\ast N_{j3}}}_{\equiv O^{\prime\prime R}_{ij}} P_R \bigg] \nino[j]
\end{align}




\subsection{Summary of Coupling Definitions}

{\renewcommand{\arraystretch}{2}
  \begin{table}[ht!]
    \centering
    \begin{tabular}{|c|c|}
      \hline
      Coupling                                            & Definition
      \\
      \hline
      \(C_{\tilde{q}_A q_g \tilde{\chi}_i^0}^{L}\)        & \(\pclosed{R^{\tilde{q}}_{A,g}}^\ast \bclosed{\pclosed{Q_e-I^3_q} t_W N_{i1} + I^3_q N_{i2}}\)           \\
      \(C_{\tilde{q}_A q_g \tilde{\chi}_i^0}^{R}\)        & \(-\pclosed{R^{\tilde{q}}_{A,g+3}}^\ast Q_e t_W N_{i1}^\ast\)                                            \\
      \hline
      \(O_{ij}^L\)                                        & \(\frac{1}{\sqrt{2}} N_{i4} V_{j2}^\ast - N_{i2}V_{j1}^\ast\)                                            \\
      \(O_{ij}^R\)                                        & \(-\frac{1}{\sqrt{2}} N_{i3}^\ast U_{j2} - N_{i2}^\ast U_{j1}\)                                          \\
      \hline
      \(O_{ij}^{\prime L}\)                               & \(\frac{1}{c_W} \pclosed{V_{i1} V_{j1}^\ast + \frac{1}{2} V_{i2}V_{j2}^\ast - \delta_{ij} s_W^2}\)       \\
      \(O_{ij}^{\prime R}\)                               & \(\frac{1}{c_W} \pclosed{U_{i1}U_{j1}^\ast + \frac{1}{2} U_{i2}U_{j2}^\ast - \delta_{ij} s_W^2}\)        \\
      \hline
      \(O^{\prime\prime L}_{ij}\)                         & \(\frac{1}{2c_W} \pclosed{N_{i4} N_{j4}^\ast - N_{i3} N_{j3}^\ast}\)                                     \\
      \(O^{\prime\prime R}_{ij}\)                         & \(-\frac{1}{2c_W} \pclosed{N_{i4}^\ast N_{j4} - N_{i3}^\ast N_{j3}}\)                                    \\
      \hline
      \(C_{\tilde{d}_A u_g \tilde{\chi}^\pm_i}^L\)        & \(\frac{1}{\sqrt{2}} U_{i1} \pclosed{R^{\tilde{d}}_{A,g}}^\ast V_{u_g d_g}^\text{CKM}\)                  \\
      \(C_{\tilde{u}_A d_g \tilde{\chi}^\pm_i}^L\)        & \(\frac{1}{\sqrt{2}} V_{i1} \pclosed{R^{\tilde{u}}_{A,g}}^\ast \pclosed{V_{u_g d_g}^\text{CKM}}^\ast\)   \\
      \(C_{\tilde{q}_A q^\prime_g \tilde{\chi}^\pm_i}^R\) & 0                                                                                                        \\
      \hline
      \(C_{qqZ}^L\)                                       & \(-\frac{I^3_q - Q_e s_W^2}{c_W}\)                                                                       \\
      \(C_{qqZ}^R\)                                       & \(\frac{Q_e s_W^2}{c_W}\)                                                                                \\
      \(C_{qq^\prime W}^L\)                               & \(-\frac{V^\text{CKM}_{q q^\prime}}{c_W}\)                                                               \\
      \(C_{qq^\prime W}^R\)                               & 0                                                                                                        \\
      \hline
      \(C_{\tilde{q}_A \tilde{q}_B Z}^L\)                 & \(-\frac{I^3_q - Q_e s_W^2}{c_W} R^{\tilde{q}}_{A,g} \pclosed{R^{\tilde{q}}_{B,g}}^\ast\)                \\
      \(C_{\tilde{q}_A \tilde{q}_B Z}^R\)                 & \(\frac{Q_e s_W^2}{c_W} R^{\tilde{q}}_{A,g+3} \pclosed{R^{\tilde{q}}_{B,g+3}}^\ast\)                     \\
      \(C_{\tilde{q}_A \tilde{q}_B^\prime W}^L\)          & \(-\frac{V^\text{CKM}_{q q^\prime}}{c_W} R^{\tilde{q}}_{A,g} \pclosed{R^{\tilde{q}^\prime}_{B,g}}^\ast\) \\
      \(C_{\tilde{q}_A \tilde{q}_B^\prime W}^R\)          & 0                                                                                                        \\
      \hline
    \end{tabular}
    \caption{A summary of the variables used in the derived Feynman rules
      and
      their definitions.}
    \label[tab]{tab:variable_definitions}
  \end{table}
}




\end{document}

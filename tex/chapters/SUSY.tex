\documentclass[../main.tex]{subfiles}

\begin{document}
\chapter{Supersymmetry}
\label{chap:susy}


\section{Introduction to Supersymmetry}
\label{susy:sec:introduction}
In this chapter, I introduce the basic ideas behind supersymmetry, what it is, and how to construct field theories that are \emph{supersymmetric}.
I will discuss the Super-Poincaré group as an extension of the Poincaré group, and introduce superspace as a vessel for supersymmetric field theories.
I go on to describe the Minimal Supersymmetric Standard Model (MSSM), the minimal (broken) supersymmetric QFT containing the Standard Model (SM) particles as a subset.
The electroweakinos, the main focus of this thesis, are introduced, including general mixing of fields into mass eigenstates, and I go into some depth to derive the interaction Feynman rules of these particles from the MSSM \emph{superlagrangian}.

This chapter makes extensive use of Weyl spinor notation and Grassmann calculus. For more details on this and the specific notation I use, I refer to \cref{app:chap:weyl_grassmann}.
Some background in group theory is necessary to follow certain parts of the chapter, but is not necessary to understand the broader ideas.

\subsection*{A Simple Supersymmetric Theory}
To illustrate what supersymmetry looks like in practice, it can be helpful to
look at a simple example. Take a Lagrangian for a massive complex scalar field
\(\phi\) and a massive Weyl spinor field \(\psi\),
\begin{equation}
  \label{susy:eq:simple_thoery}
  \L = (\partial\_{\mu} \phi)(\partial\^{\mu}\phi^\ast) + i \psi \sigma\^{\mu} \partial\_{\mu} \psi^\dagger
  - \abs{m_\phi}^2 \phi \phi^\ast - \frac{1}{2}m_\psi (\psi\psi) - \frac{1}{2} m_\psi^\ast (\psi\psi)^\dagger.
\end{equation}
To impose some symmetry between the bosonic and fermionic degrees of freedom, we want to examine a transformation of the scalar field through the spinor field and vice versa.
Imposing Lorentz invariance a general, infinitesimal, such transformation can be parametrised by
\begin{subequations}
  \begin{align}
    \delta\phi                         & = \epsilon a (\theta\psi),                                                                                                                                   \\
    \delta\phi^\ast                    & = \epsilon a^\ast (\theta\psi)^\dagger,                                                                                                                      \\
    \delta\psi\_{\alpha}               & = \epsilon \pclosed{c (\sigma\^{\mu} \theta^\dagger)\_{\alpha} \partial\_{\mu} \phi + F(\phi, \phi^\ast) \theta\_{\alpha}},                                  \\
    \delta{\psi^\dagger}\_{\dot\alpha} & = \epsilon \pclosed{c^\ast (\theta \sigma\^{\mu})\_{\dot\alpha} \partial\_{\mu} \phi^\ast + F^\ast(\phi, \phi^\ast) \theta\indices*{^\dagger_{\dot\alpha}}},
  \end{align}
\end{subequations}
where \(\epsilon\) is some infinitesimal parameter for the transformation, \(\theta\) is some Grassmann-valued Weyl spinor, \(a, c\) are complex coefficients of the transformation and \(F(\phi, \phi^\ast)\) is some linear function of \(\phi\) and \(\phi^\ast\).
The resulting change in the scalar field part of the Lagrangian is
\begin{equation}
  \label{susy:eq:dLphi}
  \delta\L_\phi / \epsilon = a (\theta\partial\_{\mu}\psi) \pclosed{\partial\^{\mu} \phi^\ast} - a\abs{m_\phi}^2 (\theta\psi) \phi^\ast + \cc,
\end{equation}
and likewise for the spinor field part
\begin{equation}
  \label{susy:eq:dLpsi}
  \delta\L_\psi/\epsilon = -i c^\ast (\psi \sigma\^{\mu}\bar{\sigma}\^{\nu}\theta) \partial\_{\mu}\partial\_{\nu} \phi^\ast + i (\psi \sigma\^{\mu} \theta^\dagger) \partial\_{\mu} F^\ast + m_\psi \bclosed{c (\psi\sigma\^{\mu}\theta^\dagger) \partial\_{\mu} \phi + (\psi\theta) F} + \cc
\end{equation}
The first term in \cref{susy:eq:dLpsi} can be rewritten using the commutativity of partial derivatives and the identity \(\tensor{\pclosed{\sigma\^{\mu} \bar{\sigma}\^{\nu} + \sigma\^{\nu} \bar{\sigma}\^{\mu}}}{_\alpha^\beta} = -2\tensor{g}{^\mu^\nu} \tensor*{\delta}{_\alpha^\beta}\) to get \(i c^\ast \pclosed{\theta \psi} \partial\_{\mu}\partial\^{\mu} \phi^\ast\).
Up to a total derivative, we can then write the change in the spinor part as
\begin{equation}
  \delta\L_\psi/\epsilon = -i c^\ast (\theta\partial\_{\mu}\psi)\partial\^{\mu} \phi^\ast + (\psi \sigma\^{\mu} \theta^\dagger) \partial\_{\mu} \pclosed{i F^\ast + m_\psi c \phi} + m_\psi (\theta\psi) F + \cc.
\end{equation}
The change of the total Lagrangian (again up to a total derivate) can then be grouped as
\begin{align}
  \nonumber
  \delta\L/\epsilon = & \pclosed{a-ic^\ast} (\theta\partial\_{\mu}\psi) \pclosed{\partial\^{\mu} \phi^\ast}
  + (\psi \sigma\^{\mu} \theta^\dagger) \partial\_{\mu} \pclosed{i F^\ast + m_\psi c \phi}                  \\
                      & + (\theta\psi) \pclosed{a\abs{m_\phi}^2\phi^\ast + m_\psi F} + \cc,
\end{align}
giving us three different conditions for the action to be invariant:
\begin{subequations}
  \begin{align}
    a - ic^\ast                          & = 0, \\
    i F^\ast + m_\psi c \phi             & = 0, \\
    a\abs{m_\phi}^2 \phi^\ast + m_\psi F & = 0.
  \end{align}
\end{subequations}
This is fulfilled if
\begin{subequations}
  \label{susy:eq:susy_transformation_requirements}
  \begin{align}
    c               & = i a^\ast,                 \\
    F               & = -a m_\psi^\ast \phi^\ast, \\
    a\abs{m_\phi}^2 & = a^\ast \abs{m_\psi}^2.
  \end{align}
\end{subequations}
What is interesting is the last condition, because it requires \(a\) to be real, as both \(\abs{m_\phi}^2\) and \(\abs{m_\psi}^2\) are real, but also requires \(\abs{m_\phi}^2 = \abs{m_\psi}^2\).
For the theory to be supersymmetric in this sense, the masses of the boson and fermion must be the same!
Since the phase of \(m_\phi\) does not appear in the Lagrangian, we are free to set \(m_\phi = m_\psi \equiv m\), suppressing any mass subscripts henceforth.

Revisiting \(F\), it can be introduced as an auxiliary field to bookkeep the
supersymmetry transformation. By including the non-dynamical term to the
Lagrangian \(\L_F = F^\ast F + m F \phi + m^\ast F^\ast \phi^\ast\), we make
sure \(F\) takes the correct value in the transformation from its equation of
motion \(\pd[F]{\L} = F^\ast + m \phi \mbeq 0\). Inserting \(F\) back into the
Lagrangian reproduces the mass term of the scalar field, allowing us to write
the original Lagrangian as
\begin{align}
  \L = & (\partial\_{\mu} \phi)(\partial\^{\mu}\phi^\ast) + i \psi \sigma\^{\mu} \partial\_{\mu} \psi^\dagger + F^\ast F + m F \phi + m^\ast F^\ast \phi^\ast - \frac{1}{2}m (\psi\psi) - \frac{1}{2}m^\ast (\psi\psi)^\dagger,
\end{align}
with the \emph{supersymmetry transformation} rules
\begin{subequations}
  \begin{align}
     & \delta\phi       = \epsilon (\theta \psi),                                                                                                                                       &
     & \delta\phi^\ast = \epsilon (\theta\psi)^\dagger,                                                                                                                                   \\
     & \delta\psi\_{\alpha} = \epsilon \pclosed{-i(\sigma\^{\mu} \theta^\dagger)\_{\alpha} \partial\_{\mu} \phi  + F \theta\_{\alpha}},                                                 &
     & \delta\psi\indices*{^\dagger_{\dot\alpha}} = \epsilon \pclosed{i(\theta \sigma\^{\mu})\_{\dot\alpha} \partial\_{\mu} \phi^\ast + F^\ast \theta\indices*{^\dagger_{\dot\alpha}}},   \\
     & \delta F         = i \epsilon \pclosed{\partial\_{\mu} \psi \sigma\^{\mu} \theta^\dagger},                                                                                       &
     & \delta F^\ast = -i \epsilon \pclosed{\theta \sigma\^{\mu} \partial\_{\mu} \psi^\dagger},
  \end{align}
\end{subequations}
where I have set \(a=1\) without loss of generality,\footnote{It can be absorbed by a redefinition of the parameter \(\epsilon\) for instance.} and found the appropriate transformation law for \(F\) such that the Lagrangian is invariant up to total derivatives.
The dynamics of this Lagrangian are the same as before, but the supersymmetry is now made manifest, i.e.~the transformation is free of any dependence on the contents of the Lagrangian as we had in \cref{susy:eq:susy_transformation_requirements}.

In fact, one can show that a general supersymmetric Lagrangian consisting of a
scalar field and a fermion field can be written
\begin{equation}
  \L = (\partial\_{\mu} \phi)(\partial\^{\mu}\phi^\ast) + i \psi \sigma\^{\mu} \partial\_{\mu} \psi^\dagger + F^\ast F + \cclosed{m F \phi - \frac{1}{2}m (\psi\psi) - \lambda \phi (\psi\psi) + \cc}
\end{equation}
up to renormalisable interactions.




\section{The Super-Poincaré Group}
To introduce more involved supersymmetric QFTs than our simple example from \cref{susy:sec:introduction}, it will be useful to first introduce a framework that will manifestly carry the supersymmetry.
This will alleviate the need to figure out the correct transformation laws, and the constraints they may carry to the parameters of the theory.
To this end, I will outline a common way of introducing supersymmetric theories -- extending our fields from representations of the Poincaré group of coordinate transformations to the super-Poincaré group.
This will hopefully give an algebraic geometrical understanding to \emph{superfields} as the building blocks of a supersymmetric field theory.

\subsection{The Poincaré and Super-Poincaré Algebras}
As we have already seen in \cref{qft:sec:yang-mills}, sets of transformations for a symmetry can be described by a group.
To introduce supersymmetry in this context, it will be clearer to study the \emph{generators} of the algebra of the group, so I would like to take a moment to motivate this change of perspective, before describing the fundamental symmetries we will be using.

The group describing the basic set of \emph{coordinate transformations} under which the fields theories we will consider are symmetric is called the \emph{Poincaré group}, denoted \(P\).
Theories that are symmetric under this group will be manifestly relativistic, and will exhibit the ordinary freedom in choice of coordinate system.
The Poincaré group consists of any transformation of space-time coordinates \(x^\mu\) such that
\begin{equation}
  {x^\prime}\^\mu = \Lambda\indices{^\mu_\nu} x\^\nu + a\^\mu,
\end{equation}
for a real, orthogonal \(4\times 4\) matrix \(\Lambda\) and real numbers \(a\^\mu\).
As a group it is the semi-direct product of Lorentz group \(O(1,3)\) and group of 4D space-time translations \(T(1,3)\)
\begin{equation}
  P \equiv O(1,3) \rtimes T(1,3).
\end{equation}
For completeness, the semi-direct product is defined such that the product of two group elements \((\Lambda_1, p_1), (\Lambda_2, p_2) \in P\) where \(\Lambda_1, \Lambda_2 \in O(1,3)\) and \(p_1, p_2 \in T(1,3)\) is
\begin{equation}
  (\Lambda_1, p_1) \circ (\Lambda_2, p_2) \equiv \pclosed{\Lambda_1 \circ_O \Lambda_2, p_1 \circ_T \Lambda_1(p_2)},
\end{equation}
where we understand \(\circ_{O/T}\) as the group multiplication operations of \(O(1,3)\) and \(T(1,3)\) respectively.\footnote{We see also that \(O(1,3)\) must also be a map \(T(1,3) \to T(1,3)\).
  We will later see that this means that the generators of translations are in a representation of the Lorentz group.}

For our purposes, it will suffice to work simply with the local structure of the Poincaré group, and being Lie groups, this can be reproduced with the exponential map we have used already in \cref{qft:eq:exponential_map} \(\text{exp} : \mathfrak{g} \to G\), where \(\mathfrak{g}\) is the \emph{Lie algebra} of the Lie group \(G\).
In this way, the algebra is said to \emph{generate} the group, and a basis set \(\cclosed{T^a}\) of the algebra \(\mathfrak{g}\) is said to be the \emph{generators} of the group.\footnote{The algebra of a Lie group can be shown to be a vector space, and as such there exists a basis set spanning the algebra.}
Accordingly, the local behaviour of the group can be inferred simply from the properties of the generators \(T^a\).
The generators of the Poincaré group can be structured by an antisymmetric Lorentz tensor \(M^{\mu\nu}\), and a four-vector \(P\^\mu\).
The properties of the algebra these generators span can be inferred from their commutation relations
\begin{subequations}
  \begin{align}
    \label{susy:eq:poincare_commutation1}
    \commutator{P\^\mu}{P\^\nu} =             & 0,                                                                                                                                                \\
    \label{susy:eq:poincare_commutation2}
    \commutator{M^{\mu\nu}}{P^\rho} =         & i\pclosed{\metric{\mu}{\sigma} P^\nu - \metric{\nu}{\sigma} P^\mu},                                                                               \\
    \label{susy:eq:poincare_commutation3}
    \commutator{M^{\mu\nu}}{M^{\rho\sigma}} = & i\pclosed{\metric{\mu}{\rho}M^{\nu\sigma} - \metric{\mu}{\sigma}M^{\nu\rho} - \metric{\nu}{\rho}M^{\mu\sigma} + \metric{\nu}{\sigma}M^{\nu\rho}}.
  \end{align}
\end{subequations}

To construct the super-Poincaré group, we can then just extend the algebra, and the rest of the group will follow.
This is done by extending the Lie algebra to a \emph{graded Lie superalgebra} by adding new generators.
A \emph{graded Lie superalgebra} is constructed from two vector spaces \(\mathfrak{l}_0, \mathfrak{l}_1\) and is denoted \(\mathfrak{l}_0 \oplus \mathfrak{l}_1\).
It is itself a vector space with a bilinear operation such that for any elements \(x_i \in \mathfrak{l}_i\) we have
\begin{align}
   & x_j \circ x_j \in \mathfrak{l}_{i+j \text{ mod } 2}, \tag{grading}                                                                     \\
  \label{susy:eq:supersymmetrisation}
   & x_i \circ x_j = -(-1)^{i\cdot j} x_j \circ x_i, \tag{supersymmetrisation}                                                              \\
  \nonumber
   & x_i \circ (x_j \circ x_k) (-1)^{i\cdot k} + x_j \circ (x_k \circ x_i) (-1)^{j \cdot i} + x_k\circ (x_i \circ x_j) (-1)^{k\cdot j} = 0. \\
  \tag{generalised Jacobi identity}
\end{align}
I note that in this case, \(\mathfrak{l}_0\) acts as an ordinary Lie algebra where \(\circ\) is the ordinary commutator, and \(\mathfrak{l}_1\) gets anti-commutator relations rather than commutator relations.\footnote{This can be seen from supersymmetrisation as for any \(x_1, x_1^\prime \in \mathfrak{l}_1\) we have that \(x_1 \circ x_1^\prime = x_1^\prime \circ x_1\).}

The \emph{super-Poincaré algebra}, denoted \(\mathfrak{sp}\), is the graded Lie superalgebra resulting from the Poincaré algebra \(\mathfrak{p}\) and the vector space \(\mathfrak{q}\).
Here \(\mathfrak{p}\) is the Lie algebra of the Poincaré group \(P\) and \(\mathfrak{q}\) is the vector space spanned by the generators \(Q_\alpha, Q^\dagger_{\dot\alpha}\) that form two Weyl spinors.
In addition to the commutation relations \cref{susy:eq:poincare_commutation1,susy:eq:poincare_commutation2,susy:eq:poincare_commutation3}, the Poincaré superalgebra is specified by the (anti-)commutator relations
\begin{subequations}
  \begin{align}
    \commutator{Q\_\alpha}{P\^\mu} = \commutator{Q\indices*{^\dagger_{\dot\alpha}}}{P\_\mu} =               & 0                                                          \\
    \commutator{Q\_\alpha}{M\^{\mu\nu}} =                                                                   & \pclosed{\sigma\^{\mu\nu}}\indices{_\alpha^\beta} Q\_\beta \\
    \anticommutator{Q\_\alpha}{Q\_\beta} = \anticommutator{Q^\dagger_{\dot\alpha}}{Q^\dagger_{\dot\beta}} = & 0,                                                         \\
    \anticommutator{Q\_\alpha}{Q^\dagger_{\dot\beta}} =                                                     & 2(\sigma\^\mu)\_{\alpha \dot\beta} P\_\mu
  \end{align}
\end{subequations}
where \(\sigma\^{\mu\nu} = \frac{i}{4}\pclosed{\sigma\^\mu \bar{\sigma}\^\nu - \sigma\^\nu \bar{\sigma}\^\mu}\), \(\sigma\^\mu = \pclosed{\eye, \sigma\^{i}}\), \(\bar\sigma\^\mu = \pclosed{\eye, -\sigma\^{i}}\) and \(\sigma\^{i}\) are the Pauli matrices.


\subsection{Superspace}
The idea behind \emph{superspace} is to create a coordinate system for which supersymmetry transformation manifest as coordinate transformations similarly to the way Poincaré transformations work on ordinary space-time coordinates.
To this end, we can start by considering a general element of the super-Poincaré group \(g \in SP\); it can be parametrised through the exponential map like this.
\begin{equation}
  g = \exptext{ix\^{\mu} P\_{\mu} + i(\theta Q) + i (\theta Q)^\dagger + \frac{i}{2}\tensor{\omega}{_\mu_\nu}\tensor{M}{^\mu^\nu}},
\end{equation}
where \(x^\mu, \theta\^{\alpha}, \theta\indices*{^\dagger_{\dot\alpha}}, \tensor{\omega}{_\mu_\nu}\) parametrise the group, and \(P_\mu, Q\_{\alpha}, Q\indices*{^\dagger^{\dot\alpha}}, \tensor{M}{^\mu^\nu}\) are the generators of the group as we have already seen.
Since the parameters \(x\^{\mu}, \theta\^{\alpha}, \theta\indices*{^\dagger_{\dot\alpha}}\) live in irreducible representations of the Lorentz algebra (four-vector and Weyl spinor representations respectively) generated by \(\tensor{M}{^\mu^\nu}\), the effect of the Lorentz part of the super-Poincaré group on the parameters can be determined easily.
Likewise, the parameters \(\tensor{\omega}{_\mu_\nu}\) are in a trivial representation of the algebra generated by \(P\_{\mu}, Q\_{\alpha}, Q\indices*{^\dagger^{\dot\alpha}}\), and need not then be considered.
It is therefore expedient to create a space with \(x\^{\mu}, \theta\^{\alpha}, \theta\indices*{^\dagger_{\dot\alpha}}\) as the coordinates, modding out the Lorentz algebra part.


We create superspace as a coordinate system with coordinates \(z\^{\pi} = (x\^{\mu}, \theta\^{\alpha}, \theta\indices*{^\dagger_{\dot\alpha}})\), and look at how they transform under super-Poincaré group transformations.
A function \(F(z)\) on superspace can then be written using the generators \(K\_\pi = (P\_\mu, Q\_\alpha, Q\indices*{^\dagger^{\dot\alpha}})\) as \(F(z) = \exptext{iz\^\pi K\_\pi} F(0)\).
Applying a super-Poincaré group element without the Lorentz generators \(\bar{g}(a, \eta) = \exptext{ia\^\mu P\_\mu + i (\eta Q) + i (\eta Q)^\dagger}\) we have
\begin{equation}
  F(z^\prime) = \exptext{i{z^\prime}\^\pi K\_\pi} F(0) = \exptext{ia\^\mu P\_\mu + i (\eta Q) + i (\eta Q)^\dagger} \exptext{iz\^\pi K\_\pi} F(0),
\end{equation}
which by the Baker-Campbell-Hausdorff formula (BCH) gives to first order in the commutators
\begin{equation}
  \label{susy:eq:supersymmetry_transformation}
  {z^\prime}\^\pi K\_\pi = (x\^\mu + a\^\mu) P\_\mu + (\theta\^\alpha + \eta\^\alpha)Q\_\alpha + (\theta\indices*{^\dagger_{\dot\alpha}} + \eta\indices*{^\dagger_{\dot\alpha}})Q\indices*{^\dagger^{\dot\alpha}} + \frac{i}{2}\bclosed{a\^\mu P\_\mu + (\eta Q) + (\eta Q)^\dagger, z\^\pi K\_\pi} + \ldots
\end{equation}
Now, \(P\_\mu\) commutes with all of \(K\_\pi\), and \(Q\_\alpha\) (\(Q\indices*{^\dagger^{\dot\alpha}}\)) anti-commute with themselves, for every combination of different \(\alpha\) (\(\dot\alpha\)), so the only relevant part of the commutator is
\begin{equation}
  \commutator{(\eta Q)}{(\theta Q)^\dagger} + \commutator{(\eta Q)^\dagger}{(\theta Q)} = -\eta\^\alpha \anticommutator{Q\_\alpha}{Q\indices*{^\dagger_{\dot\alpha}}} \theta\indices*{^\dagger^{\dot\alpha}} + (\eta \leftrightarrow \theta) = -2(\eta \sigma\^\mu \theta^\dagger)P\_\mu + (\eta \leftrightarrow \theta).
\end{equation}
Since this commutator is proportional to \(P\_\mu\) which in turn commutes with everything, all higher order commutators of BCH vanish, and we can conclude that the transformed coordinates \({z^\prime}\^\pi\) are given by
\begin{equation}
  \label{susy:eq:zprime}
  {z^\prime}\^\pi = \pclosed{x\^\mu + a\^\mu + i(\theta \sigma\^\mu \eta^\dagger) - i(\eta \sigma\^\mu \theta^\dagger), \theta\^\alpha + \eta\^\alpha, \theta\indices*{^\dagger_{\dot\alpha}} + \eta\indices*{^\dagger_{\dot\alpha}}}.
\end{equation}
This gives us a differential representation of the \(K\_\pi\) generators as
\begin{subequations}
  \begin{align}
    P\_\mu                            & = -i \partial\_\mu,                                                                   \\
    Q\_\alpha                         & = - (\sigma\^\mu \theta^\dagger)\_\alpha \partial\_\mu - i \partial\_\alpha,          \\
    Q\indices*{^\dagger_{\dot\alpha}} & = -(\theta \bar{\sigma}\^\mu)\_{\dot\alpha} \partial\_\mu - i \partial\_{\dot\alpha}.
  \end{align}
\end{subequations}

Now, to see what the these functions of superspace look like, we can expand \(F(z)\) in terms of the coordinates \(\theta\^\alpha, \theta\indices*{^\dagger_{\dot\alpha}}\), as these expansions are finite due to the fact that none of these coordinates can appear more than once per term.
Demanding that the function \(F(z)\) be invariant under Lorentz transformations, the \(x\^\mu\)-dependent coefficients of the expansion must transform such that each term is a scalar (or fully contracted Lorentz structure).
This limits a general such function of superspace to be written as
\begin{align}
  \label{susy:eq:superspace_function}
  \nonumber
  F(z) = & f(x) + \theta\^\alpha \phi\_\alpha(x) + \theta\indices*{^\dagger_{\dot\alpha}} \chi\indices*{^\dagger^{\dot\alpha}}(x) + (\theta\theta) m(x) + (\theta\theta)^\dagger n(x)                                                                              \\
         & + (\theta \sigma\^\mu \theta^\dagger) V\_\mu(x) + (\theta\theta) \theta\indices*{^\dagger_{\dot\alpha}} \lambda\indices*{^\dagger^{\dot\alpha}}(x) + (\theta\theta)^\dagger \theta\^\alpha \psi\_\alpha(x) + (\theta\theta)(\theta\theta)^\dagger d(x).
\end{align}

% \begin{temporary}
%   For future reference, a supersymmetry transformation \(\bar{g}(0, \epsilon)\) from some infinitesimal Weyl spinor \(\epsilon\_\alpha\) transforms the component fields of the general superfield \(F(z)\) according to\footnote{The transformations parameters \(a\^\mu\) are set to zero, which we can do without loss of generality, as the \(P\_\mu\) generators commute with the rest of the generators.}\needcite
%   \begin{subequations}
%     \begin{align}
%       \delta f(x) =                         & (\epsilon\phi(x)) + (\epsilon \chi(x))^\dagger                                                                                                                                                                                                               \\
%       \delta \phi\_\alpha(x) =              & 2\epsilon\_\alpha m(x) - (\sigma\^\mu \epsilon^\dagger)\_\alpha (V\_\mu(x) + i\partial_\mu f(x))                                                                                                                                                             \\
%       \delta \chi\^{\dagger\dot\alpha}(x) = & 2\epsilon\^{\dagger\dot\alpha} n(x) + (\bar\sigma\^\mu \epsilon)\^{\dagger\dot\alpha} (V\_\mu(x) - i\partial_\mu f(x))                                                                                                                                       \\
%       \delta m(x) =                         & (\epsilon \lambda(x))^\dagger - \frac{i}{2} (\epsilon^\dagger \bar\sigma\^\mu \partial\_\mu \phi(x))                                                                                                                                                         \\
%       \delta n(x) =                         & (\epsilon \psi(x)) - \frac{i}{2} (\epsilon \sigma\^\mu \partial\_\mu \chi^\dagger(x))                                                                                                                                                                        \\
%       \delta V\_\mu(x) =                    & (\epsilon \sigma\_\mu \lambda^\dagger(x)) - (\epsilon^\dagger \bar\sigma\_\mu \psi(x)) - \frac{i}{2} (\epsilon \sigma\^\nu \bar\sigma\_\mu \partial\_\nu \phi(x)) + \frac{i}{2} (\epsilon^\dagger \bar\sigma\^\nu \sigma\_\mu \partial\_\nu \chi^\dagger(x)) \\
%       \delta \psi\_\alpha =                 & 2\epsilon\_\alpha d(x) - i(\sigma\^\mu \epsilon^\dagger)\_\alpha \partial\_\mu n(x) - \frac{i}{2} (\sigma\^\nu \bar\sigma\^\mu \epsilon)\_\alpha \partial\_\mu V\_\nu(x)                                                                                     \\
%       \delta \lambda\^{\dagger\dot\alpha} = & 2\epsilon\^{\dagger\dot\alpha} d(x) - i(\bar\sigma\^\mu \epsilon)\^{\dagger\dot\alpha} \partial\_\mu m(x) + \frac{i}{2} (\bar\sigma\^\nu \sigma\^\mu \epsilon^\dagger)\^{\dot\alpha} \partial\_\mu V\_\nu(x)                                                 \\
%       \delta d(x) =                         & -\frac{i}{2} (\epsilon^\dagger \bar\sigma\^\mu \partial\_\mu \psi(x)) - \frac{i}{2} (\epsilon \partial\^\mu \partial\_\mu \lambda\^\dagger(x))
%     \end{align}
%   \end{subequations}
% \end{temporary}


\subsection{Superfields}
To construct a manifestly supersymmetric theory, it will be useful to start with finding representations of the super-Poincaré group.
This is exactly what we have already done; the functions on superspace find themselves in the representation space of a differential representation of the \(K\_\pi\) generators of the super-Poincaré group, and a scalar representation of the remaining Lorentz generators (i.e.\ the Lorentz generators leave the superspace functions unchanged).
Inside the general function on superspace \cref{susy:eq:superspace_function}, we find many component functions in different representation spaces of the Lorentz group.
Furthermore, supersymmetry transformations transform these fields into one another.
This seems like an ideal vessel for constructing supersymmetric fields theories.

We define the \emph{superfield} \(\Phi\) as an operator-valued function on
superspace.\footnote{For our purposes, it suffices to look at them simply as complex valued functions, but strictly speaking, they are operator-valued in a quantised field theory.} The general superfield from \cref{susy:eq:superspace_function} is in a
reducible representation space of the super-Poincaré group, so we define three
\emph{irreducible} representations that will be useful going forward:\footnote{I will not prove here that these in fact are irreducible representations.}
\begin{align}
  \textit{Left-handed scalar superfield:}  &  & \bar{D}\_{\dot\alpha}\Phi & = 0,    \\
  \textit{Right-handed scalar superfield:} &  & D\_{\alpha}\Phi^\dagger   & = 0,    \\
  \label{susy:eq:vector_superfield}
  \textit{Vector superfield:}              &  & \Phi^\dagger              & = \Phi.
\end{align}
Here the dagger operation refers to complex conjugation, and the differential operators \(D\_\alpha, \bar{D}\_{\dot\alpha}\) are defined as
\begin{subequations}
  \begin{align}
    D\_\alpha             & = \partial\_\alpha + i(\sigma\^\mu \theta^\dagger)\_\alpha \partial\_\mu,      \\
    \bar{D}\_{\dot\alpha} & = -\partial\_{\dot\alpha} - i (\theta\sigma\^\mu)\_{\dot\alpha} \partial\_\mu.
  \end{align}
\end{subequations}
These differential operators are covariant differentials in the sense that the commute with supersymmetry transformations, i.e. \(D\_\alpha F(z) \to {D^\prime}\_\alpha (\bar{g} F(z)) = \bar{g} \pclosed{D\_\alpha F(z)}\)
Collectively, the left- and right-handed scalar superfields are referred to as \emph{chiral superfields}.

For future reference, the general forms of a left-handed scalar superfield \(\Phi\), a right-handed scalar superfield \(\Phi^\dagger\) and a vector superfield \(V_\text{WZ}\) in the so-called Wess-Zumino gauge is~\cite{Muller-Kirsten}:
\begin{subequations}
  \begin{align}
    \label{susy:eq:lh_superfield}
    \nonumber
    \Phi(x, \theta, \theta^\dagger) =         & A(x) + i(\theta\sigma\^\mu\theta^\dagger) \partial_\mu A(x) - \frac{1}{4} (\theta\theta)(\theta\theta)^\dagger \dA A(x)                 \\
                                              & + \sqrt{2} (\theta\psi(x)) - \frac{i}{\sqrt{2}} (\theta\theta) (\partial_\mu \psi(x) \sigma\^\mu \theta^\dagger) + (\theta\theta) F(x), \\
    \label{susy:eq:rh_superfield}
    \nonumber
    \Phi^\dag(x, \theta, \theta^\dagger) =    & A^\ast(x) - i(\theta\sigma\^\mu\theta^\dagger) \partial_\mu A^\ast(x) - \frac{1}{4} (\theta\theta)(\theta\theta)^\dagger \dA A^\ast(x)  \\
                                              & + \sqrt{2} (\theta\psi(x))^\dagger + \frac{i}{\sqrt{2}} (\theta\theta)^\dagger
    (\theta \sigma\^\mu \partial_\mu \psi^\dagger(x)) + (\theta\theta)^\dagger F^\ast(x),                                                                                               \\
    \label{susy:eq:vector_superfield_WZ}
    \msub{V}{WZ}(x, \theta, \theta^\dagger) = & (\theta\sigma\^\mu\theta^\dagger) V\_\mu(x) + (\theta\theta) (\theta\lambda(x))^\dagger +
    (\theta\theta)^\dagger(\theta\lambda(x)) + \frac{1}{2}
    (\theta\theta)(\theta\theta)^\dagger D(x).
  \end{align}
\end{subequations}


\subsection{Superlagrangian}
We are now ready to define the action of a supersymmetric quantum field theory on superspace.
Given a set of superfields \(\cclosed{\Phi_i}\), we want to define an action through a Lagrangian density comprised of the component fields in \(\Phi_i\).
A function of the superfields will still be a superfield, and will therefore take the form from \cref{susy:eq:superspace_function}.
To get a supersymmetry invariant Lagrangian density, we can therefore look to extract some part of such a superspace function that at most transforms as a total derivative under a supersymmetry transformation according to \cref{susy:eq:supersymmetry_transformation}.
It can be show that the \(d(x)\) component field of \cref{susy:eq:superspace_function} transforms in such a way, and likewise for the \(F(x)\) component field of a chiral superfield \cref{susy:eq:lh_superfield,susy:eq:rh_superfield}, so projecting out these would constitute a valid Lagrangian density for a supersymmetry invariant action.
Keeping in mind that the Lagrangian density must be real, we can then get a supersymmetry invariant action through a Lagrangian density on the form\footnote{To clarify potential confusion on the capitalisation of the \(D\)-projection here -- for a vector superfield \cref{susy:eq:vector_superfield_WZ} the \(d(x)\) component field is the \(D(x)\) auxiliary component field.}
\begin{align}
  \label{susy:eq:projection_lagragnian}
  \L = \operatorname{proj}_D (V[\Phi_i]) + \operatorname{proj}_F (W[\Phi_i]) + \operatorname{proj}_{F^\dagger}(W^\dagger[\Phi_i]),
\end{align}
where \(V[\Phi_i]\) is a vector superfield and \(W[\Phi_i]\) (\(W^\dagger[\Phi_i]\)) is some left-handed (right-handed) scalar superfield.

The projection operators can be realised using Grassmann integration:\footnote{As a reminder, I detail how the calculus of Grassmann coordinates is defined in \cref{app:chap:weyl_grassmann}.}
\begin{subequations}
  \begin{align}
    \operatorname{proj}_D V[\Phi_i] =                   & \integral{^4\theta} V[\Phi_i],                        \\
    \operatorname{proj}_F W[\Phi_i] =                   & \integral{^4\theta} (\theta\theta)^\dagger W[\Phi_i], \\
    \operatorname{proj}_{F^\dagger} W^\dagger[\Phi_i] = & \integral{^4\theta} (\theta\theta) W^\dagger[\Phi_i].
  \end{align}
\end{subequations}
Accordingly, we can write down the general supersymmetry invariant action, letting \(\cclosed{\bar\Phi_i}\) be the subset of chiral superfields in \(\{\Phi_i\}\) using a Lagrangian density on the form
\begin{align}
  \label{susy:eq:general_susy_lagrangian}
  \L = \integral{^4\theta} \cclosed{V[\Phi_i] + (\theta\theta)^\dagger W[\bar\Phi_i] + (\theta\theta) W[\bar\Phi_i^\dagger]},
\end{align}
where we restrict \(W\) to be holonomic function of its argument superfields called the \emph{superpotential}.
\(W\) being holonomic in this context simply means that \(W[\bar\Phi_i]\) will be a left-handed scalar superfield and \(W[\bar\Phi_i^\dagger]\) a right-handed scalar superfield.
This leads to defining the \emph{superlagrangian} \(\tilde\L\) as a Lagrangian density analogue on superspace, where we can recognise
\begin{equation}
  \label{susy:eq:superlagrangian}
  \tilde\L = V[\Phi_i] + (\theta\theta)^\dagger W[\bar\Phi_i] + (\theta\theta) W[\bar\Phi_i^\dagger],
\end{equation}
and subsequently the action as
\begin{equation}
  \label{susy:eq:susy_action}
  S\bclosed{\cclosed{\Phi_i}} = \integral{^4x\,\mathrm{d}^4\theta} \tilde\L\pclosed{\cclosed{\Phi_i}, \cclosed{\pd[z^\pi]{\Phi_i}}, z}.
\end{equation}

Renormalisability puts severe restrictions on the form of the superlagrangian by imposing that any parameter of the theory cannot have a negative mass dimension.
Recognising that \(1 = \integral{^4\theta} (\theta\theta)(\theta\theta)^\dagger\), we must have that \([\integral{^4\theta}] = M^2\) for some mass reference scale \(M\).
Consequently, for the ordinary Lagrangian density to have mass dimension four, we must have that \([\tilde\L] = M^2\).
From \cref{susy:eq:lh_superfield} we recognise that the scalar superfield contains a scalar field term, and consequently has mass dimension \([\Phi] = M^1\).
Thus, the general form of the superpotential is
\begin{equation}
  W[\Phi_i] = \sum_i \lambda_i\Phi_i + \sum_{ij} m_{ij}\Phi_i \Phi_j + \sum_{ijk} y_{ijk} \Phi_i \Phi_j \Phi_k,
\end{equation}
and the only possible form of \(V[\Phi_i]\) only containing scalar superfields is
\begin{equation}
  V[\Phi_i] = \sum_i \Phi_i \Phi_i^\dagger,
\end{equation}
where the prefactor of the terms are set to 1, which can be done without loss of generality by rescaling the fields.


\subsection{Revisiting our Simple Supersymmetric Theory}
Now that we have developed a structure for creating manifestly supersymmetric theories using superfields, we can take a moment to revisit our simple theory from \cref{susy:eq:simple_thoery} to see what it would look like within the superspace framework.
We can use a left-handed scalar superfield \(\Phi\) as the vessel for our scalar field \(\phi\), fermionic field \(\psi\) and auxiliary field \(F\):
\begin{align}
  \nonumber
  \Phi(\theta, \theta^\dagger, x) = & \phi(x) + i(\theta\sigma\^\mu \theta^\dagger) \partial_\mu \phi(x) - \frac{1}{4} (\theta\theta)(\theta\theta)^\dagger \dA \phi(x)      \\
                                    & + \sqrt{2}(\theta \psi(x)) - \frac{i}{\sqrt{2}} (\theta\theta) (\partial_\mu \psi(x) \sigma\^\mu \theta^\dagger) + (\theta\theta)F(x).
\end{align}
The kinetic terms are reproduced through the first term in \cref{susy:eq:general_susy_lagrangian}:
\begin{align}
  \nonumber
  \L_\text{kin} = & \integral{^4\theta} \Phi^\dagger \Phi = \integral{^4\theta} \Bigg\{-\frac{1}{4}\pclosed{\phi^\ast \dA \phi + \phi \dA \phi^\ast} + (\theta \sigma\^\mu \theta^\dagger)(\theta \sigma\^\nu \theta^\dagger) \partial_\mu\phi^\ast \partial_\nu\phi \\
  \nonumber
                  & -i \bclosed{(\theta \psi)^\dagger (\theta\theta) (\partial_\mu \psi \sigma\^\mu \theta^\dagger) - (\theta\theta)^\dagger (\theta \sigma\^\mu \partial_\mu \psi^\dagger) (\theta \psi)} + (\theta\theta)(\theta\theta)^\dagger F^\ast F   \Bigg\} \\
  =               & (\partial_\mu \phi) (\partial^\mu\phi^\ast) + i(\psi \sigma\^\mu \partial_\mu \psi^\dagger) + F^\ast F.
\end{align}
The remaining mass term can be recreated by the superlagrangian term \(\frac{m}{2} (\theta\theta)^\dagger \Phi\Phi + \frac{m^\ast}{2} (\theta\theta) \Phi^\dagger\Phi^\dagger\), equivalent to a superpotential \(W[\Phi] = \frac{m}{2}\Phi \Phi\), yielding
\begin{align}
  \nonumber
  \L_\text{mass} = & \integral{^4\theta} \cclosed{\frac{m}{2} (\theta\theta)^\dagger \Phi\Phi + \cc} = \integral{^4\theta} \cclosed{\frac{m}{2} (\theta\theta)^\dagger \pclosed{2\phi (\theta\theta) F + 2(\theta\psi)(\theta\psi)} + \cc} \\
  =                & m \phi F + m^\ast \phi^\ast F^\ast + \frac{m}{2}(\psi\psi) + \frac{m^\ast}{2} (\psi\psi)^\dagger.
\end{align}
So our simple supersymmetric theory is encapsulated simply by the superlagrangian
\begin{equation}
  \tilde\L = \Phi \Phi^\dagger + \frac{m}{2} (\theta\theta)^\dagger \Phi\Phi + \frac{m^\ast}{2} (\theta\theta) \Phi^\dagger \Phi^\dagger,
\end{equation}
showing how superspace simplifies the model building considerably.




\section{Minimal Supersymmetric Standard Model}
Up to this point, the building blocks for the MSSM have been introduced, and I will now shift focus how these are put together to create the minimal supersymmetric extension of the SM\@.
I will also outline the process of spontaneous symmetry breaking, and state a general parametrisation of how this is done in the MSSM\@.

\subsection{Supersymmetric Yang-Mills Theory}
\label{susy:ssec:susy_yang_mills}
Before getting into the MSSM content, we must introduce what Yang-Mills theory
looks like at a superlagangian level. We define a \emph{supergauge
  transformation} of a left-handed scalar superfield multiplet \(\Phi\)
analogously to the ordinary case \cref{qft:eq:gauge_transformation}
\begin{equation}
  \label{susy:eq:supergauge_transformation}
  \Phi \to \exptext{i \Lambda} \Phi,
\end{equation}
where \(\Lambda \equiv \Lambda^a T^a\), \(\Lambda^a\) are the parameters of the transformation and \(T^a\) are again the generators of the gauge group.
To get a sense of what these parameters are, we can require the transformed superfield to be left-handed
\begin{align*}
  D_{\dot{\alpha}}^\dagger \exptext{i\Lambda} \Phi = & i\pclosed{D_{\dot{\alpha}}^\dagger \Lambda^a} T^a \exptext{i\Lambda^a T^a} \Phi + \exptext{i\Lambda^a T^a} D_{\dot{\alpha}}^\dagger \Phi \\
  =                                                  & i\pclosed{D_{\dot{\alpha}}^\dagger \Lambda^a} T^a \exptext{i\Lambda^a T^a} \Phi \mbeq 0,
\end{align*}
which means that we must require \(D_{\dot{\alpha}}^\dagger \Lambda^a = 0\), meaning that the parameters are themselves left-handed scalar superfields.
Examining how the kinetic term \(\Phi^\dagger \Phi\) does under this transformation we can see that\footnote{Using the Baker-Campell-Hausdorff formula (BCH) to combine the exponentials.}
\begin{equation}
  \Phi^\dagger \Phi \to \Phi^\dagger \exp{-i\Lambda^\dagger} \exp{i\Lambda} \Phi = \Phi^\dagger \exp{i\pclosed{\Lambda - \Lambda^\dagger} - \frac{1}{2} \commutator{\Lambda}{\Lambda^\dagger} + \ldots} \Phi,
\end{equation}
which is not invariant.
To remedy this, we will introduce a term to compensate for this change, like before.
For this we define a   \emph{supergauge field} \(\mathcal{V} \equiv V^a T^a\) which transforms according to\footnote{The factor of 2 in the exponential here seems arbitrary at first, and is just a matter of choice. It is chosen to be 2 here such that the transformation of law for \(\mathcal{V}\) is proportional to \(\Lambda\) without any numerical prefactors.}
\begin{equation}
  \exp{2q\mathcal{V}} \to \exp{i\Lambda^\dagger} \exp{2q\mathcal{V}} \exp{-i\Lambda}
\end{equation}
or infinitesimally
\begin{equation}
  \mathcal{V} \to \mathcal{V} - \frac{i}{2q} \pclosed{\Lambda - \Lambda^\dagger} + \frac{i}{2} \commutator{\Lambda + \Lambda^\dagger}{\mathcal{V}}.
\end{equation}
Changing the kinetic term to \(\Phi^\dagger \exp{2q\mathcal{V}} \Phi\) will then yield it invariant under supergauge transformations.
Since we require the superlagrangian term to be real, we must require \(\mathcal{V}^\dagger = \mathcal{V}\), meaning it must be a vector superfield according to \cref{susy:eq:vector_superfield}.

As before, we would also like to add dynamics to the (super)gauge field \(\mathcal{V}\).
To this end, we introduce the supersymmetric field strength \(\mathcal{W}_\alpha \equiv W\indices*{^a_\alpha} T^a\) for which we require the transformation law
\begin{equation}
  \mathcal{W}\_\alpha \to \exp{i\Lambda} \mathcal{W}\_\alpha \exp{-i\Lambda}.
\end{equation}
It can be shown that the left-handed chiral superfield construction
\begin{equation}
  \label{susy:eq:superfield_strength}
  \mathcal{W}\_\alpha = -\frac{1}{4} (\widebar{D}\widebar{D}) \pclosed{\exp{-2\mathcal{V}} D\_\alpha \exp{2\mathcal{V}}}
\end{equation}
transforms this way, and recreates field-strength tensor earlier in \cref{qft:sec:yang-mills}~\cite{Martin:1997ns}.
The gauge invariant superlagrangian kinetic term for the supergauge field becomes
\begin{equation}
  \label{susy:eq:Vkin}
  \L_{\mathcal{V}\text{-kin}} = \frac{1}{4 T(R)} \tr\cclosed{ \mathcal{W}\^\alpha \mathcal{W}\_\alpha }
\end{equation}
analogously to \cref{qft:eq:Akin}.



\subsection{Field Content}
Here I give a very brief overview of the field content and naming scheme of the MSSM -- for a more comprehensive introduction I will refer to~\cite{Martin:1997ns}.
The basic idea is to embed every SM fermion into a chiral superfield, and the vector bosons into the vector superfields arising from local gauge invariance.
Since the SM fermions are Dirac fermions, they require two different Weyl spinors, which means that two superfields are required to provide each fermion.

Consider an SM Dirac fermion
\begin{equation}
  f_D = \begin{pmatrix}
    f \\ \bar{f}^\dagger
  \end{pmatrix},
\end{equation}
where \(f\) and \(\bar{f}^\dagger\) are two \emph{different} left-handed and right-handed Weyl spinors respectively.
The left-handed Weyl spinor part \(f\) is embedded into a superfield \(f\) wherein it receives a scalar \emph{superpartner} \(\tilde{f}_L\).\footnote{The subscript \(L\) on the scalar fields carries no indication of any chirality, but rather alludes to the origin of the field as a superpartner to the left-handed chiral part of the fermion field \(f_D\).}
The superfield and Weyl spinor have the exact same name, which might seem needlessly confusing. However, it does lead to less cluttered notation, and context should clarify which is meant.
The right-handed Weyl spinor part \(\bar{f}^\dagger\) is likewise embedded into a right-handed scalar superfield \(\bar{F}^\dagger\), with a scalar superfield partner \(\tilde{f}_R\).
Furthermore, the left-handed scalar superfield \(f\) is part of an \(SU(2)_L\) doublet of superfields \(F\), matching the uppercase naming of the right-handed superfield \(\bar{F}^\dagger\).
The bar on superfields and right-handed Weyl spinors signify that they are \(SU(2)_L\) singlets, i.e.\ they do not transform under such symmetry transformations, and makes it clear that the two Weyl spinors \(f\) and \(\bar{f}^\dagger\) are separate variables belonging to the same SM fermion field.
Collectively, the scalar superpartners to the SM fermions are referred to as \emph{sfermions}.

The gauge groups of the MSSM are the same as in the SM, but the gauge fields are replaced by vector superfield gauge fields as detailed in \cref{susy:ssec:susy_yang_mills}.
This way, an SM vector boson \(V\^\mu\) is embedded in a vector superfield \(V\) where it receives a left-handed Weyl spinor superpartner \(\tilde{V}\) with its right-handed compliment \(\tilde{V}^\dagger\).

Lastly, and perhaps the most intricate, is the extension of the Higgs sector in the MSSM\@.
As it turns out, the MSSM requires two Higgs doublets for anomaly cancellation within the \(U(1)_Y\) gauge group sector, and to construct the Yukawa terms giving mass to particles with both positive and negative weak isospin.\footnote{For a more detailed explanation, I will refer the reader to~\cite{Martin:1997ns}.}
This means that there are two scalar Higgs doublets \(H_u, H_d\) before electroweak symmetry breaking (EWSB), giving mass to fermions in the upper/lower part of \(SU(2)_L\) fermion doublets respectively.
For the anomaly cancellation to work out, we must require hypercharge \(\sfrac{+1}{2}\) for \(H_u\) and \(\sfrac{-1}{2}\) for \(H_d\).
These scalar Higgs field doublets are embedded in left-handed chiral superfields \(H_{u/d}\) together with fermion superpartners.
The superfield doublet components are named according to \(H_u = (H_u^+, H_u^0)^T\) and \(H_d = (H_d^0, H_d^-)^T\), where the superscript indicates the post EWSB electric charge of the superfields.
The fermion partners to both the vector bosons and the Higgs bosons are called \emph{bosinos} collectively.
For reference all the superfields in the MSSM, their symbols and their component fields are tabulated in \cref{susy:tab:MSSM-fields}.

{
\renewcommand{\arraystretch}{1.5}
\begin{table}[ht!]
  \centering
  \begin{tabular}{|l|c|c|ccc|}
    \hline
                                             & \multicolumn{2}{|c|}{Superfield} & Boson field    & Fermion field           & Auxiliary field                                 \\
    \hline
    \multirow{4}{*}{\rotatebox{90}{Higgs}}   & \multirow{2}{*}{\(H_u\)}         & \(H_{u}^{+}\)  & \(H_{u}^{+}\)           & \(\tilde{H}_{u}^{+}\) & \(F_{H^+_{u}}\)         \\
                                             &                                  & \(H_u^0\)      & \(H_u^0\)               & \(\tilde{H}_u^0\)     & \(F_{H^0_u}\)           \\
    \cline{2-6}
                                             & \multirow{2}{*}{\(H_d\)}         & \(H_d^0\)      & \(H_d^0\)               & \(\tilde{H}_d^0\)     & \(F_{H^0_d}\)           \\
                                             &                                  & \(H_{d}^{-}\)  & \(H_{d}^{-}\)           & \(\tilde{H}_{d}^{-}\) & \(F_{H^-_{d}}\)         \\
    \hline
    \multirow{3}{*}{\rotatebox{90}{Leptons}} & \multirow{2}{*}{\(L_i\)}         & \(\nu_i\)      & \(\tilde{\nu}_{iL}\)    & \(\nu_i\)             & \(F_{\nu_i}\)           \\
                                             &                                  & \(l_i\)        & \(\tilde{l}_{iL}\)      & \(l_i\)               & \(F_{l_i}\)             \\
    \cline{2-6}
                                             & -                                & \(\wbar{E}_i\) & \(\tilde{l}_{iR}^\ast\) & \(\wbar{e}_{i}\)      & \(F_{\wbar{E}_i}^\ast\) \\
    \hline
    \multirow{4}{*}{\rotatebox{90}{Quarks}}  & \multirow{2}{*}{\(Q_i\)}         & \(u_i\)        & \(\tilde{u}_{iL}\)      & \(u_i\)               & \(F_{u_i}\)             \\
                                             &                                  & \(d_i\)        & \(\tilde{d}_{iL}\)      & \(d_i\)               & \(F_{d_i}\)             \\
    \cline{2-6}
                                             & -                                & \(\wbar{U}_i\) & \(\tilde{u}_{iR}^\ast\) & \(\wbar{u}_i\)        & \(F_{\wbar{U}_i}^\ast\) \\
                                             & -                                & \(\wbar{D}_i\) & \(\tilde{d}_{iR}^\ast\) & \(\wbar{d}_i\)        & \(F_{\wbar{D}_i}^\ast\) \\
    \hline
    \multirow{4}{*}{\rotatebox{90}{Bosons}}
                                             & -                                & \(B^0\)        & \(B^0_\mu\)             & \(\bino\)             & \(D_{B^0}\)             \\
    \cline{2-6}
                                             & \multirow{2}{*}{\(W^{k}\)}       & \(W^0\)        & \(W^0_\mu\)             & \(\wino\)             & \(D_{W^0}\)             \\
                                             &                                  & \(W^\pm\)      & \(W^\pm_\mu\)           & \(\tilde{W}^\pm\)     & \(D_{W^\pm}\)           \\
    \cline{2-6}
                                             & -                                & \(C^a\)        & \(C^a_\mu\)             & \(\tilde{g}\)         & \(D_C\)                 \\
    \hline
  \end{tabular}
  \caption{Table of superfields of the MSSM, and their component field names.
    Note that the fermion fields are left-handed Weyl spinors, in spite of any \(L\) or \(R\) in the boson field subscript.
    The conjugate superfields changes these to right-handed Weyl spinors.
    The indices \(i\) enumerate the three generations of leptons/quarks, \(k\) the three \(SU(2)_L\) gauge fields and \(a\) the eight \(SU(3)_C\) gauge fields.}
  \label{susy:tab:MSSM-fields}
\end{table}
}



\subsection{Superlagrangian and Supersymmetry Breaking}
Now that we have defined the field content of the MSSM, we need to define the interaction between them through the superlagrangian.
As has already been noted, the gauge groups of the MSSM are the same as for the SM, and all kinetic terms are defined according to the super Yang-Mills theory of \cref{susy:ssec:susy_yang_mills}.
A summary of the gauge numbers of the scalar superfields is given in \cref{susy:tab:mssm_quantum_numbers}.
This results in the kinetic part of the MSSM superlagrangian being
\begin{align}
  \nonumber
  \L^\text{MSSM}_\text{kin} = & H_u^\dagger \exp{g^\prime B + 2g\mathcal{W}} H_u + H_d^\dagger \exp{-g^\prime B + 2g\mathcal{W}} H_d + L_i^\dagger \exp{-g^\prime B + 2g\mathcal{W}} L_i + \wbar{E}_i^\dagger \exp{2g^\prime B} \wbar{E}_i             \\
  \nonumber
                              & + Q_i^\dagger \exp{\frac{1}{3}g^\prime + 2g\mathcal{W} + 2g_s \mathcal{C}} Q_i + \wbar{U}_i^\dagger \exp{-\frac{4}{3}g^\prime + 2g_s \mathcal{C}} \wbar{D}_i + \exp{\frac{2}{3}g^\prime + 2g_s \mathcal{C}} \wbar{D}_i \\
                              & \frac{1}{4} B\^\alpha B\_\alpha + \frac{1}{2} \tr\{\mathcal{W}\^\alpha \mathcal{W}\_\alpha\} + \frac{1}{2} \tr\{\mathcal{C}\^\alpha \mathcal{C}\_\alpha\},
\end{align}
where \(B\^\alpha, \mathcal{W}\^\alpha, \mathcal{C}\^\alpha\) are the supersymmetric field strengths of the gauge superfields \(B\), \(\mathcal{W} = W^k \frac{1}{2}\sigma\^k\) and \(\mathcal{C} = C^a \frac{1}{2}\lambda^a\) respectively. The matrices \(\lambda^a\) are the Gell-Mann matrices --- the generators of \(SU(3)\).

The superpotential up to gauge invariant and \(R\)-parity conserving terms is given by
\begin{equation}
  W_\text{MSSM} = \mu H_u^T i\sigma_2 H_d + y^e_{ij} (L_i^T i\sigma_2 H_d) \wbar{E}_j + y^u_{ij} (Q_i^T i\sigma_2 H_u) \wbar{U}_j + y^d_{ij} (Q_i^T i\sigma_2 H_d) \wbar{D}_j,
\end{equation}
where \(\mu\) is some complex, massive parameters and \(y^{e/u/d}_{ij}\) are the ordinary SM Yukawa couplings.
This leaves two new degrees of freedom in the MSSM superpotential beyond what is in the SM -- the phase and magnitude of \(\mu\).

Seeing as we have not discovered any particles with the same mass but opposite spin-statistics to the SM particles we know, we must conclude that supersymmetry is broken at low energy.
A mechanism for spontaneous symmetry breaking of supersymmetry would therefore be necessary.
Constructing such a mechanism in a way as to not reintroduce the hierarchy problem leads to what we call \emph{soft} breaking of supersymmetry~\cite{Martin:1997ns}.
Disregarding the high-energy completion of the theory, we can parametrise the terms that can arise in the low-energy MSSM superlagrangian up to gauge invariant and \(R\)-parity conserving terms, as
\begin{align}
  \label{susy:eq:LMSSM_soft}
  \nonumber
  \L_\text{soft}^\text{MSSM} = & (\theta\theta)(\theta\theta)^\dagger \bigg\{ -\frac{1}{4}M_1 B\^\alpha B\_\alpha - \frac{1}{2}M_2 \tr\{\mathcal{W}\^\alpha \mathcal{W}\_\alpha\} - \frac{1}{2}M_3 \tr\{\mathcal{C}\^\alpha \mathcal{C}\_\alpha\} + \cc \\
  \nonumber
                               & -\frac{1}{6}a^{e}_{ij} L_i^T i\sigma_2 H_d \wbar{E}_j - \frac{1}{6}a^{u}_{ij} Q_i^T i\sigma_2 H_u \wbar{U}_j - \frac{1}{6}a^{d}_{ij} Q_i^T i\sigma_2 H_d \wbar{D}_j + \cc                                              \\
  \nonumber
                               & -\frac{1}{2} b H_u^T i\sigma_2 H_d + \cc                                                                                                                                                                               \\
  \nonumber
                               & -(m_{ij}^L)^2 L_i^\dagger L_j - (m_{ij}^e)^2 \wbar{E}_i^\dagger \wbar{E}_j - (m_{ij}^Q)^2 Q_i^\dagger Q_j - (m_{ij}^u)^2 \wbar{U}_i^\dagger \wbar{U}_j - (m_{ij}^d)^2 \wbar{D}_i^\dagger \wbar{D}_j                    \\
                               & - m_{H_u}^2 H_u^\dagger H_u - m_{H_d}^2 H_d^\dagger H_d\bigg\}.
\end{align}
All the parameters are potentially complex numbers, although all the mass terms \(m_{ij}^2\) must be hermitian in the that \(m_{ij}^2 = (m_{ji}^2)^\ast\), which leads to \(m_{ii}^2\) having to be real.
This is the source of the great many parameters of the MSSM, as these soft-breaking parameters alone amount to 109 degrees of freedom!\footnote{A few of these can be eliminated through field redefinitions, however.}
For this reason, most searches of the MSSM focus on various simplified models~\cite{ATLAS:2023act,ATLAS:2024tqe}.
These can be based on simplifications like assuming all parameters to be real or assuming no flavour-violation as in the phenomenological MSSM (pMSSM)~\cite{pMSSM}, or by making theoretical assumptions on the specific mechanism for symmetry breaking, as in minimal supergravity (mSUGRA)~\cite{mSUGRA}, or any combinations of these.
In this thesis, I will not make any such assumptions and work with the general form of the MSSM, unless otherwise stated.
The full MSSM superlagrangian is then
\begin{equation}
  \L_\text{MSSM} = \L^\text{MSSM}_\text{kin} + (\theta\theta)^\dagger W_\text{MSSM} + (\theta\theta) W_\text{MSSM}^\dagger + \L^\text{MSSM}_\text{soft}.
\end{equation}
{
\renewcommand{\arraystretch}{1.4}
\begin{table}[ht!]
  \centering
  \begin{tabular}{|l|c|c|c|c|c|c|}
    \hline
                                             & \multicolumn{2}{|c|}{Superfield} & Hypercharge \(Y\) & Isospin \(I^3\)   & Electric Charge \(Q_e\) & Colour                  \\
    \hline
    \multirow{4}{*}{\rotatebox{90}{Higgs}}   & \multirow{2}{*}{\(H_u\)}         & \(H_u^+\)         & \(\sfrac{+1}{2}\) & \(\sfrac{+1}{2}\)       & \(+1\)            & -   \\
                                             &                                  & \(H_{u}^{0}\)     & \(\sfrac{+1}{2}\) & \(\sfrac{-1}{2}\)       & \(0\)             & -   \\
    \cline{2-7}
                                             & \multirow{2}{*}{\(H_d\)}         & \(H_d^0\)         & \(\sfrac{-1}{2}\) & \(\sfrac{+1}{2}\)       & \(0\)             & -   \\
                                             &                                  & \(H_{d}^{-}\)     & \(\sfrac{-1}{2}\) & \(\sfrac{-1}{2}\)       & \(-1\)            & -   \\
    \hline
    \multirow{3}{*}{\rotatebox{90}{Leptons}} & \multirow{2}{*}{\(L_i\)}         & \(\nu_i\)         & \(\sfrac{-1}{2}\) & \(\sfrac{+1}{2}\)       & \(0\)             & -   \\
                                             &                                  & \(l_i\)           & \(\sfrac{-1}{2}\) & \(\sfrac{-1}{2}\)       & \(-1\)            & -   \\
    \cline{2-7}
                                             & -                                & \(\wbar{E}_i\)    & \(+1\)            & -                       & \(+1\)            & -   \\
    \hline
    \multirow{4}{*}{\rotatebox{90}{Quarks}}  & \multirow{2}{*}{\(Q_i\)}         & \(u_i\)           & \(\sfrac{+1}{6}\) & \(\sfrac{+1}{2}\)       & \(\sfrac{+2}{3}\) & yes \\
                                             &                                  & \(d_i\)           & \(\sfrac{+1}{6}\) & \(\sfrac{-1}{2}\)       & \(\sfrac{-1}{3}\) & yes \\
    \cline{2-7}
                                             & -                                & \(\wbar{U}_i\)    & \(\sfrac{-2}{3}\) & -                       & \(\sfrac{-2}{3}\) & yes \\
                                             & -                                & \(\wbar{D}_i\)    & \(\sfrac{+1}{3}\) & -                       & \(\sfrac{+1}{3}\) & yes \\
    \hline
  \end{tabular}
  \caption{Summary of quantum numbers for the MSSM scalar superfields.
    The charges of barred fields \(\wbar{F}\) supplying the right-handed part of SM fermions are defined such that the charge of \(\wbar{F}^\dagger\) matches that of its left-handed compliment.
    I note that the convention for the hypercharge differs from some sources, seeing as I use \(1\) as the generator of \(U(1)_Y\) instead of \(\frac{1}{2}\) used elsewhere.
    This amounts to shuffling some factors of \(\frac{1}{2}\) around.
    The indices \(i\) enumerate the three generations of leptons/quarks.}
  \label{susy:tab:mssm_quantum_numbers}
\end{table}
}




\section{Electroweakinos}
\label{susy:sec:electroweakinos}
The focus in this thesis we be on a particular set of superpartners, namely the \emph{electroweakinos}.
These are fermion superpartners to the electroweak bosons, i.e.\ the photon, \(Z\) and \(W\) bosons and the Higgs bosons.
These are subdivided into the vector boson superpartners, the \emph{gauginos}, and the Higgs boson partners, the \emph{higgsinos}.
Before EWSB, the gauge fields naturally occurring in the Lagrangian are the \(B\)- and \(W^k\)-fields, and it is customary to work with the fermion superpartners of these fields.
These are naturally called the \emph{binos} and \emph{winos} respectively.

\subsection{Mass mixing}
After EWSB, we get two oppositely charged winos, and two mixed bino/wino states, mirroring the electroweak gauge bosons.
However, the higgsinos come in an oppositely charged pair and two neutral ones, so the gauginos and higgsinos can further mix.
So the general electroweak fermionic sector in the MSSM includes two particle-antiparticle pairs of charged Dirac fermions, and four neutral Majorana fermions, respectively referred to as \emph{charginos} and \emph{neutralinos}.
The two chargino fields are denoted with the Weyl spinors \(\tilde\chi_{i=1,2}^\pm\), and the four neutralinos are denoted with the Weyl spinors \(\tilde\chi_{i=1,2,3,4}^0\).
The indices \(i\) are numbered according to the mass hierarchy, with 1 being the lightest chargino/neutralino and 2/4 being the heaviest.

Ignoring higher order corrections, the mass terms for the gauginos and higgsinos in the MSSM Lagrangian can be structured as
\begin{equation}
  \L_{\tilde\chi\text{-mass}} = -\frac{1}{2}(\psi^0)^T M_{\tilde\chi^0} \psi^0 - \frac{1}{2}{\psi^\pm}^T M_{\tilde\chi^\pm} \psi^\pm + \cc,
\end{equation}
where \(\psi^0 = \pclosed{\bino, \wino, \hinod, \hinou}^T\), \(\psi^\pm = \pclosed{\psi^+, \psi^-}^T = \pclosed{\tilde{W}^+, \tilde{H}_u^+, \tilde{W}^-, \tilde{H}_d^-}^T\) and \(M_{\tilde\chi^0}\), \(M_{\tilde\chi^\pm}\) are the neutralino and chargino mass matrices respectively.
They are given by
\begin{align}
  \label{susy:eq:neutralino_mixing_matrix}
  M_{\tilde\chi^0} =   & \begin{bmatrix}
                           M_1              & 0                & -m_Z c_\beta s_W & m_Z s_\beta s_W  \\
                           0                & M_2              & m_Z c_\beta c_W  & -m_Z s_\beta c_W \\
                           -m_Z c_\beta s_W & m_Z c_\beta c_W  & 0                & -\mu             \\
                           m_Z s_\beta s_W  & -m_Z s_\beta c_W & -\mu             & 0
                         \end{bmatrix},             \\
  \label{susy:eq:chargino_mixing_matrix}
  M_{\tilde\chi^\pm} = & \begin{bmatrix}
                           0                   & 0                   & M_2                 & \sqrt{2}c_\beta m_W \\
                           0                   & 0                   & \sqrt{2}s_\beta m_W & \mu                 \\
                           M_2                 & \sqrt{2}s_\beta m_W & 0                   & 0                   \\
                           \sqrt{2}c_\beta m_W & \mu                 & 0                   & 0
                         \end{bmatrix},
\end{align}
where \(s_\beta \equiv \sin\beta\), \(c_\beta = \equiv \cos\beta\) and \(\beta\) is defined from the relation
\begin{eqnarray}
  \tan\beta = \frac{v_u}{v_d},
\end{eqnarray}
where \(v_{u,d}\) are the vacuum expectation values of the \(H_{u/d}\) field after electroweak symmetry breaking.

These mass matrices can be diagonalised to get the mass eigenstate \emph{neutralinos} \(\tilde\chi^0_i\) and \emph{charginos} \(\tilde\chi^\pm_i\), respectively.
Both the matrices are symmetric, but we will diagonalise them slightly differently.
The neutralino mass matrix can be diagonalised by a unitary matrix \(N\) such that
\begin{align}
  \nonumber
  \L_{\tilde\chi^0\text{-mass}} = & -\frac{1}{2}(\psi^0)^T M_{\tilde\chi^0} \psi^0 + \cc = -\frac{1}{2} \underbrace{(\psi^0)^T N^T}_{\equiv (\tilde{\chi}^{0})^T} \underbrace{N^\ast M_{\tilde\chi^0} N^\dagger}_{= \operatorname{diag}(m_{\tilde\chi^0_1}, \ldots, m_{\tilde\chi^0_4})} \underbrace{N \psi^0}_{\equiv \tilde\chi^0} + \cc \\
  =                               & -\frac{1}{2} (\tilde\chi^0)^T \operatorname{diag}(m_{\tilde\chi^0_1}, \ldots, m_{\tilde\chi^0_4}) \tilde\chi^0 + \cc
\end{align}
This factorisation is guaranteed by Takagi factorisation, which I prove in \cref{app:chap:takagi}.
When \(M_1\), \(M_2\) and \(\mu\) are real-valued, we can guarantee that the mixing matrix \(N\) is real and orthogonal -- however, this will cause at least one of the neutralino masses to have a negative sign.
This is the assumption for the SUSY Les Houches Accord (SLHA1)~\cite{SLHA1}.
In this thesis, I will allow the mixing matrices to be complex, and enforce positive neutralino masses.
Details on the realisation of the neutralino mixing matrix is given in the next section \cref{susy:sec:takagi}.

The chargino mass matrix is handled slightly differently, seeing as it has the structure \(M_{\tilde\chi^\pm} = \begin{bmatrix}
  0 & X \\ X^T & 0
\end{bmatrix}\).
Using singular value decomposition, we can write \(X = U^T D V\) for two unitary matrices \(U\), \(V\) and a diagonal matrix of positive singular values \(D = \operatorname{diag}(m_{\tilde\chi^\pm_1}, m_{\tilde\chi^\pm_2})\).
This results in
\begin{align}
  \nonumber
  \L_{\tilde\chi^\pm\text{-mass}} = & -\frac{1}{2}{\psi^\pm}^T M_{\tilde\chi^\pm} \psi^\pm + \cc = -\frac{1}{2} \begin{pmatrix}
                                                                                                                  \psi^+ \\ \psi^-
                                                                                                                \end{pmatrix}^T \begin{bmatrix}
                                                                                                                                  0       & U^T D V \\
                                                                                                                                  V^T D U & 0
                                                                                                                                \end{bmatrix} \begin{pmatrix}
                                                                                                                                                \psi^+ \\ \psi^-
                                                                                                                                              \end{pmatrix} + \cc                                                                                                              \\
  \nonumber
  =                                 & -\frac{1}{2} \underbrace{(\psi^+)^T U^T}_{\equiv (\tilde\chi^+)^T} D \underbrace{V \psi^-}_{\equiv \tilde\chi^-} - \frac{1}{2} \underbrace{(\psi^-)^T V^T}_{\equiv (\tilde\chi^-)^T} D \underbrace{U \psi^+}_{\equiv \tilde\chi^+} + \cc \\
  =                                 & -(\tilde\chi^+)^T \operatorname{diag}(m_{\tilde\chi^\pm_1}, m_{\tilde\chi^\pm_2}) \tilde\chi^- + \cc
  % =                                 & -\frac{1}{2} \begin{pmatrix}
  %                                                   (\psi^+)^T U^T & (\psi^-)^T V^T
  %                                                 \end{pmatrix} \begin{bmatrix}
  %                                                                 0 & D \\ D & 0
  %                                                               \end{bmatrix}
  % \begin{pmatrix}
  %   U \psi^+ \\ V \psi^-
  % \end{pmatrix} + \cc                                                                                                                                           \\
  % \equiv                            & -\frac{1}{2} \begin{pmatrix}
  %                                                   (\tilde\chi^+)^T & (\tilde\chi^-)^T
  %                                                 \end{pmatrix} \begin{bmatrix}
  %                                                                 0 & D \\ D & 0
  %                                                               \end{bmatrix}
  % \begin{pmatrix}
  %   \tilde\chi^+ \\ \tilde\chi^-
  % \end{pmatrix} + \cc
\end{align}
This tells us that there are two doubly degenerate mass eigenvalues of the chargino mass matrix, constituting two massive Dirac fermion particle-antiparticle pairs.

Proof of Takagi factorisation and an algorithm for realising it are done in \cref{app:chap:takagi}.




\subsection{Feynman Rules for Neutralinos}
\label{susy:sec:feynman_rules}
To calculate the cross-section for electroweakino production later on, we will need the Feynman rules of the relevant particle interactions.
I will not explicitly derive the Feynman rules for all the electroweakinos, but rather exemplify how they can be derived by deriving all the relevant neutralino interactions from the MSSM superlagrangian.
The relevant Feynman rules for the remaining electroweakino processes follow in much the same manner, and are listed in the end.

\subsubsection*{Fermion Interactions in Super Yang-Mills and Yukawa Theory}
I will start by deriving the interactions of fermions in the chiral superfields and vector superfields of a supersymmetric Yang-Mills superlagrangian.
As a reminder, the super Yang-Mills superlagrangian kinetic term is \(\Phi^\dagger_i \pclosed{\exp{2q \mathcal{V}}}_{ij}\Phi_j\).
Extracting the interaction terms containing either the fermion field multiplets \(\psi\) (\(\psi^\dagger\)) from the left-handed (right-handed) scalar superfield multiplets \(\Phi\) (\(\Phi^\dagger\)),
and the fermion fields \(\lambda \equiv \lambda^a T^a\) from vector superfields \(\mathcal{V} \equiv V^a T^a\) from terms with the appropriate amount of \(\theta\)'s to survive the projection of \cref{susy:eq:projection_lagragnian}, we have
\begin{align}
  \label{susy:eq:psi_kin_int}
  \nonumber
  \L \stackrel{\psi, \psi^\dagger, \lambda}{\supset} & 2q \sum_{ij} \bigg\{ A^\ast_i (\theta\theta)^\dagger (\theta\lambda_{ij}) \sqrt{2} (\theta\psi_j) + \sqrt{2} (\theta\psi_i)^\dagger (\theta\sigma\^\mu\theta^\dagger) \pclosed{\mathcal{V}_\mu}_{ij} \sqrt{2} (\theta\psi_j) \\
  \nonumber
                                                     & + \sqrt{2} (\theta\psi_i)^\dagger (\theta\theta) (\theta\lambda_{ij})^\dagger A_j \bigg\}                                                                                                                                    \\
  =                                                  & q(\theta\theta)(\theta\theta)^\dagger \sum_{ij} \cclosed{ -\sqrt{2} A^\ast_i (\lambda_{ij}\psi_j) + (\psi_i \sigma\^\mu \pclosed{\mathcal{V}_\mu}_{ij} \psi^\dagger_j) - \sqrt{2} (\psi_i\lambda_{ij})^\dagger A_j },
\end{align}
where I have used Weyl spinor relations listed in \cref{app:chap:weyl_grassmann}.
\medskip

There are also Yukawa terms coming from the superpotential terms of the form \(y_{ij} (\theta\theta)^\dagger \Phi_i \Phi \Phi_j + \cc\)
Here \(\Phi\) will later represent one of the Higgs superfields.
Extracting the interaction terms of fermion field \(\psi\) from \(\Phi\), we find
\begin{align}
  \label{susy:eq:psi_yuk_int}
  \nonumber
  \L & \stackrel{\psi, \psi^\dagger}{\supset} y_{ij} (\theta\theta)^\dagger \sqrt{2} (\theta\psi) \cclosed{ A_i \sqrt{2}(\theta\psi_i) + \sqrt{2}(\theta\psi_j)A_j } + \cc \\
     & = -y_{ij} (\theta\theta)(\theta\theta)^\dagger \cclosed{ A_i(\psi\psi_j) + (\psi_i\psi)A_j + \cc }
\end{align}


\subsubsection*{Wino and Bino Interactions}
First, I will look at the bino and wino interactions coming from the kinetic terms.
Writing out the \(W^a\) vector superfields in the basis \(W^\pm, W^0\), we are now only interested in the electrically neutral \(W^0\) bit.
The interactions will come from kinetic terms of scalar superfields \(\Phi\), whose relevant part can be written as
\begin{equation}
  \L = \Phi^\dagger \exp{2g\cclosed{Y t_W B^0 \pclosed{+ \frac{1}{2}\sigma_3 W^0}}} \Phi,
\end{equation}
where \(t_W \equiv \tan\theta_W\) is the tangent of the Weinberg angle, \(Y\) is the hypercharge of \(\Phi\) and the term in parentheses only appears for fields in \(SU(2)_L\) superfield doublets.
To generalise this, I will use the isospin \(I^3\), which is \(+\frac{1}{2}\) for fields in the upper part of an SU(2) doublet, \(-\frac{1}{2}\) for fields in the lower part and 0 for SU(2) singlet fields.
Then the kinetic term can be written compactly as
\begin{equation}
  \L = \Phi^\dagger \exp{2g\cclosed{(Q_e - I^3) t_W B^0 + I^3 W^0}} \Phi,
\end{equation}
where \(Q_e\) is the electric charge of \(\Phi\).
\medskip

Extracting the interactions of the fermion fields \(\bino, \wino\) in \(B^0,
W^0\) using \cref{susy:eq:psi_kin_int}, we are left with (up to appropriate
\(\theta\)'s)
\begin{equation}
  \L \stackrel{\bino, \wino}{\supset} -\sqrt{2}g (\theta\theta)(\theta\theta)^\dagger \Big\{ (Q_e - I^3)t_W (\bino \psi) A^\ast + I^3 (\wino \psi) A^\ast + \cc \Big\}.
\end{equation}

Consider an SM quark \(q\), from the scalar superfield components \(A\) and \(\psi\) contained in the superfields \(Q\) and \(\bar{Q}\), with
electric charge \(Q_e\) and weak isospins \(I^3\) and 0 respectively, we can write out the interaction as
\begin{equation}
  \L = -\sqrt{2}g \Big\{ (Q_e - I^3)t_W (\bino q) \tilde{q}_{L}^\ast + I^3 (\wino q) \tilde{q}_{L}^\ast + Q_e t_W (\bino \bar{q}) \tilde{q}_{R}^\ast + \cc \Big\}.
\end{equation}
Changing to the \(\nino\)-basis, we have that \(\bino = \sum_i N_{i1}^\ast \nino[i]\), \(\wino = \sum_i N_{i2}^\ast \nino[i]\), which together with writing out the Weyl products on Dirac spinor form yields
\begin{equation}
  \label{susy:eq:nino_sf_f_int}
  \L_{\tilde{q}q\tilde{\chi}^0} = -\sqrt{2} g \sum_{i} \barnino[i] \Big\{ \big[ \underbrace{\pclosed{Q_e - I^3}t_W N_{i1}^\ast  + I^3 N_{i2}^\ast}_{\equiv \Big(C_{\tilde{q} q \tilde{\chi}_i^0}^{L}\Big)^{\!\ast}} \big] \tilde{q}_{L}^\ast P_L \underbrace{- Q_f t_W N_{i1}}_{\equiv \Big(C_{\tilde{q} q \tilde{\chi}_i^0}^{R}\Big)^{\!\ast}} \tilde{q}_{R}^\ast P_R \Big\} q + \cc,
\end{equation}
where we understand \(\tilde\chi^0_i\) and \(q\) as Dirac spinors.

More generally, mixing can occur between the left- and right-handed chiral squark states.
The mass terms mixing the chiral states come from Yukawa terms in the superpotential and soft-breaking potential, and as such it is most prevalent in the third generation where the Yukawa couplings are larger from the SM\@.
SLHA1 standard~\cite{SLHA1} assumes no such mixing in the first two generations, but does allow for it in the last generation.

Without flavour-violation, we have that the squarks of flavour \(q\) mix such that we get the mass eigenstates by
\begin{equation}
  \tilde{q}_{A} = R_{A1}^{\tilde{q}} \tilde{q}_{L} + R_{A2}^{\tilde{q}} \tilde{q}_{R},
\end{equation}
where \(R^{\tilde{q}}\) is a \(2\times 2\) unitary matrix.
As such, we can write \(\tilde{q}_{L} = \sum_A \pclosed{R^{\tilde{q}}_{A1}}^{\!\ast} \tilde{q}_{A}\), \(\tilde{q}_{R} = \sum_A \pclosed{R^{\tilde{q}}_{A2}}^{\!\ast} \tilde{q}_{A}\) to get
\begin{equation}
  \label{susy:eq:sqqneu-interaction-L}
  \L_{\tilde{q} q \tilde{\chi}^0} = -\sqrt{2} g \sum_{i} \sum_A \barnino[i] \Big\{ \underbrace{R^{\tilde{q}}_{A1} \pclosed{C_{\tilde{q} q \tilde{\chi}_i^0}^{L}}^{\!\ast}}_{\equiv \Big(C_{\tilde{q}_A q \tilde{\chi}_i^0}^{L}\Big)^{\!\ast}} P_L + \underbrace{R^{\tilde{q}}_{A2} \pclosed{C_{\tilde{q} q \tilde{\chi}_i^0}^{R}}^{\!\ast}}_{\equiv \Big(C_{\tilde{q}_A q \tilde{\chi}_i^0}^{R}\Big)^{\!\ast}} P_R \Big\} \tilde{q}_{A}^\ast q + \cc
\end{equation}



\subsubsection*{Flavour-Violating Squark Sector}
The previous derivation was done under the assumption that squarks do not mix between fermion generations, violating flavour number.
However, this can happen if there are non-zero supersymmetry-breaking parameters coupling squarks between generations or if loop corrections are added to the squark sector.
The generalisation is fairly straight forward: Instead of one unitary, \(2 \by 2\) mixing matrix \(R^{\tilde{q}}\) for each of the six quark flavours \(q = u, d, s, c, t, b\), there is one \(6 \by 6\) mixing matrix \(R^{\tilde{q}}\) for each of the two quark types \(q = u, d\).
These mixing matrices can be defined using different conventions, but in this thesis I will follow the SLHA2 standard~\cite{SLHA2}
\begin{equation}
  \begin{pmatrix}
    \tilde{q}_1 \\ \tilde{q}_2 \\ \tilde{q}_3 \\ \tilde{q}_4 \\ \tilde{q}_5 \\ \tilde{q}_6
  \end{pmatrix}
  = R^{\tilde{q}} \begin{pmatrix}
    \tilde{q}_{1L} \\ \tilde{q}_{2L} \\ \tilde{q}_{3L} \\ \tilde{q}_{1R} \\ \tilde{q}_{2R} \\ \tilde{q}_{3R}.
  \end{pmatrix}
\end{equation}
This means that the chiral squarks in generation \(g = 1,2,3\) will rather be given by
\begin{subequations}
  \begin{align}
    \tilde{q}_{gL} = \sum_A (R^{\tilde{q}}_{A,g})^\ast \tilde{q}_A, \\
    \tilde{q}_{gR} = \sum_A (R^{\tilde{q}}_{A,g+3})^\ast \tilde{q}_A.
  \end{align}
\end{subequations}
What this means for the interaction Lagrangian in \cref{susy:eq:nino_sf_f_int} is that the sum over \(A\) changes to go from 1 to 6 and the definition of the coupling parameter changes slightly to
\begin{subequations}
  \begin{align}
    C_{\tilde{q}_A q_g \tilde{\chi}_i^0}^{L} = & \pclosed{R^{\tilde{q}}_{A,g}}^{\!\ast} \bclosed{\pclosed{Q_e-I^3}t_W N_{i1} + I^3 N_{i2}}, \\
    C_{\tilde{q}_A q_g \tilde{\chi}_i^0}^{R} = & -\pclosed{R^{\tilde{q}}_{A,g+3}}^{\!\ast} Q_e t_W N_{i1}^\ast.
  \end{align}
\end{subequations}


\subsubsection*{Higgsino Interactions}
Higgsino interaction with the squarks comes from the Yukawa terms of the superpotential.
As they are the mass-giving terms of for the quarks, they are proportional to the quark masses.
Accordingly, at the center-of-mass energies at the LHC, for which the calculations of this thesis are intended, only the last generation of squarks will have non-negiglible couplings.
However, due low parton density for the bottom quark (and non-existent for the top quark), we can safely ignore these terms in the present derivation.

The remaining relevant interaction that remains then, is that with the \(Z\)-boson.
This interaction again comes from the kinetic term, but this time for the neutral
Higgs superfields in the superfield multiplets \(H_u = \pclosed{H_u^+,
  H_u^0}^T\), \(H_d = \pclosed{H_d^0, H_d^-}^T\). The Lagrangian is of the form
\begin{equation}
  \L = \pclosed{H_{u/d}^0}^\dagger \exp{\mp g\pclosed{W^0 - t_W B^0}} H_{u/d}^0.
\end{equation}
Integrating over the Grassmann variables and using equation \cref{susy:eq:psi_kin_int} we get
\begin{equation}
  \integral{^4\theta} \L \stackrel{\tilde{H}_{u/d}^0, W^0_\mu, B^0_\mu}{=} \mp \frac{g}{2} (\tilde{H}_{u/d}^0 \sigma\^\mu (\tilde{H}_{u/d}^0)^\dagger) (W^0_\mu - t_W B_\mu^0).
\end{equation}
Switching to Dirac spinors, the mass eigenbasis for the neutralinos and the \(Z\) boson \(Z_\mu = c_W W^0_\mu - s_W B^0_\mu\), we end up with\footnote{Here I have used some of the Weyl spinor identities from \cref{app:chap:weyl_grassmann}.}
\begin{align}
  \label{susy:eq:Zneuneu-interaction-L}
  \nonumber
  \L_{Z \tilde\chi^0_i \tilde\chi^0_j} = \frac{g}{2c_W} Z_\mu \sum_{ij}\pclosed{-N_{i4}N_{j4}^\ast + N_{i3}N_{j3}^\ast} \barnino[i] \gamma^\mu P_L \nino[j] \\
  = -\frac{g}{2} Z_\mu \sum_{ij}\barnino[i] \gamma^\mu \bigg[ \underbrace{\frac{1}{2c_W}\pclosed{N_{i4}N_{j4}^\ast - N_{i3}N_{j3}^\ast}}_{\equiv O^{\prime\prime L}_{ij}} P_L \underbrace{-\frac{1}{2c_W}\pclosed{N_{i4}^\ast N_{j4} - N_{i3}^\ast N_{j3}}}_{\equiv O^{\prime\prime R}_{ij}} P_R \bigg] \nino[j]
\end{align}


\subsubsection*{Summary of Coupling Definitions}
In summary, the Feynman rules for the interactions of neutralinos with the electroweak bosons and (s)quarks are given by the interaction Lagrangians in \cref{susy:eq:Zneuneu-interaction-L,susy:eq:sqqneu-interaction-L} as
\begin{subequations}
  \begin{align}
    % \vcenter{\hbox{
    %     \inputtikz{qqZ_vertex}
    %   }}
    %  & = -ig \gamma^\mu \bclosed{C_{qqZ}^L P_L + C_{qqZ}^R P_R}                                \\
    \label{susy:eq:feynman_Z-neu-neu}
    \vcenter{\hbox{
        \inputtikz{chii-chij-Z_vertex}
      }}
     & = -ig \gamma^\mu \bclosed{O_{ij}^{\prime\prime L} P_L +
      O_{ij}^{\prime\prime R} P_R},
    \\
    \label{susy:eq:feynman_q-sq-neu}
    \vcenter{\hbox{
        \inputtikz{q-chi-sq_vertex}
      }}
     & = -i\sqrt{2}g \bclosed{ \pclosed{C_{\tilde{q}_A q_g \tilde{\chi}_i^0}^{L}}^{\!\ast} P_L
      +
      \pclosed{C_{\tilde{q}_A q_g \tilde{\chi}_i^0}^{R}}^{\!\ast} P_R }.
  \end{align}
\end{subequations}
In fact, the interactions of all electroweakinos  with \(W/Z\)-bosons and (s)quarks take the same form, and we can generalise by replacing \(O_{ij}^{\prime\prime X}\) or \(C_{\tilde{q}_A q_g \tilde\chi^0_i}^X\) with the appropriate definitions in \cref{susy:tab:variable_definitions}~\cite{HaberKane}.

A couple of remarks these Feynman rules:
The rule \cref{susy:eq:feynman_Z-neu-neu} is a factor of two greater than the corresponding Lagrangian term \cref{susy:eq:Zneuneu-interaction-L} due to the symmetry between \(i, j\) in \cref{susy:eq:Zneuneu-interaction-L}.
Furthermore, when the incoming and outgoing states are reverse, the conjugate term of the Lagrangians \cref{susy:eq:sqqneu-interaction-L,susy:eq:Zneuneu-interaction-L} must be used, effectively conjugating the couplings in \cref{susy:eq:feynman_Z-neu-neu}, and conjugating the couplings \emph{and} switching \(L \leftrightarrow R\) in \cref{susy:eq:feynman_q-sq-neu}.
{\renewcommand{\arraystretch}{2}
\begin{table}[ht!]
  \centering
  \begin{tabular}{|c|c|c|}
    \hline
    Interaction                                            & Coupling                                            & Definition
    \\
    \hline
    \multirow{2}{*}{\(\tilde{q} q \tilde\chi^0\)}          & \(C_{\tilde{q}_A q_g \tilde{\chi}_i^0}^{L}\)        & \(\pclosed{R^{\tilde{q}}_{A,g}}^\ast \bclosed{\pclosed{Q_e-I^3_q} t_W N_{i1} + I^3_q N_{i2}}\)                                                                                             \\
                                                           & \(C_{\tilde{q}_A q_g \tilde{\chi}_i^0}^{R}\)        & \(-\pclosed{R^{\tilde{q}}_{A,g+3}}^\ast Q_e t_W N_{i1}^\ast\)                                                                                                                              \\
    \hline
    \multirow{2}{*}{\(W \tilde\chi^0 \tilde\chi^\pm\)}     & \(O_{ij}^L\)                                        & \(\frac{1}{\sqrt{2}} N_{i4} V_{j2}^\ast - N_{i2}V_{j1}^\ast\)                                                                                                                              \\
                                                           & \(O_{ij}^R\)                                        & \(-\frac{1}{\sqrt{2}} N_{i3}^\ast U_{j2} - N_{i2}^\ast U_{j1}\)                                                                                                                            \\
    \hline
    \multirow{2}{*}{\(Z \tilde\chi^\pm \tilde\chi^\mp\)}   & \(O_{ij}^{\prime L}\)                               & \(\frac{1}{c_W} \pclosed{V_{i1} V_{j1}^\ast + \frac{1}{2} V_{i2}V_{j2}^\ast - \delta_{ij} s_W^2}\)                                                                                         \\
                                                           & \(O_{ij}^{\prime R}\)                               & \(\frac{1}{c_W} \pclosed{U_{i1}U_{j1}^\ast + \frac{1}{2} U_{i2}U_{j2}^\ast - \delta_{ij} s_W^2}\)                                                                                          \\
    \hline
    \multirow{2}{*}{\(Z \tilde\chi^0 \tilde\chi^0\)}       & \(O^{\prime\prime L}_{ij}\)                         & \(\frac{1}{2c_W} \pclosed{N_{i4} N_{j4}^\ast - N_{i3} N_{j3}^\ast}\)                                                                                                                       \\
                                                           & \(O^{\prime\prime R}_{ij}\)                         & \(-\frac{1}{2c_W} \pclosed{N_{i4}^\ast N_{j4} - N_{i3}^\ast N_{j3}}\)                                                                                                                      \\
    \hline
    \multirow{3}{*}{\(\tilde{q} q^\prime \tilde\chi^\pm\)} & \(C_{\tilde{d}_A u_g \tilde{\chi}^\pm_i}^L\)        & \(\frac{1}{\sqrt{2}} U_{i1} \pclosed{R^{\tilde{d}}_{A,g}}^\ast V_{u_g d_g}^\text{CKM}\)                                                                                                    \\
                                                           & \(C_{\tilde{u}_A d_g \tilde{\chi}^\pm_i}^L\)        & \(\frac{1}{\sqrt{2}} V_{i1} \pclosed{R^{\tilde{u}}_{A,g}}^\ast \pclosed{V_{u_g d_g}^\text{CKM}}^\ast\)                                                                                     \\
                                                           & \(C_{\tilde{q}_A q^\prime_g \tilde{\chi}^\pm_i}^R\) & 0                                                                                                                                                                                          \\
    \hline
    \multirow{2}{*}{\(q q Z\)}                             & \(C_{qqZ}^L\)                                       & \(-\frac{I^3_q - Q_e s_W^2}{c_W}\)                                                                                                                                                         \\
                                                           & \(C_{qqZ}^R\)                                       & \(\frac{Q_e s_W^2}{c_W}\)                                                                                                                                                                  \\
    \hline
    \multirow{2}{*}{\(q q^\prime W\)}                      & \(C_{qq^\prime W}^L\)                               & \(-\frac{V^\text{CKM}_{q q^\prime}}{c_W}\)                                                                                                                                                 \\
                                                           & \(C_{qq^\prime W}^R\)                               & 0                                                                                                                                                                                          \\
    \hline
    \multirow{2}{*}{\(\tilde{q} \tilde{q} Z\)}             & \(C_{\tilde{q}_A \tilde{q}_B Z}^L\)                 & \(-\frac{I^3_q - Q_e s_W^2}{c_W} R^{\tilde{q}}_{A,g} \pclosed{R^{\tilde{q}}_{B,g}}^\ast = C_{qqZ}^{L} R^{\tilde{q}}_{A,g} \pclosed{R^{\tilde{q}}_{B,g}}^\ast\)                             \\
                                                           & \(C_{\tilde{q}_A \tilde{q}_B Z}^R\)                 & \(\frac{Q_e s_W^2}{c_W} R^{\tilde{q}}_{A,g+3} \pclosed{R^{\tilde{q}}_{B,g+3}}^\ast = C_{qqZ}^{R} R^{\tilde{q}}_{A,g+3} \pclosed{R^{\tilde{q}}_{B,g+3}}^\ast\)                              \\
    \hline
    \multirow{2}{*}{\(\tilde{q} \tilde{q} W\)}             & \(C_{\tilde{q}_A \tilde{q}_B^\prime W}^L\)          & \(-\frac{V^\text{CKM}_{q q^\prime}}{c_W} R^{\tilde{q}}_{A,g} \pclosed{R^{\tilde{q}^\prime}_{B,g}}^\ast = C_{qq^\prime W}^L R^{\tilde{q}}_{A,g} \pclosed{R^{\tilde{q}^\prime}_{B,g}}^\ast\) \\
                                                           & \(C_{\tilde{q}_A \tilde{q}_B^\prime W}^R\)          & 0                                                                                                                                                                                          \\
    \hline
  \end{tabular}
  \caption{A summary of the variables used in the derived Feynman rules and their definitions.
    Furthermore, it is extended with the Feynman rules beyond those derived explicitly in this thesis.}
  \label{susy:tab:variable_definitions}
\end{table}
}




\section{Diagonalisation and Takagi Factorisation}
\label{susy:sec:takagi}
In this section, I will talk briefly about the diagonalisation procedure for complex, symmetric matrices due to L. Autonne~\cite{Autonne} and T. Takagi~\cite{Takagi}.
Furthermore, I will present an algorithm for finding the diagonalising matrix numerically for a given symmetric matrix.

\subsection{Numerical Diagonalisation}
Finding the diagonalisation matrix \(U\) for some matrix \(A\) is not always entirely straight-forward when done numerically.
It often entails finding solutions to sets of linear equations like
\begin{equation}
  M \vec{x} = \lambda \vec{y},
\end{equation}
for some matrix \(M\), vectors \(\vec{x}, \vec{y}\) and number \(\lambda\).
For vectors

\subsection{Takagi Factorisation}
Consider a complex, symmetric \(n\by n\) matrix \(A\).
Takagi factorisation~\cite{Horn} tells us that there exists a unitary matrix \(U\), and a real, non-negative diagonal matrix \(D\) such that
\begin{equation}
  \label{susy:eq:takagi}
  A = U^T D U.
\end{equation}
I remark that \(U\) is potentially complex, so \(U^{-1} = U^\dagger \neq U^T\), so \cref{susy:eq:takagi} should not be confused with the ordinary diagonalisation of a real matrix \(R = U^{-1} D U\) where \(U^{-1} = U^T\).


\subsubsection*{Factorisation Procedure}
I would first like to outline a practical procedure for finding such a diagonalising matrix \(U\), and consequently \(D\).
It will be based on finding vector \(\vec{v} \in  \mathbb{C}^n\) that satisfies
\begin{equation}
  \label{susy:eq:takagi_vector}
  A \vec{v}^* = \sigma \vec{v},
\end{equation}
for some real, non-negative number \(\sigma\).
A vector satisfying the modified eigenvalue relation \cref{susy:eq:takagi_vector} is called a \emph{Takagi vector} for future reference.
Existence of these vectors for any matrix \(A\), where \(AA^\ast\) only has real, non-negative eigenvalues is detailed in \cref{app:chap:takagi}.\footnote{This is always true for a symmetric matrix, as \(AA^\ast = A^T A^\dagger = (AA^\ast)^\dagger\) must be hermitian.}

To find \(U\) then, I outline a procedure based on the proof for Takagi factorisation in~\cite{Horn}.
Given a Takagi vector \(\vec{v}\) of \(A\), and an orthonormal basis \(\cclosed{\vec{v}, \vec{v}_2, \ldots, \vec{v}_n}\) of \(\mathbb{C}^n\), I show in \cref{app:chap:takagi} that we can make a diagonalisation step on \(A\), writing it as
\begin{equation}
  A = V \begin{bmatrix}
    \sigma         & \boldsymbol{0} \\
    \boldsymbol{0} & A_2
  \end{bmatrix} V^T,
\end{equation}
where \(A_2\) is a symmetric \((n-1) \by (n-1)\) matrix and \(V\) is a unitary matrix with the aforementioned orthonormal basis as its columns.
This process can be repeated with \(A_2\) and so on until we have
\begin{equation}
  A = V_1 \cdots V_{n} \begin{bmatrix}
    \sigma_1                                            & \multicolumn{2}{c}{\raisebox{-7pt}{\mbox{\Large 0}}}   \\
                                                        & \ddots                                               & \\
    \multicolumn{2}{c}{\raisebox{7pt}{\mbox{\Large 0}}} & \sigma_n
  \end{bmatrix} V_{n}^T \cdots V_{1}^T,
\end{equation}
where
\begin{equation}
  V_p = \begin{bmatrix}
    \mathbb{I}_{(p-1) \by (p-1)} & \boldsymbol{0} \\
    \boldsymbol{0}               & \tilde{V}_p,
  \end{bmatrix}
\end{equation}
and \(\tilde{V}_p\) is the unitary matrix that makes a diagonalisation step on \(A_p\).
Comparing to \cref{susy:eq:takagi}, we find that
\begin{subequations}
  \begin{align}
    U & = V_{n}^T \cdots V_{1}^T,                                  \\
    D & = \operatorname{diag}\pclosed{\sigma_1, \ldots, \sigma_n}.
  \end{align}
\end{subequations}
It is easy to show that \(U\) is unitary, as promised, as all \(V_{p}\) are so.
Furthermore, by the properties of the Takagi vector, all the values \(\sigma_p\) are real and positive.
Now the values on the diagonal of \(D\) can be permuted to any order using a permutation matrix \(P\), such that we get
\begin{equation}
  A = U_P^T D_P U_P,
\end{equation}
where \(U_P = P U\) and \(D_P = P D P^T\).
It is rather straight-forward to show that \(U_P\) will still be unitary, and \(D_P\) diagonal.

An algorithmic implementation of this procedure is shown in \cref{app:alg:takagi}.
\begin{algorithm}
  \begin{algorithmic}
    \Ensure{dim(\(A\)) \(\geq 2\)}
    \State{\(\lambda, \vec{x}\) =  eigh(\(A A^\ast\))}

    \If{\(|(A\vec{x}^\ast) \cdot \vec{x}|^2 - |A\vec{x}^\ast|^2|\vec{x}|^2\) < \epsilon}
    \State{\(\vec{v} \gets\ \frac{\vec{x}}{|\vec{x}|}\)}
    \Else\
    \State{\(\vec{v} \gets\ A \vec{x}^\ast + \sqrt{|\lambda[1]|}\vec{x}\)}
    \State{\(\vec{v} \gets\ \frac{\vec{v}}{|\vec{v}|}\)}
    \EndIf\

    \State{\(\mu \gets\ \frac{\vec{v} \cdot \vec{x}}{|\vec{x}|^2}\)}
    \State{\(\phi_\mu\) \gets\ atan2(imag(\(\mu\)), real(\(\mu\)))}
    \State{\(\vec{v}\) \gets\ exp(\(i \phi_\mu/2\)) \(\vec{v}\)}

    \State{\(I\) \gets\ diag(\(N\))}{\(i \gets\ 1\)}
    \Repeat\
    \State{\(V\) \gets\ Matrix(cols=(\(\vec{v}\),deleteColumn(\(I\), \(i\))))}
    \State{\(i \gets\ i+1\)}
    \Until{det(\(V\)) > \(\epsilon\) or \(i > N\)}

    \State{\(V\) \gets\ GramSchmidt(\(V\))}
    \State{\(U \gets\ V^\ast\)}
  \end{algorithmic}
  \caption{Diagonalisation step on an \(n\by n\) symmetric matrix \(A\) to find a matrix \(U\) s.t.
    \(A = U^T \begin{bmatrix} \sigma & \vec{0} \\ \vec{0} & A^\prime \end{bmatrix} U,\)
    where \(\sigma = \sqrt{|\lambda|}\) for some eigenvalue \(\lambda\) of \(AA^\ast\) and \(A^\prime\) is an \((n-1) \by (n-1)\) symmetric matrix.
    The algorithm relies on a function eigh(\(M\)) to give an eigenvalue with its corresponding eigenvector of a hermitian matrix \(M\), and GramSchmidt(\(M\)) to orthogonalise a complex, invertible matrix \(M\). The algorithm also relies on some machine precision parameter \(\epsilon\).}
  \label{app:alg:takagi}
\end{algorithm}


\ifSubfilesClassLoaded{%
  \bibliography{references}{}
  \bibliographystyle{style/JHEP}
}{}

\end{document}

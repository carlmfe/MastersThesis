\documentclass[../main.tex]{subfiles}

\begin{document}
\chapter{Quantum Field Theory}
In this chapter, I will go through the basics of quantum field theory (QFT) as relevant to this thesis.
I begin by formulating the perturbative approach to calculating correlation functions in interacting theories, and relate them to scattering cross-sections through the \(S\)-matrix.
In doing so, I introduce Feynman diagrams and Feynman rules, and go into some detail on dealing with fermions.
The Feynman rules will be a defining quality of the supersymmetric QFT model that I will discuss in the next chapter.

In the following section, I go through divergences in perturbative QFT, and outline the procedure of renormalisation to make the theory finite.
I then discuss Yang-Mills theory as a means for model building, which will be built on further in the next chapter.
Lastly, I briefly introduce Passarino-Veltman reduction for writing out general loop integrals that will crop up in the calculations in \cref{chap:part:parton_calculation}.





\section{Perturbative Quantum Field Theory}
In this thesis, I will use the Lagrangian framework to formulate QFT\@.
Here I will introduce the basics of how to formulate a QFT in such a way using the path integral formalism.
It is not intended as a complete introduction with proofs, but rather a summary of the tools that are used, with some simple examples.
This leads to a perturbative formulation of scattering and computation of correlation functions, which is the basis for the calculations that will be made later on.


\subsection{The Path Integral}
I will start by introducing some useful shorthands that will be used throughout this section.
Consider an action \(S[\cclosed{\Phi}]\) as a functional of some classical fields \(\cclosed{\Phi}\). Let \(\phi_i \equiv \phi(x_i)\) for some arbitrary field \(\phi \in \cclosed{\Phi}\) evaluated at some point in space-time \(x_i\)---the path integral approach to quantum field theory is built on time-ordered correlation functions through the relation~\cite{Schwartz:2014sze}
\begin{equation}
  \label{qft:eq:path_integral_correlation_function}
  \bra{\Omega} T\cclosed{\hat\phi_1(x_1) \cdots \hat\phi_n(x_n)} \ket{\Omega} = \frac{\int\!\mathcal{D}\phi\, \phi_1(x_1) \cdots \phi_n(x_n) \exp{iS[\cclosed{\Phi}]}}{\int\!\mathcal{D}\phi\, \exp{iS[\cclosed{\Phi}]}},
\end{equation}
where \(T\cclosed{.}\) denotes the time-ordering operation and \(\mathcal{D}\phi\) is the measure denoting integration over all possible \emph{field configurations}.
A field configuration here is understood as a given set of values for each of the fields \(\cclosed{\Phi}\), one for each point in space-time.
The time-ordering operation will put the fields in chronological order according to the time at which they are evaluated, with the ``first'' field being farthest to the right.
In other words, the fields are ordered by the zeroth component of their argument space-time coordinate \(x\).
In the left-hand side of \cref{qft:eq:path_integral_correlation_function} the fields are understood as operators on the Hilbert space of states in our interacting theory (denoted by their hats), and \(\ket{\Omega}\) is the vacuum state.
In this way quantum effects are encapsulated through the weighted sum of all classical \emph{paths} through configuration space, rather than just whichever one minimises the action.
Qualitatively, these correlation functions tell us about how certain field values at positions \(\vec{x}_i\) at times \(t_i\) are correlated, i.e.\ if \(\phi\) has a certain value \(\phi(t, \vec{x})\) at position \(\vec{x}\) and time \(t\), what does this tell us about the value of field \(\phi^\prime(t^\prime, \vec{x}^\prime)\) at time \(t^\prime\) and position \(\vec{x}^\prime\).



\subsection{Perturbing the Free Theory}
In interacting theories, correlation functions can be obtained through a \emph{perturbation series} by expanding them around a coupling constant \(\lambda\), denoting the strength of the interaction.
To compute time-order correlation functions with \cref{qft:eq:path_integral_correlation_function}, we need to exponentiate the action.
Assuming that the Lagrangian of the action can be written on the form \(\L = \L_0 - \lambda\L_\text{int}\), this exponentiation can be written as
\begin{align}
  \label{qft:eq:Lint_exp_expansion}
  \exp{i \integral{^4x} \pclosed{\L_0 - \lambda\L_\text{int}}} = & \exp{i \integral{^4x} \L_0} \biggl(1 - i\lambda\integral{^4x_1} \L_\text{int}(x_1)                            \\
                                                                 & - \frac{\lambda^2}{2} \integral{^4x_1}\integral{^4x_2} \L_\text{int}(x_1) \L_\text{int}(x_2) + \ldots\biggr).
\end{align}
For simplicity, let us consider a theory with just one self-interacting field \(\phi\).
Now say the interaction Lagrangian \(\L_\text{int}\) is some monomial of degree \(p\) in \(\phi\),\footnote{In the more general case where it can be expressed with some polynomial in the fields, we can look at each monomial term separately, without loss of generality.} then the interacting correlation functions can be written in terms of free-theory correlation functions!
To see this, we consider the interacting \(n\)-point function \(D^n_\text{int}(1,\ldots, n) = \bra{\Omega} T\cclosed{\hat\phi_1 \cdots \hat\phi_n} \ket{\Omega}\), and write the normalisation out in terms of the free and interacting Lagrangians:
\begin{align}
  \label{qft:eq:n-point_interactive_correlator}
  D^n_\text{int}(1,\ldots, n) = \frac{1}{\mathcal{N}} \int\!\mathcal{D}\phi\, \phi_1\cdots \phi_n \exp{i\integral{^4x}\L_0}\pclosed{1 - i\lambda \integral{^4y}\L_\text{int} + \mathcal{O}(\lambda^2)},
\end{align}
where the normalisation is given by \(\mathcal{N} = \int\!\mathcal{D}\phi\, \exp{i\integral{^4x}\L}\).
To relate this to the free-theory, let us take a moment to write this out.
Given the free-field \(n\)-point correlator \(D^n_0(1,\ldots,n) \equiv \frac{1}{\mathcal{N}_0} \int\!\mathcal{D}\phi\, \phi_i \cdots \phi_n \exp{i\integral{^4x}\L_0}\) with normalisation \(\mathcal{N}_0 = \int\!\mathcal{D}\phi\, \exp{i\integral{^4x}\L_0}\), expanding the interacting normalisation gives
\begin{equation}
  \mathcal{N} = \mathcal{N}_0 \pclosed{1 - i\lambda \integral{^4y} D_0^p(\underbrace{y, \ldots, y}_{p\text{ times}}) - \frac{\lambda^2}{2} \integral{^4y}\!\! \integral{^4z} D_0^{2p}(\underbrace{y,\ldots,y}_{p\text{ times}}, \underbrace{z,\ldots,z}_{p\text{ times}}) + \mathcal{O}(\lambda^3)}.
\end{equation}
Inserting this into \cref{qft:eq:n-point_interactive_correlator} and expanding around \(\lambda = 0\) we get
\begin{align}
  \label{qft:eq:n-point_interacting_correlator}
  \nonumber
    & D^n_\text{int}(1,\ldots, n) = \frac{D_0^n(1, \ldots, n) - i\lambda \integral{^4y} D_0^{n+p}(1, \ldots, n, \overbrace{y, \ldots, y}^{p\text{ times}}) + \mathcal{O}(\lambda^2)}{1 - i\lambda D_0^p(\underbrace{y, \ldots, y}_{p\text{ times}}) + \mathcal{O}(\lambda^2)} \\
  = & D^n_0(1,\ldots,n) - i\lambda\! \integral{^4y}\! \biggl(\!D_0^{n+p}(1, \ldots, n, \underbrace{y, \ldots, y}_{p\text{ times}}) - D_0^n(1, \ldots, n) D_0^p(\underbrace{y, \ldots, y}_{p\text{ times}})\!\biggr) + \mathcal{O}(\lambda^2).
\end{align}
We can see from this that perturbation from \(\lambda = 0\) will give corrections to the free two-point correlator that add correlations with additional space-time points that are integrated over.
These additional self-correlating points will diagrammatically form loops, as we will see an example of in the next section.
I note that the role of the second term at order \(\lambda\) is to remove all \emph{disconnected} diagrams that contribute to the first term.
Disconnected diagrams are diagrams where the self-correlating additional space-time points are not connected to the external space-time points, and as such should not be causally related to them.


\subsection{Feynman Rules in Position Space}
\label{qft:subsec:position_feynman_rules}
Let us now investigate closer what we can get from the expansion around the free theory in \cref{qft:eq:n-point_interacting_correlator}.
To do this, let us consider a concrete theory, given by the Lagrangian
\begin{equation}
  \label{qft:eq:phi4}
  \L = \underbrace{\frac{1}{2}(\partial\^\mu \phi)(\partial\_\mu\phi) - \frac{1}{2}m^2 \phi^2}_{\L_0} - \lambda \underbrace{\phi^4}_{\L_\text{int}}.
\end{equation}
Furthermore, I will use the Schwinger-Dyson equations~\cite{Schwartz:2014sze} for the non-interacting theory, which relates different \(n\)-points correlators through
\begin{equation}
  \label{qft:eq:Schwinger-Dyson_n-point}
  (\dA_x + m^2) D_0^{n+1}(x, 1, \ldots, n) = -i\sum_{i} \delta^4(x-x_i) D_0^{n-1}(1, \ldots, i-1, i+1, \ldots, n),
\end{equation}
where \(\dA = \partial\^\mu \partial\_\mu\) is the d'Alembertian operator.
It follows that
\begin{equation}
  \label{qft:eq:Schwinger-Dyson_2-point}
  (\dA_x + m^2) D^2_0(x,y) = -i \delta^4(x-y).
\end{equation}

Consider the 2-point correlator \(D_{\text{int}}^2(1,2)\) of the interacting theory.
By \cref{qft:eq:n-point_interacting_correlator}, this is given to first order in \(\lambda\) by
\begin{equation}
  \label{qft:eq:phi4_2-point_correlator_1}
  D_{\text{int}}^2(1,2) = D_0^2(1,2) - i\lambda \integral{^4y} \pclosed{D_0^6(1,2,y,y,y,y) - D_0^2(1,2)D_0^4(y,y,y,y)}.
\end{equation}
We can rewrite the six-point correlator to employ \cref{qft:eq:Schwinger-Dyson_2-point} and get
\begin{align}
  \nonumber
  D_0^6(1,2,y,y,y,y) = & \integral{^4x} \delta^4(x-x_1) D_0^6(x,2,y,y,y,y)            \\
  \nonumber
  =                    & i\integral{^4x} (\dA_x + m^2) D^2_0(x,1) D_0^6(x,2,y,y,y,y)  \\
  =                    & i\integral{^4x} D^2_0(x,1) (\dA_x + m^2) D_0^6(x,2,y,y,y,y),
\end{align}
where in the last equality I have used partial integration twice.
We can then use \cref{qft:eq:Schwinger-Dyson_n-point} to get
\begin{align}
  \nonumber
  D_0^6(1,2,y,y,y,y) = & \integral{^4x} D^2_0(x,1) \cclosed{\delta^4(x-x_2) D_0^4(y,y,y,y) + 4\delta^4(x-y)D_0^4(2,y,y,y)} \\
  =                    & D_0^2(1,2) D_0^4(y,y,y,y) + 4 D_0^2(1,y)D_0^4(2,y,y,y),
\end{align}
where I have used the property of the correlators that they are symmetric in their arguments because of the time-ordering operator.
Notice that the first term here will cancel the last term in \cref{qft:eq:phi4_2-point_correlator_1} as promised.
We can now use the same procedure to write out the four-point correlator
\begin{align}
  \nonumber
  D_0^4(2,y,y,y) = & i\integral{^4x} (\dA_x + m^2) D_0^2(x,2) D_0^4(x,y,y,y)                        \\
  =                & \integral{^4x} D_0^2(x,2) 3 \delta^4(x-y) D_0^2(y,y) = 3D_0^2(y,y) D_0^2(y,2).
\end{align}
Putting it all together in \cref{qft:eq:phi4_2-point_correlator_1}, we get that the interacting two-point correlator in this theory is
\begin{equation}
  \label{qft:eq:phi4_2-point_correlator_2}
  D_{\text{int}}^2(1, 2) = D_0^2(1, 2) - 12 i \lambda \integral{^4y} D_0^2(1, y) D_0^2(y, y) D_0^2(y, 2),
\end{equation}
to first order in the coupling constant \(\lambda\).

This procedure can be put together diagrammatically -- associating the free two-point correlators \(D_0^2(x,y)\) with edges between two points \(x, y\), and vertices where multiple edges meet with a factor of \(i k \lambda \integral{^4x}\), where \(k\) is the number of configurations of equal fields in \(\L_\text{int}\).
Due to this last point, it is therefore common to rescale the Lagrangian parameter \(\lambda \to \frac{\lambda}{k}\), such that the insertions will not contain any numerical factors.
In the case of our \(\phi^4\) theory, we should have \(k = 4!\), however, comparing to our result in \cref{qft:eq:phi4_2-point_correlator_2} we see that we only have a numerical factor of \(12 = \frac{4!}{2}\).
This is a result of the \emph{symmetry factor} of the diagram, as the loop in \(y\) is (trivially) symmetric under exchange of its endpoints.
By including the factor of \(p!\) in the vertex rule, we must divide by the symmetry factor, which are the number of ways you can change the edges of a diagram and still get the same result.
Another subtlety relates to the factor of \(\frac{1}{n!}\) associated with the \(\lambda^n\) term in the perturbation expansion \cref{qft:eq:n-point_interacting_correlator}.
This coincides with the fact that the \(n\)th order term gives equal contributions under the interchange of its internal vertices.
There are \(n!\) ways of interchanging these vertices, and so this factor cancels out neatly in the end.
\medskip

Returning to our example, we can depict \cref{qft:eq:phi4_2-point_correlator_2} diagrammatically by
\begin{align}
  D_{\text{int}}^n(1, 2) = & \inputtikz[baseline=(x1.base)]{phi4/LO_propagator} + \inputtikz[baseline=(x1.base)]{phi4/NLO_propagator}.
\end{align}
Here, I have denoted \emph{vertices} associated with the coupling constant insertion with filled dots, and external points with empty dots for clarity.
This practice will not be kept for the remainder of the thesis.
\medskip

In summary, the following rules relate a Feynman diagram to the two-point correlators of the free theory:
\begin{itemize}
  \item [(I)] A factor of \(D_0^2(x, y)\) for every edge connecting two points \(x, y\). This factor is referred to as the \emph{propagator}
  \item [(II)] A factor of \(-i\lambda \integral{^4x}\) for every internal vertex point \(x\).
  \item [(III)] An overall numerical factor of \(S^{-1}\), where \(S\) is the symmetry factor. This is the total number of ways internal lines can be exchanged without changing the diagram.
\end{itemize}
To get the full correlator to a given perturbative order, we must sum over all \emph{different}, connected Feynman diagrams to that order.


\subsection{The \textit{S}-Matrix and LSZ Reduction Formula}
To relate the correlation functions above to physical experiment, e.g.\ scattering amplitudes, we will need to more concretely define how an observable can be extracted from the theory.
Let us for a moment go to the quantum mechanical picture of Hilbert space formalism to formulate a scattering experiment.
Idealising the scenario, let us consider the amplitude of an \emph{asymptotic} in-state \(\ket{\psi_i}\) of free fields at \(t=-\infty\) evolving into an asymptotic out-state \(\ket{\psi_f}\) of some (potenitally other) fields at \(t=+\infty\).
The interaction between is captured by the \(S\)-matrix, and the amplitude is given by the inner product
\begin{equation}
  \label{qft:eq:S-matrix}
  \bra{\psi_f} S \ket{\psi_i}.
\end{equation}
Squaring this amplitude gives us the probability for the given initial configuration to end up in the given final configuration.
So, with adequately realisable in/out states, the experiment can be conducted many times to extract an experimental probability we can compare to.
For scattering experiments, we will usually look at in/out states with a given particle number in certain momentum eigenstates, for which I will now outline a procedure for relating the correlation functions above to the \(S\)-matrix amplitude of \cref{qft:eq:S-matrix}.

This procedure is due Lehmann, Symanzik and Zimmermann~\cite{LSZ}, who derived the LSZ reduction formula relating the correlation functions to the \(S\)-matrix.
Given initial fields with momenta \(p_1, \ldots, p_k\) and final fields with momenta \(p_{k+1}, \ldots, p_n\) in our scalar theory \cref{qft:eq:phi4}, it reads~\cite{Schwartz:2014sze}
\begin{align}
  \label{qft:eq:LSZ_S-matrix}
  \nonumber
  \bra{p_{k+1}\cdots p_n} S \ket{p_1 \cdots p_k} = \pclosed{\prod_{i=1}^k i \integral{^4x_i} \exp{-ip_i \cdot x_i} (\dA_{x_i} + m^2)} \\
  \times \pclosed{\prod_{i=k+1}^n i \integral{^4x_i} \exp{+ip_i \cdot x_i} (\dA_{x_i} + m^2)}
  \times \bra{\Omega} \operatorname{T}\cclosed{\phi(x_1)\cdots \phi(x_n)} \ket{\Omega}.
\end{align}

Usually, we encapsulate the information of the \(S\)-matrix in \emph{matrix elements} \(\M\) by writing it on the form
\begin{equation}
  \label{qft:eq:S-matrix_expansion}
  S = \eye + i(2\pi)^4 \delta^4(\sum_{i=1}^k p_i - \sum_{i=k+1}^n p_i) \M,
\end{equation}
essentially dropping the trivial overlap between the in- and out-states and factoring out momentum conservation.
\medskip

Let us now put this to use in our scalar toy model from \cref{qft:subsec:position_feynman_rules}, computing the transition for a field with momentum \(p_1\) to end up with momentum \(p_2\) using \cref{qft:eq:S-matrix} to first order in \(\lambda\):
\begin{align}
  \nonumber
  \bra{p_2} S \ket{p_1} = & \pclosed{i\integral{^4x_1} \exp{-ip_1 \cdot x_1} (\dA_{x_1} + m^2)} \pclosed{ i \integral{^4x_2} \exp{+ip_2 \cdot x_2} (\dA_{x_2} + m^2)} \times D_{\text{int}}^2(1, 2) \\
  \nonumber
  =                       & -\integral{^4x_1} \exp{-ip_1 \cdot x_1} (\dA_{x_1} + m^2) \integral{^4x_2} \exp{+ip_2 \cdot x_2} (\dA_{x_2} + m^2)                                                      \\
  \nonumber
                          & \times \cclosed{D_0^2(1,2) - \frac{i\lambda}{2} \integral{^4y} D_0^2(1,y)D_0^2(y,y) D_0^2(y,2)}                                                                         \\
  \nonumber
  =                       & -\integral{^4x_1} \exp{-ip_1 \cdot x_1} (\dA_{x_1} + m^2) \cclosed{i \exp{ip_2\cdot x_1} - \frac{\lambda}{2}\integral{^4y} \exp{-ip_2\cdot y} D_0^2(1,y) D_0^2(y,y)}    \\
  =                       & \integral{^4x_1} \cclosed{-i (p^2_2 - m^2)- \frac{i\lambda}{2} D_0^2(x_1,x_1)} \exp{-i(p_1-p_2) \cdot x_1},
\end{align}
where in the second equality I inserted \cref{qft:eq:phi4_2-point_correlator_2}, and in the third and fourth equalities I used the Schwinger-Dyson equation \cref{qft:eq:Schwinger-Dyson_2-point}.
By the on-shell condition on \(p_2\), the first term vanishes.
For the second term, we can use that the two-point correlator, see \cref{qft:eq:Schwinger-Dyson_2-point}, is given by its Fourier transform as
\begin{equation}
  D_0^2(x,y) = \loopintegral{k} \tilde{D}_0^2(k) \exp{ik\cdot(x-y)},
\end{equation}
to arrive at
\begin{equation}
  \bra{p_2} S \ket{p_1} = -i(2\pi)^4 \delta^4(p_1 - p_2) \frac{\lambda}{2} \loopintegral{k} \tilde{D}_0^2(k).
\end{equation}

We can relate this to the matrix element \(\M\), by assuming that the asymptotic in/out-states do not overlap, to find through the definition \cref{qft:eq:S-matrix_expansion} that
\begin{align}
  \nonumber
  \bra{p_2} i(2\pi)^4 \delta^4(p_2-p_1) \M \ket{p_1} = & -i(2\pi)^4 \delta^4(p_1 - p_2) \frac{\lambda}{2} \loopintegral{k} \tilde{D}_0^2(k) \\
  \Rightarrow i\M =                                    & - \frac{i \lambda}{2} \loopintegral{k} \tilde{D}_0^2(k).
\end{align}

This result can also be arrived at diagrammatically, leading us to define Feynman diagrams and Feynman rules in \emph{momentum space}.
The procedure for drawing diagrams is the same as before, but now every external point is associated with an external momentum, and every line carries momentum with it.
Momentum flow can be indicated with arrows for clarity.
The resulting Feynman rules are similar to before, but now give:
\begin{itemize}
  \item [(I)] A factor of \(\tilde{D}_0^2(p)\) for every \emph{internal} line associated with a momentum \(p\).
  \item [(II)] A factor of \(-i\lambda\) for every internal vertex point.
  \item [(III)] External points are associated with a factor of \(1\).
  \item [(IV)] Momentum conservation should be enforced through every vertex point.
  \item [(V)] Any undetermined momentum \(k\) of internal lines must be integrated over with a factor \(\loopintegral{k}\).
  \item [(VI)] An overall numerical factor of \(S^{-1}\), where \(S\) the symmetry factor. This is the total number of ways internal lines can be exchanged without changing the diagram.
\end{itemize}


\subsection{Feynman Rules for Fermions}
Later on, we will be considering fermion fields, which have spinor structure.
I go into more detail on the properties of spinors in \cref{app:chap:weyl_grassmann}.
When considering correlation functions of spinor fields, it is usually most convenient to work with the correlation of Lorentz-invariant contractions of the spinor field components.
This imposes some additional structure on the Feynman rules.
More generally, external fermion points will be associated with spinors, whereas propagators and vertices can be operators on spinor space.\footnote{In fact, these arguments hold for other Lorentz structures such as four-vectors or tensors of any rank.}

Let us first consider the external fermions.
Consider a Dirac spinor field \(\psi\) with a conjugate field \(\bar\psi\), and with mass \(m\). The general, quantised solution for a free Dirac spinor field is~\cite{Peskin}
\begin{equation}
  \psi(x) = \loopintegral[3]{p} \frac{1}{\sqrt{2E_{\vec{p}}}} \sum_s \pclosed{a_s(\vec{p}) u_s(p) \exp{-i p\cdot x} + b_s^\dagger(\vec{p}) v_s(p) \exp{i p\cdot x}},
\end{equation}
where \(E_{\vec{p}} = m^2 + \vec{p}^2\), \(u_s(p), v_s(p)\) are particle and antiparticle spinors of spin \(s\), and \(a_s^\dagger(\vec{p}), b_s^\dagger(\vec{p})\) are creation operators creating particle and antiparticle states with momentum \(\vec{p}\) and spin \(s\) respectively.
Normalisation is chosen such that \(a_s^\dagger(\vec{p}) \ket{0} = \sqrt{2E_{\vec{p}}} \ket{\vec{p}, s}\) and \(\braket{\vec{q}, r}{\vec{p}, s} = 2E_{\vec{p}} (2\pi)^3 \delta^3(\vec{p}-\vec{q}) \delta_{rs}\), where \(\ket{\vec{p}, s}\) denotes a fermionic particle state with momentum \(\vec{p}\) and spin \(s\).

The derivation of the LSZ reduction formula will differ slightly going from scalar theory to the fermionic one.
Particularly, one needs the inner product
\begin{align}
  \nonumber
  \bra{0}\psi(x)\ket{\vec{q},r} = & \loopintegral[3]{p} \frac{1}{\sqrt{2E_{\vec{p}}}} \sum_s \frac{1}{\sqrt{2E_{\vec{p}}}} \bra{\vec{p}, s}  u_s(p) \exp{-i p\cdot x} \ket{\vec{q}, s}                          \\
  \nonumber
  =                               & \loopintegral[3]{p} \frac{1}{\sqrt{2E_{\vec{p}}}} \sum_s \frac{1}{\sqrt{2E_{\vec{p}}}} 2E_{\vec{p}} (2\pi)^3 \delta^3(\vec{p}-\vec{q}) \delta_{rs} u_s(p) \exp{-i p\cdot x} \\
  =                               & u_r(q) \exp{-i q\cdot x},
\end{align}
where it was used that \(q^2 = p^2 = m^2\) such that when \(\vec{q} = \vec{p}\) we have \(q^0 = p^0\).
This differs from the scalar theory where \(\bra{0} \phi(x) \ket{\vec{q}} = \exp{-i q\cdot x}\).
In the end, the effect this has is that external fermion points of Feynman diagrams will be associated with the corresponding spinors, instead of the trivial factor of \(1\) for the scalar theory.
\medskip

For a Lorentz-covariant theory, the Lagrangian will only contain Lorentz invariant contractions of the spinors, all of which can be decomposed into Dirac \emph{bilinears} on the form
\begin{equation}
  \bar\psi \Gamma\^r \psi,
\end{equation}
where \(\Gamma^r\) form a basis for operators on spinor space.
One realisation of such a basis for four-component spinors is using the Dirac gamma-matrices \(\Gamma\^r = \cclosed{\eye, \gamma\^\mu, \sigma\^{\mu\nu}, \gamma\^\mu \gamma\^5, \gamma\^5}\), where \(\sigma\^{\mu\nu} = \frac{i}{2} \bclosed{\gamma\^\mu, \gamma\^\nu}\) and we understand \(\nu < \mu\).
Feynman rules for propagators and vertices will generally contain Lorentz invariant linear combinations of such operators.

With the additional spinor structure in the diagrams, it is important that the spinors and operators on spinor space are ordered correctly in the amplitudes, as to produce the intended Lorentz invariant contractions.
This is usually done by defining a \emph{fermion flow} in the diagrams, usually denoted by an arrow on the fermion lines.
The procedure goes like this:
Starting at the end of a fermion flow, insert spinors associated with external lines and spinor operators associated with internal lines and vertices until you get to the start of the fermion flow.
Repeat this for all separated fermion flows in the diagram.

Another point of subtelty is that fermionic spinor fields are anticommuting.
This causes the fermionic correlation functions to become antisymmetric under interchange of the fermions, e.g.\ \(D_\psi^2(1, 2) = -D_\psi^2(2, 1)\).
There are two important consequences of this:
Firstly, closed fermion loops get an overall factor of \(-1\).
Secondly, different diagrams get a \emph{relative sign of interfering Feynman diagrams} (RSIF).
By keeping the order of the fermionic fields in the correlation function consistent between diagrams, a different amount of (anti-)commutations might be needed for different terms in the expansion \cref{qft:eq:Lint_exp_expansion}.
The overall minus sign of the amplitude will be ambigious, depending on how you choose to order fields, but relative signs between diagrams that are added together will be absolute.
The ambiguity can be resolved by looking at the order in which external spinors come in the amplitude, giving an additional minus sign to spinor orders that are an odd permutation as compared to some arbitrary reference permutation.
\medskip

Summarised, the additional Feynman rules of fermions are
\begin{itemize}
  \item [(I)] External spinors and vertex factors are inserted starting at the end of every fermion flow, moving backwards, ending on the start of a fermion line. Incoming (outgoing) fermion particles of momentum \(p\) and spin \(s\) are associated with a factor \(u_s(p)\) (\(\bar{u}_s(p)\)). Incoming (outgoing) fermion \emph{anti}-particles of momentum \(p\) and spin \(s\) are associated with a factor \(\bar{v}_s(p)\) (\(v_s(p)\)).
  \item [(II)] A factor of \(-1\) is associated with every closed fermion loop.
  \item [(III)] A RSIF is assigned to any diagram by evaluating the eveness of the order in which spinors arise as compared to some freely chosen reference order. The reference order must be kept consistent throughout all diagrams.
\end{itemize}
\medskip

Lastly, I will note a point of added complexity when dealing with Majorana fermions, i.e.\ fermions that are their own anti-particles.
Fermion flows as we have defined them can lead to amiguities when dealing such fermions.
This is because the fermion flows can be defined to go both ways, as the same operators are used to create both particle and anti-particle spinors \(u_s(p)/v_s(p)\), creating amiguity in the RSIF of specific diagrams.
To alleviate this, I will follow a prescription due to Denner et al.~\cite{Denner:1992vza}.
Arrows on lines will be used to indicate \emph{particle number flow}, used both for Dirac fermions and particle number conserving scalar fields.
Fermion flows, however, will be denoted with an additional arrow along fermion lines.
The direction of these flows can be defined arbitrarily, but the Feynman rules will be amended by whether the flow goes with/against the particle number flow, and into/out of vertices.
By using the approriate, amended Feynman rules proposed in~\cite{Denner:1992vza}, the fermion flows can be defined any way we would like, and the RSIF due to the parity of the spinor permutation can be inserted naively.
The amended Feynman rules make sure that the RSIF will be correct, even though fermion flows are not defined consistently between the diagrams.
In fermion lines with only Majorana particles, Feynman rules are the same regardless of fermion flow, however, for Dirac fermions, the following amendments are made:
\begin{itemize}
  \item [(I)] Momentum space propagators \(\tilde{D}_0^2(p)\) for a fermion propagator is replaced by \(\tilde{D}_0^2(-p)\) if the particle number flow goes against the fermion flow. Otherwise, the ordinary propagator is used.
  \item [(II)] For vertex rules including Dirac fermions, the spinor operator \(\Gamma\) associated with the vertex must be replaced by \(C\Gamma^T C^{-1}\) if the fermion flow is defined contrary to the particle number flow. Here \(C\) is the charge conjugation matrix, which I define in \cref{app:chap:weyl_grassmann}.
\end{itemize}

I exemplify this use of arrows indicating particle number flow and fermion flow in \cref{qft:fig:feynman_flow}.

\begin{figure}[ht!]
  \centering
  \inputtikz{fermion_flow_example}
  \caption{Example of the arrow usage in this thesis.
    There is an incoming Dirac fermion \(\psi\) with a particle number flow indicated with the arrow on its line.
    Likewise, a complex scalar \(\phi\) has an arrow on its line indicating particle number flow.
    A Majorana fermion \(\chi\) has no such particle number flow indication.
    The arrow above the vertex indicates the defined fermion flow for this diagram.}
  \label{qft:fig:feynman_flow}
\end{figure}




\section{Renormalised Quantum Field Theory}
\label{qft:sec:renormalisation}
We have seen how correlation functions in interacting theories can be calculated as perturbations to the free, non-interacting theory.
In this section, I will go more into detail on the actual computation of the amplitudes, particularly touching on infinities that arise when computing higher order corrections naïvely.
This leads to the procedure of regularisation to classify the infinities, and renormalisation to get rid of them.
These methods will be necessary in the computation of higher order corrections in later chapters.


\subsection{Loop Integrals and Divergences}
Divergences appear in perturbative correlation functions in QFT, and can be
categorised into \emph{ultraviolet} (UV) divergences and \emph{infrared} (IR)
divergences. They are so named after which region of momentum space they
originate from --- high momentum for UV and low momentum for IR\@. The two types
of divergences are dealt with differently, and here I will lay out how
to deal with UV divergences through \emph{renormalisation}.
\medskip
\begin{figure}[!ht]
  \centering
  \inputtikz{ex_loop}
  \caption{Simple example of a loop diagram in a scalar theory.}
  \label{qft:fig:example_loop}
\end{figure}

Consider a loop like the one in \cref{qft:fig:example_loop}:
The propagator of a free scalar theory is given by~\cite{Schwartz:2014sze}
\begin{equation}
  \tilde{D}(p) = \frac{i}{p^2 - m^2 + i \epsilon},
\end{equation}
where the \(i\epsilon\) gives a prescription for dealing with the pole at \(p^2 = m^2\) and is taken to 0 in the end.
The prescription used here is called the Feynman prescription, and it ensures that the two-point correlator \(D(x,y) = \loopintegral{p}\tilde{D}(p) \exp{-ip\cdot(x-y)}\) is a time-ordered correlator of \(x\) and \(y\).
If we consider a massless particle in the loop, and massless external particles \(p^2 = 0\), the amplitude \cref{qft:fig:example_loop} will take the form\footnote{One can see the equivalence by introducing Feynman parametrisation as in \cref{app:sec:feynman_parametrisation} and shifting the integration variable.}
\begin{equation}
  \label{qft:eq:example_loop}
  \loopintegral[4]{q} \frac{1}{(q^2+i\epsilon)((q-p)^2+i\epsilon)} = \loopintegral[4]{q} \frac{1}{\pclosed{q^2 + i\epsilon}^2}.
\end{equation}
The Lorentz signature of the momenta, giving \(q^2 = q_0^2 - |\vec{q}|^2\) makes integration somewhat cumbersome, so we will use a trick called \emph{Wick rotation}~\cite{Schwartz:2014sze}.
This entails defomring the integration contour around \(q_0\) in the complex plane along the real and imaginary axes, making sure not to enclose the poles.
\\\todo{Make a diagram showing integration contour.}\\
By Cauchy's integral theorem, the total integral then vanishes, and the contributions going from \(+\infty\) to \(+i\infty\) and from \(-i\infty\) to \(-\infty\) vanish by Jordan's lemma.
This means that the integral contribution from \(+i\infty\) to \(-i\infty\) must be equal but opposite of the integral along the real axis.
Therefore, we can rather choose to integrate along the imaginary axis, substituting \(q_0 = iq_0^E\).
This will change the signature of the inner product such that
\begin{equation}
  q_0^2 - |\vec{q}|^2 = q^2 = -q_E^2 = -(q_0^E)^2 - |\vec{q}^E|^2 \equiv -|\vec{q}_E|^2,
\end{equation}
where I have defined \(\vec{q}^E = \vec{q}\) and \(\vec{q}_E = \pclosed{q_0^E, \vec{q}^E}\).
The result is that the loop integral \cref{qft:eq:example_loop} turns into
\begin{equation}
  \loopintegral[4]{q} \frac{1}{\pclosed{q^2 + i\epsilon}^2} = i\loopintegral[4]{\vec{q}_E} \frac{1}{|\vec{q}_E|^4} = \frac{i\Omega_4}{(2\pi)^4} \integral[_0^\infty]{q_E} \frac{1}{q_E},
\end{equation}
where after Wick rotation I took the limit of \(\epsilon \to 0\), and I changed to four-dimensional spherical coordinates in the last equality where \(\Omega_4 = 2\pi^2\) is the surface are of a unit sphere in four dimensions.
For reference, the \(d\)-dimensional spherical surface is given by
\begin{equation}
  \Omega_d = \frac{2\pi^{\sfrac{d}{2}}}{\Gamma(\sfrac{d}{2})},
\end{equation}
where \(Gamma(n)\) is the Gamma-function, analytically continuing the factorial to arbitrary real numbers, which will prove useful shortly.
The factor of \(i\) comes from the change of variable from \(q_0 \to q^E_0 = -iq_0\).
This diverges for both low and high momenta \(q_E\).
Had the particle been massive, the momentum would have a non-zero lower limit, and the IR divergence from the low-momentum limit would disappear.
However, the UV divergence must be handled differently.


\subsection{Regularisation}
A first step to handle the divergences is to deform our theory in some way to
make the loop integral formally finite, but recovering the divergence in the
limit that the deformation disappears. An intuitive deformation would be to cap
the momentum integral at some \(\Lambda\), recovering our original theory in
the limit \(\Lambda \to \infty\). To illustrate the procedure of regularisation
and subsequently renormalisation, it will be useful to have an example, for
which I choose a scalar Lagrangian
\begin{equation}
  \label{qft:eq:example_lagrangian}
  \L = \frac{1}{2} \pclosed{\partial_\mu\phi}^2 - \frac{1}{2}m_\phi^2 \phi^2 - \frac{\lambda}{3!} \phi^3.
\end{equation}
Regularising the IR divergence in \cref{qft:eq:example_loop} by giving our scalar a mass \(m\) to the loop particle, and the UV divergence with a momentum cap \(\Lambda\), we are left with
\begin{align}
  \label{qft:eq:cut-off_loop_example}
  \nonumber
  \int_{\abs{q} < \Lambda}\! \frac{\mathrm{d}^4q}{(2\pi)^4}
  \frac{1}{\pclosed{(q^2-m^2)+i\epsilon}^2} = \frac{i}{(2\pi)^4}
  \integral{\Omega_4} \integral[_0^\Lambda]{q_E} \frac{q_E^3}{\pclosed{q_E^2
      -
  m^2}^2} \\
  = \frac{i}{16\pi^2} \cclosed{\ln\pclosed{1 + \frac{\Lambda^2}{m^2}} -
    \frac{\Lambda^2}{\Lambda^2 + m^2}},
\end{align}
where now evidently the divergence manifest as a logarithm.
\medskip

\subsubsection*{Dimensional Regularisation}
Another popular choice of regularisation, which I will use in this thesis, is
\emph{dimensional regularisation}. It entails analytically continuing the
number of space-time dimension from the ordinary 4 dimensions to \(d
= 4-2\epsilon\) dimensions for some small \(\epsilon\).\footnote{The reason for choosing \(2\epsilon\) is purely aesthetical, making some expressions neater later on.} This removes much of
the intuition for what we are doing, but turns out to be computationally very
efficient. Our loop integral \cref{qft:eq:example_loop} will then turn into
\begin{align}
  \label{qft:eq:dimreg_loop_example}
  \nonumber
    & \loopintegral[d]{q} \frac{1}{\pclosed{q^2+i\epsilon}^2} = \frac{i 2\pi^{\sfrac{d}{2}}}{(2\pi)^d} \frac{1}{\Gamma(\sfrac{d}{2})} \integral[_0^\infty]{q_E} q_E^{d-5}                                                                                                                                                 \\
  = & \frac{i 2\pi^{2-\epsilon}}{(2\pi)^{4-2\epsilon}} \frac{1}{\Gamma\pclosed{2-\epsilon}} \cclosed{ \integral[_0^\mu]{q} \frac{1}{q_E^{1+2\epsilon}} + \integral[_\mu^\infty]{q} \frac{1}{q_E^{1+2\epsilon}} } = \frac{i}{16\pi^2} \pclosed{\frac{1}{\epsilon_\text{IR}} - \frac{1}{\epsilon_\text{UV}}} + O(\epsilon),
\end{align}
where in the second equality, the momentum integral is split into a low-energy and high-energy part at some scale \(\mu\).
Here a trick was performed, as the low-energy part requires \(\epsilon < 0\) to be convergent, whereas the high-energy part requires \(\epsilon>0\).
The two different divergences thus require different deformations of the theory to be finite, and should be handled separately, hence the subscripts.
It is a general result that divergences coming from the low-energy parts of momentum integrals require \(\epsilon_{\text{IR}} < 0\), whereas high-energy divergences require \(\epsilon_{\text{UV}} > 0\), and this can be used to identify the source of divergences when using dimensional regularisation.
In the end divergences when using dimensional regularisation come out as \(\frac{1}{\epsilon^p}\)-terms, for some power \(p\).
\medskip

To achieve dimensional regularisation, we must analytically continue the definition of our theory in a neighbourhood around \(d = 4\).
The action will change accordingly to
\begin{equation}
  S = \integral{^d x} \L,
\end{equation}
and so the mass dimension of the Lagrangian in the deformed theory must therefore be \(\bclosed{\L} = \mu^d\), for some mass scale \(\mu\).
From the kinetic term in our toy model \cref{qft:eq:example_lagrangian}, we can see we must require the field to have mass dimension \(\bclosed{\phi} = \mu^{\sfrac{(d-2)}{2}}\), which will imply mass dimension for the coupling \(\bclosed{\lambda} = \mu^{3-\sfrac{d}{2}}\).
Thus, the coupling has changed from mass dimension \(\mu^1 \to \mu^{3-\sfrac{d}{2}}\) -- insisting on keeping the mass dimension of the coupling constant as the theory is deformed will lead to a parametrisation of the deformation by changing the coupling to
\begin{equation}
  \lambda \to \lambda \mu^{\sfrac{(4-d)}{2}}.
\end{equation}
The arbitrary mass scale \(\mu\) has been incorporating into the theory, and although it vanishes in the limit \(d \to 4\) in the above expression, dependence on it will not vanish completely when it is multiplied by poles in \(\epsilon\) such as those in \cref{qft:eq:dimreg_loop_example}.
It has been argued that the scale \(\mu\) heuristically mimics the cut-off scale \(\Lambda\) introduced in \cref{qft:eq:cut-off_loop_example}~\cite{Collins:2011zzd}.\footnote{Generally, it will carry some proportionality to \(\Lambda\), however, in certain \emph{renormalisation schemes} we will discuss shortly (namely \MSbar{}), the proportionality is unity.}





\subsection{Counterterm Renormalisation}
\label{qft:subsec:counterterms}
To take care of UV divergences, we note that there is freedom in how we define the contents of our Lagrangian.
We should be able to rescale our fields \(\phi_0 = \sqrt{Z_\phi} \phi\), and rescale our couplings by \(m_{\phi,0}^2 = Z_m m_\phi^2\) and \(\lambda_0 = Z_\lambda \mu^{\sfrac{(4-d)}{2}} \lambda\).\footnote{The mass scale \(\mu\) included in the rescaling of the coupling is there to let the renormalised coupling have the same mass dimension as the undeformed theory in \(d = 4\) dimensions.}
Although suggestively naming terms such as \emph{mass term} with mass \(m_\phi^0\) implies a connection to the mass of a particle, we have yet to define what that would mean experimentally.
Thus, rescaling our parameters and fields parametrises the way in which we can tune our theory, allowing us freedom in choosing the way our theory connects to experiments.

This approach of rescaling actually allows us to make a perturbative scheme for fixing our (re)normalisations of the fields and couplings to experiment.
There are many choices for how to connect theory to experiment, but one common approach for field and mass renormalisation is to identify the pole in momentum space of the two-point correlation function \(D_\text{int}^n(x, y)\) of a particle to the mass resonance  associated with the on-shell production of a particle in scattering experiments.
This allows us to perturbatively calculate the two-point correlator, and then fix our normalisations accordingly, such that our mass resonance condition above holds at every order in the perturbation series.
We achieve this systematically with \emph{counterterms}, which in essence are additional Feynman rules added to the theory. By expanding the renormalisation parameters as \(Z = 1 + \delta\), the \(\delta\) will carry the correction to the normalisation to any given order in a coupling constant so that the physical predictions are finite.
To one-loop order, the self-energy of our scalar theory from \cref{qft:eq:example_lagrangian} is diagrammatically given by
\begin{equation*}
  \inputtikz[baseline=(in.base)]{SE_LO} +
  \inputtikz[baseline=(in.base)]{SE_NLO} +
  \inputtikz[baseline=(in.base)]{SE_cnt},
\end{equation*}
where the crossed dot represents an insertion of the \(\delta\) into the LO amplitude.
Since the free theory is divergence free, the counterterms only needs to carry corrections from the first order in the coupling constant that loops arise.
They are therefore understood to be of one-loop order.
\medskip

This perturbative expansion can be done for any \(n\)-point correlator calculable in the theory, and the counterterm insertions can be determined from the vertices in the bare Lagrangian.
Returning to our toy model in \cref{qft:eq:example_lagrangian}, we can expand the three-point vertex
\begin{equation}
  \lambda_0 \phi_0^3 = Z_\lambda Z_\phi^{\sfrac{3}{2}} \lambda \phi^3 = \mu^{\sfrac{(4-d)}{2}} \lambda \phi^3 + (\delta_\lambda(\lambda) + \frac{3}{2} \delta_\phi(\lambda)) \lambda \phi^3 + \ldots,
\end{equation}
where the dots denote terms of higher order in \(\lambda\).
Using the renormalised coupling \(\lambda\) for the vertex Feynman rules, we must add the vertex counterterm insertions
\begin{equation}
  \inputtikz[baseline=(v.base)]{phi3_renormalisation/vertex_ct} = -i \pclosed{\delta_\lambda + \frac{3}{2} \delta_\phi}.
\end{equation}


\subsection{On-Shell Renormalisation}
\label{qft:subsec:on_shell_renormalsiation}
Categorising all higher order contributions that can arise to the LO
self-energy of a massive particle, they come in the form of combinations of
\emph{one-particle-irreducible} (1PI) diagrams.
Self-energy in this context are diagrams with the same particle coming in and out, being the quantum corrections to the free two-point correlator.
1PI diagrams are diagrams where all lines with loop momentum running through are connected.
Other diagrams can be reconstructed as the sum of 1PI diagrams.
Denoting the leading order correlator \(\mathcal{G}_0(p)\) and the contribution from one insertion of all 1PI
diagrams \(i\Sigma(p)\), we get an infinite series that can be summed\footnote{A note on the argument \(p\)
  of these functions: The two-point-correlators in momentum space depend on
  the
  four-momentum \(p^\mu\) in such a way that when it is put in between the
  external particle representations, i.e.\ 1 for scalars, spinors for
  fermions
  and polarisation vectors for vector bosons, the result will be Lorentz
  invariant. This means that the correlator could in principle carry Lorentz
  indices too, which will be suppressed here for simplicity.}
\begin{align}
  \nonumber
  i\mathcal{G}(p) = & \inputtikz[baseline=(in.base)]{SE_LO} +
  \inputtikz[baseline=(in.base)]{SE_1PI} +
  \inputtikz[baseline=(in.base)]{SE_2PI} + \ldots
  \\
  \nonumber
  =                 & i\mathcal{G}_0(p) + i\mathcal{G}_0(p) i\Sigma(p)
  i\mathcal{G}_0(p) + i\mathcal{G}_0(p) i\Sigma(p) i\mathcal{G}_0(p)
  i\Sigma(p)
  i\mathcal{G}_0(p) + \ldots
  \\
  \nonumber
  =                 & i\mathcal{G}_0(p) \bclosed{i\Sigma(p) i\mathcal{G}_0(p)
    + \pclosed{i\Sigma(p) i\mathcal{G}_0(p)}^2 + \ldots}
  \\
  =                 & i\mathcal{G}_0(p) \frac{1}{1 + \Sigma(p)
    \mathcal{G}_0(p)} = \frac{i}{\mathcal{G}_0^{-1}(p) + \Sigma(p)}.
\end{align}
So the computation of the two-point correlator to any order can be done simply
by computing the sum of the 1PI diagrams to that order. These contributions
will generally diverge, but then we can take into account the renormalisation
parameters. Since this is a \emph{bare} function, i.e.\ using the
non-renormalised quantities, we can get the renormalised two-point correlator
\(\mathcal{G}^\text{R}(p)\) through
\begin{equation}
  \mathcal{G}^\text{R}(p) = \frac{1}{Z_\psi} \mathcal{G}^\text{bare}(p) =
  \frac{1}{1 + \delta_\psi} \mathcal{G}^\text{bare}(p),
\end{equation}
for any field \(\psi\), seeing as the two-point correlator is quadratic in
\(\psi\) and thereby quadratic in \(\sqrt{Z_\psi}\).
\medskip

On-shell mass renormalisation seeks to identify the pole of the two-point correlator with the physical mass as observed in experiment.
This is a generalisation of the property of the free theory two-point function to the perturbative interacting two-point function at any order.
It yields two conditions:
\begin{center}
  \begin{itemize}
    \item [(I)] \(\evalat{\bclosed{(1+\delta_\psi)\pclosed{\mathcal{G}^\text{bare}_0(p)}^{-1}
              + \Sigma(p)}}{p^2 = m_\text{pole}^2} = 0\),
    \item [(II)] \(\operatorname{Res}\cclosed{\mathcal{G}^\text{R}(p), p^2
            = m_\text{pole}^2} = 1\),
  \end{itemize}
\end{center}
where \(\operatorname{Res}\cclosed{f(z), z = z_0}\) is the residue of the
function \(f\) at \(z_0\).
\medskip

For our scalar theory, where the leading order \emph{bare} two-point correlator is
\(\mathcal{G}_0^\text{bare}(p) =\frac{1}{p^2 - m_0^2}\), this means that we get the
relations
\begin{center}
  \begin{itemize}
    \item [(I)] \(\delta_m m_\phi^2 = \Sigma(m_\phi^2)\),
    \item [(II)] \(\delta_\phi = -\evalat{\d[p^2]{}\Sigma(p^2)}{p^2 =
            m_\phi^2}\).
  \end{itemize}
\end{center}
\medskip

Later on, I will make use of \emph{chiral} on-shell renormalisation.
This happens in chiral theories where the left-handed and right-handed degrees of freedom in fermion fields are treated differently in the Lagrangian.
This means that divergent corrections to the two-point correlator can be different between the fermion chiralities.
Still using Dirac spinor notation, we can then rescale a fermion \(\psi\)
\begin{equation}
  \psi^0 = \sqrt{Z_L} P_L \psi + \sqrt{Z_R} P_R \psi,
\end{equation}
where \(P_{L/R} = \frac{1}{2}(1 \mp \gamma^5)\) are the chiral projection operators, and work out and renormalise the two-point correlators separately.
Expanding \(Z_{L/R} = 1 + \delta_{L/R}\) and writing the \(\Sigma(\slashed{p}) = \Sigma_L(\slashed{p})P_L + \Sigma_R(\slashed{p})P_R\) we end up with three conditions analogous to the case above:
\begin{center}
  \begin{itemize}
    \item [(I)] \(\delta_m m_\psi = \evalat{\Sigma}{\slashed{p}=m_\psi}\),
    \item [(II)] \(\delta_L = -\evalat{\d[\slashed{p}]{}\Sigma_L(\slashed{p})}{\slashed{p} = m_\psi}\),
    \item [(III)] \(\delta_R = -\evalat{\d[\slashed{p}]{}\Sigma_R(\slashed{p})}{\slashed{p} = m_\psi}\).
  \end{itemize}
\end{center}

% \feynmandiagram [horizontal=in to out, baseline=(current bounding box.center)] {
% in -- [scalar] blob1 [label=center:1PI, blob, fill=n@terracotta]
% -- [scalar] blob2 [label=center:1PI, blob, fill=n@terracotta]
% -- [scalar] out
% };



\subsection{Running Couplings and Renormalisation Group}
So far, we have concerned ourselves with renormalisation of the fields and mass terms, but the interaction couplings of a theory also demand renormalisation.
Given an interaction term in the Lagrangian with a rescaled coupling \(\lambda_0 = Z_\lambda \mu^{\sfrac{(4-d)}{2}} \lambda\), the counterterm from expanding \(Z_\lambda = 1 + \delta_\lambda\) is used to cancel divergences arising from higher order corrections to the LO interaction vertex.
Diagrammatically to NLO in the scalar example from \cref{qft:eq:example_lagrangian}, we can have
\begin{equation}
  \label{qft:eq:vertex_counterterm_renormalisation}
  \inputtikz[baseline=(v.base)]{phi3_renormalisation/vertex_LO} + \inputtikz[baseline=(v.base)]{phi3_renormalisation/vertex_loop} + \inputtikz[baseline=(v.base)]{phi3_renormalisation/vertex_ct},
\end{equation}
where the loop diagram together with the counterterm is UV finite.
However, as UV divergences are swallowed into the definition of the coupling, it will inherit dependence on the artificial scale \(\mu\).
The bare, unrenormalised Lagrangian contains no dependence on the arbitrary mass scale, and as such, we should expect \(\d[\mu]\ \lambda_0 = 0\).
This is the condition that leads to the \emph{renormalisation group equations} that govern the dependence of the coupling on the \emph{renormalisation scale} \(\mu\).
In \(d = 4-2\epsilon\) dimensions, this condition gives us
\begin{equation}
  \label{qft:eq:RGE}
  0 \stackrel{!}{=} \mu \d[\mu]\ \lambda_0 = \mu \d[\mu]\ \pclosed{\mu^{\epsilon} Z_\lambda \lambda} = \mu^\epsilon Z_\lambda \lambda \pclosed{\epsilon + \frac{\mu}{\lambda} \d[\mu]{\lambda} + \frac{\mu}{Z_\lambda}\d[\mu]{Z_\lambda}},
\end{equation}
which can be rearranged into
\begin{equation}
  \label{qft:eq:RGE_rearranged}
  \d[\mu]{\lambda} = -\pclosed{\frac{\epsilon}{\mu} -\frac{1}{Z_\lambda} \d[\mu]{Z_\lambda}} \lambda.
\end{equation}

The running of the coupling is therefore dependent on the conditions imposed to fix the rescaling \(Z_\lambda\).
In other words, the running of the coupling is dependent on the \emph{renormalisation scheme}.
In this thesis, I will employ a variation of the Minimal Subtraction (MS) scheme, known as \MSbar\@.
As the name suggests, this entails subtracting the minimal amount necessary to make the theory UV finite.
In practice, this means setting the counterterm to be the \(\frac{1}{\epsilon^p}\)-poles arising from calculating loop amplitudes such as the one in \cref{qft:eq:vertex_counterterm_renormalisation}.
The \MSbar\ scheme additionally subtracts a factor which commonly arises together with the \(\frac{1}{\epsilon}\)-poles in dimensional regularisation, setting the counterterm proportional to\footnote{At least to first order of divergence.}
\begin{equation}
  \frac{1}{\bar\epsilon} = \frac{1}{\epsilon} - \gamma_E + \ln 4\pi,
\end{equation}
where \(\gamma_E\) is the Euler-Mascheroni constant.
This fixes the counterterm, which in turn fixes the scale dependence of the renormalised coupling.
\medskip

Finally, the renormalised coupling is fixed through a renormalisation condition as before.
This amounts to setting a boundary condition for which to solve \cref{qft:eq:RGE_rearranged}.
A common procedure is to associate the LO vertex in \cref{qft:eq:vertex_counterterm_renormalisation} with a given experimental result.
To LO the diagram only contributes \(\M = -i\lambda\), so the squared amplitude simply becomes \(\lambda^2\).
If we could concoct an experiment that measures this three-point correlator at a given energy scale, we could then fix \(\lambda\) to be this value, in essence encapsulating the result to all orders at this energy.
A natural way would be to measure the decay of particle, where the natural energy scale would be the particle mass.
Seeing as the decay \(\Gamma \propto |\M|^2\)~\cite{Schwartz:2014sze}, the renormalisation condition would be
\begin{equation}
  \lambda^2(m) = C \Gamma,
\end{equation}
where \(C\) is some calculable kinematic constant.
However, in our toy model with only one particle, such a decay is not kinematically possible, so the coupling would have to be fixed by looking at some other process, like the four-point correlator, where the kinematics are a bit more complicated.


\subsection{IR Divergences}
\todo{Formulate a few words on IR divergences.}


\section{Yang-Mills Theories}
\label{qft:sec:yang-mills}
Having gone through the basics of renormalised perturbative QFT, I will turn to the construction of a quantum field theory.
Here I will outline imposing certain symmetries between the fields in a theory will naturally lead to an interacting theory of multiple fields.
This is the basis for gauge theory, which are the foundations of modern QFTs such as the Standard Model (SM).
This section does assume some background group theory.
\medskip

Gauge theory in QFT is based on imposing \emph{internal symmetries} on the Lagrangian.
Internal symmetries are symmetries separate from \emph{external symmetries} in that they are not symmetries of coordinate transformations, but rather symmetries based on transformations of the fields in themselves.
Typically, the field transformations under which the Lagrangian is invariant are Lie groups, and are referred to as the \emph{gauge group}.
A collection of fields that transform into each other under a particular representation\footnote{More on this later.} is called a \emph{multiplet}.

Let us consider a complex scalar field theory to illustrate.
Let \(\phi_i\) be a multiplet of complex scalar fields, and let the gauge group be a general non-Abelian Lie group, locally defined by a set of hermitian generators \(T^a\).
Locally, the group elements can then be described using the exponential map as\needcite\
\begin{equation}
  \label{qft:eq:exponential_map}
  g(\vec{\alpha}) = \exptext{i \alpha^a T^a},
\end{equation}
for a set of real parameters \(\vec{\alpha}\).\footnote{I will use bold
  notation \(\vec\alpha\) to refer to the collection of parameters
  \(\alpha^a\), of which there is one for each generator \(T^a\).}
This way of parametrising the group is convenient in that the inverse of the group elements are the hermitian conjugate, i.e.\ \(g^{-1}(\vec\alpha) = g^\dagger(\vec{\alpha})\).
The transformation law for \(\Phi = \pclosed{\phi_1, \ldots}^T\) is
\begin{equation}
  \label{qft:eq:gauge_transformation}
  \Phi \to g(\vec{\alpha}) \Phi = \exptext{i \alpha^a T^a} \Phi,
\end{equation}
which for an infinitesimal set of parameters \(\epsilon^a\) becomes
\begin{equation}
  \Phi \to \pclosed{1 + i \epsilon^a T^a} \Phi.
\end{equation}
\medskip


Now, we would like to categorise the Lagrangian terms that are invariant under
such transformations.
The ordinary free Klein-Gordon Lagrangian
\begin{equation}
  \L_\text{KG} = \partial^\mu \Phi^\dagger \partial_\mu \Phi - m^2 \Phi^\dagger \Phi
\end{equation}
is invariant.
However, if we promote our gauge symmetry to be a local symmetry, i.e.\ let the
parameters become space-time-dependent \(\vec{\alpha} \to \vec{\alpha}(x)\),
this is no longer the case.
Since space-time coordinates are unchanged under gauge transformations, it
follows that so too is the derivative \(\partial_\mu\).
However, it will be useful to rewrite this in as somewhat convoluted way,
letting it ``transform'' according to\footnote{It can be shown to be equivalent
  to \(\partial_\mu\) when applied to any field (whether they transform under
  the
  gauge transformations or not).}\textsuperscript{,}\footnote{In the following I suppress the
  argument so that \(g = g\pclosed{\vec{\alpha}(x)}\).}
\begin{equation}
  \partial_\mu \to \partial_\mu = g \partial_\mu g^{-1} +
  \pclosed{\partial_\mu g} g^{-1},
\end{equation}
which in turn makes the field derivative transform to
\begin{equation}
  \partial_\mu \Phi \to g \partial_\mu \Phi + \pclosed{\partial_\mu g} \Phi,
\end{equation}
which does \emph{not} leave the kinetic term invariant.
So we must rethink the kinetic term of the Lagrangian.
To get the right transformation properties of the derivative term, we need a
\emph{covariant derivate } \(D_\mu\) such that \(D_\mu \Phi \to g D_\mu
\Phi\).
In order to create such a \(D_\mu\), we must require that it transforms as
\(D_\mu \to g D_\mu g^{-1}\).
This can be done by introducing the \emph{gauge field} \(\mathcal{A}_\mu(x)
\equiv A_\mu^a(x) T^a\) which transforms according to
\begin{equation}
  \mathcal{A}_\mu \to g \mathcal{A}_\mu g^{-1} - \frac{i}{q}
  \pclosed{\partial_\mu g} g^{-1}.
\end{equation}
The last term can compensate for the extra term in the ``transformation'' law
of \(\partial_\mu\).
We can then define the covariant derivative \(D_\mu = \partial_\mu - iq
\mathcal{A}_\mu\) to achieve this.
\medskip

In summary, with a local gauge symmetry, a gauge field \(\mathcal{A}_\mu\) must
be introduced such that kinetic terms in the original Lagrangian can be
invariant under the gauge transformation. In our case this amounts to adding
the interaction term
\begin{equation}
  \L_{\mathcal{A}\Phi\text{-int}} = -iq\bclosed{(\partial^\mu \Phi^\dagger)
    \mathcal{A}_\mu \Phi - \Phi^\dagger \mathcal{A}^\mu (\partial_\mu
    \Phi)} + q^2
  \Phi^\dagger \mathcal{A}^\mu \mathcal{A}_\mu \Phi
\end{equation}
to the Klein-Gordon Lagrangian \(\L_{KG}\).
Note that for the Lagrangian to be real-valued, we must require \(\mathcal{A}\_\mu\) to be hermitian, or equivalently, the components \(A_\mu^a\) must be real-valued seeing the generators \(T^a\) already are hermitian.

Now, the Lagrangian is gauge invariant, but there still remains to add dynamics
to the gauge field \(\mathcal{A}_\mu\) through a kinetic term.
To this end, we can make a field-strength tensor \(\mathcal{F}_{\mu\nu} \equiv
F_{\mu\nu}^a T^a\) that transforms as \(\mathcal{F}_{\mu\nu} \to g
\mathcal{F}_{\mu\nu} g^{-1}\).
The covariant derivative already has this property, and so we can define
\(\mathcal{F}_{\mu\nu} = \frac{i}{q} \commutator{D_\mu}{D_\nu}\), which will include
derivative terms for the \(\mathcal{A}_\mu\) gauge field and let us construct a
gauge invariant kinetic term \(\tr\cclosed{\mathcal{F}^{\mu\nu} \mathcal{F}_{\mu\nu}}\).
Antisymmetrising \(D_\mu D_\nu \to \commutator{D_\mu}{D_\nu}\) serves to get rid of the \(\partial_\mu \partial_\nu\)-term which would result in third derivatives of the gauge field.
The kinetic term can be shown to be gauge invariant using the transformation law the field-strength tensor and the cyclic property of the trace
\begin{equation}
  \tr\{\mathcal{F}^{\mu\nu} \mathcal{F}_{\mu\nu}\} \to
  \tr\{g \mathcal{F}^{\mu\nu} \mathcal{F}_{\mu\nu} g^{-1}\} =
  \tr\{g^{-1} g \mathcal{F}^{\mu\nu} \mathcal{F}_{\mu\nu}\} =
  \tr\{\mathcal{F}^{\mu\nu} \mathcal{F}_{\mu\nu}\}.
\end{equation}
This results in a kinetic term for the \(\mathcal{A}_\mu\)-field
\begin{equation}
  \label{qft:eq:Akin}
  \L_{\mathcal{A}\text{-kin}} = -\frac{1}{4T(R)} \tr\{\mathcal{F}^{\mu\nu} \mathcal{F}_{\mu_\nu}\} = -\frac{1}{4} F^{a\,\mu\nu} F^a_{\mu\nu},
\end{equation}
where \(T(R)\) is the Dynkin index of the representation \(R\) of the group defined by the relation \(\tr\{T^a T^b\} = T(R) \delta^{ab}\) when \(T^a\) are the generators of the group in that representation.




\section{Passarino-Veltman Loop Integrals}
\label{qft:sec:pave}
Lastly, I will go through a categorisation of loop integrals due to Passarino and Veltman~\cite{PaVe}, known as Passarino-Veltman reduction.
This provides a useful tool for symbolically handling loop integrals, by creating basis loop integral functions with which you can write arbitrary loop integrals.
\medskip

By Lorentz invariance, there are a limited set of forms that loop integrals can take.
These can be categorised according to the number of propagator terms they include, which corresponds to the number of externally connected points there are in the loop.
A general scalar \(N\)-point loop integral takes the form in \(d\) space-time dimensions can be written as
\begin{align}
  T^N_0\pclosed{p_i^2, (p_i-p_j)^2; m_0^2, m_i^2} & = \frac{\pclosed{2\pi
      \mu}^{4-d}}{i \pi^2} \integral{^d q} \mathcal{D}_0
  \prod_{i=1}^{N-1}
  \mathcal{D}_i,
\end{align}
where \(\mathcal{D}_0 = \bclosed{q^2-m_0^2}^{-1}\) and \(\mathcal{D}_i =
\bclosed{\pclosed{q+p_i}^2-m_i^2}^{-1}\).
I note that the renormalisation scale \(\mu\) coming from the coupling constant has been swallowed into the definition of these loop integrals.
The first 4 scalar loop integrals are named accordingly
\begin{align}
  T^1_0 & \equiv A_0(m_0^2)
  \\
  T^2_0 & \equiv B_0(p_1^2; m_0^2, m_1^2)
  \\
  T^3_0 & \equiv C_0(p_1^2, p_2^2, (p_1-p_2)^2; m_0^2, m_1^2, m_2^2)
  \\
  T^4_0 & \equiv D_0(p_1^2, p_2^2, p_3^2, (p_1-p_2)^2, (p_1-p_3)^2,
  (p_2-p_3)^2; m_0^2, m_1^2, m_2^2)
\end{align}

\begin{figure}[ht!]
  \centering
  \begin{subfigure}{0.49\linewidth}
    \centering
    \inputtikz{triangle_loop}
    \caption{}
    \label{qft:fig:PV_triloop_conventions}
  \end{subfigure}
  \begin{subfigure}{0.49\linewidth}
    \centering
    \inputtikz{box_loop}
    \caption{}
    \label{qft:fig:PV_boxloop_conventions}
  \end{subfigure}
  \caption{Illustration of the momentum conventions for loop diagrams used in
    the Passarino-Veltman functions.}
  \label{qft:fig:PV_loop_conventions}
\end{figure}

More complicated Lorentz structure can be obtained in loop integrals, however,
these can still be related to the scalar integrals by exploiting the possible
tensorial structure they can have. Defining an arbitrary loop integral
\begin{equation}
  T^N_{\mu_1\cdots\mu_P}\pclosed{p_i^2, (p_i-p_j)^2; m_0^2, m_i^2} =
  \frac{\pclosed{2\pi \mu}^{4-d}}{i \pi^2} \integral{^d q} q_{\mu_1}\cdots
  q_{\mu_P} \mathcal{D}_0 \prod_{i=1}^{N-1} \mathcal{D}_i,
\end{equation}
where the \(q\_{\mu_i}\) is the loop momentum with for some Lorentz indices \(\mu_i\),
these tensors can only depend on the metric \(\tensor{g}{^\mu^\nu}\) and the
external momenta \(p_i\).
The possible structures up to four-point loops are as following:
\begin{subequations}
  \begin{align}
    B^\mu                & = p_1^\mu B_1,
    \\
    B^{\mu\nu}           & = g^{\mu\nu} B_{00} + p_1^\mu p_1^\nu B_{11},
    \\
    C^\mu                & = \sum_{i=1}^2 p_i^\mu C_i,
    \\
    C^{\mu\nu}           & = g^{\mu\nu} C_{00} + \sum_{i,j=1}^{2} p_i^\mu
    p_j^\nu C_{ij},
    \\
    C^{\mu\nu\rho}       & = \sum_{i=1}^2 (g^{\mu\nu} p_i^{\rho} +
    g^{\mu\rho} p_i^{\nu} + g^{\nu\rho} p_i^{\mu}) C_{00i} +
    \sum_{i,j,k=1}^2
    p_i^\mu p_j^\nu p_k^\rho C_{ijk},
    \\
    D^\mu                & = \sum_{i=1}^3 p_i^\mu D_i,
    \\
    D^{\mu\nu}           & = g^{\mu\nu} D_{00} + \sum_{i,j=1}^{3} p_i^\mu
    p_j^\nu D_{ij},
    \\
    D^{\mu\nu\rho}       & = \sum_{i=1}^3 (g^{\mu\nu} p_i^{\rho} +
    g^{\mu\rho} p_i^{\nu} + g^{\nu\rho} p_i^{\mu}) D_{00i} +
    \sum_{i,j,k=1}^3
    p_i^\mu p_j^\nu p_k^\rho D_{ijk},
    \\
    \nonumber
    D^{\mu\nu\rho\sigma} & = (g^{\mu\nu}g^{\rho\sigma} +
    g^{\mu\rho}g^{\nu\sigma} + g^{\mu\sigma}g^{\nu\rho})D_{0000}
    \\
    \nonumber
                         & + \sum_{i,j=1}^3 (g_{\mu\nu}p_i^\rho p_j^\sigma
    + g_{\mu\nu}p_i^\sigma p_j^\rho + g_{\mu\rho}p_i^\nu p_j^\sigma +
    g_{\mu\rho}p_i^\sigma p_j^\nu + g_{\mu\sigma}p_i^\rho p_j^\nu +
    g_{\mu\nu}p_i^\nu p_j^\rho) D_{00ij}
    \\
                         & + \sum_{i,j,k,l=1}^3 p_i^\mu p_j^\nu p_k^\rho
    p_l^\sigma D_{ijkl},
  \end{align}
\end{subequations}
where all coefficients with lower case Latin indices must be completely symmetric in \(i,j,k,l\).


\ifSubfilesClassLoaded{%
  \bibliography{references}{}
  \bibliographystyle{style/JHEP}
}{}

\end{document}

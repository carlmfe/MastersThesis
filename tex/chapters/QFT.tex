\documentclass[../main.tex]{subfiles}

\begin{document}
\chapter{Quantum Field Theory}





\section{Perturbative Quantum Field Theory}
\phpar[Here I want to introduce the Lagrangian formulation of quantum field
  theory, and the types of fields we will be working with.
  Furthermore, I want to introduce the perturbation series through the path
  integral, and talk about the nature of quantum effects.
  Also derive Feynman rules from path integral.]

\begin{TODO}
  \item Introduce perturbative QFT.
  \item Talk about reading Feynman diagrams and special care to take with
  Majorana fermions.
\end{TODO}

In this thesis, I will use the Lagrangian framework to formulate QFT.
Here I will introduce the basics of how to formulate a QFT in such a way using the path integral formalism.
This leads to a perturbative formulation of scattering and computation of correlation functions, which is the basis for the calculations that will be made. \medskip






\subsection{The Path Integral}
I will start by introducing some useful shorthands that will be used throughout this section.
Consider an action \(S[\cclosed{\Phi}]\) as a functional of some fields \(\cclosed{\Phi}\). Let \(\phi_i \equiv \phi(x_i)\) for some arbitrary field \(\phi \in \cclosed{\Phi}\) evaluated at some point in space-time \(x_i\)---the path integral approach to quantum field theory is built on time-ordered correlation functions through the relation\needcite\
\begin{equation}
  \label{qft:eq:path_integral_correlation_function}
  \bra{\Omega} T\cclosed{\hat\phi_1 \cdots \hat\phi_n} \ket{\Omega} = \frac{\int\!\mathcal{D}\phi\, \phi_1 \cdots \phi_n \exp{iS[\cclosed{\Phi}]}}{\int\!\mathcal{D}\phi\, \exp{iS[\cclosed{\Phi}]}},
\end{equation}
where \(T\cclosed{.}\) denotes the time-ordering operation and \(\mathcal{D}\phi\) is the measure denoting integration over all possible \emph{field configurations}.
A field configuration here is understood as a given set of values for the fields \(\cclosed{\Phi}\), one for each point in space-time.
The time-ordering operation will put the fields in chronological order according to the time at which they are evaluated, with the ``first'' field being farthest to the right.
The left-hand side of \cref{qft:eq:path_integral_correlation_function} the fields are understood as operators on the Hilbert space of states in our interacting theory (denoted by their hats), whereas on the right-hand side they are considered as classical fields.
In this way quantum effects are encapsulated through the weighted sum of all classical \emph{paths} through configuration space, rather than just whichever one minimises the action.




\subsection{Feynman Rules}
In interacting theories, correlation functions can be obtained through a \emph{perturbation series} by expanding them around their coupling constant, here denoted \(\lambda\).
If the Lagrangian of the action can be written on the form \(\L = \L_0 + \lambda\L_\text{int}\), then the exponentiation of the action can be written as
\begin{align}
  \exp{i \integral{^4x} \pclosed{\L_0 + \L_\text{int}}} = \exp{i \integral{^4x} \L_0} \pclosed{1 + \lambda\integral{^4x_1} \L_\text{int}(x_1) + \frac{\lambda^2}{2} \integral{^4x_1}\integral{^4x_2} \L_\text{int}(x_1) \L_\text{int}(x_2) + \ldots}.
\end{align}
Now say the interaction Lagrangian \(\L_\text{int}\) is some monomial of degree \(p\) in the fields \(\cclosed{\Phi}\),\footnote{In the more general case where it can be expressed with some polynomial in the fields, we can look at each monomial term separately, without loss of generality.} then the interacting correlation functions can be written in terms of free-theory correlation functions!
To see this, we consider the interacting \(n\)-point function \(D^n_\text{int}(1,\ldots, n) = \bra{\Omega} T\cclosed{\hat\phi_1 \cdots \hat\phi_n} \ket{\Omega}\), and write it out in terms of the free and interacting Lagrangians:
\begin{align}
  \label{qft:eq:n-point_interactive_correlator}
  D^n_\text{int}(1,\ldots, n) = \frac{1}{\mathcal{N}} \int\!\mathcal{D}\phi\, \phi_1\cdots \phi_n \exp{i\integral{^4x}\L_0}\pclosed{1 + i\lambda \integral{^4y}\L_\text{int} + \mathcal{O}(\lambda^2)},
\end{align}
where the normalisation is given by \(\mathcal{N} = \int\!\mathcal{D}\phi\, \exp{i\integral{^4x}\L}\).
To relate this to the free-theory, let us take a moment to write this out.
Given the free-field \(n\)-point correlator \(D^n_0(1,\ldots,n) \equiv \frac{1}{\mathcal{N}_0} \int\!\mathcal{D}\phi\, \phi_i \cdots \phi_n \exp{i\integral{^4x}\L_0}\) with normalisation \(\mathcal{N}_0 = \int\!\mathcal{D}\phi\, \exp{i\integral{^4x}\L_0}\), expand the interacting normalisation as
\begin{equation}
  \mathcal{N} = \mathcal{N}_0 \pclosed{1 + i\lambda \integral{^4y} D_0^p(\underbrace{y, \ldots, y}_{p\text{ times}}) - \frac{\lambda^2}{2} \integral{^4y}\!\! \integral{^4z} D_0^{2p}(\underbrace{y,\ldots,y}_{p\text{ times}}, \underbrace{z,\ldots,z}_{p\text{ times}}) + \mathcal{O}(\lambda^3)}.
\end{equation}
Inserting this into \cref{qft:eq:n-point_interactive_correlator} and expanding around \(\lambda = 0\) we get
\begin{align}
  \nonumber
  D^n_\text{int}(1,\ldots, n) = \frac{D_0^n(1, \ldots, n) + i\lambda \integral{^4y} D_0^{n+p}(1, \ldots, n, \overbrace{y, \ldots, y}^{p\text{ times}}) + \mathcal{O}(\lambda^2)}{1 + i\lambda D_0^p(\underbrace{y, \ldots, y}_{p\text{ times}}) + \mathcal{O}(\lambda^2)} \\
  = D^n_0(1,\ldots,n) + i\lambda \pclosed{D_0^{n+p}(1, \ldots, n, \underbrace{y, \ldots, y}_{p\text{ times}}) - D_0^n(1, \ldots, n) D_0^p(\underbrace{y, \ldots, y}_{p\text{ times}})} + \mathcal{O}(\lambda^2)
\end{align}


\begin{equation}
  \pclosed{\dA_x - m^2} D^2_0(x,y) = -i \delta^4(x-y)
\end{equation}





\section{Renormalised Quantum Field Theory}
\phpar[Talk about loop integrals, divergences, regularisation and
  renormalisation.]

\todo{Mention Wick rotation and evaluation of loop integrals?}

Divergences appear in perturbative correlation functions in QFT, and can be
categorised into \emph{ultraviolet} (UV) divergences and \emph{infrared} (IR)
divergences. They are so named after which region of momentum space they
originate from --- high momentum for UV and low momentum for IR. The two types
of divergences are dealt with entirely differently, and here I will lay out how
to deal with UV divergences through \emph{renormalisation}. \medskip

Consider a loop like the one in \cref{qft:fig:example_loop}: If we consider a
massless particle in the loop, the loop integral will take the form
\begin{equation}
  \label{qft:eq:example_loop}
  \loopintegral[4]{q} \frac{1}{\pclosed{q^2 + i\epsilon}^2} =
  \frac{i}{(2\pi)^4} \integral{\Omega_4} \integral[_0^\infty]{q_E}
  \frac{1}{q_E},
\end{equation}
which diverges for both low and high momenta. \comment{Mention what went into
this integral, i.e.\ Wick rotation and \(i\epsilon\).} Had the particle been
massive, the momentum would have a non-zero lower limit, and the IR divergence
would disappear.
However, the UV divergence must be handled differently.





\subsection{Regularisation}

\begin{figure}[!ht]
  \centering
  \inputtikz{ex_loop}
  \caption{Simple example of a loop diagram in a scalar theory.}
  \label{qft:fig:example_loop}
\end{figure}

A first step to handle the divergences is to deform our theory in some way to
make the loop integral formally finite, but recovering the divergence in the
limit that the deformation disappears. An intuitive deformation would be to cap
the momentum integral at some \(\Lambda\), recovering our original theory in
the limit \(\Lambda \to \infty\). To illustrate the procedure of regularisation
and subsequently renormalisation, it will be useful to have an example, for
which I choose a scalar Lagrangian \(\L = \frac{1}{2} \pclosed{\partial_\mu
  \phi}^2 - \frac{1}{2}m_\phi^2 \phi^2 - \frac{\lambda}{3!} \phi^3\).
\comment{Perhaps introduce this model earlier?} Regularising the IR divergence
in \cref{qft:eq:example_loop} by giving our scalar a mass \(m\), and the UV
divergence with a momentum cap \(\Lambda\), we are left with
\begin{align}
  \nonumber
  \int_{\abs{q} < \Lambda}\! \frac{\mathrm{d}^4q}{(2\pi)^4}
  \frac{1}{\pclosed{(q^2-m^2)+i\epsilon}^2} = \frac{i}{(2\pi)^4}
  \integral{\Omega_4} \integral[_0^\Lambda]{q_E} \frac{q_E^3}{\pclosed{q_E^2
      -
  m^2}^2} \\
  = \frac{i}{16\pi^2} \cclosed{\ln\pclosed{1 + \frac{\Lambda^2}{m^2}} -
    \frac{\Lambda^2}{\Lambda^2 + m^2}},
\end{align}
where now evidently the divergences manifest as a logarithm.
\medskip

Another popular choice of regularisation, which I will use in this thesis, is
\emph{dimensional regularisation}. It entails analytically continuing the
number of space-time dimension from the ordinary 4 dimensions to \(d
= 4-2\epsilon\) dimensions for some small \(\epsilon\).\footnote{The reason for choosing \(2\epsilon\) is purely aesthetical, making some expressions neater.} This removes much of
the intuition for what we are doing, but turns out to be computationally very
efficient. Our loop integral \cref{qft:eq:example_loop} will then turn into
\begin{align}
  \nonumber
  \loopintegral[d]{q} \frac{1}{\pclosed{q^2+i\epsilon}^2} = \frac{i
    2\pi^{\sfrac{d}{2}}}{(2\pi)^d} \frac{1}{\Gamma(\sfrac{d}{2})}
  \integral[_0^\infty]{q} q_E^{d-5} \\
  = \frac{i 2\pi^{2-\epsilon}}{(2\pi)^{4-2\epsilon}}
  \frac{1}{\Gamma\pclosed{2-\epsilon}} \cclosed{ \integral[_0^\mu]{q}
  \frac{1}{q_E^{1+2\epsilon}} + \integral[_\mu^\infty]{q}
  \frac{1}{q_E^{1+2\epsilon}} } = \frac{i}{16\pi^2}
  \pclosed{\frac{1}{\epsilon_\text{IR}} - \frac{1}{\epsilon_\text{UV}}} +
  O(\epsilon),
\end{align}
where in the second equality, the momentum integral is split into a low-energy
and high-energy part with some scale \(\mu\). Here a trick was performed, as
the low-energy part requires \(\epsilon < 0\) to be convergent, whereas the
high-energy part requires \(\epsilon>0\). The two different divergences thus
require different deformations of the theory to be finite, and should be
handled separately, hence the subscripts. In the end, the divergences when using
dimensional regularisation come out as \(\frac{1}{\epsilon}\)-terms.






\subsection{Counterterm Renormalisation}
To take care of UV divergences, we note that there is freedom in how we define
the contents of our Lagrangian. We should be able to rescale our fields like \(\phi_0 =
\sqrt{Z_\phi} \phi\), and rescale our couplings like \(m_{\phi,0}^2 = Z_m
m_\phi^2\) and \(\lambda_0 = Z_\lambda \lambda\). Although suggestively naming
terms such as \emph{mass term} with mass \(m_\phi^0\) implies a connection to
the mass of a particle, we have yet to define what that would mean
experimentally. Thus, rescaling our parameters and fields parametrises the way
in which we can tune our theory, allowing us freedom in choosing the way our
theory connects to experiments.

This approach actually allows us to make a perturbative scheme for fixing our
(re)normalisations of the fields and couplings. There are many choices for how
to connect theory to experiment, but one common approach for field and mass
renormalisation is to identify the pole of the two-point correlation function
\(\mathcal{G}(x, y)\) of a particle to the mass resonance measurable in
experiment.\margincomment{Is there a better explanation for this experiment
  than `mass resonance'?} This allows us to perturbatively calculate the
two-point correlator, and then fix our normalisations accordingly, such that
our imposed condition on it holds at every order in the perturbation series. We
achieve this systematically with \emph{counterterms}, which in essence are
additional Feynman rules added to the theory. By expanding the renormalisation
parameters as \(Z = 1 + \delta\), the \(\delta\) will carry the correction to
the normalisation to any given order in a coupling constant. To one-loop order,
the self-energy of our \needcite[scalar theory] is diagrammatically given by
\begin{equation*}
  \inputtikz[baseline=(in.base)]{SE_LO} +
  \inputtikz[baseline=(in.base)]{SE_NLO} +
  \inputtikz[baseline=(in.base)]{SE_cnt},
\end{equation*}
where the crossed dot represents an insertion of the \(\delta\) into the LO
amplitude. \cringe{Since the \(\delta\) carries corrections proportional to the
  NLO amplitudes, it should be seen as coming in at NLO}.





\subsection{On-Shell Renormalisation}
Categorising all higher order contributions that can arise to the LO
self-energy of a massive particle, they come in the form of
\emph{one-particle-irreducible} (1PI) diagrams. These are diagrams where all
lines with loop momentum running through are connected. Other diagrams can be
reconstructed as the sum of 1PI diagrams. Denoting the leading order correlator
\(\mathcal{G}_0(p)\) and the contribution from one insertion of all 1PI
diagrams \(i\Sigma(p)\), we get a series\footnote{A note on the argument \(p\)
  of these functions: The two-point-correlators in momentum space depend on
  the
  four-momentum \(p^\mu\) in such a way that when it is put in between the
  external particle representations (i.e.\ 1 for scalars, spinors for
  fermions
  and polarisation vectors for vector bosons) the result will be Lorentz
  invariant. This means in principle that the correlator could carry Lorentz
  indices too, which will be suppressed here for simplicity.}
\begin{align}
  \nonumber
  i\mathcal{G}(p) = & \inputtikz[baseline=(in.center)]{SE_LO} +
  \inputtikz[baseline=(in.center)]{SE_1PI} +
  \inputtikz[baseline=(in.center)]{SE_2PI} + \ldots
  \\
  \nonumber
  =                 & i\mathcal{G}_0(p) + i\mathcal{G}_0(p) i\Sigma(p)
  i\mathcal{G}_0(p) + i\mathcal{G}_0(p) i\Sigma(p) i\mathcal{G}_0(p)
  i\Sigma(p)
  i\mathcal{G}_0(p) + \ldots
  \\
  \nonumber
  =                 & i\mathcal{G}_0(p) \bclosed{i\Sigma(p) i\mathcal{G}_0(p)
    + \pclosed{i\Sigma(p) i\mathcal{G}_0(p)}^2 + \ldots}
  \\
  =                 & i\mathcal{G}_0(p) \frac{1}{1 + \Sigma(p)
    \mathcal{G}_0(p)} = \frac{i}{\mathcal{G}_0^{-1}(p) + \Sigma(p)}.
\end{align}
So the computation of the two-point correlator to any order can be done simply
by computing the sum of the 1PI diagrams to that order. These contributions
will generally diverge, but then we can take into account the renormalisation
parameters. Since this is a \emph{bare} function, i.e.\ using the
non-renormalised quantities, we can get the renormalised two-point correlator
\(\mathcal{G}^\text{R}(p)\) through
\begin{equation}
  \mathcal{G}^\text{R}(p) = \frac{1}{Z_\psi} \mathcal{G}^\text{bare}(p) =
  \frac{1}{1 + \delta_\psi} \mathcal{G}^\text{bare}(p),
\end{equation}
for any field \(\psi\), seeing as the two-point correlator is quadratic in
\(\psi\) and thereby quadratic in \(\sqrt{Z_\psi}\).
\medskip

On-shell mass renormalisation seeks to identify the pole of the two-point correlator with the physical mass as observed in experiment.
This is a generalisation the property of the free theory two-point function to the perturbative interacting two-point function at any order.
It yields two conditions:
\begin{center}
  \begin{itemize}
    \item [(I)] \(\evalat{\bclosed{(1+\delta_\psi)\pclosed{\mathcal{G}^\text{bare}_0(p)}^{-1}
              + \Sigma(p)}}{p^2 = m_\text{pole}^2} = 0\),
    \item [(II)] \(\operatorname{Res}\cclosed{\mathcal{G}^\text{R}(p), p^2
            = m_\text{pole}^2} = 1\),
  \end{itemize}
\end{center}
where \(\operatorname{Res}\cclosed{f(z), z = z_0}\) is the residue of the
function \(f\) at \(z_0\).
\medskip

For our scalar theory, where the leading order \emph{bare} two-point correlator is
\(\mathcal{G}_0^\text{bare}(p) =\frac{1}{p^2 - m_0^2}\), this means that we get the
relations
\begin{center}
  \begin{itemize}
    \item [(I)] \(\delta_m m_\phi^2 = \Sigma(m_\phi^2)\),
    \item [(II)] \(\delta_\phi = -\evalat{\d[p^2]{}\Sigma(p^2)}{p^2 =
            m_\phi^2}\).
  \end{itemize}
\end{center}
\provethis{}

\todo{Outline chiral mass renormalisation.}

% \feynmandiagram [horizontal=in to out, baseline=(current bounding box.center)] {
% in -- [scalar] blob1 [label=center:1PI, blob, fill=n@terracotta]
% -- [scalar] blob2 [label=center:1PI, blob, fill=n@terracotta]
% -- [scalar] out
% };





\subsection{Renormalised Parton Distribution Functions}




\section{Yang-Mills Theories}
\label{qft:sec:yang-mills}

Gauge theory in QFT is based on imposing \emph{internal symmetries} on the Lagrangian.
Internal symmetries are symmetries separate from \emph{external symmetries} in that they are not symmetries of coordinate transformations, but rather symmetries based on transformations of the fields.
Typically, the field transformations under which the Lagrangian is invariant are Lie groups, and are referred to as the \emph{gauge group}.
A collection of fields that transform into each other under a particular representation\footnote{More on this later.} is called a \emph{multiplet}.

Let us consider a complex scalar field theory to illustrate.
Let \(\phi_i\) be a multiplet of complex scalar fields, and let the gauge group be a general non-Abelian Lie group, locally defined by a set of hermitian generators \(T^a\).
Locally, the group elements can then be described using the exponential map as\needcite\
\begin{equation}
  \label{qft:eq:exponential_map}
  g(\vec{\alpha}) = \exptext{i \alpha^a T^a},
\end{equation}
for a set of real parameters \(\vec{\alpha}\).\footnote{I will use bold
  notation \(\vec\alpha\) to refer to the collection of parameters
  \(\alpha^a\), of which there is one for each generator \(T^a\).}
This way of parametrising the group is convenient in that the inverse of the group elements are the hermitian conjugate, i.e.\ \(g^{-1}(\vec\alpha) = g^\dagger(\vec{\alpha})\).
The transformation law for \(\Phi = \pclosed{\phi_1, \ldots}^T\) is
\begin{equation}
  \label{qft:eq:gauge_transformation}
  \Phi \to g(\vec{\alpha}) \Phi = \exptext{i \alpha^a T^a} \Phi,
\end{equation}
which for an infinitesimal set of parameters \(\epsilon^a\) becomes
\begin{equation}
  \Phi \to \pclosed{1 + i \epsilon^a T^a} \Phi.
\end{equation}
\medskip


Now, we would like to categorise the Lagrangian terms that are invariant under
such transformations.
The ordinary free Klein-Gordon Lagrangian
\begin{equation}
  \L_\text{KG} = \partial^\mu \Phi^\dagger \partial_\mu \Phi - m^2 \Phi^\dagger \Phi
\end{equation}
is invariant.
However, if we promote our gauge symmetry to be a local symmetry, i.e.\ let the
parameters become spacetime-dependent \(\vec{\alpha} \to \vec{\alpha}(x)\),
this is no longer the case.
Since space-time coordinates are unchanged under gauge transformations, it
follows that so too is the derivative \(\partial_\mu\).
However, it will be useful to rewrite this in as somewhat convoluted way,
letting it ``transform'' according to\footnote{It can be shown to be equivalent
  to \(\partial_\mu\) when applied to any field (whether they transform under
  the
  gauge transformations or not).}\textsuperscript{,}\footnote{In the following I suppress the
  argument so that \(g = g\pclosed{\vec{\alpha}(x)}\).}
\begin{equation}
  \partial_\mu \to \partial_\mu = g \partial_\mu g^{-1} +
  \pclosed{\partial_\mu g} g^{-1},
\end{equation}
which in turn makes the field derivative transform to
\begin{equation}
  \partial_\mu \Phi \to g \partial_\mu \Phi + \pclosed{\partial_\mu g} \Phi,
\end{equation}
which does \emph{not} leave the kinetic term invariant.
So we must rethink the kinetic term of the Lagrangian.
To get the right transformation properties of the derivative term, we need a
\emph{covariant derivate } \(D_\mu\) such that \(D_\mu \Phi \to g D_\mu
\Phi\).
In order to create such a \(D_\mu\), we must require that it transforms as
\(D_\mu \to g D_\mu g^{-1}\).
This can be done by introducing the \emph{gauge field} \(\mathcal{A}_\mu(x)
\equiv A_\mu^a(x) T^a\)\margincomment{Here it might be prudent to mention that
  \(\mathcal{A}_\mu\) is a real-valued field somehow.} which transforms according to
\begin{equation}
  \mathcal{A}_\mu \to g \mathcal{A}_\mu g^{-1} - \frac{i}{q}
  \pclosed{\partial_\mu g} g^{-1}.
\end{equation}
The last term can compensate for the extra term in the ``transformation'' law
of \(\partial_\mu\).
We can then define the covariant derivative \(D_\mu = \partial_\mu - iq
\mathcal{A}_\mu\) to achieve this.
\medskip

In summary, with a local gauge symmetry, a gauge field \(\mathcal{A}_\mu\) must
be introduced such that kinetic terms in the original Lagrangian can be
invariant under the gauge transformation. In our case this amounts to adding
the interaction term
\begin{equation}
  \L_{\mathcal{A}\Phi\text{-int}} = -iq\bclosed{(\partial^\mu \Phi^\dagger)
    \mathcal{A}_\mu \Phi - \Phi^\dagger \mathcal{A}^\mu (\partial_\mu
    \Phi)} + q^2
  \Phi^\dagger \mathcal{A}^\mu \mathcal{A}_\mu \Phi
\end{equation}
to the Klein-Gordon Lagrangian \(\L_{KG}\).

Now, the Lagrangian is gauge invariant, but there still remains to add dynamics
to the gauge field \(\mathcal{A}_\mu\) through a kinetic term.
To this end, we can make a field-strength tensor \(\mathcal{F}_{\mu\nu} \equiv
F_{\mu\nu}^a T^a\) that transforms as \(\mathcal{F}_{\mu\nu} \to g
\mathcal{F}_{\mu\nu} g^{-1}\).
The covariant derivative already has this property, and so we can define
\(\mathcal{F}_{\mu\nu} = \frac{i}{q} \commutator{D_\mu}{D_\nu}\), which will include
derivative terms for the \(\mathcal{A}_\mu\) gauge field and let us construct a
gauge invariant kinetic term \(\tr\cclosed{\mathcal{F}^{\mu\nu} \mathcal{F}_{\mu\nu}}\).
Antisymmetrising \(D_\mu D_\nu \to \commutator{D_\mu}{D_\nu}\) serves to get rid of the \(\partial_\mu \partial_\nu\)-term which would result in third derivatives of the gauge field.
The kinetic term can be shown to be gauge invariant using the transformation law the field-strength tensor and the cyclic property of the trace
\begin{equation}
  \tr\{\mathcal{F}^{\mu\nu} \mathcal{F}_{\mu\nu}\} \to
  \tr\{g \mathcal{F}^{\mu\nu} \mathcal{F}_{\mu\nu} g^{-1}\} =
  \tr\{g^{-1} g \mathcal{F}^{\mu\nu} \mathcal{F}_{\mu\nu}\} =
  \tr\{\mathcal{F}^{\mu\nu} \mathcal{F}_{\mu\nu}\}.
\end{equation}
This results in a kinetic term for the \(\mathcal{A}_\mu\)-field
\begin{equation}
  \label{qft:eq:Akin}
  \L_{\mathcal{A}\text{-kin}} = -\frac{1}{4T(R)} \tr\{\mathcal{F}^{\mu\nu} \mathcal{F}_{\mu_\nu}\} = -\frac{1}{4} F^{a\,\mu\nu} F^a_{\mu\nu},
\end{equation}
where \(T(R)\) is the Dynkin index of the representation \(R\) of the group defined by the relation \(\tr\{T^a T^b\} = T(R) \delta^{ab}\) when \(T^a\) are the generators of the group in that representation.




\section{The Standard Model}
\phpar[Introduce the \emph{relevant} fields of the Standard Model and its construction. Maybe mention the Higgs mechanism?]




\section{Loop Integrals and Regularisation}
\phpar[Introduce loop integrals, how to calculate them, where divergences appear and how to regularise them.]
\begin{TODO}
  \item Introduce \DRbar\ renormalisation scheme and talk about Yukawa counterterm in relation to SUSY breaking.
\end{TODO}



\subsection{Dimensional Regularisation}




\subsection{Passarino-Veltman Loop Integrals}
By Lorentz invariance, there are a limited set of forms that loop integrals
can
take.\needcite[Why is this?] These can be categorised according to the number
of propagator terms they include, which corresponds to the number of externally
connected points there are in the loop. A general scalar \(N\)-point loop
integral takes the form
\begin{align}
  T^N_0\pclosed{p_i^2, (p_i-p_j)^2; m_0^2, m_i^2} & = \frac{\pclosed{2\pi
      \mu}^{4-d}}{i \pi^2} \integral{^d q} \mathcal{D}_0
  \prod_{i=1}^{N-1}
  \mathcal{D}_i,
\end{align}
where \(\mathcal{D}_0 = \bclosed{q^2-m_0^2}^{-1}\) and \(\mathcal{D}_i =
\bclosed{\pclosed{q+p_i}^2-m_i^2}^{-1}\).
The first 4 scalar loop integrals are named accordingly
\begin{align}
  T^1_0 & \equiv A_0(m_0^2)
  \\
  T^2_0 & \equiv B_0(p_1^2; m_0^2, m_1^2)
  \\
  T^3_0 & \equiv C_0(p_1^2, p_2^2, (p_1-p_2)^2; m_0^2, m_1^2, m_2^2)
  \\
  T^4_0 & \equiv D_0(p_1^2, p_2^2, p_3^2, (p_1-p_2)^2, (p_1-p_3)^2,
  (p_2-p_3)^2; m_0^2, m_1^2, m_2^2)
\end{align}

\begin{figure}[ht!]
  \centering
  \begin{subfigure}{0.49\linewidth}
    \centering
    \inputtikz{triangle_loop}
    \caption{}
    \label{qft:fig:PV_triloop_conventions}
  \end{subfigure}
  \begin{subfigure}{0.49\linewidth}
    \centering
    \inputtikz{box_loop}
    \caption{}
    \label{qft:fig:PV_boxloop_conventions}
  \end{subfigure}
  \caption{Illustration of the momentum conventions for loop diagrams used in
    the Passarino-Veltman functions.}
  \label{qft:fig:PV_loop_conventions}
\end{figure}

More complicated Lorentz structure can be obtained in loop integrals, however,
these can still be related to the scalar integrals by exploiting the possible
tensorial structure they can have. Defining an arbitrary loop integral
\begin{equation}
  T^N_{\mu_1\cdots\mu_P}\pclosed{p_i^2, (p_i-p_j)^2; m_0^2, m_i^2} =
  \frac{\pclosed{2\pi \mu}^{4-d}}{i \pi^2} \integral{^d q} q_{\mu_1}\cdots
  q_{\mu_P} \mathcal{D}_0 \prod_{i=1}^{N-1} \mathcal{D}_i.
\end{equation}
These tensors can only depend on the metric \(\tensor{g}{^\mu^\nu}\) and the
external momenta \(p_i\).
The possible structures up to four-point loops are as following:
\begin{subequations}
  \begin{align}
    B^\mu                & = p_1^\mu B_1,
    \\
    B^{\mu\nu}           & = g^{\mu\nu} B_{00} + p_1^\mu p_1^\nu B_{11},
    \\
    C^\mu                & = \sum_{i=1}^2 p_i^\mu C_i,
    \\
    C^{\mu\nu}           & = g^{\mu\nu} C_{00} + \sum_{i,j=1}^{2} p_i^\mu
    p_j^\nu C_{ij},
    \\
    C^{\mu\nu\rho}       & = \sum_{i=1}^2 (g^{\mu\nu} p_i^{\rho} +
    g^{\mu\rho} p_i^{\nu} + g^{\nu\rho} p_i^{\mu}) C_{00i} +
    \sum_{i,j,k=1}^2
    p_i^\mu p_j^\nu p_k^\rho C_{ijk},
    \\
    D^\mu                & = \sum_{i=1}^3 p_i^\mu D_i,
    \\
    D^{\mu\nu}           & = g^{\mu\nu} D_{00} + \sum_{i,j=1}^{3} p_i^\mu
    p_j^\nu D_{ij},
    \\
    D^{\mu\nu\rho}       & = \sum_{i=1}^3 (g^{\mu\nu} p_i^{\rho} +
    g^{\mu\rho} p_i^{\nu} + g^{\nu\rho} p_i^{\mu}) D_{00i} +
    \sum_{i,j,k=1}^3
    p_i^\mu p_j^\nu p_k^\rho D_{ijk},
    \\
    \nonumber
    D^{\mu\nu\rho\sigma} & = (g^{\mu\nu}g^{\rho\sigma} +
    g^{\mu\rho}g^{\nu\sigma} + g^{\mu\sigma}g^{\nu\rho})D_{0000}
    \\
    \nonumber
                         & + \sum_{i,j=1}^3 (g_{\mu\nu}p_i^\rho p_j^\sigma
    + g_{\mu\nu}p_i^\sigma p_j^\rho + g_{\mu\rho}p_i^\nu p_j^\sigma +
    g_{\mu\rho}p_i^\sigma p_j^\nu + g_{\mu\sigma}p_i^\rho p_j^\nu +
    g_{\mu\nu}p_i^\nu p_j^\rho) D_{00ij}
    \\
                         & + \sum_{i,j,k,l=1}^3 p_i^\mu p_j^\nu p_k^\rho
    p_l^\sigma D_{ijkl},
  \end{align}
\end{subequations}
where all coefficients must be completely symmetric in \(i,j,k,l\).


\end{document}

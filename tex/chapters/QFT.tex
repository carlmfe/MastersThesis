\section{Lagrangian Formalism and the Path Integral}
    \phpar[Here I want to introduce the Lagrangian formulation of quantum field theory, and the types of fields we will be working with.
        Furthermore, I want to introduce the perturbation series through the path integral, and talk about the nature of quantum effects.
        Also derive Feynman rules from path integral.]

\section{Renormalised Quantum Field Theory}
    \phpar[Talk about loop integrals, divergences, regularisation and renormalisation.]

\section{The Standard Model}
    \phpar[Introduce the \emph{relevant} fields of the Standard Model and its construction.
        Maybe mention the Higgs mechanism?]

\section{Loop Integrals and Regularisation}
    \phpar[Introduce loop integrals, how to calculate them, where divergences appear and how to regularise them.]

    \subsection{Dimensional Regularisation}

    \subsection{Passarino-Veltman Loop Integrals}
        By Lorentz invariance, there are a limited set of forms that loop integrals can take.\needcite[Why is this?]
        These can be categorised according to the number of propagator terms they include, which corresponds to the number of externally connected points there are in the loop.
        A general scalar \(N\)-point loop integral takes the form
        \begin{align}
            T^N_0\pclosed{p_i^2, (p_i-p_j)^2; m_0^2, m_i^2} & = \frac{\pclosed{2\pi \mu}^{4-d}}{i \pi^2} \integral{^d q} \mathcal{D}_0 \prod_{i=1}^{N-1} \mathcal{D}_i,
        \end{align}
        where \(\mathcal{D}_0 = \bclosed{q^2-m_0^2}^{-1}\) and \(\mathcal{D}_i = \bclosed{\pclosed{q+p_i}^2-m_i^2}^{-1}\).
        The first 4 scalar loop integrals are named accordingly
        \begin{align}
            T^1_0 & \equiv A_0(m_0^2)                                                                           \\
            T^2_0 & \equiv B_0(p_1^2; m_0^2, m_1^2)                                                             \\
            T^3_0 & \equiv C_0(p_1^2, p_2^2, (p_1-p_2)^2; m_0^2, m_1^2, m_2^2)                                  \\
            T^4_0 & \equiv D_0(p_1^2, p_2^2, p_3^2, (p_1-p_2)^2, (p_1-p_3)^2, (p_2-p_3)^2; m_0^2, m_1^2, m_2^2)
        \end{align}

        \begin{figure}[ht!]
            \centering
            \begin{subfigure}{0.49\linewidth}
                \centering
                \inputtikz{triangle_loop}
                \caption{}
                \label{qft:fig:PV_triloop_conventions}
            \end{subfigure}
            \begin{subfigure}{0.49\linewidth}
                \centering
                \inputtikz{box_loop}
                \caption{}
                \label{qft:fig:PV_boxloop_conventions}
            \end{subfigure}
            \caption{Illustration of the momentum conventions for loop diagrams used in the Passarino-Veltman functions.}
            \label{qft:fig:PV_loop_conventions}
        \end{figure}

        More complicated Lorentz structure can be obtained in loop integrals, however, these can still be related to the scalar integrals by exploiting the possible tensorial structure they can have.
        Defining an arbitrary loop integral
        \begin{equation}
            T^N_{\mu_1\cdots\mu_P}\pclosed{p_i^2, (p_i-p_j)^2; m_0^2, m_i^2} = \frac{\pclosed{2\pi \mu}^{4-d}}{i \pi^2} \integral{^d q} q_{\mu_1}\cdots q_{\mu_P} \mathcal{D}_0 \prod_{i=1}^{N-1} \mathcal{D}_i.
        \end{equation}
        These tensors can only depend on the metric \(\tensor{g}{^\mu^\nu}\) and the external momenta \(p_i\).
        The possible structures up to four-point loops are as following:
        \begin{subequations}
            \begin{align}
                B^\mu                & = p_1^\mu B_1,                                                                                                                                                                                                         \\
                B^{\mu\nu}           & = g^{\mu\nu} B_{00} + p_1^\mu p_1^\nu B_{11},                                                                                                                                                                          \\
                C^\mu                & = \sum_{i=1}^2 p_i^\mu C_i,                                                                                                                                                                                            \\
                C^{\mu\nu}           & = g^{\mu\nu} C_{00} + \sum_{i,j=1}^{2} p_i^\mu p_j^\nu C_{ij},                                                                                                                                                         \\
                C^{\mu\nu\rho}       & = \sum_{i=1}^2 (g^{\mu\nu} p_i^{\rho} + g^{\mu\rho} p_i^{\nu} + g^{\nu\rho} p_i^{\mu}) C_{00i} + \sum_{i,j,k=1}^2 p_i^\mu p_j^\nu p_k^\rho C_{ijk},                                                                    \\
                D^\mu                & = \sum_{i=1}^3 p_i^\mu D_i,                                                                                                                                                                                            \\
                D^{\mu\nu}           & = g^{\mu\nu} D_{00} + \sum_{i,j=1}^{3} p_i^\mu p_j^\nu D_{ij},                                                                                                                                                         \\
                D^{\mu\nu\rho}       & = \sum_{i=1}^3 (g^{\mu\nu} p_i^{\rho} + g^{\mu\rho} p_i^{\nu} + g^{\nu\rho} p_i^{\mu}) D_{00i} + \sum_{i,j,k=1}^3 p_i^\mu p_j^\nu p_k^\rho D_{ijk},                                                                    \\
                \nonumber
                D^{\mu\nu\rho\sigma} & = (g^{\mu\nu}g^{\rho\sigma} + g^{\mu\rho}g^{\nu\sigma} + g^{\mu\sigma}g^{\nu\rho})D_{0000}                                                                                                                             \\
                \nonumber
                                     & + \sum_{i,j=1}^3 (g_{\mu\nu}p_i^\rho p_j^\sigma + g_{\mu\nu}p_i^\sigma p_j^\rho + g_{\mu\rho}p_i^\nu p_j^\sigma + g_{\mu\rho}p_i^\sigma p_j^\nu + g_{\mu\sigma}p_i^\rho p_j^\nu + g_{\mu\nu}p_i^\nu p_j^\rho) D_{00ij} \\
                                     & + \sum_{i,j,k,l=1}^3 p_i^\mu p_j^\nu p_k^\rho p_l^\sigma D_{ijkl},
            \end{align}
        \end{subequations}
        where all coefficients must be completely symmetric in \(i,j,k,l\).

\documentclass[../main.tex]{subfiles}


\begin{document}
\chapter{Electroweakino Pair Production in Proton--Proton Collisions}
\label{had:chap:hadron_calculation}
In this chapter, I go through the computation of the hadronic cross-section.
This entails renormalising infrared divergences the parton level cross-sections and the parton distribution functions (PDFs).
I briefly go over the parton model, the hadron level kinematics and the renormalisation procedure, before presenting the concrete integral to be computed numerically in the next chapter.


\section{The Parton Model and PDFs}
So far, we have worked with the parton level cross-section, figuring out the contribution of the individual constituents of a proton to the cross-section for our final state.
These do not individually result in any observable, as the partons are confined to the proton, and can therefore not be singled out in an experiment.
To get an observable quantity comparable to experiment, we must sew the individual contributions together.
This is done with the \emph{parton model}, where scattering interactions with the proton is modelled with the interaction of free constituent particles inside.
The parton model builds on the concept of \emph{factorisation} which, owing to the weakening of the QCD coupling at high-energies, divides interactions with colour-neutral particles into a high-energy \emph{hard} part between partons of the colliding protons, and a low-energy regime constituing interaction between the partons in a single proton.
The hard part constitutes the partonic cross-sections calculated in the previous chapter, whereas the low-energy interactions are encapsulated in the PDFs which are fitted to experiment.
The low-energy regime dictates that partons each carry a fraction of the total momentum of the proton, and the probability of encountering a given parton with said momentum fraction.


\subsection{Hadronic kinematics}
Consider the scattering of two protons with momenta \(P_1^\mu\) and \(P_2^\mu\) respectively into a set of final state particles \(\chi, \chi^\prime, X\) where \(X\) is some collection of unlabelled particles.
\cref{had:tab:had_kinematic_variables} lists the definitions of kinematic variables at the hadronic level and their relation of the partonic kinematic variables defined in \cref{pc:sec:NLO}.
I define the centre-of-mass energy of the protons \(S \equiv (P_1 + P_2)^2\).
The cross-section for a given process is then given in terms of the cross-section of two partons \(i, j\) with momenta \(k_i = x_1 P_1\) and \(k_j = x_2 P_2\) where \(x_1, x_2 \in [0, 1]\) are the respective fractions of the proton momenta the partons carry.
The \emph{hadronic cross-section} differential in the squared mass \(Q^2\) of two final state particles \(\chi\) and \(\chi^\prime\) is then given by
\begin{align}
  \label{had:eq:dsigma_dQ2}
  \nonumber
  \d[Q^2]{\sigma}(PP \to \chi \chi^\prime + X) = \sum_ {ij} \integral[_0^1]{x_1}\!\integral[_0^1]{x_2} \theta\pclosed{\hat{s} - Q^2} f_i(x_1) f_j(x_2) \d[Q^2]{\hat\sigma}(ij \to \chi\chi^\prime + X) \\
  = \sum_ {ij} \integral[_\tau^1]{x_1}\!\integral[_{\tau/x_1}^1]{x_2} f_i(x_1) f_j(x_2) \d[Q^2]{\hat\sigma}(ij \to \chi\chi^\prime + X).
\end{align}
The Heaviside function \(\theta\pclosed{\hat{s}-Q^2} = \theta\pclosed{x_1 x_2 - \tau}\) ensures that there is enough energy between the scattering partons to produce the final state \(\chi \chi^\prime\)-pair with centre-of-mass energy \(Q^2 = \tau S\).

\begin{table}[ht!]
  \centering
  \begin{tabular}{|c|c|}
    \hline
    Partonic variable & Definition in terms of hadronic variables \\
    \hline
    \(k_i^\mu\)       & \(x_1 P_1^\mu\)                           \\
    \(k_j^\mu\)       & \(x_2 P_2^\mu\)                           \\
    \(\hat{s}\)       & \(x_1 x_2 S\)                             \\
    \(z\)             & \(\frac{\tau}{x_1x_2}\)                   \\
    \hline
  \end{tabular}
  \caption{List of relations between hadronic and partonic kinematic variables.}
  \label{had:tab:had_kinematic_variables}
\end{table}



\subsection{Integration over PDFs}
Practically, the two-dimensional integration over the parton momentum fractions \(x_1, x_2\) can be alleviated by the fact that partonic cross-section contains terms proportional to either \(\delta(1-z)\) or plus distributions \(f_+(z)\) as we have seen in \cref{chap:part:parton_calculation}.
Let us consider these types of integrals in some generality.
Let \(g(x_1, x_2)\) be some function of \(x_1, x_2\), consider the integral
\begin{equation}
  \int_{\tau/x_1}^1 \! \frac{\mathrm{d}x_2}{x_2} g(x_1, x_2) \delta(1-z).
\end{equation}
Switching variables to \(z = \frac{\tau}{x_1 x_2}\) while keeping \(x_1\) constant yields
\begin{align}
  \int_{\tau/x_1}^1 \! \frac{\mathrm{d}z}{z}  g(x_1, \frac{\tau}{x_1 z}) \delta(1-z) = g(x_1, \frac{\tau}{x_1}).
\end{align}

The plus-distributions are somewhat more complicated.
Keeping in mind their definition
\begin{equation}
  \integral[_0^1]{z} g(z) f_+(z) = \integral[_0^1]{z} (g(z)-g(1)) f(z),
\end{equation}
we have that
\begin{align}
  \label{had:eq:plus_integral}
  \nonumber
  \int_{\tau/x_1}^1 \! \frac{\mathrm{d}x_2}{x_2} g(x_1, x_2) f_+(z) = \int_{\tau/x_1}^1 \! \frac{\mathrm{d}z}{z} g(x_1, \frac{\tau}{x_1z}) f_+(z)                                           \\
  \nonumber
  = \int_0^1 \! \frac{\mathrm{d}z}{z} g(x_1, \frac{\tau}{x_1 z}) f_+(z) - \int_0^{\tau/x_1} \! \frac{\mathrm{d}z}{z} g(x_1, \frac{\tau}{x_1 z}) f_+(z)                                      \\
  \nonumber
  = \int_0^1 \! \mathrm{d}z\, \pclosed{\frac{1}{z} g(x_1, \frac{\tau}{x_1 z}) - g(x_1, \frac{\tau}{x_1})} f(z) - \int_0^{\tau/x_1} \! \frac{\mathrm{d}z}{z} g(x_1, \frac{\tau}{x_1 z}) f(z) \\
  = \int_{\tau/x_1}^1 \! \frac{\mathrm{d}z}{z} \pclosed{g(x_1, \frac{\tau}{x_1 z}) - zg(x_1, \frac{\tau}{x_1})} f(z) - g(x_1, \frac{\tau}{x_1}) \integral[_0^{\tau/x_1}]{z} f(z)            \\
  = \int_0^1 \! \mathrm{d}z\, \pclosed{\frac{1}{z} g(x_1, \frac{\tau}{x_1 z}) - g(x_1, \frac{\tau}{x_1})} f(z) - \int_0^{\tau/x_1} \! \frac{\mathrm{d}z}{z} g(x_1, \frac{\tau}{x_1 z}) f(z) \\
  = \int_{\tau/x_1}^1 \! \frac{\mathrm{d}x_2}{x_2} \pclosed{g(x_1, x_2) - \frac{\tau}{x_1 x_2}g(x_1, \frac{\tau}{x_1})} f(z) - g(x_1, \frac{\tau}{x_1}) F(x_1),
\end{align}
where in the third line we have used that \(f_+(z) = f(z)\) for \(z < 1\) and in the last equality I switched integration variable back to \(x_2\).
Now, the only plus distributions that have cropped up thus far have been \(\plusdist{\frac{1}{1-z}}\) and \(\plusdist{\frac{\ln(1-z)}{1-z}}\), so the last integral in \cref{had:eq:plus_integral} can be done analytically to get
\begin{equation}
  F(x_1) \equiv \integral[_0^{\tau/x_1}]{z} f(z) = \begin{cases}
    -\ln(1-\frac{\tau}{x_1})              & \text{if } f(z) = \frac{1}{1-z}        \\
    -\frac{1}{2}\ln^2(1-\frac{\tau}{x_1}) & \text{if } f(z) = \frac{\ln(1-z)}{1-z}
  \end{cases}.
\end{equation}
Together, this reduces the integration over the parton momentum fractions into a 1-dimensional and a 2-dimensional integral, easing the computational power necessary to compute it numerically.
Writing the parton level differential cross-sections as
\begin{equation}
  \label{had:eq:parton_xsec_decomposition}
  \d[Q^2]{\hat\sigma_{ij}}(x_1, x_2) = \frac{\hat\sigma_c}{x_1 x_2} \cclosed{w_{ij}^\text{rad}(z) + w_{ij}^\text{soft}(z) \delta(1-z) + \sum_{f} w_{ij}^{f_+}(z) f_+(z)},
\end{equation}
where \(\hat\sigma_c\) is some common prefactor, we can factor the \(x_1, x_2\) integrals into
\begin{align}
  \label{had:eq:hadron_xsec_integral}
  \nonumber
  \d[Q^2]{\sigma}(Q^2) = & \hat\sigma_c \int_{\tau}^1 \! \frac{\mathrm{d}x_1}{x_1} f_i(x_1) \Bigg\{ f_j(\frac{\tau}{x_1})\biggl[w^\text{soft}_{ij}(1) - \sum_f w^{f_+}_{ij}(1) F(x_1)\biggr]                           \\
  \nonumber
                         & -  \sum_f f_j(\frac{\tau}{x_1}) w^{f_+}_{ij}(1) \int_{\tau/x_1}^1\!\frac{\mathrm{d}x_2}{x_2}\, \frac{\tau}{x_1 x_2} f(\frac{\tau}{x_1x_2})                                                  \\
                         & + \int_{\tau/x_1}^1\!\frac{\mathrm{d}x_2}{x_2}\, f_j(x_2) \biggl[w^\text{rad}_{ij}(\frac{\tau}{x_1 x_2}) + \sum_f w^{f_+}_{ij}(\frac{\tau}{x_1 x_2}) f(\frac{\tau}{x_1 x_2})\biggr] \Bigg\}
\end{align}
I note that the sum over \(f\) used here means a sum over the different plus-distributions that show up in the cross-section.


\subsection{Renormalised PDFs}
\label{had:subsec:renormalised_pdfs}
As mentioned earlier in \cref{pc:subsec:real_emission}, there remain collinear divergences in the partonic cross-sections we have calculated thus far.
These can be absorbed into the definition of the PDFs through a renormalisation procedure akin to that which is done when renormalising the fields and couplings.
Renormalisation conditions can be imposed, defining the parton distributions in such a way as to produce a finite observable cross-section at a reference energy \(\mu_F\) called the factorisation scale.
The trade-off is that the PDFs inherit scale-dependence as observables are computed at other energy levels.
This scale dependence is encapsulated in the DGLAP equations, due to Docshitzer, Gribov, Liapov, Altarelli and Parisi~\cite{DGLAP-D,DGLAP-GL,Altarelli:1977zs}.
Abbreviating the factorisation scale \(\mu_F \equiv \mu\), the DGLAP equations for the renormalised distribution PDFs of quarks/gluons \(f_{q/g}(x, \mu)\) read\footnote{A sum over the quark flavours \(q^\prime\) is implied here.}
\begin{equation}
  \label{had:eq:DGLAP}
  \mu \d[\mu]\ \begin{pmatrix} f_q(x, \mu) \\ f_g(x, \mu) \end{pmatrix} = \frac{\alpha_s}{\pi} \int_x^1 \! \frac{\mathrm{d}\xi}{\xi} \begin{bmatrix} P_{qq^\prime}(\frac{x}{\xi}) & P_{qg}(\frac{x}{\xi}) \\ P_{gq^\prime}(\frac{x}{\xi}) & P_{gg}(\frac{x}{\xi}) \end{bmatrix} \begin{pmatrix}
    f_{q^\prime}(\xi, \mu) \\ f_g(\xi, \mu)
  \end{pmatrix},
\end{equation}
where \(P_{ij}(z)\) are know as the DGLAP splitting functions, or sometimes just the splitting functions, and the sum goes over all quark flavours \(q^\prime\).
A heuristic interpretation \(P_{ij}(\frac{x}{\xi})\) is that they function in \cref{had:eq:DGLAP} to encapsulate the probability of an incoming parton \(j\) with momentum fraction \(\xi\) to radiate of some energy, leaving parton \(i\) with momentum fraction \(x\) instead.
To leading order, they are given by~\cite{Schwartz:2014sze}
\begin{subequations}
  \begin{align}
    P_{qq}(z) = & C_F \bclosed{\frac{1+z^2}{\plusdist{1-z}} + \frac{3}{2} \delta(1-z)},                             \\
    P_{qg}(z) = & T_F \bclosed{1 - 2z(1-z)},                                                                        \\
    P_{gq}(z) = & C_F \bclosed{\frac{z^2 + 2(1-z)}{z}},                                                             \\
    P_{gg}(z) = & 2C_A \bclosed{\frac{z}{\plusdist{1-z}} + \frac{1-z}{z} + z(1-z)} + \frac{\beta_0}{2} \delta(1-z),
  \end{align}
\end{subequations}
where \(\beta_0 = \frac{11}{3}C_A + \frac{4}{3} T_F n_f\).
Here, \(n_f\) is the number of light quark flavours, i.e. the quark flavours present in the hadron in question.
\medskip

Now, let us investigate concretely how the divergences are absorbed, and the scale dependence arises in the PDFs.
The idea is that when naively calculating the partonic cross-section, divergences can appear as a consequence of implicitly defining the factorisation between the soft and hard processes as we have.
The divergences in the hard scattering cross-section should cancel with soft effects in the hadron.
In some sense, the partonic cross-section we have is a \emph{bare} cross-section, and so are the PDFs.
Denoting the bare partonic cross-section for partons \(i, j\) as \(\d[Q^2]{\hat\sigma^\text{bare}_{ij}}\), we get the renormalised, finite cross-section \(\d[Q^2]{\hat\sigma^R_{ij}}\) by the rescaling
\begin{equation}
  \label{had:eq:renormalised_xsec}
  \d[Q^2]{\hat\sigma^\text{bare}_{ij}}(\eta) =\sum_{kl} \integral[_0^1]{z_1} \integral[_0^1]{z_2} Z_{ik}(z_1) Z_{jl}(z_2) \d[Q^2]{\hat\sigma^R_{kl}}(\frac{\eta}{z_1 z_2}),
\end{equation}
where \(Z_{ij}(z)\) are transition functions that hold the singular structure of \(\d[Q^2]{\hat\sigma^\text{bare}_{ij}}\) and work akin to the renormalisation constants \(Z\) used for the fields and couplings in \cref{qft:subsec:counterterms}.
These transition functions also rescale the PDFs through a convolution, defining the renormalised PDFs as
\begin{equation}
  \label{had:eq:pdf_renormalised}
  f_i^R(x) = \sum_{j} \integral[_0^1]{\xi \mathrm{d}z} \delta(x - \xi z) f_j^\text{bare}(\xi) Z_{ji}(z) = \sum_{j}\int_x^1 \! \frac{\mathrm{d}z}{z} f_j^\text{bare}(\frac{x}{z}) Z_{ji}(z).
\end{equation}
The non-singular hadronic cross-section is then given by the finite renormalised PDFs and cross-section as
\begin{equation}
  \label{had:eq:dsigma_dQ2_renormalised}
  \d[Q^2]{\sigma}(\tau) = \sum_ {ij} \integral[_\tau^1]{x_1}\!\integral[_{\tau/x_1}^1]{x_2} f_i^R(x_1) f_j^R(x_2) \d[Q^2]{\hat\sigma^R_{ij}}(\frac{\tau}{x_1 x_2}).
\end{equation}
The relation to the bare quantities can be seen by inserting the renormalised PDF definition \cref{had:eq:pdf_renormalised} into \cref{had:eq:dsigma_dQ2_renormalised}:
\begin{align}
  \nonumber
  \d[Q^2]{\sigma}(\tau) = & \sum_ {ij} \integral[_\tau^1]{x_1}\!\integral[_{\tau/x_1}^1]{x_2} \sum_{kl} \integral[_0^1]{\xi_1\mathrm{d}z_1} \delta(x_1 - \xi_1 z_1) Z_{ki}(z_1) f_k^\text{bare}(\xi_1)          \\
  \nonumber
                          & \times \integral[_0^1]{\xi_2\mathrm{d}z_2} \delta(x_2 - \xi_2 z_2) Z_{lj}(z_2) f_l^\text{bare}(\xi_2) \d[Q^2]{\hat\sigma^R_{ij}}(\frac{\tau}{x_1 x_2})                              \\
  \nonumber
  =                       & \sum_{ijkl} \integral[_0^1]{\xi_1 \mathrm{d}\xi_2 \mathrm{d}z_1 \mathrm{d}z_2} Z_{ki}(z_1) Z_{lj}(z_2) f_k^\text{bare}(\xi_1) f_l^\text{bare}(\xi_2)                                \\
  \nonumber
                          & \times \d[Q^2]{\hat\sigma_{ij}^R}(\frac{\tau}{z_1 z_2 \xi_1 \xi_2}) \theta(\xi_2 z_2 - \frac{\tau}{\xi_1 z_1}) \theta(\xi_1 z_1 - \tau)                                             \\
  \nonumber
  =                       & \sum_{kl} \integral[_\tau^1]{\xi_1} \integral[_{\tau/\xi_1}^1]{\xi_2} f_k^\text{bare}(\xi_1) f_l^\text{bare}(\xi_2)                                                                 \\
  \nonumber
                          & \times \integral[_{\tau/(\xi_1 \xi_2)}^1]{z_1} \integral[_{\tau/(z_1 \xi_1 \xi_2)}^1]{z_2} \sum_{ij} Z_{ki}(z_1) Z_{lj}(z_2) \d[Q^2]{\hat\sigma_{kl}^R}(z_1z_2 \xi_1\xi_2 S)        \\
  =                       & \sum_{kl} \integral[_\tau^1]{\xi_1} \integral[_{\tau/\xi_1}^1]{\xi_2} f_k^\text{bare}(\xi_1) f_l^\text{bare}(\xi_2) \d[Q^2]{\hat\sigma^\text{bare}_{kl}}(\frac{\tau}{\xi_1 \xi_2}),
\end{align}
where in the second equality, I changed the order of integration and integrated over \(x_{1,2}\)
This shows that if we take the partonic cross-section we have worked with so far to be the bare cross-section integrated over bare PDFs in \cref{had:eq:dsigma_dQ2}, then the restatement \cref{had:eq:dsigma_dQ2_renormalised} with a finite cross-section and finite PDFs is equivalent.
In other words, the divergences cancel between the partonic cross-section and the PDFs.
\medskip

What remains is to find the insertion functions \(Z_{ij}\) to get the renormalised, finite quantities.
Let us assume that we can expand the bare and renormalised cross-sections and the insertion functions as a power series in \(\frac{\alpha_s}{2\pi}\).
Furthermore, since the LO cross-section is already finite, we can choose the LO insertion function to be \(\delta_{ij} \delta(1-z)\) leaving the LO contribution unchanged.
\begin{subequations}
  \begin{eqnarray}
    \d[Q^2]{\hat\sigma^\text{bare}_{ij}}(z) = & \sigma^\text{LO}_{ij}(z) + \frac{\alpha_s}{2\pi} \sigma^\text{NLO}_{ij}(z) + \ldots, \\
    \d[Q^2]{\hat\sigma^R_{ij}}(z) = & \bar\sigma^\text{LO}_{ij}(z) + \frac{\alpha_s}{2\pi} \bar\sigma^\text{NLO}_{ij}(z) + \ldots, \\
    Z_{ij}(z) = & \delta_{ij} \delta(1-z) + \frac{\alpha_s}{2\pi} Z_{ij}^\text{NLO}(z) + \ldots
  \end{eqnarray}
\end{subequations}
Expanding both sides of \cref{had:eq:renormalised_xsec} we have to NLO
\begin{align}
  \nonumber
  \sigma^\text{LO}_{ij}(\eta)+ \frac{\alpha}{2\pi} \sigma^\text{NLO}_{ij}(\eta) = & \bar\sigma^\text{LO}_{ij}(\eta) + \frac{\alpha}{2\pi}                                                                                                                      \Bigl\{ \bar\sigma^\text{NLO}_{ij}(\eta) + \sum_{l}\integral[_0^1]{z_2} Z_{jl}^\text{NLO}(z_2) \bar\sigma^\text{LO}_{il}(\frac{\eta}{z_2}) \\
                                                                                  & + \sum_{k}\integral[_0^1]{z_1} Z_{ik}^\text{NLO}(z_1) \bar\sigma^\text{LO}_{kj}(\frac{\eta}{z_1}) \Bigr\}.
\end{align}
As anticipated, we see that the renormalised partonic LO cross-section is the same as the bare one.
However, the renormalised NLO cross-section is given by
\begin{equation}
  \bar\sigma^\text{NLO}_{ij}(\eta) = \sigma^\text{NLO}_{ij}(\eta) - \sum_{k} \integral[_0^1]{z} \cclosed{Z^\text{NLO}_{jk}(z) \bar\sigma^\text{LO}_{ik}(\frac{\eta}{z}) + Z^\text{NLO}_{ik}(z) \bar\sigma^\text{LO}_{kj}(\frac{\eta}{z})}.
\end{equation}
The similarity between the insertion functions and the counterterms in field/coupling renormalisation is more evident here.
There are many schemes for defining the insertion functions explicitly, but in this thesis, I will be using the \MSbar\ scheme, in which they are defined such that they only remove terms proportional to \(\frac{1}{\bar\epsilon} = \frac{1}{\epsilon} - \gamma_E - \ln 4\pi\).
Choice of scheme affects the PDFs through \cref{had:eq:pdf_renormalised}, and as such, care must be taken when using an explicit PDF set defined in a certain renormalisation scheme.
In the next chapter, I will be using numerical values for the PDFs taken from \verb|LHAPDF|~\cite{LHAPDF}, where they have been defined in the \MSbar\ scheme, so it is most convenient to do so here too.


\section{Total Hadronic Cross-Section Result}
\label{had:sec:total_cross_section}
We can now put together the results from the previous section to get the full inclusive hadronic cross-section for neutralino pair production to NLO\@.
To leading order, we have the partonic contributions
\begin{equation}
  \d[Q^2]{\hat\sigma^\text{LO}_{q\bar{q}}} = \hat\sigma^0 \frac{\delta(1-z)}{\hat{s}}.
\end{equation}
In the language of \cref{had:eq:parton_xsec_decomposition}, we can factor out \(\hat\sigma_c = \hat\sigma^0\), leaving
\begin{equation}
  w_{q\bar{q}}^\text{soft}(z) = \frac{1}{S},
\end{equation}
to LO\@.
As a reminder, \(\hat\sigma^0\) is the partonic LO cross-section defined in \cref{pc:eq:born}.

Recalling the NLO contribution \cref{pc:eq:all_NLO_contributions} we have
\begin{subequations}
  \begin{align}
    \d[Q^2]{\hat\sigma_{q\bar{q}}^\text{NLO}} = & \d[Q^2]{\hat\sigma_v^\text{non-SUSY}} + \d[Q^2]{\hat\sigma_v^\text{SUSY}} + \d[Q^2]{\hat\sigma_{r,g}}, \\
    \d[Q^2]{\hat\sigma_{qg}^\text{NLO}} =       & \d[Q^2]{\hat\sigma_{r,q}},                                                                             \\
    \d[Q^2]{\hat\sigma_{\bar{q}g}^\text{NLO}} = & \d[Q^2]{\hat\sigma_{r,\bar{q}}}.
  \end{align}
\end{subequations}
From the explicit expressions for these contributions \cref{pc:eq:nonSUSY_sigma_v,pc:eq:SUSY_sigma_v,pc:eq:sigma_rg,pc:eq:sigma_rq}, we can extract the prefactor \(\hat\sigma_c = \hat\sigma^0\) and get the partonic functions\footnote{In the higgsino scenario we assume for these NLO contributions \(\hat\sigma^0 = \hat\sigma_B F_{Z}^{\prime\prime}\).}
\begin{subequations}
  \label{had:eq:w-functions_unrenormalised}
  \begin{eqnarray}
    w_{q\bar{q}}^\text{soft}(z) & = & \frac{\alpha_s C_F}{\pi S}  P(\epsilon) \Bigl\{-\frac{3}{2\epsilon} + \frac{\pi^2}{3} - 4 + \tilde{C}_Z^{\prime\prime}\Bigr\}, \\
    w_{q\bar{q}}^\text{rad}(z) & = & -\frac{\alpha_s C_F}{\pi S}  P(\epsilon) \frac{(1+z^2)\ln z}{1-z}, \\
    w_{q\bar{q}}^{1/(1-z)_+} & = & -\frac{\alpha_s C_F}{\pi S}  P(\epsilon) \frac{1}{\epsilon} (1+z^2),     \\
    w_{q\bar{q}}^{(\ln(1-z)/(1-z))_+} & = & \frac{\alpha_s C_F}{\pi S}  P(\epsilon) 2(1+z^2),\\
    \nonumber
    w_{qg}^\text{rad}(z) & = & \frac{\alpha_s T_F}{2\pi S} P(\epsilon) \biggl\{-\frac{1}{\epsilon}(1 - 2z(1-z))     \\
    && + \frac{1}{2}(1 + 6z - 7z^2) + (1-2z(1-z)) \ln\frac{(1-z)^2}{z}\biggl\},    \\
    w_{\bar{q}g}^\text{rad}(z) & = & w_{qg}^\text{rad}(z).
  \end{eqnarray}
\end{subequations}

Renormalising the NLO contributions in the \MSbar\ scheme, we choose the insertion functions such that the \(\frac{1}{\epsilon}\)-poles in \cref{had:eq:w-functions_unrenormalised} are removed, together with the canonical \MSbar\ factors \(-\gamma_E + \ln 4\pi\).
In practice, when taking the (now non-diverging) limit \(\epsilon\) we replace \(\frac{1}{\epsilon} P(\epsilon) \to \ln\frac{\mu_F^2}{Q^2}\) and \(P(\epsilon) \to 1\).
As a reminder, we had
\begin{equation}
  \label{had:eq:P_of_epsilon}
  P(\epsilon) = 1 + \pclosed{\ln\frac{\mu_F^2}{Q^2} - \gamma_E + \ln 4\pi - 1}\epsilon + \mathcal{O}(\epsilon^2),
\end{equation}
so there is actually a factor of \(-\epsilon\) in \(P(\epsilon)\) that is not accounted for in the pole terms in the ordinary \MSbar\ scheme.
These can be removed by redefining the factorisation scale \(\mu_F \to \exp{\sfrac{1}{2}} \mu_F\), which can be seen by inserting into \cref{had:eq:P_of_epsilon}.
In practice, this is what has been done, and the factorisation scale dependence of the PDFs has been defined accordingly.

Denoting the renormalised partonic cross-section functions with a bar, we get
\begin{subequations}
  \label{had:eq:w-functions_renormalised}
  \begin{eqnarray}
    \bar{w}_{q\bar{q}}^\text{soft}(z) & = & \frac{\alpha_s C_F}{\pi} \frac{\hat\sigma_B}{S} F_Z^{\prime\prime} \Bigl\{-\frac{3}{2} \ln\frac{\mu_F^2}{Q^2} + \frac{\pi^2}{3} - 4 + \tilde{C}_Z^{\prime\prime}\Bigr\}, \\
    \bar{w}_{q\bar{q}}^\text{rad}(z) & = & -\frac{\alpha_s C_F}{\pi} \frac{\hat\sigma_B}{S} F_Z^{\prime\prime} \frac{(1+z^2)\ln z}{1-z}, \\
    \bar{w}_{q\bar{q}}^{1/(1-z)_+} & = & -\frac{\alpha_s C_F}{\pi} \frac{\hat\sigma_B}{S} F_Z^{\prime\prime} \ln\frac{\mu_F^2}{Q^2} (1+z^2),     \\
    \bar{w}_{q\bar{q}}^{(\ln(1-z)/(1-z))_+} & = & \frac{\alpha_s C_F}{\pi} \frac{\hat\sigma_B}{S} F_Z^{\prime\prime} 2(1+z^2),\\
    \nonumber
    \bar{w}_{qg}^\text{rad}(z) & = & \frac{\alpha_s T_F}{2\pi} \frac{\hat\sigma_B}{S} F_{Z}^{\prime\prime} \biggl\{-\ln\frac{\mu_F^2}{Q^2}(1 - 2z(1-z))     \\
    && + \frac{1}{2}(1 + 6z - 7z^2) + (1-2z(1-z)) \ln\frac{(1-z)^2}{z}\biggl\},    \\
    \bar{w}_{\bar{q}g}^\text{rad}(z) & = & \bar{w}_{qg}^\text{rad}(z).
  \end{eqnarray}
\end{subequations}
These functions are then inserted into \cref{had:eq:hadron_xsec_integral} for numeric integration, which I will perform in the next chapter.
To get the full, non-differential cross-section, we simply integrate over \(Q^2\) with the integration limits given in \cref{pc:eq:Q2_limits}.
% \begin{subequations}
%   \begin{align}
%     \sum_k \integral[_0^1]{z} Z_{qk}^{\text{NLO}}(z) \bar\sigma_{\bar{q}k}^{\text{LO}}(\frac{\eta}{z}) = -\frac{\hat\sigma^0}{\hat{s}} \frac{\alpha_s C_F}{\pi} \pclosed{\frac{3}{2}\delta(1-z) + \frac{1+z^2}{\plusdist{1-z}}} \frac{1}{\bar\epsilon}, \\
%     \sum_k \integral[_0^1]{z} Z_{gk}^{\text{NLO}}(z) \bar\sigma_{qk}^{\text{LO}}(\frac{\eta}{z}) = -\frac{\hat\sigma^0}{\hat{s}} \frac{\alpha_s T_F}{2\pi} \frac{1}{\bar\epsilon} (1-2z(1-z)),
%   \end{align}
% \end{subequations}
% which is realisable with
% \begin{subequations}
%   \begin{align}
%     Z_{qk}^\text{NLO}(z) = & -\frac{1}{z} \delta_{qk} \frac{\alpha_s C_F}{\pi} 2P_{qq}(z) \frac{1}{\bar\epsilon} \\
%     Z_{gk}^\text{NLO}(z) = & -\frac{1}{z} \delta_{gk} \frac{\alpha_s T_F}{2\pi} P_{qg}(z) \frac{1}{\bar\epsilon}
%   \end{align}
% \end{subequations}


\ifSubfilesClassLoaded{%
  \bibliography{references}{}
  \bibliographystyle{style/JHEP}
}{}

\end{document}

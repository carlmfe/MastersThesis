\documentclass[../main.tex]{subfiles}


\begin{document}
\chapter{Proton--Proton Electroweakino Pair Production}

\begin{TODO}
  \item Describe parton model.
\end{TODO}

\section{The Parton Model and pdf's}
So far, we have worked with the parton level cross-section, figuring out the contribution of the individual constituents of a proton to the cross-section for our final state.
Now these do not individually result in any observable, as the partons are confined to the proton, and can therefore not be singled out in an experiment.
To get an observable quantity comparable to experiment, we must sew the individual contributions together.
This is done with the \emph{parton model}, where scattering interactions with the proton is modelled with the interaction of free constituent particles inside.
The parton model builds on the concept of \emph{factorisation} which, owing to the weakening of the QCD coupling at high-energies, divides interactions with colour-neutral particles into a high-energy and a low-energy regime that are treated separately.
The low-energy regime dictates that partons each carry a fraction of the total momentum of the proton, and the probability of encountering a given parton with said momentum fraction.

\subsection{Hadronic kinematics}
Consider the scattering of two protons with momenta \(P_1^\mu\) and \(P_2^\mu\) respectively into a set of final state particles \(\chi, \chi^\prime, X\) where \(X\) is some collection of unlabelled particles.
\cref{had:tab:had_kinematic_variables} lists the definitions of kinematic variables at the hadronic level and their relation of the partonic kinematic variables defined in \cref{chap:part:parton_calculation}.
I define the centre-of-mass energy \(S \equiv (P_1 + P_2)^2\).
The cross-section for a given process is then given in terms of the cross-section of two partons \(i, j\) with momenta \(k_i = x_1 P_1\) and \(k_j = x_2 P_2\) where \(x_1, x_2 \in [0, 1]\) are the respective fractions of the proton momenta the partons carry.
The \emph{hadronic cross-section} differential in the squared mass \(Q^2\) of two final state particles \(\chi\) and \(\chi^\prime\) is then given by
\begin{align}
  \label{had:eq:dsigma_dQ2}
  \nonumber
  \d[Q^2]{\sigma}(PP \to \chi \chi^\prime + X) = \sum_ {ij} \integral[_0^1]{x_1}\!\integral[_0^1]{x_2} \theta\pclosed{\hat{s} - Q^2} f_i(x_1) f_j(x_2) \d[Q^2]{\hat\sigma}(ij \to \chi\chi^\prime + X) \\
  = \sum_ {ij} \integral[_\tau^1]{x_1}\!\integral[_{\tau/x_1}^1]{x_2} f_i(x_1) f_j(x_2) \d[Q^2]{\hat\sigma}(ij \to \chi\chi^\prime + X).
\end{align}
The Heaviside function \(\theta\pclosed{\hat{s}-Q^2} = \theta\pclosed{x_1 x_2 - \tau}\) ensures that there is enough energy between the scattering partons to produce the final state \(\chi \chi^\prime\)-pair with centre-of-mass energy \(Q^2 = \tau S\).

\begin{table}[ht!]
  \centering
  \begin{tabular}{|c|c|}
    \hline
    Partonic variable & Definition in terms of hadronic variables \\
    \hline
    \(k_i^\mu\)       & \(x_1 P_1^\mu\)                           \\
    \(k_j^\mu\)       & \(x_2 P_2^\mu\)                           \\
    \(\hat{s}\)       & \(x_1 x_2 S\)                             \\
    \(z\)             & \(\frac{\tau}{x_1x_2}\)                   \\
    \hline
  \end{tabular}
  \caption{List of relations between hadronic and partonic kinematic variables.}
  \label{had:tab:had_kinematic_variables}
\end{table}



\subsection{Integration over pdf's}
Practically, the two-dimensional integration over the parton momentum fractions \(x_1, x_2\) can be alleviated by the fact that partonic cross-section contains terms proportional to either \(\delta(1-z)\) or plus distributions \(f_+(z)\) as we have seen in \cref{chap:part:parton_calculation}.
Let us consider these types of integrals in some generality.
Let \(g(x_1, x_2)\) be some function of \(x_1, x_2\), consider the integral
\begin{equation}
  \int_{\tau/x_1}^1 \! \frac{\mathrm{d}x_2}{x_2} g(x_1, x_2) \delta(1-z).
\end{equation}
Switching variables to \(z = \frac{\tau}{x_1 x_2}\) and keeping \(x_1\) constant yields
\begin{align}
  \int_{\tau/x_1}^1 \! \frac{\mathrm{d}z}{z}  g(x_1, \frac{\tau}{x_1 z}) \delta(1-z) = g(x_1, \frac{\tau}{x_1}).
\end{align}

The plus-distributions are somewhat more complicated.
Keeping in mind their definition
\begin{equation}
  \integral[_0^1]{z} g(z) f_+(z) = \integral[_0^1]{z} (g(z)-g(1)) f(z),
\end{equation}
we have that
\begin{align}
  \label{had:eq:plus_integral}
  \nonumber
  \int_{\tau/x_1}^1 \! \frac{\mathrm{d}x_2}{x_2} g(x_1, x_2) f_+(z) = \int_{\tau/x_1}^1 \! \frac{\mathrm{d}z}{z} g(x_1, \frac{\tau}{x_1z}) f_+(z)                                           \\
  \nonumber
  = \int_0^1 \! \frac{\mathrm{d}z}{z} g(x_1, \frac{\tau}{x_1 z}) f_+(z) - \int_0^{\tau/x_1} \! \frac{\mathrm{d}z}{z} g(x_1, \frac{\tau}{x_1 z}) f_+(z)                                      \\
  \nonumber
  = \int_0^1 \! \mathrm{d}z\, \pclosed{\frac{1}{z} g(x_1, \frac{\tau}{x_1 z}) - g(x_1, \frac{\tau}{x_1})} f(z) - \int_0^{\tau/x_1} \! \frac{\mathrm{d}z}{z} g(x_1, \frac{\tau}{x_1 z}) f(z) \\
  = \int_{\tau/x_1}^1 \! \frac{\mathrm{d}z}{z} \pclosed{g(x_1, \frac{\tau}{x_1 z}) - zg(x_1, \frac{\tau}{x_1})} f(z) - g(x_1, \frac{\tau}{x_1}) \integral[_0^{\tau/x_1}]{z} f(z),
\end{align}
where in the third line we have used that \(f_+(z) = f(z)\) for \(z < 1\).
Now, the only plus distribution that have cropped up thus far have been \(\plusdist{\frac{1}{1-z}}\) and \(\plusdist{\frac{\ln(1-z)}{1-z}}\), so the last integral in \cref{had:eq:plus_integral} can be done analytically.
\begin{temporary}
  \margincomment{Should I define this like Tore?}
  Defining
  \begin{equation}
    F(x_1) \equiv \integral[_0^{\tau/x_1}]{z} f(z) = \begin{cases}
      -\ln(1-\frac{\tau}{x_1})              & \text{if } f(z) = \frac{1}{1-z}        \\
      -\frac{1}{2}\ln^2(1-\frac{\tau}{x_1}) & \text{if } f(z) = \frac{\ln(1-z)}{1-z}
    \end{cases}
  \end{equation}
\end{temporary}

Together, this reduces the integration over the parton momentum fractions into a 1-dimensional and a 2-dimensional integral, easing on the computational power necessary to compute it numerically.
Writing the parton level differential cross-sections as
\begin{equation}
  \d[Q^2]{\hat\sigma_{ij}}(x_1, x_2) = \frac{1}{x_1 x_2} \cclosed{w_{ij}^\text{rad}(z) + w_{ij}^\text{soft}(z) \delta(1-z) + \sum_{f} w_{ij}^{f_+}(z) f_+(z)},
\end{equation}
we can factor the \(x_1, x_2\) integrals into
\begin{align}
  \nonumber
  \d[Q^2]{\sigma}(Q^2) = \int_{\tau}^1 \! \frac{\mathrm{d}x_1}{x_1} f_i(x_1) \Bigg\{ f_j(\frac{\tau}{x_1})w^\text{soft}_{ij}(1) + \sum_f w^{f_+}_{ij}(1) F(x_1) \\
  + \int_{\tau/x_1}^1\!\frac{\mathrm{d}x_2}{x_2}\, \cclosed{f_j(x_2) w^\text{rad}_{ij}(\frac{\tau}{x_1 x_2}) +  \sum_f \pclosed{f_j(x_2) w^{f_+}_{ij}(\frac{\tau}{x_1 x_2}) - f_j(\frac{\tau}{x_1}) \frac{\tau}{x_1 x_2} w^{f_+}_{ij}(1)} f(\frac{\tau}{x_1 x_2})} \bigg\}
\end{align}

\ifSubfilesClassLoaded{%
  \bibliography{references}{}
  \bibliographystyle{style/JHEP}
}{}

\end{document}

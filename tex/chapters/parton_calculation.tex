\documentclass[../main.tex]{subfiles}

\begin{document}

\chapter{Electroweakino Pair Production at Parton Level}
\label{chap:part}

\begin{TODO}
  \item Formulate a section on the dipole formalism used in Debove et al.\ and make a comparison.
  \item Mention Mathematica scripts.
\end{TODO}

\section{Phase Space and Kinematics in Scattering Processes}
To start off, it will be useful to introduce some procedure for going forward
in the phase space of an inclusive \(2\to 2(+1)\) scattering process.
The phase space of 2-body and 3-body final states are quite different as there are
more degrees of freedom in the 3-body final state.
In the end, these extra degrees of freedom will be need to be integrated over to make an additive
comparison between the 2-body and 3-body processes, however, exactly how we
choose to parametrise and subsequently integrate over the extra degrees of
freedom can matter quite a bit.
\medskip

\todo{Quickly go over shorthand notation, and neutralino pair production etc.}
Furthermore, I will use the shorthand \(m_{i,j}\) for the neutralino masses \(p_{i,j}^2\) and \(m_A\) for squark masses \(m_{\tilde{q}_A}\).

To start out, let us count the degrees
of freedom of a scattering problem involving \(N\) four-momenta
\(p_{i=1,\ldots,N}\).
Assuming our end result to be Lorentz invariant, there are \(N(N+1)/2\) different scalar products that can be produced using \(N\) different four-momenta.
Momentum conservation allows us to eliminate one momentum, such that we have \(N(N-1)/2\) possible scalar products.
Denoting the scalar products by \(m^2_{ij} \equiv (p_i + p_j)^2\) for \(j \neq i\), and \(m^2_i \equiv p_i^2\), we can find a relation between scalar products by using momentum conservation.
\begin{align}
  \label{pc:eq:generalised_mandelstam_relation}
  \nonumber
  m^2_{ij} & = \pclosed{p_i - \sum_{k\neq j} p_k}^2 =
  \pclosed{\sum_{k\neq i,j} p_k}^2 = \sum_{k\neq i,j} \sum_{l\neq i,j}
  p_k\cdot
  p_l
  \\
  \nonumber
           & = \sum_{k\neq i,j} \sum_{l\neq i,j,k} \frac{m^2_{kl} - m^2_k
  - m^2_l}{2} + \sum_{k\neq i,j} m^2_k
  \\
  \nonumber
           & = \sum_{k\neq i,j} \sum_{\substack{l\neq i,j
  \\l>k}} m^2_{kl}
  - \frac{1}{2} \sum_{k\neq i, j} (N-3) m^2_k - \frac{1}{2} \sum_{l\neq
    i,j}
  (N-3) m^2_l + \sum_{k\neq i,j} m^2_k
  \\
           & = \sum_{k\neq i,j} \sum_{\substack{l\neq i,j
  \\l>k}} m^2_{kl}
  - (N-4) \sum_{k\neq i, j} m^2_k.
\end{align}
\comment{This little generalised relation might not be immediately necessary\ldots}

To count the degrees of freedom in an \(N\)-body final state, we need to classify how many scalar products need to be specified for every scalar product to be defined.
We assign the \(N\) scalar products \(m^2_i\) to the invariant masses of the incoming and outgoing particles, thereby not counting them as kinematic degrees of freedom, which leaves us with \(n_\text{dof} = \frac{N(N-3)}{2}\) degrees of freedom.\footnote{I note that we
  often consider the invariant mass of the incoming bodies to be fixed, which would reduce our degrees of freedom by one.}
This means that in a \(2\to 2\) process, we must specify 2 kinematic variables, and in a \(2\to 3\) process we must specify 5.
For instance, in the \(2\to 2\) case, the canonical Mandelstam variables \(s, t, u\) can be used together with the restriction that \(s+t+u = \sum_i m_i^2\).

For reference, I give the general formula for the Lorentz invariant differential phase space for a process of \(n_i\) particles with four-momenta \(k_i\) going to \(n_f\) particles with four-momenta \(p_j\) in \(d\) space-time dimensions~\cite{Schwartz:2014sze}:
\begin{equation}
  \label{pc:eq:dPi_d}
  \mathrm{d}\Pi_{n_i \to n_f} =  (2\pi)^d \delta^d(\sum_{i=1}^{n_i} k_i - \sum_{j=1}^{n_f} p_j) \prod_{j=1}^{n_f} \frac{\mathrm{d}^{d-1}\vec{p}_j}{(2\pi)^{d-1}} \frac{1}{2E_j},
\end{equation}
where the particles with four-momenta \(p_j\) are understood to be  on-shell, i.e. \(E_j^2 = m_j^2 + \vec{p}_j^2\), where \(m_j\) is the mass of the particle.
In the case of four space-time dimensions, this reduces to
\begin{equation}
  \label{pc:eq:dPi_4}
  \mathrm{d}\Pi_{n_i \to n_f} =  (2\pi)^4 \delta^4(\sum_{i=1}^{n_i} k_i - \sum_{j=1}^{n_f} p_j) \prod_{j=1}^{n_f} \frac{\mathrm{d}^{3}\vec{p}_j}{(2\pi)^{3}} \frac{1}{2E_j}.
\end{equation}

\subsection{2-body Phase Space}
From \cref{pc:eq:dPi_d} above, the Lorentz invariant phase space differential for a 2-body final state with
four-momenta \(p_i, p_j\) in \(d\) dimensions is
\begin{equation}
  \mathrm{d}\Pi_{2\to 2} = \pclosed{2\pi}^d \delta^d\pclosed{P - p_i -
    p_j} \frac{\mathrm{d}^{d-1} \vec{p}_i}{\pclosed{2\pi}^{d-1}}
  \frac{1}{2E_i}
  \frac{\mathrm{d}^{d-1} \vec{p}_j}{\pclosed{2\pi}^{d-1}} \frac{1}{2E_j}.
\end{equation}
Going to the centre-of-mass frame of the incoming particles, we have \(P\^\mu =
\pclosed{\sqrt{s}, 0, 0, 0}\), where \(s \equiv P^2\).
This allows us to integrate over the spatial part of Dirac delta-function to arrive at
\begin{equation}
  \mathrm{d}\Pi_{2\to 2} = \frac{1}{\pclosed{2\pi}^{d-2}}
  \mathrm{d}^{d-1} \vec{p} \frac{1}{4E_i E_j} \delta\pclosed{\sqrt{s} -
    E(p, m_i)
    - E(p, m_j)},
\end{equation}
where the \(E(p, m) = \sqrt{p^2 + m^2}\).
We can write out the differential of the spatial component of \(p_i\) in
spherical coordinates as \(\mathrm{d}^{d-1} \vec{p} = \mathrm{d}\Omega_{d-1}
\mathrm{d}p \, p^{d-2} = \mathrm{d}\Omega_{d-2} \sin^{d-3}\theta \,
\mathrm{d}\theta \, \mathrm{d}p \, p^{d-2}\).
As a \(2\to 2\) process is restricted to planar motion, we can always go to a
frame of reference such that any amplitude we calculate will not be dependent
on the spatial angles \(\mathrm{d} \Omega_{d-2}\), allowing us to integrate
over them using that \(\integral{\Omega_{d-2}} = 2 \pi^{\frac{d-2}{2}}
\frac{1}{\Gamma\pclosed{\frac{d-2}{2}}}\) to get
\begin{equation}
  \label{pc:eq:dPi_dtheta}
  \mathrm{d}\Pi_{2\to 2} = \frac{1}{\pclosed{4\pi}^{\frac{d-2}{2}}}
  \frac{1}{\Gamma\pclosed{\frac{d-2}{2}}} \frac{p^{d-3}}{2 \sqrt{s}}
  \sin^{d-3}\theta \,\mathrm{d}\theta,
\end{equation}
where we understand the momentum to be given by
\begin{equation}
  \label{pc:eq:p}
  p = \frac{\sqrt{\lambda(s, m_i^2, m_j^2)}}{2\sqrt{s}}
\end{equation}
The \(\lambda\) function is known as the Källén function and is defined as
\begin{equation}
  \label{pc:eq:kallen}
  \lambda(x, y, z) = x^2 + y^2 + z^2  - 2xy - 2xz - 2yz.
\end{equation}
In \(d=4\) dimensions, it is often convenient to change to the Mandelstam
variable \(t\), which for massless initial state particles becomes
\(t=\frac{1}{2}\pclosed{-s+m_i^2+m_j^2+\sqrt{\lambda(s,m_i^2,m_j^2)}\cos\theta}\).
Making the change of variable, the differential phase space reduces to
\begin{equation}
  \label{pc:eq:dPi_dt}
  \evalat{\mathrm{d}\Pi_{2\to 2}}{d=4} = \frac{1}{8\pi s} \,\mathrm{d}t
\end{equation}


\subsection{3-body Phase Space}
\todo{Fill out an introduction here.}\\
The differential Lorentz invariant phase space for a 3-body final state with
four-momenta \(p_i, p_j, k\), where \(k^2=0\) in \(d\) dimensions is
\begin{equation}
  \mathrm{d}\Pi_{2\to 3} = (2\pi)^d \delta^d(P - p_i - p_j - k)
  \frac{\mathrm{d}^{d-1} \vec{p}_i}{(2\pi)^{d-1}} \frac{1}{2E_i}
  \frac{\mathrm{d}^{d-1} \vec{p}_j}{(2\pi)^{d-1}} \frac{1}{2E_j}
  \frac{\mathrm{d}^{d-1} \vec{k}}{(2\pi)^{d-1}} \frac{1}{2\omega}.
\end{equation}
First, it will be useful to write out the differential in \(\vec{k}\) in
spherical coordinates where it reads \(\mathrm{d}^{d-1}\vec{k} = \omega^{d-2}
\mathrm{d}\Omega_{d-1} \mathrm{d}\omega\).
The differentials in \(\vec{p}_{i/j}\) together with the delta-function are
easier to compute in the centre-of-mass frame of the neutralinos where we have
\(P - k = \pclosed{Q, 0, 0, 0}\).
This leaves
\begin{equation}
  \mathrm{d}\Pi_{2\to 3} = \frac{1}{8} \frac{1}{\pclosed{2\pi}^{2d-3}}
  \delta(Q - E_i^\star - E_j^\star) \delta^{d-1}(\vec{p}^\star_i +
  \vec{p}_j^\star) \frac{\omega^{d-3}}{E_i^\star E_j^\star}
  \mathrm{d}^{d-1}\vec{p}_i^\star\, \mathrm{d}^{d-1}\vec{p}_j^\star\,
  \mathrm{d}\Omega_{d-1}\, \mathrm{d}\omega,
\end{equation}
where the stars denote quantities calculated in the aforementioned reference
frame.
Integrating trivially over \(\vec{p}_j^\star\) using the delta-function, and
using polar coordinates \(\mathrm{d}^{d-1}\vec{p}_i =
\mathrm{d}\Omega_{d-1}^\star \, \mathrm{d}\!\abs{\vec{p}_i^\star}
\abs{\vec{p}_i^\star}^{d-2}\) to integrate over \(\delta\pclosed{Q - E_i^\star
  - E_j^\star}\), we get
\begin{equation}
  \mathrm{d}\Pi_{2\to 3} = \frac{1}{\pclosed{2\pi}^{2d-3}}
  \frac{\omega^{d-3} \abs{\vec{p}_i^\star}^{d-3}}{8Q}
  \mathrm{d}\Omega_{d-1}^\star\, \mathrm{d}\Omega_{d-1}\,
  \mathrm{d}\omega.
\end{equation}
Here, we understand the magnitude of the three-momenta to be given by
\(\abs{\vec{p}_i^\star} = \frac{\sqrt{\lambda(Q^2, m_i^2, m_j^2)}}{2Q}\) and
\(\omega = \frac{s-Q^2}{2\sqrt{s}}\).'
It will also be useful to make a change of integration variable to \(Q^2\),
leaving us finally with
\begin{equation}
  \mathrm{d}\Pi_{2\to 3} = \frac{1}{\pclosed{2\pi}^{2d-3}}
  \frac{\omega^{d-3} \abs{\vec{p}_i^\star}^{d-3}}{16Q\sqrt{s}}
  \mathrm{d}\Omega_{d-1}^\star\, \mathrm{d}\Omega_{d-1}\, \mathrm{d}Q^2.
\end{equation}

\todo{Comment on integration boundaries.}\\
\begin{temporary}
  \[(m_i + m_j)^2 \leq Q^2 \leq s\]
\end{temporary}

With two initial state momenta, the amplitude will be independent of the
azimuthal angle in the centre-of-mass frame of the initial partons. This lets
us integrate over it for a factor of \(2\pi\).

\begin{equation}
  \mathrm{d}\Pi_{2\to 3} = \frac{1}{\pclosed{2\pi}^{2d-3}}
  \frac{1}{\Gamma\pclosed{\frac{d-2}{2}}}
  \frac{1}{2^d\pi^{\frac{3d-4}{2}}}
  \frac{\lambda^{\frac{d-3}{2}\pclosed{Q^2,m_i^2,m_j^2}}}{s}
  \frac{\pclosed{1-z}^{\frac{d-3}{2}}}{z^{\frac{d-2}{2}}}
  \pclosed{y(1-y)}^{\frac{d-4}{2}} \mathrm{d}y\,
  \mathrm{d}\Omega_{d-1}^\star\,
  \mathrm{d}Q^2.
\end{equation}
% \comment{Complete this calculation.}\\
% It is important to note that there is an interdependence between \(Q^2\) and \(\omega\), which can be found using that \(P^2 = s\).
% In the centre-of-mass frame of the neutralinos, the magnitude of the three-momentum of \(P\) must be \(\omega\), so we have the relation \(P_0^2 = s + \omega^2\), which together with momentum conservation \(P\_0 - \omega = Q\) yields
% \begin{equation}
%     \omega = \frac{s-Q^2}{2\sqrt{s}}.
% \end{equation}
% Switching integration variables to \(Q^2\) and \(y = \frac{1}{2}\pclosed{1 + \cos\theta}\), we finally get
% \begin{align}
%     \nonumber
%     \mathrm{d}\Pi_{2\to 3} = \frac{1}{8} \frac{1}{\pclosed{4\pi}^{\frac{3d-4}{2}}} \frac{1}{\Gamma\pclosed{\frac{d-2}{2}}} \frac{\pclosed{\lambda\pclosed{Q^2, m_i^2, m_j^2}}^{\frac{d-3}{2}}}{Q^{2d-2}} \pclosed{s^2-Q^4}\pclosed{s-Q^2}^{d-4} \\
%     \pclosed{y(1-y)}^{\frac{d-4}{2}} \mathrm{d}y\, \mathrm{d}Q^2\, \sin^{d-3}\theta^\ast \mathrm{d}\theta^\ast\, \mathrm{d}\Omega^\ast_{d-2}.
% \end{align}

% \begin{align}
%     \frac{\mathrm{d}^{d-1}\vec{k}}{(2\pi)^{d-1}} \frac{1}{2\omega} = -\frac{1}{(4\pi)^{\sfrac{d}{2}}} \frac{(s-Q^2)^{d-3}}{s^{\sfrac{(d-2)}{2}}} \pclosed{y(1-y)}^{\frac{d-4}{2}} \mathrm{d}y \, \mathrm{d}Q^2
% \end{align}
% \begin{align}
%     \frac{\mathrm{d}^{d-1} \vec{p}_i}{(2\pi)^{d-1}} \frac{1}{2E_i} \frac{\mathrm{d}^{d-1} \vec{p}_j}{(2\pi)^{d-1}} \frac{1}{2E_j} (2\pi)^d \delta^d(P - p_i - p_j - k) \\
%     = \frac{1}{(8\pi)^{\sfrac{d}{2}}} \frac{1}{\Gamma\!\pclosed{\frac{d-2}{2}}} \frac{\pclosed{\lambda(Q^2, m_i^2, m_j^2)}^{\frac{d-3}{2}}}{8 Q^{d-2}} \sin^{d-3}\theta^\ast \,\mathrm{d}\theta^\ast \, \mathrm{d}\phi^\ast
% \end{align}

% \begin{align}
%     \mathrm{d}\Pi_{2\to 3} = -\frac{1}{2^{\sfrac{3d}{2}} (2\pi)^d} \frac{1}{\Gamma\!\pclosed{\frac{d-2}{2}}} \frac{(s-Q^2)^{d-3}}{s^{\sfrac{(d-2)}{2}}} \pclosed{y(1-y)}^{\frac{d-4}{2}} \\
%     \frac{\pclosed{\lambda(Q^2, m_i^2, m_j^2)}^{\frac{d-3}{2}}}{8 Q^{d-2}} \sin^{d-3}\theta^\ast \,\mathrm{d}\theta^\ast \, \mathrm{d}\phi^\ast \, \mathrm{d}y \, \mathrm{d}Q^2
% \end{align}

% \begin{align}
%     \mathrm{d}\sigma = -\frac{1}{2^{\sfrac{3d}{2}} (2\pi)^d} \frac{1}{\Gamma\!\pclosed{\frac{d-2}{2}}} \frac{(s-Q^2)^{d-3}}{s^{\sfrac{(d-2)}{2}}} \pclosed{y(1-y)}^{\frac{d-4}{2}} \\
%     \frac{\pclosed{\lambda(Q^2, m_i^2, m_j^2)}^{\frac{d-3}{2}}}{16 s Q^{d-2}} \abs{\M}^2 \sin^{d-3}\theta^\ast \,\mathrm{d}\theta^\ast \, \mathrm{d}\phi^\ast \, \mathrm{d}y \, \mathrm{d}Q^2
% \end{align}

% \begin{align}
%     \mathrm{d}\Pi_{2\to 3} = \frac{1}{\pclosed{4\pi}^{\frac{3d-4}{2}}} \frac{1}{\Gamma\pclosed{\frac{d-2}{2}}} \frac{\pclosed{\lambda\pclosed{Q^2, m_i^2, m_j^2}}^{\frac{d-3}{2}}}{Q^{d-2}} \sin^{d-3}\theta \mathrm{d}\theta\, \omega^{d-3} \mathrm{d}\omega\, \sin^{d-3}\theta^\ast \mathrm{d}\theta^\ast\, \mathrm{d}\Omega^\ast_{d-2}
% \end{align}

Parametrising the free variables in a \(2 \to 3\) process can be tricky. I will
define some natural variables in two different frames of reference, and
rediscover the Lorentz transformation between them to parametrise all scalar
products in terms of the variables in these reference frames. First, we will
consider the lab frame, or the centre-of-mass frame of the incoming partons
with momenta \(\vec{k}_{i,j}\). We can reduce this to an ordinary \(2 \to 2\)
scattering by considering the outgoing neutralinos with momenta
\(\vec{p}_{i,j}\) as a single system. This lets us write the momenta as
\begin{subequations}
  \begin{align}
    k_i^\mu            & = \frac{\sqrt{s}}{2} \pclosed{1, 0, 0, 1},
    \\
    k_j^\mu            & = \frac{\sqrt{s}}{2} \pclosed{1, 0, 0,
      -1},
    \\
    k\^{\mu}           & = \frac{\sqrt{s}}{2} (1-z) \pclosed{1,
      \sin\theta, 0, \cos\theta},
    \\
    (p_i + p_j)\^{\mu} & = \frac{\sqrt{s}}{2} \pclosed{(1+z),
      -(1-z)\sin\theta, 0, -(1-z)\cos\theta}.
  \end{align}
\end{subequations}

The centre-of-mass frame of the neutralinos is defined by \((p^\ast_i +
p^\ast_k)\^{\mu} = \pclosed{\sqrt{zs}, 0, 0, 0}\).\footnote{I will from now on
  always put a star on quantities pertaining to the centre-of-mass frame
  of the
  neutralinos.} We find the transformation to this frame then by making
appropriate boosts and rotations of this four-vector. Let us start by rotating
the 3-momentum to lie along the positive \(z\)-direction. As the
\(y\)-component is already zero in the lab-frame, we only require a rotation
around the \(y\)-axis, we can be parametrised by the following matrix
\begin{equation}
  \text{Rot}_y(\alpha) = \begin{pmatrix}
    1 & 0           & 0 & 0          \\
    0 & \cos\alpha  & 0 & \sin\alpha \\
    0 & 0           & 1 & 0          \\
    0 & -\sin\alpha & 0 & \cos\alpha
  \end{pmatrix}.
\end{equation}
Using \(\alpha = -\theta-\pi\) we get that \(\text{Rot}_y(-\theta-\pi) (p_i +
p_j)\^{\mu} = \frac{\sqrt{s}}{2} \pclosed{(1+z), 0, 0, (1-z)}\).
We can subsequently boost along the \(z\)-axis to eliminate the
\(z\)-component.
Such a boost can be parametrised by
\begin{equation}
  \text{Boost}_z(\beta) = \begin{pmatrix}
    \gamma      & 0 & 0 & \gamma\beta \\
    0           & 1 & 0 & 0           \\
    0           & 0 & 1 & 0           \\
    \gamma\beta & 0 & 0 & \gamma
  \end{pmatrix},
\end{equation}
where \(\gamma = \pclosed{1 - \beta^2}^{-\sfrac{1}{2}}\).
The \(z\)-component is eliminated using \(\beta = -\frac{1-z}{1+z}\), such that
we end up with
\[(p_i^\ast + p_j^\ast)\^\mu \equiv \text{Boost}_z\pclosed{-\frac{1-z}{1+z}}
  \text{Rot}_y\pclosed{-\theta-\pi} \pclosed{p_i + p_j}^\mu =
  \pclosed{\sqrt{zs},
    0, 0, 0}\]
as we expected.

Now we can parametrise \({p_{i,j}^\ast}\^\mu\) in this frame using two angular
variables \(\theta^\ast, \phi^\ast\), knowing that \(\vec{p}_i + \vec{p}_j =
0\),
\begin{subequations}
  \begin{align}
    {p_i^\ast}\^\mu & = \pclosed{E_i, p\sin\theta^\ast
      \cos\phi^\ast, p\sin\theta^\ast \sin\phi^\ast, p
    \cos\theta^\ast},                                   \\
    {p_j^\ast}\^\mu & = \pclosed{E_j, -p\sin\theta^\ast
      \cos\phi^\ast, -p\sin\theta^\ast \sin\phi^\ast, -p
      \cos\theta^\ast}.
  \end{align}
\end{subequations}
To find what \(E_{i,j}\) and \(p\) need to be, we can transform \(k^\mu\) and
\(k_{i,j}^\mu\) to this frame of reference, finding
\begin{subequations}
  \begin{align}
    {k^\ast}\^\mu                      & = \frac{\sqrt{s}}{2}
    \frac{1-z}{\sqrt{z}} \pclosed{1, 0, 0, -1},                \\
    \pclosed{k_i^\ast + k_j^\ast}\^\mu & = \frac{s}{2\sqrt{z}}
    \pclosed{1+z, 0, 0, -(1-z)},
  \end{align}
\end{subequations}
and use conservation of momentum and the fact that \({p_{i,j}^\ast}^2 =
m^2_{i,j}\) to get that
\begin{subequations}
  \begin{align}
    E_{i,j}(z) & = \frac{zs + m^2_{i,j} - m^2_{j,i}}{2 \sqrt{zs}},
    \\
    p(z)       & = \frac{\sqrt{\lambda\pclosed{zs, m^2_i,
          m^2_j}}}{2 \sqrt{zs}}.
  \end{align}
\end{subequations}

Now to get all momenta in the lab frame, we can apply the reverse
transformations on \({p_{i,j}^\ast}\^\mu\) using that
\(\text{Rot}^{-1}_y(\alpha) = \text{Rot}_y(-\alpha)\) and
\(\text{Boost}_z^{-1}(\beta) = \text{Boost}_z(-\beta)\):
\begin{equation}
  p_{i,j}^\mu = \text{Rot}_y\pclosed{\theta+\pi}
  \text{Boost}_z\pclosed{\frac{1-z}{1+z}} {p_{i,j}^\ast}\^\mu.
\end{equation}

\begin{figure}
  \centering
  \begin{subfigure}{.49\textwidth}
    \centering
    \inputtikz[auto, >=latex, scale=1.1]{kinematics_lab_frame}
    \caption{Angular definition in the centre-of-mass frame of the
      initial particles with momenta \(\vec{k}_{i,j}\).}
  \end{subfigure}
  \hfill
  \begin{subfigure}{.49\textwidth}
    \centering
    \inputtikz[auto, >=latex, scale=1.1]{kinematics_com_frame}
    \caption{Angular definitions in the centre-of-mass frame of the
      outgoing particles with momenta \(\vec{p}_{i,j}\).}
  \end{subfigure}
  \caption{}
\end{figure}


\subsection{Differential Cross-Section}
Consider a scattering process of set of two initial state particles \(\cclosed{i}\) going to a set of \(n\) final particles \(\cclosed{f}\).
Given a transition amplitude \(\M(\cclosed{i} \to \cclosed{f})\), the differential cross-section is given by~\cite{Schwartz:2014sze}
\begin{equation}
  \mathrm{d}\sigma = \frac{1}{4|E_2 \vec{k}_1 - E_1\vec{k}_2|} \abs{\M(\cclosed{i} \to \cclosed{f})}^2 \mathrm{d}\Pi_{2 \to n},
\end{equation}
where \(k_{1/2}^\mu = \pclosed{E_{1/2}, \vec{k}_{1/2}}\) are the four-momenta of the initial particles and \(\mathrm{d}\Pi_{2 \to n}\) is the differential phase space of the final state particles as given by \cref{pc:eq:dPi_d}.
In the case where the initial particles are massless, this simplifies to
\begin{equation}
  \label{pc:eq:diff_xsec_formula}
  \mathrm{d}\sigma = \frac{1}{2s} \abs{\M(\cclosed{i} \to \cclosed{f})}^2 \mathrm{d}\Pi_{2 \to n},
\end{equation}
where \(s = (k_1 + k_2)^2\).

For future reference, the differential cross-section in four dimensions for a \(2\to 2\) scattering process with massless initial state particles using \cref{pc:eq:dPi_dt} is
\begin{equation}
  \label{pc:eq:dsigma_dt}
  \d[t]{\sigma} = \frac{1}{16\pi} \frac{1}{s^2} \abs{\M}^2.
\end{equation}
In \(d\) dimension, switching to the variable \(y = \frac{1}{2}(1+\cos\theta)\) from \cref{pc:eq:dPi_dtheta}, the differential cross-section is given by\footnote{The seemingly arbitrary change of variable is a pre-emptive change anticipating some integrals that will arise later on.}
\begin{equation}
  \label{pc:eq:dsigma_dy}
  \d[y]{\sigma^d} = \frac{1}{\pclosed{4\pi}^{\frac{d-2}{2}}} \frac{1}{\Gamma\pclosed{\frac{d-2}{2}}} \frac{p^{d-3}}{8s\sqrt{s}} \abs{\M}^2 (y(1-y))^{d-4},
\end{equation}
where the momentum \(p\) is given by \cref{pc:eq:p}.
The limits on the integration variables are
\begin{subequations}
  \begin{align}
    -p\sqrt{s} - \frac{1}{2}\pclosed{s - m_i^2 - m_j^2} \leq & t \leq p\sqrt{s} - \frac{1}{2}\pclosed{s - m_i^2 - m_j^2}, \\
    0 \leq                                                   & y \leq 1.
  \end{align}
\end{subequations}


% Averaged over spin and colour, and taking account of symmetry if the particles
% are identical, the differential cross-section in \(d=4\) dimensions is.
% \begin{equation}
%   \label{pc:eq:dxsec}
%   \mathrm{d}\hat{\sigma} = \pclosed{\frac{1}{2}}^{\delta_{ij}}
%   \frac{1}{64 N_C^2 \pi} \frac{1}{\hat{s}^2}
%   \sum_{\substack{\text{spin}\\\text{colour}}} \abs{\M}^2
%   \mathrm{d}\hat{t}
% \end{equation}
% \clearpage
% \begin{equation}
%     \lambda(x, y, z) = x^2 + y^2 + z^2 - 2xy - 2xz - 2yz
% \end{equation}

% \begin{align}
%     \nonumber
%     t_{ii}(z, y, \theta^\ast, \phi^\ast) \equiv (k_i - p_i)^2 = \frac{1}{2z} \bigg\{ (m_i^2 + m_j^2)z - (m_i^2 - m_j^2)y(1-z) + sz(yz - y - z) \\
%     + 2z \sqrt{y(1-y)} \sqrt{\lambda(Q^2,m_i^2,m_j^2)}\sin\theta^\ast \cos\phi^\ast + (z-y-yz)\sqrt{\lambda(Q^2,m_i^2,m_j^2)} \cos\theta^\ast \bigg\}
% \end{align}

% \begin{align}
%     \nonumber
%     m^2_{gi}(z, \theta^\ast) \equiv (p_i + k)^2 = \frac{1}{2z} \bigg\{ (m_i^2-m_j^2) + (m_i^2+m_j^2)z \\
%     + sz(1-z) + \sqrt{\lambda(Q^2, m_i^2, m_j^2)} (1-z) \cos\theta^\ast \bigg\}
% \end{align}

% \begin{align}
%     \abs{\M}^2 \supset \frac{\pclosed{t_{ii}(z, y, \theta^\ast, \phi^\ast) - m_i^2} \pclosed{m^2_{gi}(z, \theta^\ast) - m_i^2 - (1-z)s}}{s y (1-z) \pclosed{t_{ii}(z, y, \theta^\ast, \phi^\ast) - m_A^2} \pclosed{t_{ii}(z, y, \theta^\ast, \phi^\ast) - m_B^2}}
% \end{align}



\section{Leading Order Cross-Section}

\begin{TODO}
  \item Comment on reason for using Breit-Wigner approximation.
\end{TODO}

For the leading order contributions, there are no divergences that need regulating, so we will use \(d=4\).

\subsection{Kinematic Definitions}
Before getting into the details of the calculation, it will be helpful to present some definitions of the variables we will need.
I will make use of the shorthand notation for the spinors \(w_{i/j} = w(p_{i/j}), w_{1,2} = w(k_{i/j})\) where \(w\) is either \(u\) or \(v\).
We will also need to define an appropriate set of kinematic variables.
Seeing as the inclusive scattering cross-section is only a \(2 \to 2\) process to leading order, I will make use of the Mandelstam variables, which in this case will be defined as
\begin{subequations}
  \label{pc:eq:mandelstam_LO}
  \begin{align}
    \hat{s} \equiv & (k_i+k_j)^2 = (p_i+p_j)^2, \\
    \hat{t} \equiv & (k_i-p_i)^2 = (k_j-p_j)^2, \\
    \hat{u} \equiv & (k_i-p_j)^2 = (k_j-p_i)^2,
  \end{align}
\end{subequations}
which by \cref{pc:eq:generalised_mandelstam_relation} is constrained by \(\hat{s} + \hat{t} + \hat{u} = m_i^2 + m_j^2\).
For clarity later on when we will be working with hadron-level kinematics in the next chapter, I will put a hat on variables that are defined at parton level which have an unhatted, hadron-level counterparts.
This includes the Mandelstam variables above and the cross-sections.


\subsection{The Matrix Elements}
\begin{figure} [ht!]
  \centering
  \begin{subfigure}{0.3\linewidth}
    \centering
    \inputtikz{s_channel}
    \caption{\(\hat{s}\)-channel}
  \end{subfigure}
  \begin{subfigure}{0.3\linewidth}
    \centering
    \inputtikz{t_channel}
    \caption{\(\hat{t}\)-channel}
  \end{subfigure}
  \begin{subfigure}{0.3\linewidth}
    \centering
    \inputtikz{u_channel}
    \caption{\(\hat{u}\)-channel}
  \end{subfigure}
  \caption{The leading order diagrams contributing to neutralino pair
    production at parton-level.}
  \label{pc:fig:tree_level_diagrams}
\end{figure}

At leading order the contributing diagrams to the parton-level process are shown in \cref{pc:fig:tree_level_diagrams}.
The resulting amplitudes, using the Feynman rules in \needcite[Feynman rules section], are then
\begin{subequations}
  \begin{align}
    \nonumber
    \M_{\hat{s}} = -g^2 D_Z(\hat{s})                         & \bclosed{ \bar{u}_i\gamma^\mu\pclosed{O^{\prime\prime L}_{ij}P_L + O^{\prime\prime R}_{ij}P_R} v_j }                                              \\
    \label{pc:eq:Ms}
                                                             & \times \bclosed{ \bar{v}_2 \gamma_\mu \pclosed{C^L_{qqZ}P_L + C^R_{qqZ}P_R} u_1 },                                                                \\
    \nonumber
    \M_{\hat{t}} = -\sum_A 2g^2 D_{\tilde{q}_A}(\hat{t})     & \bclosed{ \bar{u}_i \pclosed{\pclosed{C_{q_g \tilde{q}_A \nino[i]}^L}^{\!\ast} P_L + \pclosed{C_{q_g \tilde{q}_A \nino[i]}^R}^{\!\ast} P_R} u_1 } \\
    \label{pc:eq:Mt}
                                                             & \times \bclosed{ \bar{v}_2 \pclosed{C_{q_g \tilde{q}_A \nino[j]}^R P_L + C_{q_g \tilde{q}_A \nino[j]}^L P_R} v_j },                               \\
    \nonumber
    \M_{\hat{u}} = (-1)-\sum_A 2g^2 D_{\tilde{q}_A}(\hat{u}) & \bclosed{ \bar{u}_j \pclosed{\pclosed{C_{q_g \tilde{q}_A \nino[j]}^L}^{\!\ast} P_L + \pclosed{C_{q_g \tilde{q}_A \nino[j]}^R}^{\!\ast} P_R} u_1 } \\
    \label{pc:eq:Mu}
                                                             & \times \bclosed{ \bar{v}_2 \pclosed{C_{q_g \tilde{q}_A \nino[i]}^R P_L + C_{q_g \tilde{q}_A \nino[i]}^L P_R} v_i },
  \end{align}
\end{subequations}
where \(D_p(q^2) = \frac{1}{q^2 - m_p^2 (+ i\Gamma_p m_p)}\) is the
Breit-Wigner propagator~\cite{Schwartz:2014sze} of a particle with mass \(m_p\) and decay
width \(\Gamma_p\) and the extra minus sign in \(\mathcal{M}_{\hat{u}}\) comes from it being an odd permutation of the external spinors to the other two amplitudes.
Among other things, the Breit-Wigner propagator regularises divergences near the resonance where an intermediate particle goes on-shell, e.g. when \(hat{s} \to m_Z^2\).
As it turns out, such divergences will not appear in the integrated cross-section for the \(\hat{t}\)- and \(\hat{u}\)-channels, and so I will use \(\Gamma_{\tilde{q}_A} = 0\) for the remainder of this thesis.
\\\comment{Maybe point to somewhere for further explanation/details?}

These matrix elements can be expanded using the \textit{effective charges} defined by
\begin{subequations}
  \label{pc:eq:eff_charges}
  \begin{align}
    Z^{XY}   & = D_Z(\hat{s}) C_{qqZ}^X O_{ij}^{\prime\prime Y},                                   \\
    Q_A^{XY} & = C_{q_g \tilde{q}_A \nino[i]}^{X} \pclosed{C_{q_g \tilde{q}_A \nino[j]}^{Y}}^\ast,
  \end{align}
\end{subequations}
% and the Dirac quadrilinears
% \begin{subequations}
% 	\begin{align}
% 		q_S^{XY}(w_a, w_b, w_c, w_d) = & \bclosed{\bar{w}_a P_X w_b} \bclosed{\bar{w}_c P_Y w_d}                       \\
% 		q_V^{XY}(w_a, w_b, w_c, w_d) = & \bclosed{\bar{w}_a \gamma^\mu P_X w_b} \bclosed{\bar{w}_c \gamma_\mu P_Y w_d}
% 	\end{align}
% \end{subequations}
and the \emph{Dirac bilinears}
\begin{subequations}
  \begin{align}
    b_{L/R}(w_a, w_b) =     & \bar{w}_a P_{L/R} w_b,            \\
    b_{L/R}^\mu(w_a, w_b) = & \bar{w}_a \gamma^\mu P_{L/R} w_b,
  \end{align}
\end{subequations}
to arrive at
\begin{subequations}
  \label{pc:eq:MLO}
  \begin{align}
    \nonumber
    \M_{\hat{s}} = -g^2 \Big[                                  &
    Z^{LL} b_L^\mu(v_2, u_1){b_L}\_{\mu}(u_i, v_j) + Z^{LR} b_L^\mu(v_2, u_1){b_R}\_{\mu}(u_i, v_j)                 \\
    +                                                          & Z^{LR} b_{R}^\mu(v_2, u_1){b_L}\_{\mu}(u_i, v_j) +
    Z^{RR} b_R^\mu(v_2, u_1){b_R}\_{\mu}(u_i, v_j) \Big],                                                           \\
    \nonumber
    \M_{\hat{t}} = -\sum_A 2g^2 D_{\tilde{q}_A}(\hat{t}) \Big[ &
    \pclosed{Q_A^{LR}}^\ast b_L(u_i, u_1) b_L(v_2, v_j) +
    \pclosed{Q_A^{LL}}^\ast
    b_L(u_i, u_1) b_R(v_2, v_j)                                                                                     \\
    +                                                          &
    \pclosed{Q_A^{RR}}^\ast b_R(u_i, u_1) b_L(v_2, v_j) +
    \pclosed{Q_A^{RL}}^\ast
    b_R(u_i, u_1) b_R(v_2, v_j) \Big],                                                                              \\
    \nonumber
    \M_{\hat{u}} = \sum_A 2g^2 D_{\tilde{q}_A}(\hat{u}) \Big[  &
    Q_A^{RL} b_L(v_2, v_i) b_L(u_j, u_1) + Q_A^{RR} b_L(v_2, v_i)
    b_R(u_j, u_1)
    \\
    +                                                          &
    Q_A^{LL} b_R(v_2, v_i) b_L(u_j, u_1) + Q_A^{LR} b_R(v_2, v_i)
    b_R(u_j, u_1)
    \Big].
  \end{align}
\end{subequations}

To square the amplitudes we will need to use that the complex conjugate of the Dirac bilinears is
\begin{subequations}
  \label{pc:eq:bilinear_conjugate}
  \begin{align}
    \pclosed{b_{L/R}(w_a, w_b)}^\dagger =     & b_{R/L}(w_b, w_a),
    \\
    \pclosed{b_{L/R}^\mu(w_a, w_b)}^\dagger = & b_{L/R}^\mu(w_b,
    w_a).
  \end{align}
\end{subequations}
Furthermore, when summing over the spins of the various spinors in the
bilinears, they have the sum identities
\begin{align}
  \label{pc:eq:scalar_bilinear_trace}
  \sum_{\text{spins}} b_X(w_a, w_b) b_Y(w_b, w_a) =         & 2 \Big[
    (1-\delta_{XY}) (p_a \cdot p_b) + \operatorname{rsgn}
    \delta_{XY} m_a m_b
    \Big],
  \\
  \nonumber
  \label{pc:eq:vector_bilinear_trace}
  \sum_{\text{spins}} b_X^\mu(w_a, w_b) b_Y^\nu(w_b, w_a) = & 2\Big[
    \delta_{XY}\pclosed{p_a^\mu p_b^\nu - \g (p_a \cdot p_b) +
      p_a^\nu
      p_b^\mu +
      (-1)^{\delta_{XL}} i \epsilon^{\mu\nu\alpha\beta}
      \tensor{(p_a)}{_\alpha}
  \tensor{(p_b)}{_\beta}}                                             \\
                                                            & +
    (1-\delta_{XY})\operatorname{rsgn} m_a m_b \g \Big],
\end{align}
where \(\operatorname{rsgn}\) is 1 if \(w_a, w_b\) are spinors of the same
type, e.g. both are \(u\)-spinors, and -1 otherwise.


\subsection{Differential Result}
Averaging the cross-section over spin the two spins and \(N_C\) colour charges of the initial quark, and taking account of symmetry if the final neutralinos are identical, we get from \cref{pc:eq:dsigma_dt}
\begin{equation}
  \label{pc:eq:dxsec}
  \d[\hat{t}]{\hat{\sigma}} = \pclosed{\frac{1}{2}}^{\delta_{ij}} \frac{1}{64 N_C^2 \pi} \frac{1}{\hat{s}^2} \sum_{\substack{\text{spin}\\\text{colour}}} \abs{\M}^2
\end{equation}
Now, squaring the amplitudes and inserting them, the partonic cross-section differential in \(\hat{t}\) can be shown to be\footnote{I note that the amplitudes in \cref{pc:eq:Ms,pc:eq:Mt,pc:eq:Mu} have the quark colour indices suppressed, including a Kronecker-delta in the vertex rule for \(qqZ\) and in the squark propagators. In the end, summing over the colours of the squared amplitudes amounts to a sum over this Kronecker-delta, producing a factor of \(N_C\).}
\begin{align}
  \label{pc:eq:dsigma0_dt}
  \nonumber
  \d[\hat{t}]{\hat{\sigma}^0} = & \frac{\pi \alpha_W^2}{N_C \hat{s}^2}
  \pclosed{\frac{1}{2}}^{\delta_{ij}} \Bigg\{ \sum_{X,Y}
  \bclosed{\abs{C^{XY}_{\hat{t}}}^2
    \pclosed{\hat{t}-m_i^2}\pclosed{\hat{t}-m_j^2} +
    \abs{C^{XY}_{\hat{u}}}^2
  \pclosed{\hat{u}-m_i^2}\pclosed{\hat{u}-m_j^2}}                      \\
                                & - \sum_X
  \bclosed{2\Re{\pclosed{C_{u}^{XX}}^\ast C_{t}^{XX}} m_i m_j \hat{s} -
    2\Re{\pclosed{C_{u}^{XX^\prime}}^\ast C_{t}^{XX^\prime}}
    \pclosed{\hat{t}\hat{u} - m_i^2m_j^2}} \Bigg\},
\end{align}
where I have defined
\begin{subequations}
  \begin{align}
    C_{\hat{t}}^{XY} = & -\delta^{XY} (Z^{XX^\prime})^\ast + \sum_{A}\frac{Q_A^{XY}}{t-m_A^2}, \\
    C_{\hat{u}}^{XY} = & \delta^{XY} (Z^{XX})^\ast + \sum_{A}\frac{(Q_A^{XY})^\ast}{t-m_A^2}.  \\
  \end{align}
\end{subequations}
The sum over \(X, Y\) go over \(L, R\), and \(L^\prime/R^\prime = R/L\).

The result \cref{pc:eq:dsigma0_dt} has been compared and verified to be equivalent to other results~\cite{Debove:2010kf} in the literature symbolically in Mathematica programs.


\subsection{Phase Space Integral}
To get the full cross-section, we will need to integrate over the \(\hat{t}\)-variable.
To do this, we can classify the types of integrals that will arise.
After inserting \(\hat{u} = m_i^2 + m_j^2 - \hat{s} - \hat{t}\), all the integrals take the form
\begin{equation}
  T^p(\Delta_1, \Delta_2) \equiv \int_{t_-}^{t_+} \!\mathrm{d}\hat{t}\,
  \frac{\hat{t}^p}{(\hat{t}-\Delta_1)(\hat{t}-\Delta_2)}
\end{equation}
for some \(\Delta_{1,2}\) dependent on \(\hat{s}\), the neutralino masses and
the squark masses, and \(p\) is some non-negative integer.

Using the integral limits are \(t_\pm = -\frac{\hat{s} - m_i^2 - m_j^2}{2} \pm p\sqrt{\hat{s}}\), we get that the possible integrals evaluate to
\begin{subequations}
  \begin{align}
    T^2(0, 0) =                         & 2p\sqrt{\hat{s}},
    \\
    T^3(0, 0) =                         &
    -p\sqrt{\hat{s}}\pclosed{\hat{s} - m_i^2 - m_j^2},
    \\
    T^4(0, 0) =                         & p\sqrt{\hat{s}}
    \pclosed{\frac{8}{3}\hat{s}p^2 + 2m_i^2m_j^2},
    \\
    T^1\!\pclosed{\Delta, 0} =          & -L(\Delta),
    \\
    T^2\!\pclosed{\Delta, 0} =          & 2p\sqrt{\hat{s}} - \Delta L(\Delta),
    \\
    T^3\!\pclosed{\Delta, 0} =          &
    p\sqrt{\hat{s}} \pclosed{2\Delta - (\hat{s}-m_i^2-m_j^2)} - \Delta^2 L(\Delta),
    \\
    T^0\!\pclosed{\Delta_1, \Delta_2} = & \begin{cases} \frac{1}{\Delta_2 - \Delta_1}
                                            \cclosed{L(\Delta_1) - L(\Delta_2)}                                           & \text{if } \Delta_1\neq \Delta_2             \\
                                            \frac{2p \sqrt{\hat{s}}}{\Delta^2 + \Delta(\hat{s}-m_i^2-m_j^2) + m_i^2m_j^2} & \text{if } \Delta_1 = \Delta_2 \equiv \Delta\end{cases},
    \\
    T^1\!\pclosed{\Delta_1, \Delta_2} = & \begin{cases} \frac{1}{\Delta_2 - \Delta_1}
                                            \cclosed{\Delta_1 L(\Delta_1) - \Delta_2 L(\Delta_2)}                                            & \text{if } \Delta_1 \neq \Delta_2            \\
                                            \frac{2\Delta p \sqrt{\hat{s}}}{\Delta^2 + \Delta(\hat{s}-m_i^2-m_j^2) + m_i^2m_j^2} - L(\Delta) & \text{if } \Delta_1 = \Delta_2 \equiv \Delta\end{cases},
    \\
    \nonumber
    T^2\!\pclosed{\Delta_1, \Delta_2} = & \begin{cases} 2p\sqrt{\hat{s}} +
                                            \frac{1}{\Delta_2 - \Delta_1} \cclosed{\Delta_1^2 L(\Delta_1) -
                                            \Delta_2^2 L(\Delta_2)}                                                                                                                                  & \text{if } \Delta_1 \neq \Delta_2            \\
                                            \frac{2(2\Delta^2 + \Delta(\hat{s}-m_i^2-m_j^2) + m_i^2m_j^2) p \sqrt{\hat{s}}}{\Delta^2 + \Delta(\hat{s}-m_i^2-m_j^2) + m_i^2m_j^2} - 2\Delta L(\Delta) & \text{if } \Delta_1 = \Delta_2 \equiv \Delta\end{cases},
  \end{align}
\end{subequations}
where I have defined \(L(\Delta) = \log\frac{\Delta + \frac{1}{2}(\hat{s}-m_i^2-m_j^2) + p\sqrt{\hat{s}}}{\Delta + \frac{1}{2}(\hat{s}-m_i^2-m_j^2) - p\sqrt{\hat{s}}}\).
The two non-zero arguments to these functions that will arise are \(\Delta^{\hat{t}}_A = m_A^2\) and \(\Delta^{\hat{u}}_A = -(\hat{s}-m_i^2-m_j^2) - m_A^2\), and I note that \(L(\Delta^{\hat{u}}_A) = -L(\Delta^{\hat{t}}_A)\).

Putting it all together, this lets us write the total cross-section
\begin{align}
  \label{pc:eq:born}
  \hat\sigma^0 = & \frac{4 \pi p \alpha_W^2}{\hat{s}^{\sfrac{3}{2}} N_C} \pclosed{\frac{1}{2}}^{\delta_{ij}} \pclosed{F^{\prime\prime}_{\tilde{q}} + F^{\prime\prime}_Z + F^{\prime\prime}_{\tilde{q}Z}} \equiv \hat{\sigma}_B \pclosed{F^{\prime\prime}_{\tilde{q}} + F^{\prime\prime}_Z + F^{\prime\prime}_{\tilde{q}Z}},
\end{align}
with effective couplings defined as
\\\comment{Should I perhaps be explicit in the neutralino indices \(i,j\) in these coulpings?}
\begin{align}
  \label{pc:eq:eff_coupling_Fq}
  \nonumber
  F^{\prime\prime}_{\tilde{q}} = \sum_{A,B,X,Y} & \biggl\{\Re{Q_A^{XY} (Q_B^{XY})^\ast} \pclosed{1 - L_2^{AB}} + \delta_{XY}\Re{Q_A^{XX} Q_B^{XX}} L_1^{AB} \\
                                                & + \delta_{XY} \Re{Q_A^{XX^\prime} Q_B^{X^\prime X}} \pclosed{1 - L_3^{AB}} \biggr\},
\end{align}
\begin{align}
  \label{pc:eq:eff_coupling_FZ}
  F^{\prime\prime}_{Z} = & \sum_{X,Y} \cclosed{\frac{1}{12} \pclosed{\hat{s}(\hat{s}-m_i^2-m_j^2) + \hat{s}^2 - (m_i^2-m_j^2)^2} |Z^{XY}|^2 + \delta_{XY} \hat{s}m_i m_j \Re{Z^{XX} (Z^{XX^\prime})^\ast}},
\end{align}
and
\begin{align}
  \label{pc:eq:eff_coupling_FqZ}
  \nonumber
  F^{\prime\prime}_{\tilde{q}Z} = & \frac{1}{2} \sum_{A, X} \biggl\{
  \Bigl[\hat{s} m_i m_j \Re{Q_A^{XX} \bigl(Z^{XX} - (Z^{XX^\prime})^\ast\bigr)}                                                                                       \\
  \nonumber
                                  & \quad - (m_A^2-m_i^2)(m_A^2-m_j^2) \Re{Q_A^{XX}\bigl((Z^{XX})^\ast - Z^{XX^\prime}\bigr)}\Bigr] \frac{L(m_A^2)}{p \sqrt{\hat{s}}} \\
                                  & - (\hat{s} + m_i^2 + m_j^2 - 2m_A^2) \Re{Q_A^{XX}\bigl((Z^{XX})^\ast - Z^{XX^\prime}\bigr)}
  \biggr\},
\end{align}
where \(X,Y \in L,R\), \(L^\prime/R^\prime = R/L\) and I have defined the shorthands for some functions of the kinematics
\begin{subequations}
  \begin{align}
    L_1^{AB} = & \frac{\hat{s}m_i m_j}{m_A^2+m_B^2 + \hat{s} - m_i^2 - m_j^2} \frac{L(m_A^2)}{p \sqrt{\hat{s}}},                                                                                                                                                         \\
    L_2^{AB} = & \begin{cases}\frac{(m_A^2-m_i^2)(m_A^2-m_j^2)}{m_A^2-m_B^2} \frac{L(m_A^2)}{p \sqrt{\hat{s}}} & \text{if } m_A \neq m_B \\ \frac{1}{2}(2 m_A^2 - m_i^2 - m_j^2) \frac{L(m_A^2)}{p \sqrt{\hat{s}}} & \text{if } m_A = m_B \end{cases}, \\
    L_3^{AB} = & \frac{m_A^4 + m_A^2(\hat{s}-m_i^2-m_j^2) + m_i^2 m_j^2}{m_A^2+m_B^2 + \hat{s} + m_i^2 - m_j^2} \frac{L(m_A^2)}{p \sqrt{\hat{s}}}.
  \end{align}
\end{subequations}
The sums over the squark mass eigenstates with indices \(A, B\) go to two for the non-flavour violating case, and from one to six in the flavour violating SLHA2 case.


\subsection{Generalising to all Electroweakinos}
So far, we have only calculated the cross-section for production of a pair of neutralinos.
However, the amplitude structure is very similar both for pair production of charginos and production of a neutralino with a chargino.
With a few modifications, we can thus generalise the result from \cref{pc:eq:born} to any electroweakino pair.
The LO diagrams for the other electroweakino processes at parton level are shown in \cref{pc:fig:NC_diagrams,pc:fig:CC_diagrams}.

\begin{figure}[ht!]
  \centering
  \inputtikz{NC_s_channel_W}
  \inputtikz{NC_t_channel}
  \inputtikz{NC_u_channel}
  \caption{}
  \label{pc:fig:NC_diagrams}
\end{figure}

\begin{figure}[ht!]
  \centering
  \inputtikz{CC_s_channel_Z}
  \inputtikz{CC_s_channel_A}
  \inputtikz{CC_t_channel}
  \caption{}
  \label{pc:fig:CC_diagrams}
\end{figure}

\subsubsection*{Neutralino and Chargino production}
Producing a chargino together with a neutralino requires the total charge of the process to differ from zero.
The partonic processes that contribute are on the form \(q \bar{q}^\prime \to \tilde\chi^0_i \tilde\chi^\pm_j\).
Now the indices \(j\) will refer to chargino mass eigenstates and \(m_j = m_{\tilde\chi^\pm_j}\).
Furthermore, since both up- and down-type quarks are involved in the process, I will be explicit in the squark indices \(A, B\) whether they refer to up- or down-type squarks.
Using the Feynman rules from \cref{susy:sec:feynman_rules} we get the amplitudes
\begin{subequations}
  \begin{align}
    \nonumber
    \M_{\hat{s}} = -g^2 D_W(\hat{s})                                & \bclosed{ \bar{u}_i\gamma^\mu\pclosed{O^{L}_{ij}P_L + O^{R}_{ij}P_R} v_j }
    \\
    \label{pc:eq:NC_Ms}
                                                                    & \times \bclosed{ \bar{v}_2 \gamma_\mu C^L_{qq^\prime W}P_L u_1 },
    \\
    \nonumber
    \M_{\hat{t}} = -\sum_A 2g^2 D_{\tilde{q}_A}(\hat{t})            & \bclosed{ \bar{u}_i \pclosed{\pclosed{C_{q \tilde{q}_A \nino[i]}^L}^{\!\ast} P_L + \pclosed{C_{q \tilde{q}_A \nino[i]}^R}^{\!\ast} P_R} u_1 }
    \\
    \label{pc:eq:NC_Mt}
                                                                    & \times \bclosed{ \bar{v}_2 C_{q^\prime \tilde{q}_A \tilde\chi^\pm_j}^L P_R v_j },
    \\
    \nonumber
    \M_{\hat{u}} = (-1)-\sum_A 2g^2 D_{\tilde{q}_A^\prime}(\hat{u}) & \bclosed{ \bar{u}_j \pclosed{C_{q \tilde{q}_A^\prime \tilde\chi^\pm_j}^L}^{\!\ast} P_L u_1 }
    \\
    \label{pc:eq:NC_Mu}
                                                                    & \times \bclosed{ \bar{v}_2 \pclosed{C_{q^\prime \tilde{q}_A^\prime \nino[i]}^R P_L + C_{q^\prime \tilde{q}_A^\prime \nino[i]}^L P_R} v_i },
  \end{align}
\end{subequations}
from the diagrams in \cref{pc:fig:NC_diagrams}.
Redefining the effective charges in \cref{pc:eq:eff_charges} to
\begin{subequations}
  \begin{align*}
    Z^{XY} =     & D_W(\hat{s}) C_{qq^\prime W}^X O_{ij}^Y,                                        \\
    Q_{A}^{XY} = & C_{q \tilde{q}_A \tilde\chi^0_i}^X C_{q^\prime \tilde{q}_A \tilde\chi^\pm_j}^Y,
  \end{align*}
\end{subequations}
we can rewrite the amplitudes to
\begin{subequations}
  \begin{align}
    \M_{\hat{s}} = -g^2 \Big[                                        &
      Z^{LL} b_L^\mu(v_2, u_1){b_L}\_{\mu}(u_i, v_j) + Z^{LR} b_L^\mu{(v_2, u_1)b_R}\_{\mu}(u_i, v_j)
    \Big],                                                             \\
    \M_{\hat{t}} = -\sum_A 2g^2 D_{\tilde{q}_A}(\hat{t}) \Big[       &
      \pclosed{Q_{A}^{LL}}^\ast b_L(u_i, u_1) b_R(v_2, v_j) + \pclosed{Q_{A}^{RL}}^\ast b_R(u_i, u_1) b_R(v_2, v_j)
    \Big] \Big|_{q=q},                                                 \\
    \M_{\hat{u}} = \sum_A 2g^2 D_{\tilde{q}^\prime_A}(\hat{u}) \Big[ &
      Q_{A}^{RL} b_L(v_2, v_i) b_L(u_j, u_1) + Q_{A}^{LL} b_R(v_2, v_i) b_L(u_j, u_1)
      \Big] \Big|_{q=q^\prime},
  \end{align}
\end{subequations}
mimicking the structure of \cref{pc:eq:MLO}.
It is worth noting that the structure is not entirely the same, as the charges and propagator in \(\M_{\hat{t}}\) and \(\M_{\hat{u}}\) are not the same, owing to a different squark type being mediated.
Nevertheless, we can follow the calculation for the neutralinos to arrive at a similar cross-section to the one in \cref{pc:eq:born}, but altering the effective couplings slightly.
We get
\begin{equation}
  \hat\sigma^0(q\bar{q}^\prime \to \tilde\chi^0_i \tilde\chi^\pm_j) = \frac{4 \pi p \alpha_W^2}{\hat{s}^{\sfrac{3}{2}} N_C} \pclosed{F_{\tilde{u}} + F_{\tilde{d}} + F_W + F_{\tilde{u}W} + F_{\tilde{d}W}},
\end{equation}
where I note that the identical particle factor from \cref{pc:eq:born} is no longer necessary, and with the effective couplings defined as\footnote{The observant reader will notice that there is one term missing from the definitions of \(F_{\tilde{u}/\tilde{d}}\) as compared to \cref{pc:eq:eff_coupling_Fq} --- this term disappears due to \(Q_{qA}^{XR} = 0\) for \(X = L/R\).}
\\\comment{Should I combine these two to one?}
\begin{subequations}
  \begin{align}
    F_{\tilde{u}} = & \frac{1}{2} \sum_{A,B,X,Y} (1 - L_{u2}^{AB}) \Re{Q_{A}^{XY} (Q_{B}^{XY})^\ast}\Big|_{q=u} + \delta_{XY} L_{u_1}^{AB} \Re{Q_{A}^{XX}|_{q=u} Q_{dB}^{XX}|_{q=d}}, \\
    F_{\tilde{d}} = & \frac{1}{2} \sum_{A,B,X,Y} (1 - L_{d2}^{AB}) \Re{Q_{A}^{XY} (Q_{B}^{XY})^\ast}\Big|_{q=d} + \delta_{XY} L_{d1}^{AB} \Re{Q_{A}^{XX}|_{q=d} Q_{dB}^{XX}|_{q=u}},
  \end{align}
\end{subequations}
\begin{align}
  F_{W} = & \sum_{X,Y} \cclosed{\frac{1}{12} \pclosed{2\hat{s}(\hat{s}-m_i^2-m_j^2) + m_i^4 + m_j^4} |Z^{XY}|^2 + \delta_{XY} \hat{s}m_i m_j \Re{Z^{XX} (Z^{XX^\prime})^\ast}},
\end{align}
and
\\\comment{Should I combine these two to one too?}
\begin{subequations}
  \begin{align}
    \nonumber
    F_{\tilde{u}W} = & \frac{1}{2} \sum_{A,X} \biggl\{
    \hat{s} m_i m_j \Re{Q_{uA}^{XX} (Z^{XX^\prime})^\ast}\frac{L(m_{\tilde{u}_A}^2)}{p \sqrt{\hat{s}}}                                                                                                                              \\
                     & \hspace{-1cm}+ \Bigl[(m_{\tilde{u}_A}^2-m_i^2)(m_{\tilde{u}_A}^2-m_j^2) \frac{L(m_{\tilde{u}_A}^2)}{p \sqrt{\hat{s}}} + (\hat{s} + m_i^2 + m_j^2 - 2m_{\tilde{u}_A}^2)\Bigr] \Re{Q_{uA}^{XX}(Z^{XX})^\ast}
    \biggr\},                                                                                                                                                                                                                       \\
    \nonumber
    F_{\tilde{d}W} = & -\frac{1}{2} \sum_{A,X} \biggl\{
    \hat{s} m_i m_j \Re{Q_{dA}^{XX} Z^{XX}}                                                                                                                                                                                         \\
                     & \hspace{-1cm} + \Bigl[(m_{\tilde{d}_A}^2-m_i^2)(m_{\tilde{d}_A}^2-m_j^2) \frac{L(m_{\tilde{d}_A}^2)}{p \sqrt{\hat{s}}} + (\hat{s} + m_i^2 + m_j^2 - 2m_{\tilde{d}_A}^2) \Bigr]\Re{Q_{dA}^{XX} Z^{XX^\prime}}
    \biggr\},
  \end{align}
\end{subequations}
and where I have defined slightly altered kinematic functions\footnote{The only difference in these definitions is that the squark type is specified. Make note that both squark types (up and down) are used in \(L_{q1}^{AB}\).}
\begin{align}
  L_{q1}^{AB} = & \frac{\hat{s} m_i m_j}{m_{\tilde{q}_A}^2 + m_{\tilde{q}^\prime_B}^2 + \hat{s} - m_i^2 - m_j^2} \frac{L(m_{\tilde{q}_A}^2)}{p \sqrt{\hat{s}}},                                                                                                                                                                                                                                                                                                                                                                                      \\
  L_{q2}^{AB} = & \begin{cases}\frac{(m_{\tilde{q}_A}^2-m_i^2)(m_{\tilde{q}_A}^2-m_j^2)}{m_{\tilde{q}_A}^2-m_{\tilde{q}_B}^2} \frac{L(m_{\tilde{q}_A}^2)}{p \sqrt{\hat{s}}} & \text{if } m_{\tilde{q}_A} \neq m_B \\ \frac{1}{2}(2 m_{\tilde{q}_A}^2 - m_i^2 - m_j^2) \frac{L(m_{\tilde{q}_A}^2)}{p \sqrt{\hat{s}}} & \text{if } m_{\tilde{q}_A} = m_{\tilde{q}_B} \end{cases}.
\end{align}


\subsubsection*{Chargino Pair Production}
The above derivation can be repeated for the chargino pair production process.
This time, the total charge is zero, and the partonic processes are all on the form \(q\bar{q} \to \tilde\chi^\pm_i \tilde\chi^\mp_j\).
Now both indices \(i, j\) will refer to the chargino eigenstates, and \(m_{i,j} = m_{\tilde\chi^\pm_{i,j}}\).
I will also not the in the squark exchange is of opposite type to the quarks in the initial state.
From the diagrams in \cref{pc:fig:CC_diagrams} and the Feynman rules from \cref{susy:sec:feynman_rules} we get the amplitudes
\begin{subequations}
  \begin{align}
    \nonumber
    \M_{\hat{s}} = -g^2 D_Z(\hat{s})                            & \bclosed{ \bar{u}_i\gamma^\mu\pclosed{O^{\prime L}_{ij}P_L + O^{\prime R}_{ij}P_R} v_j } \bclosed{ \bar{v}_2 \gamma_\mu \pclosed{C^L_{qqZ}P_L + C^R_{qqZ}P_R} u_1 }
    \\
    \label{pc:eq:CC_Ms}
    -\frac{g^2}{\hat{s}}                                        & \bclosed{ \bar{u}_i\gamma^\mu \delta_{ij} s_W v_j } \bclosed{ \bar{v}_2 \gamma_\mu Q_e s_W u_1 },                                                                          \\
    \label{pc:eq:CC_Mt}
    \M_{\hat{t}} = -\sum_A 2g^2 D_{\tilde{q}^\prime_A}(\hat{t}) & \bclosed{ \bar{u}_i \pclosed{C_{q_g \tilde{q}_A^\prime \tilde\chi^\pm_i}^L}^{\!\ast} P_L u_1 } \bclosed{ \bar{v}_2 C_{q_g \tilde{q}_A^\prime \tilde\chi^\pm_j}^L P_R v_j}, \\
    \label{pc:eq:CC_Mu}
    \M_{\hat{u}} = 0,                                           &
  \end{align}
\end{subequations}
again mimicking the same structure of \cref{pc:eq:Ms,pc:eq:Mt}.
However, this time there is no \(\hat{u}\)-channel analogue.
Redefining the effective charges from \cref{pc:eq:eff_charges} to be
\begin{subequations}
  \begin{align*}
    Z^{XY} =   & D_Z(\hat{s}) C_{qqZ}^X O_{ij}^{\prime Y} + \frac{1}{\hat{s}} Q_e s_W^2 \delta_{ij}                               \\
    Q_A^{XY} = & C_{q_g \tilde{q}_A^\prime \tilde\chi^\pm_i}^X \bigl(C_{q_g \tilde{q}_A^\prime \tilde\chi^\pm_j}^Y\bigr)^{\!\ast}
  \end{align*}
\end{subequations}
we can rewrite the amplitudes to
\begin{subequations}
  \begin{align}
    \nonumber
    \M_{\hat{s}} = -g^2 \Big[ &
    Z^{LL} b_L^\mu(v_2, u_1){b_L}\_{\mu}(u_i, v_j) + Z^{LR} b_L^\mu(v_2, u_1){b_R}\_{\mu}(u_i, v_j)                         \\
    +                         & Z^{LR} b_{R}^\mu(v_2, u_1){b_L}\_{\mu}(u_i, v_j) +
    Z^{RR} b_R^\mu(v_2, u_1){b_R}\_{\mu}(u_i, v_j) \Big],                                                                   \\
    \M_{\hat{t}} = -2g^2      & \sum_A D_{\tilde{q}^\prime_A}(\hat{t}) \pclosed{Q_A^{LL}}^\ast b_L(u_i, u_1) b_R(v_2, v_j), \\
    \M_{\hat{u}} =            & 0.
  \end{align}
\end{subequations}

Following the procedure for calculating the total cross-section again, we find that it can be written as
\begin{equation}
  \hat\sigma^0(q\bar{q} \to \tilde\chi^\pm_i \tilde\chi^\mp_j) = \frac{4 \pi p \alpha_W^2}{\hat{s}^{\sfrac{3}{2}} N_C} \pclosed{F^\prime_{\tilde{q}^\prime} + F^\prime_Z + F^\prime_{\tilde{q}^\prime Z}},
\end{equation}
where the effective couplings are defined as\footnote{Again, two terms are missing from \(F^{\prime}_{\tilde{q}^\prime}\) as compared to \cref{pc:eq:eff_coupling_Fq}, this time due to lack of any interference between \(\hat{t}\) and \(\hat{u}\) channels. I also note that the sum over \(X, Y\) in \(F^\prime_{\tilde{q}^\prime}\) is purely to align with the earlier results, the only non-zero contribution of \(Q_A^{XY}\) is for \(X = Y = L\).}
\begin{align}
  \label{pc:eq:CC_eff_coupling_Fq}
  F^{\prime}_{\tilde{q}^\prime} = \frac{1}{2} \sum_{A,B,X,Y} & \Re{Q_A^{XY} (Q_B^{XY})^\ast} \pclosed{1 - L_2^{AB}},
\end{align}
\begin{align}
  \label{pc:eq:CC_eff_coupling_FZ}
  F^{\prime}_{Z} = & \sum_{X,Y} \cclosed{\frac{1}{12} \pclosed{\hat{s}(\hat{s}-m_i^2-m_j^2) + \hat{s} - (m_i^2-m_j^2)^2} |Z^{XY}|^2 + \delta_{XY} \hat{s}m_i m_j \Re{Z^{XX} (Z^{XX^\prime})^\ast}},
\end{align}
and
\begin{align}
  \label{pc:eq:CC_eff_coupling_FqZ}
  \nonumber
  F^\prime_{\tilde{q}^\prime Z} = & -\frac{1}{2} \sum_{A,X} \biggl\{
  \hat{s} m_i m_j \Re{Q_{A}^{XX} Z^{XX}}\frac{L(m_{\tilde{q}^\prime_A}^2)}{p \sqrt{\hat{s}}}                                                                                                                                                                               \\
                                  & \hspace{-1cm}+ \Bigl[(m_{\tilde{q}^\prime_A}^2-m_i^2)(m_{\tilde{q}^\prime_A}^2-m_j^2) \frac{L(m_{\tilde{q}^\prime_A}^2)}{p \sqrt{\hat{s}}} + (\hat{s} + m_i^2 + m_j^2 - 2m_{\tilde{q}^\prime_A}^2)\Bigr] \Re{Q_{A}^{XX} Z^{XX^\prime}}
  \biggr\}.
\end{align}

Lastly, a summary of the effective charges are summarised in \cref{pc:tab:eff_charges}.

{\renewcommand{\arraystretch}{2}
\begin{table}
  \centering
  \begin{tabular}{|c|cc|}
    \hline
                                          & \(Z^{XY}\)                                                                             & \(Q_A^{XY}\)                                                                                                   \\
    \hline
    \(\tilde\chi^0_i \tilde\chi^0_j\)     & \(D_Z(\hat{s}) C_{qqZ}^X O_{ij}^{\prime\prime Y}\)                                     & \(C_{q\tilde{q}_A \tilde\chi^0_i}^X \bigl(C_{q\tilde{q}_A \tilde\chi^0_j}^Y\bigr)^{\!\ast}\)                   \\
    % \hline
    \(\tilde\chi^0_i \tilde\chi^\pm_j\)   & \(D_W(\hat{s}) C_{qqW}^X O_{ij}^{Y}\)                                                  & \(C_{q\tilde{q}_A \tilde\chi^0_i}^X \bigl(C_{q^\prime \tilde{q}_A \tilde\chi^\pm_j}^Y\bigr)^{\!\ast}\)         \\
    % \hline
    \(\tilde\chi^\pm_i \tilde\chi^\mp_j\) & \(D_Z(\hat{s}) C_{qqZ}^X O_{ij}^{\prime Y} + \frac{1}{\hat{s}} Q_e s_W^2 \delta_{ij}\) & \(C_{q\tilde{q}^\prime_A \tilde\chi^\pm_i}^X \bigl(C_{q\tilde{q}^\prime_A \tilde\chi^\pm_j}^Y\bigr)^{\!\ast}\) \\
    \hline
  \end{tabular}
  \caption{Table of the effective charges defined for each type of electroweakino pair production process.}
  \label{pc:tab:eff_charges}
\end{table}


}



\section{NLO Corrections}
In this section, I will take a look at the NLO QCD corrections associated with the production of a pair of neutralinos.
I will perform the computation of the NLO corrections to the higgsino cross-section, as this goes through an intermediate \(Z\)-boson state, making it kinematically simpler.
The corrections to the gaugino cross-section is beyond the scope of this thesis, but I will comment on a procedure for doing it.

\subsection{Factorisation}
As we will only look at NLO contributions to the \(\hat{s}\)-channel contribution through a \(Z\)-boson, we can do a trick to simplify the process and its corrections.
This trick is factorisation, which involves splitting the total cross-section into the two separate processes of the production of an off-shell \(Z\)-boson, and its subsequent decay into two neutralinos.
Seeing as we are calculating the inclusive cross-section, I include the potential emission of another particle (gluon or quark) along with the \(Z\)-boson production.

To start off, we can factorise the \(d\)-dimensional differential \(2\to 3\) phase space into two processes by adding an intermediate momentum \(q\) with `mass' squared \(Q^2\).
We end up with
\begin{align}
  \label{pc:eq:dPi_factorised}
  \nonumber
  \mathrm{d}q \delta^d\pclosed{k+q-P} \mathrm{d}Q^2 \delta(q^2-Q^2) \mathrm{d}\Pi_{2\to 3} = \frac{1}{(2\pi)^{2d-3}}\mathrm{d}^{d-1}\vec{p}_i\,\mathrm{d}^{d-1}\vec{p}_j\,\mathrm{d}^{d-1}\vec{k}\,\mathrm{d}^{d-1}\vec{q}\,\,\mathrm{d}Q^2 \\
  \nonumber
  \times \frac{1}{16E_iE_j\omega q^0} \delta^d\pclosed{q+k - k_i-k_j}\delta^d\pclosed{p_i+p_j+k - k_i-k_j}                                                                                                                                  \\
  \equiv \frac{1}{2\pi} \mathrm{d}\Pi_H\, \mathrm{d}\Pi_N\, \mathrm{d}Q^2,\hspace{6.7cm}
\end{align}
where
\begin{subequations}
  \begin{align}
    \label{pc:eq:Pi_H}
    \mathrm{d}\Pi_H = & \frac{\mathrm{d}^{d-1}\vec{k}\, \mathrm{d}^{d-1}\vec{q}}{(2\pi)^{d-2}} \frac{1}{4\omega q^0} \delta^d(q+k-k_i-k_j), \\
    \label{pc:eq:Pi_N}
    \mathrm{d}\Pi_N = & \frac{\mathrm{d}^{d-1}\vec{p}_i\, \mathrm{d}^{d-1}\vec{p}_j}{(2\pi)^{d-2}} \frac{1}{4E_iE_j} \delta^d(p_i+p_j-q),
  \end{align}
\end{subequations}
which are recognisable as differential phase spaces for a \(2\to 2\) processes going from momenta \(k_i+k_j \to q+k\) and a \(1\to 2\) phase space going from \(q\ \to p_i+p_j\).
The total phase space integrates over all possible off-shell masses \(Q^2\) for the intermediate momentum \(q\).

So, we have factorised the differential phase space of the differential cross-section \cref{pc:eq:diff_xsec_formula}, but it remains to factorise the amplitude part \(\abs{\M}^2\) into part only dependent on either \(q, k\) or \(p_i, p_j\).
Looking at the tree-level amplitudes \cref{pc:eq:Ms,pc:eq:Mt,pc:eq:Mu} that this happens neatly with the \(\hat{s}\)-channel contribution \cref{pc:eq:Ms}.
It has the Lorentz structure \(\M_s = D_Z(\hat{s}) g_{\mu\nu} \bclosed{\bar{v}(k_j)\Gamma_{qqZ}^\mu u(k_i)} \bclosed{\bar{u}(p_i)\Gamma_{Z\tilde{\chi}_i^0 \tilde{\chi}_j^0}^\nu v(p_j)}\).
The two terms in brackets are individually only dependent on couplings and the momenta of either the initial partons or the final neutralinos.
In fact, they individually take the form of the processes
\begin{subequations}
  \begin{align}
    \label{pc:eq:qq_Z}
    \inputtikz[baseline=(blob.base)]{qqZ}   & = \bclosed{\bar{v}(k_j) i\Gamma_{qqZ}^\mu u(k_i)} \epsilon^\ast_\mu(q)                           \\
    \label{pc:eq:Z_chichi}
    \inputtikz[baseline=(blob.base)]{chijZ} & = \bclosed{\bar{u}(p_i) i\Gamma_{Z\tilde{\chi}_i^0 \tilde{\chi}_j^0}^\mu v(p_j)} \epsilon_\mu(q)
  \end{align}
\end{subequations}

Squaring it, we can write a differential cross-section from \cref{pc:eq:diff_xsec_formula} as
\begin{equation}
  \d[Q^2]{\hat\sigma} = \frac{1}{4\pi \hat{s}} \abs{D_Z(\hat{s})}^2 H^{\mu\nu}N_{\mu\nu},
\end{equation}
where
\begin{subequations}
  \begin{align}
    \epsilon_{\mu}(q)\epsilon^\ast_{\nu}(q) H^{\mu\nu} = & \integral{\Pi_H} \abs{\M(q\bar{q} \to Z^\ast)}^2,          \\
    \epsilon_{\mu}^\ast(q)\epsilon_{\nu}(q) N^{\mu\nu} = & \integral{\Pi_N} \abs{\M(Z^\ast \to \nino[i] \nino[j])}^2.
  \end{align}
\end{subequations}
Writing it in this way showcases that \(Q^2\) is not a real degree of freedom in the cross-section, but rather put in by hand.
However, it will become relevant when we look at the \emph{inclusive} cross-section, where we take into account any contributions from processes resulting in the neutralino pair and something else.
Seeing as at least one more vertex is necessary to produce another particle, these contributions must come in at NLO.
The NLO QCD contribution from a process producing neutralino pair and a strongly interacting particle can only come from the quark tensor \(H\^{\mu\nu}\), as the neutralino tensor \(N\^{\mu\nu}\) contains no strongly interacting particles at LO.
\(Q^2\) will then parametrise the extra degree of freedom when going from a \(2\to 1\) phase space to a \(2\to 2\) phase space in the hadronic tensor.
\medskip

Since the neutralino tensor will not receive any higher order corrections to the order we will calculate, we can take a look at it already.
It describes the decay of an off-shell \(Z\)-boson, and as such it can only depend on its four-momentum and on the metric, so it can be parametrised as
\begin{equation}
  \label{pc:eq:neutralino_tensor}
  N\^{\mu\nu} = N_0(Q^2) \g + N_1(Q^2) \frac{q\^\mu q\^\nu}{Q^2},
\end{equation}
where \(N_0, N_1\) are scalar functions of the off-shell \(Z\)-boson mass \(Q\).
Generalising the hadronic tensor to include the inclusive production of some strongly interacting particle(s) \(X\), we can write it as
\begin{equation}
  H\^{\mu\nu} = \integral{\Pi_H} \M\^\mu (\M\^\nu)^\ast,
\end{equation}
where \(\M\^\mu\) is defined such that \(\epsilon^\ast_\mu(q) \M\^\mu = \M(q\bar{q} \to Z^\ast (+ X))\).
Now, the form of the neutralino tensor \cref{pc:eq:neutralino_tensor} is particularly convenient, as due to the Ward identity, \(q\_\mu \M\^\mu(q\bar{q} \to Z^\ast (+ X)) = 0\).
This means that only the coefficient \(N_0\) is necessary to calculate.
Using that \(\sum_\text{pol.} \epsilon\_\mu(q) \epsilon_\nu^\ast(q) = -g\_{\mu\nu}\) we can then write the cross-section on a convenient form
\begin{equation}
  \label{pc:eq:dsigma_dQ2_1}
  \d[Q^2]{\hat\sigma} = -\frac{1}{4\pi \hat{s}} |D_Z(\hat{s})|^2 N_0 \integral{\Pi_H} \sum_\text{pol.}|\M(q\bar{q} \to Z^\ast (+ X))|^2.
\end{equation}

We can calculate the coefficient \(N_0(Q^2)\) already, and given that there are no loops or divergences that need regularisation, we can compute it in \(d=4\) dimensions.
From \cref{pc:eq:neutralino_tensor} we can find it is given by
\begin{equation}
  N_0(Q^2) = \frac{1}{3} \pclosed{g\_{\mu\nu} - \frac{q\_\mu q\_\nu}{Q^2}} N\^{\mu\nu}.
\end{equation}
Furthermore \(N\^{\mu\nu} = \integral{\Pi_N} \M\^\mu (\M\^\nu)^\ast\), where \(\epsilon\_\mu(q) \M\^\mu = \M(Z^\ast \to \tilde\chi^0_i \tilde\chi^0_j)\), and is given by
\begin{equation}
  \M\^\mu = g \bclosed{(O_{ij}^{\prime\prime L})^\ast b^\mu_L(u_i, v_j) + (O_{ij}^{\prime\prime R})^\ast b^\mu_R(u_i, v_j)}.
\end{equation}
Summing over the spins, using the relations \cref{pc:eq:bilinear_conjugate,pc:eq:vector_bilinear_trace}, we get
\begin{align}
  \nonumber
  N\^{\mu\nu} = & g^2 \integral{\Pi_N} \sum_\text{spin} \bclosed{(O_{ij}^{\prime\prime L})^\ast b^\mu_L(u_i, v_j) + (O_{ij}^{\prime\prime R})^\ast b^\mu_R(u_i, v_j)} \bclosed{O_{ij}^{\prime\prime L} b^\nu_L(v_j, u_i) + O_{ij}^{\prime\prime R} b^\nu_R(v_j, u_i)} \\
  \nonumber
  =             & 2g^2 \integral{\Pi_N} \Bigl\{\pclosed{|O_{ij}^{\prime\prime L}|^2 + |O_{ij}^{\prime\prime R}|^2} \pclosed{p_i^\mu p_j^\nu - (p_i\cdot p_j)\g + p_i^\nu p_j^\mu}                                                                                     \\
                & \qquad - 2\Re{O_{ij}^{\prime\prime L} (O_{ij}^{\prime\prime R})^\ast} m_i m_j \g \Bigr\}
\end{align}
The contraction of the four-momenta are fixed by momentum conservation \(q = p_i + p_j\), meaning \(p_i \cdot p_j = \frac{1}{2}(Q^2-m_i^2-m_j^2)\), leaving
\begin{align}
  N_0(Q^2) = & -g^2 \integral{\Pi_N} \Bigl\{\frac{Q^2(Q^2-m_i^2-m_j^2) + Q^4 - (m_i^2-m_j^2)^2}{3Q^2} \pclosed{|O_{ij}^{\prime\prime L}|^2+|O_{ij}^{\prime\prime R}|^2} \\
             & \qquad + 4m_i m_j \Re{O_{ij}^{\prime\prime L} (O_{ij}^{\prime\prime R})^\ast} \Bigr\}.
\end{align}
As we have worked the expression so far, we have used the momentum-conserving Dirac delta-function in the phase space integral \(\integral{\Pi_N}\), so there is no longer any dependence on \(p_{i/j}\).
This lets us do the integral over the phase space in \cref{pc:eq:Pi_N} in isolation, leaving in the centre-of-mass frame of the \(Z\)-boson
\begin{align}
  \integral{\Pi_N} = & \int\! \frac{\mathrm{d}^3\vec{p}_i \mathrm{d}^3\vec{p}_j}{(2\pi)^2} \frac{1}{4E_i E_j} \delta(E_i+E_j-Q) \delta^3(p_i+p_j) \\
  =                  & \frac{1}{16\pi^2} \integral{^3\vec{p}_i} \frac{1}{E_i E_j} \delta(E_i + E_j - Q)|_{\vec{p}_j = -\vec{p}_i}                 \\
  =                  & \frac{p}{4\pi Q},
\end{align}
where in the last equality we switch to polar coordinates and use the identity \(\integral{x} f(x) \delta(g(x)) = \sum_{x_0} \frac{f(x_0)}{|g^\prime(x_0)|}\) where \(x_0\) are the zero(es) of \(g(x)\).
The momentum is understood to be
\begin{equation}
  p(Q^2) = \frac{\sqrt{\lambda}(Q^2, m_i, m_j)}{2Q}.
\end{equation}
Finally, inserting this into the cross-section expression \cref{pc:eq:dsigma_dQ2_1}, averaging over initial the four quark spins and \(N_C^2\) colours, and inserting a symmetry factor of a half in the case where the final state particles are identical, we get
\begin{align}
  \label{pc:eq:dsigma_dQ2_2}
  \nonumber
  \d[Q^2]{\hat\sigma} = & \frac{\alpha_W p}{16 N_C^2 \pi \hat{s} Q} \pclosed{\frac{1}{2}}^{\delta_{ij}} |D_Z(\hat{s})|^2 \pclosed{K_1\pclosed{|O_{ij}^{\prime\prime L}|^2+|O_{ij}^{\prime\prime R}|^2} + K_2 \Re{O_{ij}^{\prime\prime L} (O_{ij}^{\prime\prime R})^\ast}} \\
                        & \times \integral{\Pi_H} \sum_{\stackrel{\text{spins}}{\text{colours}}} \sum_\text{pol.}|\M(q\bar{q} \to Z^\ast (+ X))|^2,
\end{align}
where \(\alpha_W = \frac{g^2}{4\pi}\) and I have defined
\begin{subequations}
  \begin{align}
    K_1 = & \frac{Q^2(Q^2-m_i^2-m_j^2) + Q^4 - (m_i^2-m_j^2)^2}{3Q^2}, \\
    K_2 = & 4m_i m_j.
  \end{align}
\end{subequations}


\subsection{Higgsino Corrections/Quark Tensor Renormalisation}

% \begin{figure}[ht!]
%   \centering
%   \inputtikz{s_channel_se_1}
%   \inputtikz{s_channel_se_2}
%   \caption{}
%   \label{pc:fig:s_channel_se}
% \end{figure}
\begin{figure}[ht!]
  \centering
  \begin{subfigure}{.49\textwidth}
    \centering
    \inputtikz{quark_renormalisation/gluon_triangle}
    \caption{}
    \label{pc:subfig:gluon_triangle}
  \end{subfigure}
  \begin{subfigure}{.49\textwidth}
    \centering
    \inputtikz{quark_renormalisation/gluino_triangle}
    \caption{}
    \label{pc:subfig:gluino_triangle}
  \end{subfigure}
  \caption{The two triangle diagrams contributing to NLO QCD corrections to the quark tensor.}
  \label{pc:fig:NLO_triangles}
\end{figure}

Let us first look at the amplitude for the exchange of a virtual gluon between the quarks.
Labelling the quark colours \(a, b\) and the gluon indices \(i\), we get
\begin{align}
  % \M_{g} = & -i g_s g \bar{v}_2 (\gamma^\sigma T^a_{jk}) \loopintegral[d]{q} \frac{1}{q_l^2} \frac{\slashed{q}_l+\slashed{k}_j}{(k_j+q_l)} (C_{qqZ}^L \gamma^\mu P_L + C_{qqZ}^R \gamma^\mu P_R) \frac{\slashed{q} - \slashed{q}_l - \slashed{k}_j}{(q-q_l-k_j)^2} (\gamma^\nu T^a_{ki}) u_1
  \M_g = & -i\delta_{ab} C_F \epsilon_\mu^\ast(q) \mu^{4-d} g_s^2 g \loopintegral[d]{\ell} \frac{\bar{v}_2 \gamma^\nu (\slashed{p}-\slashed{k}_j) \gamma^\mu (\slashed{p}+\slashed{k}_i) \gamma_\nu u_1}{\ell^2(\ell+k_i)^2(\ell-k_j)^2},
\end{align}
where I have summed over the \(SU(3)\) generators \(T_{ac}^k T_{cb}^k = C_F \delta_{ab}\).
Doing Passerino-Veltman reduction from \cref{qft:subsec:pave} using the \verb|FeynCalc| package, we can rewrite this to
\begin{equation}
  \M_g = \frac{C_F g_s^2}{16\pi^2} \bclosed{(7-d)B_0(\hat{s},0,0) - 4B_0(0,0,0) + 2\hat{s}C_0(0,\hat{s},0,0,0,0)} \M_0,
\end{equation}
and I have defined
\begin{equation}
  \label{pc:eq:LO_qq_Z}
  \M_0 = \delta_{ab} g \epsilon_\mu^\ast(q) \bclosed{C_{qqZ}^L b_L^\mu(v_2, u_1) + C_{qqZ}^R b_R^\mu(v_2, u_1)}.
\end{equation}
This last bit \cref{pc:eq:LO_qq_Z} is recognisable as the tree-level amplitude for \(q\bar{q} \to Z^\ast\).
The phase space is simpler in this case without any real emission: In the centre-of-mass frame of the quarks we have
\begin{equation}
  \integral{\Pi_H} = \loopintegral[d-1]{\vec{q}} \frac{1}{2\sqrt{Q^2+|\vec{q}|^2}} (2\pi)^d \delta(Q - \sqrt{\hat{s}}) \delta^{d-1}(\vec{q}),
\end{equation}
which w.r.t the integration variable \(Q^2\) in \cref{pc:eq:dPi_factorised} becomes
\begin{equation}
  \integral{\Pi_H} = \frac{2\pi}{\hat{s}} \delta(1-z),
\end{equation}
where \(z = \frac{Q^2}{\hat{s}}\).
The contribution to the quark tensor will to NLO come from the interference between \(\M_g\) and \(\M_0\), and so it follows that the contribution to the total cross-section contribution from \cref{pc:eq:dsigma_dQ2_2} becomes
\begin{align}
  \nonumber
  \d[Q^2]{\hat\sigma_g} = & \frac{C_F \alpha_s}{\pi} \pclosed{\frac{4\pi \mu^2}{\hat{s}}}^\epsilon \frac{\Gamma(1-\epsilon)}{\Gamma(1-2\epsilon)} (1-\epsilon)                                       \\
                          & \times \Re{2B_0(0,0,0) - \Bigl(\frac{3}{2}+\epsilon\Bigr)B_0(\hat{s},0,0) - \hat{s} C_0(0,\hat{s},0,0,0,0)} \hat\sigma_B F^{\prime\prime}_Z \frac{\delta(1-z)}{\hat{s}},
\end{align}
where \(\alpha_s = \frac{g_s^2}{4\pi}\).

Doing the same for the gluino exchange diagrams \cref{pc:subfig:gluino_triangle} is a bit more involved, as the loop particles are massive, and the couplings are complex.
Working it out in Mathematica, we get
\begin{align}
  \nonumber
  \M_{\tilde{g}} = & i2\delta_{ab} C_F \epsilon_\mu^\ast(q) \mu^{4-d} g_s^2 g (C_{qqZ}^L (R_{A1}^{\tilde{q}})^\ast R_{B1}^{\tilde{q}} + C_{qqZ}^R (R_{A2}^{\tilde{q}})^\ast R_{B2}^{\tilde{q}}) \\
                   & \loopintegral[d]{\ell} \frac{\bar{v}_2 N (2\ell-k_i+k_j)\^\mu u_1}{(\ell^2-m_{\tilde{g}}^2) ((\ell-k_i)^2 - m_A^2)((\ell+k_j)^2 - m_B^2)},
\end{align}
where
\begin{equation}
  N = \bclosed{\pclosed{R_{A1}^{\tilde{q}}(R_{B1}^{\tilde{q}})^\ast\slashed{\ell} - m_{\tilde{g}}R_{A1}^{\tilde{q}}(R_{B2}^{\tilde{q}})^\ast}P_L + \pclosed{R_{A2}^{\tilde{q}}(R_{B2}^{\tilde{q}})^\ast\slashed{\ell} - m_{\tilde{g}}R_{A2}^{\tilde{q}}(R_{B1}^{\tilde{q}})^\ast}P_R}.
\end{equation}

Putting it together and doing Passerino-Veltman reduction in Mathematica, we end up with a hadronic tensor
\begin{align}
  H^{\mu\nu}_{\tilde{g}} = & 8\pi (d-2) C_F \alpha_s \alpha_W \delta_{ab} \sum_{AB} \left|R^{\tilde{q}}_{A1}(R^{\tilde{q}}_{B1})^\ast C_{qqZ}^L + |R^{\tilde{q}}_{A2}(R^{\tilde{q}}_{B2})^\ast C_{qqZ}^R\right|^2 \\
                           & \times \Re{C_{00}(0,0,\hat{s};m_A^2, m_{\tilde{g}}^2, m_B^2)} \frac{\delta(1-z)}{\hat{s}}.
\end{align}
The total contribution to the cross-section is then
\begin{align}
  \d[Q^2]{\hat\sigma_{\tilde{g}}} = & \frac{C_F \alpha_s}{\pi} \pclosed{\frac{4\pi \mu^2}{\hat{s}}}^\epsilon \frac{\Gamma(1-\epsilon)}{\Gamma(1-2\epsilon)} (1-\epsilon) \hat\sigma_B \frac{F^{\prime\prime}_Z}{((C_{qqZ}^L)^2 + (C_{qqZ}^R)^2)} \tilde{C}^{\prime\prime}_Z \frac{\delta(1-z)}{\hat{s}},
\end{align}
where
\begin{align}
  \tilde{C}^{\prime\prime}_Z = 2\sum_{AB} \pclosed{R^{\tilde{q}}_{A1}(R^{\tilde{q}}_{B1})^\ast C_{qqZ}^L + R^{\tilde{q}}_{A2}(R^{\tilde{q}}_{B2})^\ast C_{qqZ}^R}^2 \Re{C_{00}(0,0,\hat{s},m_A^2,m_{\tilde{g}}^2,m_B^2)}.
\end{align}

For future reference, the total virtual contribution to the cross-section is
\begin{equation}
  \label{pc:eq:sigma_v}
  \d[Q^2]{\hat\sigma_v} = \d[Q^2]{\hat\sigma_g} + \d[Q^2]{\hat\sigma_{\tilde{g}}}.
\end{equation}


\subsection{Counterterms}
\begin{figure}[ht!]
  \centering
  \begin{subfigure}{.49\textwidth}
    \centering
    \inputtikz{quark_renormalisation/quarkSE_gluon}
    \caption{}
    \label{pc:subfig:quarkSE_gluon}
  \end{subfigure}
  \begin{subfigure}{.49\textwidth}
    \centering
    \inputtikz{quark_renormalisation/quarkSE_gluino}
    \caption{}
    \label{pc:subfig:quarkSE_gluino}
  \end{subfigure}
  \caption{The QCD quark self-energy diagrams used for extracting quark renormalisations.}
  \label{pc:fig:quarkSE}
\end{figure}
\todo{Mention lack of vertex renormalisation. Why is mass renormalisation the only renormalisation required?}\\
The UV divergences in the virtual contribution \cref{pc:eq:sigma_v} can be cancelled through renormalisation of the quark field wave functions.
Renormalisation of the \(Z\)-boson field wave function will not contribute at order \(\alpha_s\), so it will suffice to renormalise the quarks.
I will follow the on-shell renormalisation procedure of the seciton \cref{qft:subsec:counterterms}, but I will not make use of any mass counterterm.
Reading off the amplitudes in \cref{pc:fig:quarkSE} and inserting the Passerino-Veltman loop functions we have
\begin{align}
  \label{pc:eq:quarkSE}
  \nonumber
  \Sigma_q(\slashed{p}) = & \frac{g_s^2 C_F \delta_{ab}}{16\pi^2} \biggl\{
  \frac{d-2}{2} \slashed{p} B_0(0,0,0)                                                                                                                                                \\
  \nonumber
                          & - 2\sum_A \bigl[
  \slashed{p} (|R^{\tilde{q}}_{A1}|^2P_L + |R^{\tilde{q}}_{A2}|^2P_R)B_1(0,m_{\tilde{g}}^2,m_A^2)                                                                                     \\
                          & \qquad + m_{\tilde{g}} (R^{\tilde{q}}_{A1} (R^{\tilde{q}}_{A2})^\ast P_L + R^{\tilde{q}}_{A2} (R^{\tilde{q}}_{A1})^\ast P_R) B_0(0, m_{\tilde{g}}, m_A^2)
  \bigr]\biggr\}.
\end{align}
Renormalising the quark field wave function chirally, we have
\begin{equation}
  \psi \to \sqrt{Z_L} P_L \psi + \sqrt{Z_R} P_R \psi,
\end{equation}
which expanded into counterterms \(Z_{L/R} = 1 + \delta_{L/R}\) yields through the on-shell conditions from \cref{qft:subsec:on_shell_renormalsiation}\footnote{The keen reader will notice that the counterterms do not cancel the divergences proportional to \(m_{\tilde{g}}\) in \cref{pc:eq:quarkSE}. However, this is not necessary, as due to the unitarity of the quark mixing matrices, the divergences cancel after summing over squark eigenstates \(A\).}
\begin{subequations}
  \begin{align}
    \Re{\delta_L} = & \frac{g_s^2 C_F \delta_{ab}}{16\pi^2} \biggl\{
    -\frac{d-2}{2} B_0(0,0,0) + 2\sum_A|R^{\tilde{q}}_{A1}|^2B_1(0,m_{\tilde{g}}^2,m_A^2)
    \biggr\} ,                                                       \\
    \Re{\delta_R} = & \frac{g_s^2 C_F \delta_{ab}}{16\pi^2} \biggl\{
    -\frac{d-2}{2} B_0(0,0,0) + 2\sum_A|R^{\tilde{q}}_{A2}|^2B_1(0,m_{\tilde{g}}^2,m_A^2)
    \biggr\}.
  \end{align}
\end{subequations}
The counterterm amplitude then becomes
\begin{equation}
  \M_{c.t.} = g \delta_{ab} \epsilon_\mu^\ast(q) \bclosed{\Re{\delta_L} C_{qqZ}^L b_L^\mu(v_2,u_1) + \Re{\delta_R} C_{qqZ}^R b_R^\mu(v_2,u_1)},
\end{equation}
essentially turning \(C_{qqZ}^X \to \Re{\delta_X} C_{qqZ}^X\) or, consequently, the effective charge \(Z^{XY} \to \Re{\delta_X} Z^{XY}\).
The contribution to the total cross-section comes again through interference with the tree-level amplitude \cref{pc:eq:LO_qq_Z}, yielding
\begin{equation}
  \d[Q^2]{\hat\sigma_{c.t.}} = \frac{C_F \alpha_s}{2\pi} \hat\sigma_B \tilde{F}^{\prime\prime}_Z,
\end{equation}
where the new effective coupling is defined
\begin{align}
  \nonumber
  \tilde{F}^{\prime\prime}_Z = & -(1-\epsilon)B_0(0,0,0) F^{\prime\prime}_Z + 2 \sum_A B_1(0, m_{\tilde{g}}^2, m_A^2)                                                    \\
  \nonumber
                               & \times \biggl\{
  \frac{1}{12} \pclosed{\hat{s}(\hat{s}-m_i^2-m_j^2) + \hat{s} - (m_i^2-m_j^2)^2} \sum_{X} \pclosed{|R^{\tilde{q}}_{A1}|^2|Z^{LX}|^2 + |R^{\tilde{q}}_{A2}|^2|Z^{RX}|^2} \\
                               & + \hat{s}m_i m_j \pclosed{|R^{\tilde{q}}_{A1}|^2\Re{Z^{LL} (Z^{LR})^\ast} + |R^{\tilde{q}}_{A2}|^2\Re{Z^{RR} (Z^{RL})^\ast}}
  \biggr\}.
\end{align}

The cancellation of UV divergences between \(\d[Q^2]{\hat\sigma_v}\) and \(\d[Q^2]{\hat\sigma_{c.t.}}\) has been confirmed symbolically in Mathematica.


\subsection{Real Emission}
% \begin{figure}[ht!]
%   \centering
%   \begin{subfigure}{0.32\textwidth}
%     \centering
%     \inputtikz{real_emission/gluon_emission}
%     \caption{}
%     \label{pc:subfig:gluon_emission}
%   \end{subfigure}
%   \begin{subfigure}{0.32\textwidth}
%     \centering
%     \inputtikz{real_emission/quark_emission}
%     \caption{}
%     \label{pc:subfig:quark_emission}
%   \end{subfigure}
%   \begin{subfigure}{0.32\textwidth}
%     \centering
%     \inputtikz{real_emission/antiquark_emission}
%     \caption{}
%     \label{pc:subfig:antiquark_emission}
%   \end{subfigure}
%   \caption{Emission diagrams of two partons producing an of-shell \(Z\)-boson with either a gluon or quark.}
%   \label{pc:fig:real_emission}
% \end{figure}

The massless gluon and quarks in the loop of \cref{pc:subfig:gluon_triangle} will result in IR divergences.
These divergences are cancelled by soft emission of strongly coupling massless particles.
For two incoming partons \(i, j\) we shall take into account the inclusive production of a neutralino pair with either a gluon or a quark.
Again, the only QCD NLO contributions can come from adjusting the quark tensor \(H\^{\mu\nu}\) as no strong interaction vertex can be added to the neutralino tensor.

\subsubsection*{Gluon Emission}
\begin{figure}[ht!]
  \centering
  \begin{subfigure}{0.49\textwidth}
    \centering
    \inputtikz{real_emission/gluon_emission_t}
    \caption{}
    \label{pc:subfig:gluon_emission_t}
  \end{subfigure}
  \begin{subfigure}{0.49\textwidth}
    \centering
    \inputtikz{real_emission/gluon_emission_u}
    \caption{}
    \label{pc:subfig:gluon_emission_u}
  \end{subfigure}
  \caption{Gluon emission diagrams}
  \label{pc:fig:gluon_emission}
\end{figure}

The production of a gluon with four-momentum \(k\^\mu\) together with an off-shell Z-boson with \(q\^\mu\) through two partons goes through a quark-antiquark pair as seen in \cref{pc:fig:gluon_emission}.
The matrix element for this is
\begin{align}
  \nonumber
  \M_r = & g_s g T_{ab}^k \bar{v}_2 \biggl\{
  \gamma^\mu \bigl(C_{qqZ}^L P_L + C_{qqZ}^R P_R\bigr) \frac{\slashed{q}-\slashed{k}_j}{(q-k_j)^2} \gamma^\nu            \\
         & + \gamma^\nu \frac{\slashed{k}-\slashed{k}_j}{(k-k_j)^2} \gamma^\mu \bigl(C_{qqZ}^L P_L + C_{qqZ}^R P_R\bigr)
  \biggr\} u_1 \epsilon_\mu^\ast(q) \epsilon_\nu^\ast(k),
\end{align}
where \(a,b\) are the colour indices of the quark and antiquark respectively and \(k\) is the gluon index.
Now, defining the Mandelstam variables
\begin{subequations}
  \label{pc:eq:mandelstam_factorised}
  \begin{align}
    \hat{s} \equiv & (k_i+k_j)^2 = (q+k)^2, \\
    \hat{t} \equiv & (k_i-k)^2 = (k_j-q)^2, \\
    \hat{u} \equiv & (k_i-q)^2 = (k_j-k)^2,
  \end{align}
\end{subequations}
where I note that \(\hat{t}\) and \(\hat{u}\) are \emph{not} the same as the definitions \cref{pc:eq:mandelstam_LO}.
Squaring the matrix element and summing over spins, external polarisations and colours, we get
\begin{align}
  |\M_r|^2 = & (d-2) N_C C_F g_s^2 g_W^2 ((C_{qqZ}^L)^2 + (C_{qqZ}^R)^2) \pclosed{2(d-4) + (d-2)\left(\frac{\hat{t}}{\hat{u}}+\frac{\hat{u}}{\hat{t}}\right) + \frac{4Q^2\hat{s}}{\hat{t}\hat{u}}}.
\end{align}
The phase space is a \(2\to 2\) process as in the LO case, so we can use the differential cross-section definition from \cref{pc:eq:dsigma_dy}.
Defining the variable \(z\) such that \(Q^2 = z \hat{s}\) and using the \(y\) variable, we can express the Mandelstam variables as
\begin{subequations}
  \begin{align}
    \hat{s} = & \frac{Q^2}{z},            \\
    \hat{t} = & -\frac{Q^2}{z}(1-z)(1-y), \\
    \hat{u} = & -\frac{Q^2}{z} (1-z)y.
  \end{align}
\end{subequations}
The phase space integral then looks like
\begin{align}
  \nonumber
  \integral{\Pi_H} \sum_{\substack{\text{spin} \\\text{colour}}} |\M_r|^2 = \frac{N_C C_F \mu^{4-d} g_s^2 g_W^2}{\pclosed{4\pi}^{\frac{d-2}{2}}} \frac{1}{\Gamma\pclosed{\frac{d-2}{2}}} \frac{p^{d-3}}{8\hat{s}\sqrt{\hat{s}}} (d-2) ((C_{qqZ}^L)^2 + (C_{qqZ}^R)^2) \\
  \times \integral[_0^1]{y} (y(1-y))^{d-4} \pclosed{2(d-4) + (d-2)\pclosed{\frac{1-y}{y} + \frac{y}{1-y}} + \frac{4z}{(1-z)^2 y(1-y)}},
\end{align}
which evaluated in \(d=4-2\epsilon\) spacetime dimensions becomes
\begin{align}
  \nonumber
  \integral{\Pi_H} \sum_{\substack{\text{spin} \\\text{colour}}} |\M_r|^2 = (1-\epsilon) \frac{2 \pi^{\sfrac{3}{2}} N_C C_F \alpha_s \alpha_W}{\Gamma(1-\epsilon) \Gamma(\frac{3}{2}-\epsilon)} ((C_{qqZ}^L)^2 + (C_{qqZ}^R)^2) \pclosed{\frac{4\pi \mu^2}{Q^2}}^{\epsilon} \pclosed{\frac{4z}{(1-z)^2}}^\epsilon \\
  \times \Gamma(-\epsilon) \pclosed{(1+z)^2 \epsilon^2 - (5z^2-6z+7)\epsilon - 4z(1-z) + 2},
\end{align}
where we have used the integral definition of the Euler-Beta function
\begin{equation}
  \integral[_0^1]{y} y^{a-1}(1-y)^{b-1} = \frac{\Gamma(a)\Gamma(b)}{\Gamma(a+b)}.
\end{equation}
% Expanding
% \[\frac{\Gamma(-\epsilon)}{\Gamma(1-\epsilon)\Gamma(\sfrac{3}{2}-\epsilon)} = -\frac{2}{\sqrt{\pi}}\pclosed{\frac{1}{\epsilon} + \psi^0(\sfrac{3}{2}) + O(\epsilon)},\]
% we get
Expanding \(z^{\epsilon} = 1 + \epsilon \ln z + O(\epsilon^2)\) and using the definition of the plus distribution to write
\begin{equation}
  (1-z)^{1-2\epsilon} = -\frac{1}{2\epsilon} \delta(1-z) + \plusdist{\frac{1}{1-z}} -2\epsilon \plusdist{\frac{\ln(1-z)}{1-z}} + O(\epsilon^2),
\end{equation}
we can rewrite to
\begin{align}
  \nonumber
  \integral{\Pi_H} \sum_{\substack{\text{spin} \\\text{colour}}} |\M_r|^2 = 16 \pi N_C C_F \alpha_s \alpha_W ((C_{qqZ}^L)^2 + (C_{qqZ}^R)^2) \pclosed{\frac{4\pi \mu^2}{Q^2}}^{\epsilon} (1-\epsilon) \frac{\Gamma(1-\epsilon)}{\Gamma(1-2\epsilon)} \\
  \times \pclosed{\frac{1}{\epsilon^2} \delta(1-z) - \frac{1}{\epsilon} (1+z^2)\plusdist{\frac{1}{1-z}} - \frac{(1+z^2)\ln z}{1-z} + 2(1+z^2)\plusdist{\frac{\ln(1-z)}{1-z}}}.
\end{align}
This yields a total contribution to the cross-section
\begin{align}
  \nonumber
  \d[Q^2]{\hat\sigma_{qq}} = & \frac{\hat\sigma_B}{\hat{\sigma}} \frac{\alpha_s C_F}{\pi} \pclosed{\frac{4\pi \mu^2}{Q^2}}^{\epsilon} (1-\epsilon) \frac{\Gamma(1-\epsilon)}{\Gamma(1-2\epsilon)} F^{\prime\prime}_Z \\
                             & \times \pclosed{\frac{1}{\epsilon^2} \delta(1-z) - \frac{1}{\epsilon} (1+z^2)\plusdist{\frac{1}{1-z}} - \frac{(1+z^2)\ln z}{1-z} + 2(1+z^2)\plusdist{\frac{\ln(1-z)}{1-z}}}.
\end{align}

\subsubsection*{Quark Emission}
\begin{figure}[ht!]
  \centering
  \begin{subfigure}{0.49\textwidth}
    \centering
    \inputtikz{real_emission/quark_emission_s}
    \caption{}
    \label{pc:subfig:quark_emission_s}
  \end{subfigure}
  \begin{subfigure}{0.49\textwidth}
    \centering
    \inputtikz{real_emission/quark_emission_t}
    \caption{}
    \label{pc:subfig:quark_emission_t}
  \end{subfigure}
  \caption{Quark emission diagrams}
  \label{pc:fig:quark_emission}
\end{figure}
It is also possible for two partons to produce a quark or an antiquark together with an off-shell \(Z\)-boson.
This process goes through a gluon and a(n) quark/antiquark, as shown in \cref{pc:fig:quark_emission}.
The amplitude for this looks like
\begin{align}
  \nonumber
  \M_q = & g g_s T_{ab}^k \epsilon_\mu^\ast(q) \epsilon\_\nu(k_j) \bar{u}(k) \biggl\{
  \frac{C_{qqZ}^L (\gamma\^\mu \slashed{k}_j \gamma\^\nu + 2k_i^\nu \gamma\^\mu )P_L + C_{qqZ}^R (\gamma\^\mu \slashed{k}_j \gamma\^\nu + 2k_i^\nu \gamma\^\mu )P_R }{\hat{s}}                 \\
         & \qquad - \frac{C_{qqZ}^L (\gamma\^\nu \slashed{k}_j \gamma\^\mu - 2k\^\nu \gamma\^\mu )P_L + C_{qqZ}^R (\gamma\^\nu \slashed{k}_j \gamma\^\mu - 2k\^\nu \gamma\^\mu )P_R }{\hat{u}}
  \biggr\}u(k_i).
\end{align}
Again evaluating in Mathematica, the squared matrix element becomes, after averaging over the 2 initial quark spins and \(N_C\) colours, and 2 initial gluon polarisations and \(N_C^2-1\) gluon states,
\begin{align}
  \nonumber
  \frac{1}{4N_C(N_C^2-1)} \sum_{\stackrel{\text{spin/pol.}}{\text{colour}}}|\M_q|^2 = & \frac{g^2 \mu^{2\epsilon} g_s^2 T_F}{N_C} ((C_{qqZ}^L)^2 + (C_{qqZ}^R)^2) (1-\epsilon) \\
                                                                                      & \times\Bigl\{
  -\frac{2Q^2 \hat{t}}{\hat{s} \hat{u}} + 2\epsilon - (1-\epsilon)\pclosed{\frac{\hat{s}}{\hat{u}} + \frac{\hat{u}}{\hat{s}}}
  \Bigr\},
\end{align}
where \(T_F = \frac{T^k_{ab} T^{k}{ba}}{N_C^2-1}\) is the index of the fundamental representation of \(SU(N_C)\).
Doing the phase space integral and inserting into \cref{pc:eq:dsigma_dQ2_2} with the averaging factor replacing the colour average \(\frac{1}{N_C^2} \to \frac{1}{N_C(N_C^2-1)}\) we get
\begin{align}
  \label{pc:eq:sigma_qg}
  \nonumber
  \d[Q^2]{\hat\sigma_{qg}} = & \frac{\hat\sigma_B}{\hat{s}} \frac{T_F \alpha_s}{2\pi} \pclosed{\frac{4\pi \mu^2}{Q^2}}^\epsilon (1-\epsilon) \frac{\Gamma(1-\epsilon)}{\Gamma(1-2\epsilon)} F^{\prime\prime}_Z \\
                             & \times \cclosed{-\frac{1-2z(1-z)}{\epsilon} + \frac{1}{2}(1+6z-7z^2) + (1-2z(1-z))\ln\frac{(1-z)^2}{z}}
\end{align}

For further reference, the real emission contribution will be referred to as
\begin{equation}
  \d[Q^2]{\hat\sigma_r} = \d[Q^2]{\hat\sigma_{qq}} + \d[Q^2]{\hat\sigma_{qg}}.
\end{equation}


\subsection{Gaugino Corrections}
The gaugino corrections at NLO in QCD come from corrections to the \(\hat{t}\)- and \(\hat{u}\)-channels in \cref{pc:fig:tree_level_diagrams}.
These come in the form of a gluon or gluino between the initial quarks, creating the box diagrams in \cref{pc:fig:bwino_boxes}, the vertex corrections to the quark-squark-neutralino vertices in \cref{pc:fig:bwino_vertices} or self-energy insertions corrections to the squark propagator in \cref{pc:fig:squark_SE}.

The IR-divergences from exchanging real gluons in these diagrams are cancelled by taking into account the real gluon emission diagrams in \cref{pc:fig:tu_gluon_emission}. Furthermore, quark emission should be included in the inclusive cross-section with the gaugino-like vertices enabling the diagrams in \cref{pc:fig:tu_quark_emission}.
\medskip

Two subtleties must be adressed when computing these contributions.
First, when using dimensional regularisation, the dimensionality of vector fields go from \(4 \to d\), whereas Dirac spinors remain four-dimensional.
This comes from the fact that the Dirac spinors are generated from the Dirac gamma-matrices defined by the Clifford algebra with the anticommutation relation
\begin{equation}
  \cclosed{\gamma\^\mu, \gamma\^\nu} = 2\g \eye,
\end{equation}
of which there is no true realisation in fractional dimensions.
The result is that the degrees of freedom between the SM gauge bosons and their fermion partners are no longer the same --- breaking supersymmetry.
Others have shown that this can be remedied by introducing a counterterm to the weak coupling of the quark-squark-neutralino interaction \(g \to \hat{g} = g\pclosed{1 - \frac{\alpha_s CF}{8\pi}}\).
\\\comment{Cite this!}\\
The other subtlety comes from the real quark emission of the leftmost diagrams in \cref{pc:fig:tu_quark_emission}, going through an intermediate squark state.
This squark can go on-shell if \(m_{i/j} > m_A\) and \(\hat{s} > (m_{i/j} + m_A)^2\), which experimentally coincides with the production of a neutralino and a squark followed by the subsequent decay of the squark into a neutralino and a quark.
To avoid this double-counting, we must subtract the cross-section for neutralino-squark produciton with the corresponding branching ratio:
\begin{equation}
  \sum_A \cclosed{\d[Q^2]{\hat\sigma}(qg \to \tilde\chi^0_i \tilde{q}_A) \operatorname{BR}(\tilde{q}_A \to \tilde\chi^0_j q) + (i \leftrightarrow j)}.
\end{equation}



\begin{figure}[ht!]
  \centering
  \begin{subfigure}{0.24\textwidth}
    \centering
    \inputtikz{NLO/box1}
    \caption{}
  \end{subfigure}
  \begin{subfigure}{0.24\textwidth}
    \centering
    \inputtikz{NLO/box2}
    \caption{}
  \end{subfigure}
  \begin{subfigure}{0.24\textwidth}
    \centering
    \inputtikz{NLO/box3}
    \caption{}
  \end{subfigure}
  \begin{subfigure}{0.24\textwidth}
    \centering
    \inputtikz{NLO/box4}
    \caption{}
  \end{subfigure}
  \caption{Box diagrams contributing to NLO QCD corrections to the gaugino part of neutralino pair production.}
  \label{pc:fig:bwino_boxes}
\end{figure}

\begin{figure}[ht!]
  \centering
  \begin{subfigure}{0.24\textwidth}
    \centering
    \inputtikz{NLO/vertex1t}
    \caption{}
  \end{subfigure}
  \begin{subfigure}{0.24\textwidth}
    \centering
    \inputtikz{NLO/vertex1u}
    \caption{}
  \end{subfigure}
  \begin{subfigure}{0.24\textwidth}
    \centering
    \inputtikz{NLO/vertex2t}
    \caption{}
  \end{subfigure}
  \begin{subfigure}{0.24\textwidth}
    \centering
    \inputtikz{NLO/vertex2u}
    \caption{}
  \end{subfigure}
  \caption{Vertex diagrams contributing to NLO QCD corrections to the gaugino part of neutralino pair production. The blobs are defined in \cref{pc:fig:bwino_vertex_insertions}.}
  \label{pc:fig:bwino_vertices}
\end{figure}

\begin{figure}
  \centering
  \begin{equation*}
    \inputtikz[baseline=(blob.base)]{NLO/vertex_blob} = \inputtikz[baseline=(v1.base)]{NLO/vertex1} + \inputtikz[baseline=(v1.base)]{NLO/vertex2}
  \end{equation*}
  \caption{The vertex insertions used in \cref{pc:fig:bwino_vertices}.}
  \label{pc:fig:bwino_vertex_insertions}
\end{figure}

\begin{figure}[ht!]
  \centering
  \begin{align*}
    \inputtikz[baseline=(in.base)]{NLO/squark_SE} = & \inputtikz[baseline=(in.base)]{NLO/squark_SE1} + \inputtikz[baseline=(in.base)]{NLO/squark_SE2} \\
                                                    & + \inputtikz[baseline=(in.base)]{NLO/squark_SE3}
  \end{align*}
  % \begin{subfigure}{0.32\textwidth}
  %   \centering
  %   \inputtikz{NLO/squark_SE1}
  %   \caption{}
  % \end{subfigure}
  % \begin{subfigure}{0.32\textwidth}
  %   \centering
  %   \inputtikz{NLO/squark_SE2}
  %   \caption{}
  % \end{subfigure}
  % \begin{subfigure}{0.32\textwidth}
  %   \centering
  %   \inputtikz{NLO/squark_SE3}
  %   \caption{}
  % \end{subfigure}
  \caption{NLO QCD diagrams contributing to the self-energy of the squark.}
  \label{pc:fig:squark_SE}
\end{figure}

\begin{figure}[ht!]
  \centering
  \begin{subfigure}{0.32\textwidth}
    \centering
    \inputtikz{NLO/gluon_em1}
    \caption{}
  \end{subfigure}
  \begin{subfigure}{0.32\textwidth}
    \centering
    \inputtikz{NLO/gluon_em2}
    \caption{}
  \end{subfigure}
  \begin{subfigure}{0.32\textwidth}
    \centering
    \inputtikz{NLO/gluon_em3}
    \caption{}
  \end{subfigure}
  \begin{subfigure}{0.32\textwidth}
    \centering
    \inputtikz{NLO/gluon_em4}
    \caption{}
  \end{subfigure}
  \begin{subfigure}{0.32\textwidth}
    \centering
    \inputtikz{NLO/gluon_em5}
    \caption{}
  \end{subfigure}
  \begin{subfigure}{0.32\textwidth}
    \centering
    \inputtikz{NLO/gluon_em6}
    \caption{}
  \end{subfigure}
  \caption{Gluon emission diagrams with gaugino-like vertices for neutralino pair production.}
  \label{pc:fig:tu_gluon_emission}
\end{figure}

\begin{figure}[ht!]
  \centering
  \begin{subfigure}{0.32\textwidth}
    \centering
    \inputtikz{NLO/quark_em1}
    \caption{}
  \end{subfigure}
  \begin{subfigure}{0.32\textwidth}
    \centering
    \inputtikz{NLO/quark_em2}
    \caption{}
  \end{subfigure}
  \begin{subfigure}{0.32\textwidth}
    \centering
    \inputtikz{NLO/quark_em3}
    \caption{}
  \end{subfigure}
  \begin{subfigure}{0.32\textwidth}
    \centering
    \inputtikz{NLO/quark_em4}
    \caption{}
  \end{subfigure}
  \begin{subfigure}{0.32\textwidth}
    \centering
    \inputtikz{NLO/quark_em5}
    \caption{}
  \end{subfigure}
  \begin{subfigure}{0.32\textwidth}
    \centering
    \inputtikz{NLO/quark_em6}
    \caption{}
  \end{subfigure}
  \caption{Quark emission diagrams with gaugino-like vertices for neutralino pair production.}
  \label{pc:fig:tu_quark_emission}
\end{figure}


\subsection{Catani-Seymour Dipole Formalism}
The loops in these virtual exchange diagrams will contain dependence on the Mandelstam variables \cref{pc:eq:mandelstam_LO}, which will complicate the phase space integral over \(\hat{t}\) considerably.
Doing the numerically will require some procedure for regularising and extracting these divergences such that they can cancel.
For the IR-divergences to cancel, it will require matching between the phase space integrals of the \(2 \to 2\) process of virtual gluon/quark exchange with the \(2 \to 3\) of the real emission.
The Catani-Seymour Dipole formalism rewrites these integrals such that the divergences cancel at the level of the integrand of the two phase space integrals seperately.
This allows for the numerical implementation of the two phase space integrals.



\end{document}

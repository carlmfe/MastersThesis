\documentclass[../main.tex]{subfiles}

\begin{document}

\chapter{Neutralino Pair Production at Parton Level}

\begin{TODO}
	\item Formulate a section on the dipole formalism used in Debove et al.
	and make a
	comparison.
\end{TODO}

\section{Kinematics}
To start off, it will be useful to introduce some procedure for going forward
in the phase space of an inclusive \(2\to 2(+1)\) cross-section process. The
phase space of 2-body and 3-body final states are quite different as there are
more degrees of freedom in the 3-body final state. In the end, these extra
degrees of freedom will be need to be integrated over to make an additive
comparison between the 2-body and 3-body processes, however, exactly how we
choose to parametrise and subsequently integrate over the extra degrees of
freedom can matter quite a bit. \medskip To start out, let us count the degrees
of freedom of a scattering problem involving \(N\) four-momenta
\(p_{i=1,\ldots,N}\). Assuming our end result to be Lorentz invariant, there
are \(N(N+1)/2\) different scalar products that can be produced using \(N\)
different four-momenta. Momentum conservation allows us to eliminate one
momentum, such that we have \(N(N-1)/2\) possible scalar products. Denoting the
scalar products by \(m^2_{ij} \equiv (p_i + p_j)^2\) for \(j \neq i\), and
\(m^2_i \equiv p_i^2\), we can find a relation between scalar products by using
momentum conservation.
\begin{align} \nonumber
	m^2_{ij} & = \pclosed{p_i - \sum_{k\neq j} p_k}^2 =
	\pclosed{\sum_{k\neq i,j} p_k}^2 = \sum_{k\neq i,j} \sum_{l\neq i,j}
	p_k\cdot
	p_l
	\\
	\nonumber
	         & = \sum_{k\neq i,j} \sum_{l\neq i,j,k} \frac{m^2_{kl} - m^2_k
	- m^2_l}{2} + \sum_{k\neq i,j} m^2_k
	\\
	\nonumber
	         & = \sum_{k\neq i,j} \sum_{\substack{l\neq i,j
	\\l>k}} m^2_{kl}
	- \frac{1}{2} \sum_{k\neq i, j} (N-3) m^2_k - \frac{1}{2} \sum_{l\neq
		i,j}
	(N-3) m^2_l + \sum_{k\neq i,j} m^2_k
	\\
	         & = \sum_{k\neq i,j} \sum_{\substack{l\neq i,j
	\\l>k}} m^2_{kl}
	- (N-4) \sum_{k\neq i, j} m^2_k.
\end{align}
\comment{This little generalised relation might not be immediately
necessary\ldots}
Furthermore, we assign the \(N\) scalar products \(m^2_i\) to the invariant
masses of the incoming and outgoing particles, thus not counting them as
degrees of freedom, leaving us with
\(n_\text{dof} = \frac{N(N-3)}{2}\) degrees of freedom.\footnote{I note that we
	often consider the invariant mass of the incoming bodies to be fixed,
	which
	would reduce our degrees of freedom by one.}
This means that in a \(2\to 2\) process, we have 2 degrees of freedom, and in a
\(2\to 3\) process we have 5.

\subsection{2-body Phase Space}
The Lorentz invariant phase space differential for a 2-body final state with
four-momenta \(p_i, p_j\) in \(d\) dimensions is
\begin{equation}
	\mathrm{d}\Pi_{2\to 2} = \pclosed{2\pi}^d \delta^d\pclosed{P - p_i -
		p_j} \frac{\mathrm{d}^{d-1} \vec{p}_i}{\pclosed{2\pi}^{d-1}}
	\frac{1}{2E_i}
	\frac{\mathrm{d}^{d-1} \vec{p}_j}{\pclosed{2\pi}^{d-1}} \frac{1}{2E_j}.
\end{equation}
Going to the centre-of-mass frame of the incoming partons, we have \(P\^\mu =
\pclosed{\sqrt{s}, 0, 0, 0}\), allowing us to integrate over the spatial part
of Dirac delta-function to arrive at
\begin{equation}
	\mathrm{d}\Pi_{2\to 2} = \frac{1}{\pclosed{2\pi}^{d-2}}
	\mathrm{d}^{d-1} \vec{p} \frac{1}{4E_i E_j} \delta\pclosed{\sqrt{s} -
		E(p, m_i)
		- E(p, m_j)},
\end{equation}
where the \(E(p, m) = \sqrt{p^2 + m^2}\).
We can write out the differential of the spatial component of \(p_i\) in
spherical coordinates as \(\mathrm{d}^{d-1} \vec{p} = \mathrm{d}\Omega_{d-1}
\mathrm{d}p \, p^{d-2} = \mathrm{d}\Omega_{d-2} \sin^{d-3}\theta \,
\mathrm{d}\theta \, \mathrm{d}p \, p^{d-2}\).
As a \(2\to 2\) process is restricted to planar motion, we can always go to a
frame of reference such that any amplitude we calculate will not be dependent
on the spatial angles \(\mathrm{d} \Omega_{d-2}\), allowing us to integrate
over them using that \(\integral{\Omega_{d-2}} = 2 \pi^{\frac{d-2}{2}}
\frac{1}{\Gamma\pclosed{\frac{d-2}{2}}}\) to get
\begin{equation}
	\mathrm{d}\Pi_{2\to 2} = \frac{1}{\pclosed{4\pi}^{\frac{d-2}{2}}}
	\frac{1}{\Gamma\pclosed{\frac{d-2}{2}}} \frac{p^{d-3}}{2 \sqrt{s}}
	\sin^{d-3}\theta \,\mathrm{d}\theta,
\end{equation}
where we understand the momentum to be given by \(p = \frac{\sqrt{\lambda(s,
		m_i^2, m_j^2)}}{2\sqrt{s}}\).
In \(d=4\) dimensions, it is often convenient to change to the Mandelstam
variable \(t\), which for massless initial state particles becomes
\(t=\frac{1}{2}
\pclosed{-s+m_i^2+m_j^2+\sqrt{\lambda(s,m_i^2,m_j^2)}\cos\theta}\).
Making the change of variable, the differential phase space reduces to
\begin{equation}
	\evalat{\mathrm{d}\Pi_{2\to 2}}{d=4} = \frac{1}{8\pi s} \,\mathrm{d}t
\end{equation}

\subsection{3-body Phase Space}
\todo{Fill out an introduction here.}\\
The differential Lorentz invariant phase space for a 3-body final state with
four-momenta \(p_i, p_j, k\), where \(k^2=0\) in \(d\) dimensions is
\begin{equation}
	\mathrm{d}\Pi_{2\to 3} = (2\pi)^d \delta^d(P - p_i - p_j - k)
	\frac{\mathrm{d}^{d-1} \vec{p}_i}{(2\pi)^{d-1}} \frac{1}{2E_i}
	\frac{\mathrm{d}^{d-1} \vec{p}_j}{(2\pi)^{d-1}} \frac{1}{2E_j}
	\frac{\mathrm{d}^{d-1} \vec{k}}{(2\pi)^{d-1}} \frac{1}{2\omega}.
\end{equation}
First, it will be useful to write out the differential in \(\vec{k}\) in
spherical coordinates where it reads \(\mathrm{d}^{d-1}\vec{k} = \omega^{d-2}
\mathrm{d}\Omega_{d-1} \mathrm{d}\omega\).
The differentials in \(\vec{p}_{i/j}\) together with the delta-function are
easier to compute in the centre-of-mass frame of the neutralinos where we have
\(P - k = \pclosed{Q, 0, 0, 0}\).
This leaves
\begin{equation}
	\mathrm{d}\Pi_{2\to 3} = \frac{1}{8} \frac{1}{\pclosed{2\pi}^{2d-3}}
	\delta(Q - E_i^\star - E_j^\star) \delta^{d-1}(\vec{p}^\star_i +
	\vec{p}_j^\star) \frac{\omega^{d-3}}{E_i^\star E_j^\star}
	\mathrm{d}^{d-1}\vec{p}_i^\star\, \mathrm{d}^{d-1}\vec{p}_j^\star\,
	\mathrm{d}\Omega_{d-1}\, \mathrm{d}\omega,
\end{equation}
where the stars denote quantities calculated in the aforementioned reference
frame.
Integrating trivially over \(\vec{p}_j^\star\) using the delta-function, and
writing using polar coordinates \(\mathrm{d}^{d-1}\vec{p}_i =
\mathrm{d}\Omega_{d-1}^\star \, \mathrm{d}\!\abs{\vec{p}_i^\star}
\abs{\vec{p}_i^\star}^{d-2}\) to integrate over \(\delta\pclosed{Q - E_i^\star
	- E_j^\star}\), we get
\begin{equation}
	\mathrm{d}\Pi_{2\to 3} = \frac{1}{\pclosed{2\pi}^{2d-3}}
	\frac{\omega^{d-3} \abs{\vec{p}_i^\star}^{d-3}}{8Q}
	\mathrm{d}\Omega_{d-1}^\star\, \mathrm{d}\Omega_{d-1}\,
	\mathrm{d}\omega.
\end{equation}
Here, we understand the magnitude of the three-momenta to be given by
\(\abs{\vec{p}_i^\star} = \frac{\sqrt{\lambda(Q^2, m_i^2, m_j^2)}}{2Q}\) and
\(\omega = \frac{s-Q^2}{2\sqrt{s}}\).'
It will also be useful to make a change of integration variable to \(Q^2\),
leaving us finally with
\begin{equation}
	\mathrm{d}\Pi_{2\to 3} = \frac{1}{\pclosed{2\pi}^{2d-3}}
	\frac{\omega^{d-3} \abs{\vec{p}_i^\star}^{d-3}}{16Q\sqrt{s}}
	\mathrm{d}\Omega_{d-1}^\star\, \mathrm{d}\Omega_{d-1}\, \mathrm{d}Q^2.
\end{equation}

\todo{Comment on integration boundaries.}\\
\begin{temporary}
	\[(m_i + m_j)^2 \leq Q^2 \leq s\]
\end{temporary}

With two initial state momenta, the amplitude will be independent of the
azimuthal angle in the centre-of-mass frame of the initial partons. This lets
us integrate over it for a factor of \(2\pi\).

\begin{equation}
	\mathrm{d}\Pi_{2\to 3} = \frac{1}{\pclosed{2\pi}^{2d-3}}
	\frac{1}{\Gamma\pclosed{\frac{d-2}{2}}} \frac{1}{2^d
		\pi^{\frac{3d-4}{2}}}
	\frac{\lambda^{\frac{d-3}{2}\pclosed{Q^2,

				m_i^2,m_j^2}}}{s}\frac{\pclosed{1-z}^{\frac{d-3}{2}}}{z^{\frac{d-2}{2}}}
	\pclosed{y(1-y)}^{\frac{d-4}{2}} \mathrm{d}y\,
	\mathrm{d}\Omega_{d-1}^\star\,
	\mathrm{d}Q^2.
\end{equation}
% \comment{Complete this calculation.}\\
% It is important to note that there is an interdependence between \(Q^2\) and \(\omega\), which can be found using that \(P^2 = s\).
% In the centre-of-mass frame of the neutralinos, the magnitude of the three-momentum of \(P\) must be \(\omega\), so we have the relation \(P_0^2 = s + \omega^2\), which together with momentum conservation \(P\_0 - \omega = Q\) yields
% \begin{equation}
%     \omega = \frac{s-Q^2}{2\sqrt{s}}.
% \end{equation}
% Switching integration variables to \(Q^2\) and \(y = \frac{1}{2}\pclosed{1 + \cos\theta}\), we finally get
% \begin{align}
%     \nonumber
%     \mathrm{d}\Pi_{2\to 3} = \frac{1}{8} \frac{1}{\pclosed{4\pi}^{\frac{3d-4}{2}}} \frac{1}{\Gamma\pclosed{\frac{d-2}{2}}} \frac{\pclosed{\lambda\pclosed{Q^2, m_i^2, m_j^2}}^{\frac{d-3}{2}}}{Q^{2d-2}} \pclosed{s^2-Q^4}\pclosed{s-Q^2}^{d-4} \\
%     \pclosed{y(1-y)}^{\frac{d-4}{2}} \mathrm{d}y\, \mathrm{d}Q^2\, \sin^{d-3}\theta^\ast \mathrm{d}\theta^\ast\, \mathrm{d}\Omega^\ast_{d-2}.
% \end{align}

% \begin{align}
%     \frac{\mathrm{d}^{d-1}\vec{k}}{(2\pi)^{d-1}} \frac{1}{2\omega} = -\frac{1}{(4\pi)^{\sfrac{d}{2}}} \frac{(s-Q^2)^{d-3}}{s^{\sfrac{(d-2)}{2}}} \pclosed{y(1-y)}^{\frac{d-4}{2}} \mathrm{d}y \, \mathrm{d}Q^2
% \end{align}
% \begin{align}
%     \frac{\mathrm{d}^{d-1} \vec{p}_i}{(2\pi)^{d-1}} \frac{1}{2E_i} \frac{\mathrm{d}^{d-1} \vec{p}_j}{(2\pi)^{d-1}} \frac{1}{2E_j} (2\pi)^d \delta^d(P - p_i - p_j - k) \\
%     = \frac{1}{(8\pi)^{\sfrac{d}{2}}} \frac{1}{\Gamma\!\pclosed{\frac{d-2}{2}}} \frac{\pclosed{\lambda(Q^2, m_i^2, m_j^2)}^{\frac{d-3}{2}}}{8 Q^{d-2}} \sin^{d-3}\theta^\ast \,\mathrm{d}\theta^\ast \, \mathrm{d}\phi^\ast
% \end{align}

% \begin{align}
%     \mathrm{d}\Pi_{2\to 3} = -\frac{1}{2^{\sfrac{3d}{2}} (2\pi)^d} \frac{1}{\Gamma\!\pclosed{\frac{d-2}{2}}} \frac{(s-Q^2)^{d-3}}{s^{\sfrac{(d-2)}{2}}} \pclosed{y(1-y)}^{\frac{d-4}{2}} \\
%     \frac{\pclosed{\lambda(Q^2, m_i^2, m_j^2)}^{\frac{d-3}{2}}}{8 Q^{d-2}} \sin^{d-3}\theta^\ast \,\mathrm{d}\theta^\ast \, \mathrm{d}\phi^\ast \, \mathrm{d}y \, \mathrm{d}Q^2
% \end{align}

% \begin{align}
%     \mathrm{d}\sigma = -\frac{1}{2^{\sfrac{3d}{2}} (2\pi)^d} \frac{1}{\Gamma\!\pclosed{\frac{d-2}{2}}} \frac{(s-Q^2)^{d-3}}{s^{\sfrac{(d-2)}{2}}} \pclosed{y(1-y)}^{\frac{d-4}{2}} \\
%     \frac{\pclosed{\lambda(Q^2, m_i^2, m_j^2)}^{\frac{d-3}{2}}}{16 s Q^{d-2}} \abs{\M}^2 \sin^{d-3}\theta^\ast \,\mathrm{d}\theta^\ast \, \mathrm{d}\phi^\ast \, \mathrm{d}y \, \mathrm{d}Q^2
% \end{align}

% \begin{align}
%     \mathrm{d}\Pi_{2\to 3} = \frac{1}{\pclosed{4\pi}^{\frac{3d-4}{2}}} \frac{1}{\Gamma\pclosed{\frac{d-2}{2}}} \frac{\pclosed{\lambda\pclosed{Q^2, m_i^2, m_j^2}}^{\frac{d-3}{2}}}{Q^{d-2}} \sin^{d-3}\theta \mathrm{d}\theta\, \omega^{d-3} \mathrm{d}\omega\, \sin^{d-3}\theta^\ast \mathrm{d}\theta^\ast\, \mathrm{d}\Omega^\ast_{d-2}
% \end{align}

Parametrising the free variables in a \(2 \to 3\) process can be tricky. I will
define some natural variables in two different frames of reference, and
rediscover the Lorentz transformation between them to parametrise all scalar
products in terms of the variables in these reference frames. First, we will
consider the lab frame, or the centre-of-mass frame of the incoming partons
with momenta \(\vec{k}_{i,j}\). We can reduce this to an ordinary \(2 \to 2\)
scattering by considering the outgoing neutralinos with momenta
\(\vec{p}_{i,j}\) as a single system. This lets us write the momenta as
\begin{subequations}
	\begin{align}
		k_i^\mu            & = \frac{\sqrt{s}}{2} \pclosed{1, 0, 0, 1},
		\\
		k_j^\mu            & = \frac{\sqrt{s}}{2} \pclosed{1, 0, 0,
			-1},
		\\
		k\^{\mu}           & = \frac{\sqrt{s}}{2} (1-z) \pclosed{1,
			\sin\theta, 0, \cos\theta},
		\\
		(p_i + p_j)\^{\mu} & = \frac{\sqrt{s}}{2} \pclosed{(1+z),
			-(1-z)\sin\theta, 0, -(1-z)\cos\theta}.
	\end{align}
\end{subequations}

The centre-of-mass frame of the neutralinos is defined by \((p^\ast_i +
p^\ast_k)\^{\mu} = \pclosed{\sqrt{zs}, 0, 0, 0}\).\footnote{I will from now on
	always put a star on quantities pertaining to the centre-of-mass frame
	of the
	neutralinos.} We find the transformation to this frame then by making
appropriate boosts and rotations of this four-vector. Let us start by rotating
the 3-momentum to lie along the positive \(z\)-direction. As the
\(y\)-component is already zero in the lab-frame, we only require a rotation
around the \(y\)-axis, we can be parametrised by the following matrix
\begin{equation}
	\text{Rot}_y(\alpha) = \begin{pmatrix}
		1 & 0           & 0 & 0          \\
		0 & \cos\alpha  & 0 & \sin\alpha \\
		0 & 0           & 1 & 0          \\
		0 & -\sin\alpha & 0 & \cos\alpha
	\end{pmatrix}.
\end{equation}
Using \(\alpha = -\theta-\pi\) we get that \(\text{Rot}_y(-\theta-\pi) (p_i +
p_j)\^{\mu} = \frac{\sqrt{s}}{2} \pclosed{(1+z), 0, 0, (1-z)}\).
We can subsequently boost along the \(z\)-axis to eliminate the
\(z\)-component.
Such a boost can be parametrised by
\begin{equation}
	\text{Boost}_z(\beta) = \begin{pmatrix}
		\gamma      & 0 & 0 & \gamma\beta \\
		0           & 1 & 0 & 0           \\
		0           & 0 & 1 & 0           \\
		\gamma\beta & 0 & 0 & \gamma
	\end{pmatrix},
\end{equation}
where \(\gamma = \pclosed{1 - \beta^2}^{-\sfrac{1}{2}}\).
The \(z\)-component is eliminated using \(\beta = -\frac{1-z}{1+z}\), such that
we end up with
\[(p_i^\ast + p_j^\ast)\^\mu \equiv \text{Boost}_z\pclosed{-\frac{1-z}{1+z}}
	\text{Rot}_y\pclosed{-\theta-\pi} \pclosed{p_i + p_j}^\mu =
	\pclosed{\sqrt{zs},
		0, 0, 0}\]
as we expected.

Now we can parametrise \({p_{i,j}^\ast}\^\mu\) in this frame using two angular
variables \(\theta^\ast, \phi^\ast\), knowing that \(\vec{p}_i + \vec{p}_j =
0\),
\begin{subequations}
	\begin{align}
		{p_i^\ast}\^\mu & = \pclosed{E_i, p\sin\theta^\ast
			\cos\phi^\ast, p\sin\theta^\ast \sin\phi^\ast, p
		\cos\theta^\ast},                                   \\
		{p_j^\ast}\^\mu & = \pclosed{E_j, -p\sin\theta^\ast
			\cos\phi^\ast, -p\sin\theta^\ast \sin\phi^\ast, -p
			\cos\theta^\ast}.
	\end{align}
\end{subequations}
To find what \(E_{i,j}\) and \(p\) need to be, we can transform \(k^\mu\) and
\(k_{i,j}^\mu\) to this frame of reference, finding
\begin{subequations}
	\begin{align}
		{k^\ast}\^\mu                      & = \frac{\sqrt{s}}{2}
		\frac{1-z}{\sqrt{z}} \pclosed{1, 0, 0, -1},                \\
		\pclosed{k_i^\ast + k_j^\ast}\^\mu & = \frac{s}{2\sqrt{z}}
		\pclosed{1+z, 0, 0, -(1-z)},
	\end{align}
\end{subequations}
and use conservation of momentum and the fact that \({p_{i,j}^\ast}^2 =
m^2_{i,j}\) to get that
\begin{subequations}
	\begin{align}
		E_{i,j}(z) & = \frac{zs + m^2_{i,j} - m^2_{j,i}}{2 \sqrt{zs}},
		\\
		p(z)       & = \frac{\sqrt{\lambda\pclosed{zs, m^2_i,
					m^2_j}}}{2 \sqrt{zs}}.
	\end{align}
\end{subequations}

Now to get all momenta in the lab frame, we can apply the reverse
transformations on \({p_{i,j}^\ast}\^\mu\) using that
\(\text{Rot}^{-1}_y(\alpha) = \text{Rot}_y(-\alpha)\) and
\(\text{Boost}_z^{-1}(\beta) = \text{Boost}_z(-\beta)\):
\begin{equation}
	p_{i,j}^\mu = \text{Rot}_y\pclosed{\theta+\pi}
	\text{Boost}_z\pclosed{\frac{1-z}{1+z}} {p_{i,j}^\ast}\^\mu.
\end{equation}

\begin{figure}
	\centering
	\begin{subfigure}{.49\textwidth}
		\centering
		\inputtikz[auto, >=latex, scale=1.1]{kinematics_lab_frame}
		\caption{Angular definition in the centre-of-mass frame of the
			initial particles with momenta \(\vec{k}_{i,j}\).}
	\end{subfigure}
	\hfill
	\begin{subfigure}{.49\textwidth}
		\centering
		\inputtikz[auto, >=latex, scale=1.1]{kinematics_com_frame}
		\caption{Angular definitions in the centre-of-mass frame of the
			outgoing particles with momenta \(\vec{p}_{i,j}\).}
	\end{subfigure}
	\caption{}
\end{figure}

\subsection{Differential Cross-Section}
\begin{equation}
	\mathrm{d}\hat{\sigma}^d = \frac{1}{\pclosed{4\pi}^{\frac{d-2}{2}}}
	\frac{1}{\Gamma\pclosed{\frac{d-2}{2}}}
	\frac{p^{d-3}}{4\hat{s}\sqrt{\hat{s}}}
	\abs{\M}^2 \sin^{d-3}\theta\, \mathrm{d}\theta
\end{equation}

\begin{equation}
	\mathrm{d}\hat{\sigma} = \frac{1}{16\pi} \frac{1}{\hat{s}^2} \abs{\M}^2
	\mathrm{d}\hat{t}
\end{equation}

Averaged over spin and colour, and taking account of symmetry if the particles
are identical, the differential cross-section in \(d=4\) dimensions is.
\begin{equation}
	\mathrm{d}\hat{\sigma} = \pclosed{\frac{1}{2}}^{\delta_{ij}}
	\frac{1}{64 N_C^2 \pi} \frac{1}{\hat{s}^2}
	\sum_{\substack{\text{spin}\\\text{colour}}} \abs{\M}^2
	\mathrm{d}\hat{t}
\end{equation}
% \clearpage
% \begin{equation}
%     \lambda(x, y, z) = x^2 + y^2 + z^2 - 2xy - 2xz - 2yz
% \end{equation}

% \begin{align}
%     \nonumber
%     t_{ii}(z, y, \theta^\ast, \phi^\ast) \equiv (k_i - p_i)^2 = \frac{1}{2z} \bigg\{ (m_i^2 + m_j^2)z - (m_i^2 - m_j^2)y(1-z) + sz(yz - y - z) \\
%     + 2z \sqrt{y(1-y)} \sqrt{\lambda(Q^2,m_i^2,m_j^2)}\sin\theta^\ast \cos\phi^\ast + (z-y-yz)\sqrt{\lambda(Q^2,m_i^2,m_j^2)} \cos\theta^\ast \bigg\}
% \end{align}

% \begin{align}
%     \nonumber
%     m^2_{gi}(z, \theta^\ast) \equiv (p_i + k)^2 = \frac{1}{2z} \bigg\{ (m_i^2-m_j^2) + (m_i^2+m_j^2)z \\
%     + sz(1-z) + \sqrt{\lambda(Q^2, m_i^2, m_j^2)} (1-z) \cos\theta^\ast \bigg\}
% \end{align}

% \begin{align}
%     \abs{\M}^2 \supset \frac{\pclosed{t_{ii}(z, y, \theta^\ast, \phi^\ast) - m_i^2} \pclosed{m^2_{gi}(z, \theta^\ast) - m_i^2 - (1-z)s}}{s y (1-z) \pclosed{t_{ii}(z, y, \theta^\ast, \phi^\ast) - m_A^2} \pclosed{t_{ii}(z, y, \theta^\ast, \phi^\ast) - m_B^2}}
% \end{align}

\section{Leading Order Cross-Section}

\begin{TODO}
	\item Comment on reason for using Breit-Wigner approximation.
\end{TODO}

\subsection{The Matrix Elements}
\begin{figure} [ht!]
	\centering
	\begin{subfigure}{0.3\linewidth}
		\centering
		\inputtikz{s_channel}
		\caption{\(s\)-channel}
	\end{subfigure}
	\begin{subfigure}{0.3\linewidth}
		\centering
		\inputtikz{t_channel}
		\caption{\(t\)-channel}
	\end{subfigure}
	\begin{subfigure}{0.3\linewidth}
		\centering
		\inputtikz{u_channel}
		\caption{\(u\)-channel}
	\end{subfigure}
	\caption{The leading order diagrams contributing to neutralino pair
		production at parton-level.}
	\label{pc:fig:tree_level_diagrams}
\end{figure}

At leading order the contributing diagrams to the parton-level process are
shown in \cref{pc:fig:tree_level_diagrams}. In the following, I will make use
of the shorthand notation for the spinors \(w_{i/j} = w(p_{i/j}), w_{1,2} =
w(k_{i/j})\) where \(w\) is either \(u\) or \(v\). The resulting amplitudes,
using the Feynman rules in \needcite[Feynman rules section], are then
\begin{subequations}
	\begin{align}
		\nonumber
		\M_{\hat{s}} = -\frac{g^2}{2} D_Z(\hat{s})           &
		\bclosed{ \bar{u}_i\gamma^\mu\pclosed{O^{\prime\prime
						L}_{ij}P_L +
				O^{\prime\prime R}_{ij}P_R} v_j }
		\\
		                                                     & \times
		\bclosed{ \bar{v}_2 \gamma_\mu \pclosed{C^L_{Zqq}P_L +
		C^R_{Zqq}P_R} u_1 },
		\\
		\nonumber
		\M_{\hat{t}} = -\sum_A 2g^2 D_{\tilde{q}_A}(\hat{t}) &
		\bclosed{ \bar{u}_i
			\pclosed{C_{\nino[i]\tilde{q}_Aq}^{L\ast}P_L +
				C_{\nino[i]\tilde{q}_Aq}^{R\ast} P_R} u_1 }
		\\
		                                                     & \times
		\bclosed{ \bar{v}_2 \pclosed{C^R_{\nino[j]\tilde{q}_Aq}P_L +
		C^L_{\nino[j]\tilde{q}_Aq}P_R} v_j },
		\\
		\nonumber
		\M_{\hat{u}} = -\sum_B 2g^2 D_{\tilde{q}_B}(\hat{u}) &
		\bclosed{ \bar{u}_j
			\pclosed{C_{\nino[j]\tilde{q}_Bq}^{L\ast}P_L +
				C_{\nino[j]\tilde{q}_Bq}^{R\ast} P_R} u_1 }
		\\
		                                                     & \times
		\bclosed{ \bar{v}_2 \pclosed{C^R_{\nino[i]\tilde{q}_Bq}P_L +
		C^L_{\nino[i]\tilde{q}_Bq}P_R} v_i },
	\end{align}
\end{subequations}
where \(D_p(q^2) = \frac{1}{q^2 - m_p^2 + i\Gamma_p m_p}\) is the
\needcite[Breit-Wigner propagator] of a particle with mass \(m_p\) and decay
width \(\Gamma_p\).

These matrix elements can be expanded using the \textit{supercharges}
\begin{subequations}
	\begin{align}
		Z^X      & = C_{qqZ}^X O_{ij}^{\prime\prime X},
		\\
		Q_A^{XY} & = C_{\nino[i] \tilde{q}_A q}^{X}
		\pclosed{C_{\nino[j] \tilde{q}_A q}^{Y}}^\ast,
	\end{align}
\end{subequations}
% and the Dirac quadrilinears
% \begin{subequations}
% 	\begin{align}
% 		q_S^{XY}(w_a, w_b, w_c, w_d) = & \bclosed{\bar{w}_a P_X w_b} \bclosed{\bar{w}_c P_Y w_d}                       \\
% 		q_V^{XY}(w_a, w_b, w_c, w_d) = & \bclosed{\bar{w}_a \gamma^\mu P_X w_b} \bclosed{\bar{w}_c \gamma_\mu P_Y w_d}
% 	\end{align}
% \end{subequations}
and the Dirac bilinears
\begin{subequations}
	\begin{align}
		b_{L/R}(w_a, w_b) =     & \bar{w}_a P_{L/R} w_b,            \\
		b_{L/R}^\mu(w_a, w_b) = & \bar{w}_a \gamma^\mu P_{L/R} w_b,
	\end{align}
\end{subequations}
to arrive at
\begin{subequations}
	\begin{align}
		\nonumber
		\M_{\hat{s}} = -\frac{g^2}{2} D_Z(\hat{s}) \Big[           &
		Z^L b_L^\mu(u_i, v_j){b_L}\_{\mu}(v_2, u_1) -
		\pclosed{Z^R}^\ast b_L^\mu(u_i,
		v_j){b_R}\_{\mu}(v_2, u_1)                                   \\
		-                                                          &
		\pclosed{Z^L}^\ast b_{R}^\mu(u_i, v_j){b_L}\_{\mu}(v_2, u_1) +
		Z^R b_R^\mu(u_i,
		v_j){b_R}\_{\mu}(v_2, u_1) \Big]                             \\
		\nonumber
		\M_{\hat{t}} = -\sum_A 2g^2 D_{\tilde{q}_A}(\hat{t}) \Big[ &
		\pclosed{Q_A^{LR}}^\ast b_L(u_i, u_1) b_L(v_2, v_j) +
		\pclosed{Q_A^{LL}}^\ast
		b_L(u_i, u_1) b_R(v_2, v_j)                                  \\
		+                                                          &
		\pclosed{Q_A^{RR}}^\ast b_R(u_i, u_1) b_L(v_2, v_j) +
		\pclosed{Q_A^{RL}}^\ast
		b_R(u_i, u_1) b_R(v_2, v_j) \Big]                            \\
		\nonumber
		\M_{\hat{u}} = -\sum_A 2g^2 D_{\tilde{q}_A}(\hat{u}) \Big[ &
		Q_A^{RL} b_L(v_2, v_i) b_L(u_j, u_1) + Q_A^{RR} b_L(v_2, v_i)
		b_R(u_j, u_1)
		\\
		+                                                          &
		Q_A^{LL} b_R(v_2, v_i) b_L(u_j, u_1) + Q_A^{LR} b_R(v_2, v_i)
		b_R(u_j, u_1)
		\Big].
	\end{align}
\end{subequations}

To get the matrix element squared we can use that the complex conjugate of the
Dirac bilinears is given by
\begin{subequations}
	\begin{align}
		\pclosed{b_{L/R}(w_a, w_b)}^\dagger =     & b_{R/L}(w_b, w_a),
		\\
		\pclosed{b_{L/R}^\mu(w_a, w_b)}^\dagger = & b_{L/R}^\mu(w_b,
		w_a).
	\end{align}
\end{subequations}
Furthermore, when summing over the spins of the various spinors in the
bilinears, they have the sum identities
\begin{align}
	\sum_{\text{spins}} b_X(w_a, w_b) b_Y(w_b, w_a) =         & 2 \Big[
		(1-\delta_{XY}) (p_a \cdot p_b) + \operatorname{rsgn}
		\delta_{XY} m_a m_b
		\Big],
	\\
	\nonumber
	\sum_{\text{spins}} b_X^\mu(w_a, w_b) b_Y^\nu(w_b, w_a) = & 2\Big[
	\delta_{XY}\pclosed{p_a^\mu p_b^\nu - \g (p_a \cdot p_b) + p_a^\nu
		p_b^\mu +
		(-1)^{\delta_{XY}} i \epsilon^{\mu\nu\alpha\beta}
		\tensor{(p_a)}{_\alpha}
	\tensor{(p_b)}{_\beta}}                                             \\
	                                                          & +
	(1-\delta_{XY})\operatorname{rsgn} m_a m_b \g \Big],
\end{align}
where \(\operatorname{rsgn}\) is 1 if \(w_a, w_b\) are spinors of the same
type, e.g. both are \(u\)-spinors, and -1 otherwise.

\begin{temporary}

	\begin{subequations}
		\begin{align}
			\nonumber
			\M^\dagger_{\hat{s}} = -\frac{g^2}{2} D_Z^\ast(\hat{s})
			\Big[                               &
			\pclosed{Z^L}^\ast b_L^\mu(v_j, u_i){b_L}\_{\mu}(u_1,
			v_2) -
			Z^R b_L^\mu(v_j, u_i){b_R}\_{\mu}(u_1, v_2)
			\\
			-
			                                    & Z^L b_R^\mu(v_j,
			u_i){b_L}\_{\mu}(u_1, v_2) +
			\pclosed{Z^R}^\ast b_R^\mu(v_j, u_i){b_R}\_{\mu}(u_1,
			v_2) \Big]
			\\
			\nonumber
			\M^\dagger_{\hat{t}} = -\sum_A 2g^2
			D^\ast_{\tilde{q}_A}(\hat{t}) \Big[ & Q_A^{RL} b_L(u_1,
			u_i) b_L(v_j, v_2) +
			Q_A^{RR} b_L(u_1, u_i) b_R(v_j, v_2)
			\\
			+
			                                    & Q_A^{LL} b_R(u_1,
			u_i) b_L(v_j, v_2) + Q_A^{LR} b_R(u_1, u_i)
			b_R(v_j, v_2) \Big]
			\\
			\nonumber
			\M^\dagger_{\hat{u}} = -\sum_B 2g^2
			D^\ast_{\tilde{q}_B}(\hat{u}) \Big[ &
			\pclosed{Q_A^{LR}}^\ast b_L(v_i, v_2)
			b_L(u_1, u_j) + \pclosed{Q_A^{LL}}^\ast b_L(v_i, v_2)
			b_R(u_1, u_j)
			\\
			+
			                                    &
			\pclosed{Q_A^{RR}}^\ast b_R(v_i, v_2) b_L(u_1, u_j) +
			\pclosed{Q_A^{RL}}^\ast b_R(v_i, v_2) b_R(u_1, u_j)
			\Big].
		\end{align}
	\end{subequations}

\end{temporary}

\subsection{Differential Result}
\begin{align}
	\nonumber
	\d[\hat{t}]{\hat{\sigma}_0} = & \frac{\pi \alpha_W^2}{N_C \hat{s}^2}
	\pclosed{\frac{1}{2}}^{\delta_{ij}} \Bigg\{ \sum_{X,Y}
	\bclosed{\abs{Q^{XY}_{\hat{t}}}^2
		\pclosed{\hat{t}-m_i^2}\pclosed{\hat{t}-m_j^2} +
		\abs{Q^{XY}_{\hat{u}}}^2
	\pclosed{\hat{u}-m_i^2}\pclosed{\hat{u}-m_j^2}}                      \\
	                              & - \sum_X
	\bclosed{2\Re{\pclosed{Q_{u}^{XX}}^\ast Q_{t}^{XX}} m_i m_j \hat{s} -
		2\Re{\pclosed{Q_{u}^{XX^\prime}}^\ast Q_{t}^{XX^\prime}}
		\pclosed{\hat{t}\hat{u} - m_i^2m_j^2}} \Bigg\}
\end{align}

\subsection{Phase Space Integral}
To integrate over the variable \(\hat{t}\), we can classify the types of
integrals that will arise. All the integrals take the form
\begin{equation}
	T^p(\Delta_1, \Delta_2) \equiv \int_{t_-}^{t_+} \!\mathrm{d}\hat{t}\,
	\frac{\hat{t}^p}{(\hat{t}-\Delta_1)(\hat{t}-\Delta_2^\ast)}
\end{equation}
for some \(\Delta_{1,2}\) dependent on \(\hat{s}\), the neutralino masses and
the squark masses and decay rates, and \(p\) is some integer.

\todo{Talk about complex logarithms and define log-difference function.}\\
\todo{Maybe substitute in \(p\).}\\
Using the the integral limits are \(t_\pm = \frac{\hat{s} - m_i^2 - m_j^2}{2}
\pm \sqrt{\hat{s}}p\), we get that the possible integrals evaluate to
\begin{subequations}
	\begin{align}
		T^2(0, 0) =                         & 2\sqrt{\hat{s}}p
		\\
		T^3(0, 0) =                         &
		\sqrt{\hat{s}}p\pclosed{\hat{s} - m_i^2 - m_j^2}
		\\
		T^4(0, 0) =                         & \sqrt{\hat{s}}p
		\pclosed{\frac{8}{3}hat{s}p^2 + 2m_i^2mj^2}
		\\
		T^1\!\pclosed{\Delta, 0} =          &
		\operatorname{dLog}\pclosed{t_+ - \Delta, t_- - \Delta}
		\\
		T^2\!\pclosed{\Delta, 0} =          & 2\sqrt{\hat{s}}p + \Delta
		\operatorname{dLog}\pclosed{t_+ - \Delta, t_- - \Delta}
		\\
		T^3\!\pclosed{\Delta, 0} =          &
		\frac{\sqrt{\hat{s}}p}{2}\pclosed{\hat{s} - m_i^2 - m_j^2} +
		2\Delta\sqrt{\hat{s}}p	+ \Delta^2
		\operatorname{dLog}\pclosed{t_+ - \Delta,
			t_- - \Delta}
		\\
		T^0\!\pclosed{\Delta_1, \Delta_2} = & \frac{1}{\Delta_1 -
			\Delta_2^\ast} \cclosed{\operatorname{dLog}\pclosed{t_+
				- \Delta_1, t_- -
				\Delta_1} - \operatorname{dLog}\pclosed{t_+ -
				\Delta_2^\ast, t_- -
				\Delta_2^\ast}}
		\\
		T^1\!\pclosed{\Delta_1, \Delta_2} = & \frac{1}{\Delta_1 -
			\Delta_2^\ast} \cclosed{\Delta_1
			\operatorname{dLog}\pclosed{t_+ - \Delta_1,
				t_- - \Delta_1} - \Delta_2^\ast
			\operatorname{dLog}\pclosed{t_+ -
				\Delta_2^\ast, t_- - \Delta_2^\ast}}
		\\
		\nonumber
		T^2\!\pclosed{\Delta_1, \Delta_2} = & 2\sqrt{\hat{s}}p	+
		\frac{1}{\Delta_1 - \Delta_2^\ast}
		\\
		                                    & \times
		\cclosed{\Delta_1^2 \operatorname{dLog}\pclosed{t_+ - \Delta_1,
				t_- - \Delta_1}
			- \pclosed{\Delta_2^\ast}^2
			\operatorname{dLog}\pclosed{t_+ - \Delta_2^\ast,
				t_- - \Delta_2^\ast}}
	\end{align}
\end{subequations}
In here, I have defined \(\operatorname{dLog}(z, w)\) to be the log-difference
between to complex number \(z, w\) such that
\begin{equation}
	\operatorname{dLog}(z, w) \equiv \ln\abs{\frac{z}{w}} +
	i\pclosed{\arctan\frac{\Im{z}}{\Re{z}} - \arctan\frac{\Im{w}}{\Re{w}}}.
\end{equation}
The only non-zero arguments that will appear in these integrals are
\(\Delta^{\hat{t}}_A = m_{\tilde{q}_A}^2 - i\Gamma_{\tilde{q}_A}
m_{\tilde{q}_A}\) and \(\Delta^{\hat{u}}_A = m_i^2 + m_j^2 - \hat{s} -
m_{\tilde{q}_A}^2 + i\Gamma_{\tilde{q}_A} m_{\tilde{q}_A}\).

\begin{temporary}
	\begin{align}
		\hat{\sigma}^0 = & \frac{\pi \alpha^2}{\hat{s}^2 c_W^4 N_C}
		\pclosed{\frac{1}{2}}^{\delta_{ij}}
		\\
		\times           & \Bigg\{
		\frac{2\sqrt{\hat{s}}p\pclosed{\abs{Z^L}^2 + \abs{Z^R}^2}
			\pclosed{\frac{32}{3}\hat{s} p^2 - (m_i^2-m_j^2)^2 +
				8m_i^2 m_j^2 +
				hat{s}(m_i^2+m_j^2)}}{\abs{\hat{s}-\Delta_Z}^2}
		\\
		                 & - \frac{4\sqrt{\hat{s}}p\Re{(Z^L)^2 +
				(Z^R)^2} \hat{s} m_i m_j}{\abs{s-\Delta_Z}^2}
		\\
		                 & \sum_{X,Y} 4\operatorname{Re}\Bigg\{
		\operatorname{dLog}(t_+ - \Delta_A, t_- - \Delta_A) \bigg[
		\frac{\delta_{XY}\Re{Q_A^{XX}Q_B^{XX}}
		\hat{s}m_im_j}{\Delta_A
		+ \Delta_B^\ast + \hat{s} - m_i^2 - m_j^2}
		\\
		                 & \frac{(1-\delta_{XY})\Re{Q_A^{XY}Q_B^{YX}}
		\pclosed{\Delta_A(\Delta_A-\hat{s}+m_i^2+m_j^2) + m_i^2
			m_j^2}}{\Delta_A +
		\Delta_B^\ast + \hat{s} - m_i^2 - m_j^2}
		\\
		                 & + \frac{\Re{Q_A^{XY}
		\pclosed{Q_B^{XY}}^\ast} (\Delta_A - m_i^2)(\Delta_A -
		m_j^2)}{\Delta_A -
		\Delta_B^\ast} \bigg] \Bigg\}
		\Bigg\}
	\end{align}
\end{temporary}

\section{NLO Corrections}
\begin{TODO}
	\item Discuss supersymmetry breaking in dimensional regularisation and
	its effect on
	the cross-section.
\end{TODO}
\subsection{Self-Energy Contributions}

\begin{figure}
	\centering
	\inputtikz{s_channel_se_1}
	\inputtikz{s_channel_se_2}
	\caption{}
	\label{pc:fig:s_channel_se}
\end{figure}

\subsection{Vertex Corrections}

\subsection{Box Diagrams}

\subsection{Real Emission}

\end{document}

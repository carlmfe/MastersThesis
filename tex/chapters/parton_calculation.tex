\begin{TODO}
    \item Formulate a section on the dipole formalism used in Debove et al. and make a comparison.
\end{TODO}

\section{Kinematics}
    To start off, it will be useful to introduce some procedure for going forward in the phase space of an inclusive \(2\to 2(+1)\) cross-section process.
    The phase space of 2-body and 3-body final states are quite different as there are more degrees of freedom in the 3-body final state.
    In the end, these extra degrees of freedom will be need to be integrated over to make an additive comparison between the 2-body and 3-body processes, however, exactly how we choose to parametrise and subsequently integrate over the extra degrees of freedom can matter quite a bit.
    \medskip
    To start out, let us count the degrees of freedom of a scattering problem involving \(N\) four-momenta \(p_{i=1,\ldots,N}\).
    Assuming our end result to be Lorentz invariant, there are \(N(N+1)/2\) different scalar products that can be produced using \(N\) different four-momenta.
    Momentum conservation allows us to eliminate one momentum, such that we have \(N(N-1)/2\) possible scalar products.
    Denoting the scalar products by \(m^2_{ij} \equiv (p_i + p_j)^2\) for \(j \neq i\), and \(m^2_i \equiv p_i^2\), we can find a relation between scalar products by using momentum conservation.
    \begin{align} \nonumber
        m^2_{ij} & = \pclosed{p_i - \sum_{k\neq j} p_k}^2 = \pclosed{\sum_{k\neq i,j} p_k}^2 = \sum_{k\neq i,j} \sum_{l\neq i,j} p_k\cdot p_l \\
        \nonumber
                 & = \sum_{k\neq i,j} \sum_{l\neq i,j,k} \frac{m^2_{kl} - m^2_k - m^2_l}{2} + \sum_{k\neq i,j} m^2_k                          \\
        \nonumber
                 & = \sum_{k\neq i,j} \sum_{\substack{l\neq i,j                                                                               \\l>k}} m^2_{kl} - \frac{1}{2} \sum_{k\neq i, j} (N-3) m^2_k - \frac{1}{2} \sum_{l\neq i,j} (N-3) m^2_l + \sum_{k\neq i,j} m^2_k \\
                 & = \sum_{k\neq i,j} \sum_{\substack{l\neq i,j                                                                               \\l>k}} m^2_{kl} - (N-4) \sum_{k\neq i, j} m^2_k.
    \end{align}
    \comment{This little generalised relation might not be immediately necessary\ldots}
    Furthermore, we assign the \(N\) scalar products \(m^2_i\) to the invariant masses of the incoming and outgoing particles, thus not counting them as degrees of freedom, leaving us with
    \(n_\text{dof} = \frac{N(N-3)}{2}\) degrees of freedom.\footnote{I note that we often consider the invariant mass of the incoming bodies to be fixed, which would reduce our degrees of freedom by one.}
    This means that in a \(2\to 2\) process, we have 2 degrees of freedom, and in a \(2\to 3\) process we have 5.


    \subsection{3-body Phase Space}
        \comment{Fill out an introduction here.}\\
        The differential Lorentz invariant phase space for a 3-body final state with four-momenta \(p_i, p_j, k\) in \(d\) dimensions is
        \begin{align}
            \mathrm{d}\Pi_\text{3-body} = \delta^d(P - p_i - p_j - k) \frac{\mathrm{d}^{d-1} \vec{p}_i}{(2\pi)^{d-1}} \frac{1}{2E_i} \frac{\mathrm{d}^{d-1} \vec{p}_j}{(2\pi)^{d-1}} \frac{1}{2E_j} \frac{\mathrm{d}^{d-1} \vec{k}}{(2\pi)^{d-1}} \frac{1}{2\omega}
        \end{align}



        \begin{align}
            \frac{\mathrm{d}^{d-1}\vec{k}}{(2\pi)^{d-1}} \frac{1}{2\omega} = -\frac{1}{(4\pi)^{\sfrac{d}{2}}} \frac{(s-Q^2)^{d-3}}{s^{\sfrac{(d-2)}{2}}} \pclosed{y(1-y)}^{\frac{d-4}{2}} \mathrm{d}y \, \mathrm{d}Q^2
        \end{align}
        \begin{align}
            \frac{\mathrm{d}^{d-1} \vec{p}_i}{(2\pi)^{d-1}} \frac{1}{2E_i} \frac{\mathrm{d}^{d-1} \vec{p}_j}{(2\pi)^{d-1}} \frac{1}{2E_j} (2\pi)^d \delta^d(P - p_i - p_j - k) \\
            = \frac{1}{(8\pi)^{\sfrac{d}{2}}} \frac{1}{\Gamma\!\pclosed{\frac{d-2}{2}}} \frac{\pclosed{\lambda(Q^2, m_i^2, m_j^2)}^{\frac{d-3}{2}}}{8 Q^{d-2}} \sin^{d-3}\theta^\ast \,\mathrm{d}\theta^\ast \, \mathrm{d}\phi^\ast
        \end{align}

        \begin{align}
            \mathrm{d}\Pi_\text{3-body} = -\frac{1}{2^{\sfrac{3d}{2}} (2\pi)^d} \frac{1}{\Gamma\!\pclosed{\frac{d-2}{2}}} \frac{(s-Q^2)^{d-3}}{s^{\sfrac{(d-2)}{2}}} \pclosed{y(1-y)}^{\frac{d-4}{2}} \\
            \frac{\pclosed{\lambda(Q^2, m_i^2, m_j^2)}^{\frac{d-3}{2}}}{8 Q^{d-2}} \sin^{d-3}\theta^\ast \,\mathrm{d}\theta^\ast \, \mathrm{d}\phi^\ast \, \mathrm{d}y \, \mathrm{d}Q^2
        \end{align}

        \begin{align}
            \mathrm{d}\sigma = -\frac{1}{2^{\sfrac{3d}{2}} (2\pi)^d} \frac{1}{\Gamma\!\pclosed{\frac{d-2}{2}}} \frac{(s-Q^2)^{d-3}}{s^{\sfrac{(d-2)}{2}}} \pclosed{y(1-y)}^{\frac{d-4}{2}} \\
            \frac{\pclosed{\lambda(Q^2, m_i^2, m_j^2)}^{\frac{d-3}{2}}}{16 s Q^{d-2}} \abs{\M}^2 \sin^{d-3}\theta^\ast \,\mathrm{d}\theta^\ast \, \mathrm{d}\phi^\ast \, \mathrm{d}y \, \mathrm{d}Q^2
        \end{align}


        Parametrising the free variables in a \(2 \to 3\) process can be tricky.
        I will define some natural variables in two different frames of reference, and rediscover the Lorentz transformation between them to parametrise all scalar products
        in terms of the variables in these reference frames.
        First, we will consider the lab frame, or the centre-of-mass frame of the incoming partons with momenta \(\vec{k}_{i,j}\).
        We can reduce this to an ordinary \(2 \to 2\) scattering by considering the outgoing neutralinos with momenta \(\vec{p}_{i,j}\) as a single system.
        This lets us write the momenta as
        \begin{subequations}
            \begin{align}
                k_i^\mu            & = \frac{\sqrt{s}}{2} \pclosed{1, 0, 0, 1},                                   \\
                k_j^\mu            & = \frac{\sqrt{s}}{2} \pclosed{1, 0, 0, -1},                                  \\
                k\^{\mu}           & = \frac{\sqrt{s}}{2} (1-z) \pclosed{1, \sin\theta, 0, \cos\theta},           \\
                (p_i + p_j)\^{\mu} & = \frac{\sqrt{s}}{2} \pclosed{(1+z), -(1-z)\sin\theta, 0, -(1-z)\cos\theta}.
            \end{align}
        \end{subequations}

        The centre-of-mass frame of the neutralinos is defined by \((p^\ast_i + p^\ast_k)\^{\mu} = \pclosed{\sqrt{zs}, 0, 0, 0}\).\footnote{I will from now on always put a star on quantities pertaining to the centre-of-mass frame of the neutralinos.}
        We find the transformation to this frame then by making appropriate boosts and rotations of this four-vector.
        Let us start by rotating the 3-momentum to lie along the positive \(z\)-direction.
        As the \(y\)-component is already zero in the lab-frame, we only require a rotation around the \(y\)-axis, we can be parametrised by the following matrix
        \begin{equation}
            \text{Rot}_y(\alpha) = \begin{pmatrix}
                1 & 0           & 0 & 0          \\
                0 & \cos\alpha  & 0 & \sin\alpha \\
                0 & 0           & 1 & 0          \\
                0 & -\sin\alpha & 0 & \cos\alpha
            \end{pmatrix}.
        \end{equation}
        Using \(\alpha = -\theta-\pi\) we get that \(\text{Rot}_y(-\theta-\pi) (p_i + p_j)\^{\mu} = \frac{\sqrt{s}}{2} \pclosed{(1+z), 0, 0, (1-z)}\).
        We can subsequently boost along the \(z\)-axis to eliminate the \(z\)-component.
        Such a boost can be parametrised by
        \begin{equation}
            \text{Boost}_z(\beta) = \begin{pmatrix}
                \gamma      & 0 & 0 & \gamma\beta \\
                0           & 1 & 0 & 0           \\
                0           & 0 & 1 & 0           \\
                \gamma\beta & 0 & 0 & \gamma
            \end{pmatrix},
        \end{equation}
        where \(\gamma = \pclosed{1 - \beta^2}^{-\sfrac{1}{2}}\).
        The \(z\)-component is eliminated using \(\beta = -\frac{1-z}{1+z}\), such that we end up with
        \[(p_i^\ast + p_j^\ast)\^\mu \equiv \text{Boost}_z\pclosed{-\frac{1-z}{1+z}} \text{Rot}_y\pclosed{-\theta-\pi} \pclosed{p_i + p_j}^\mu = \pclosed{\sqrt{zs}, 0, 0, 0}\]
        as we expected.

        Now we can parametrise \(\vec{p}_{i,j}^\ast\) in this frame using two angular variables \(\theta^\ast, \phi^\ast\) as
        \begin{subequations}
            \begin{align}
                {p_i^\ast}\^\mu & = \pclosed{E_i, p\sin\theta^\ast \cos\phi^\ast, p\sin\theta^\ast \sin\phi^\ast, p \cos\theta^\ast},    \\
                {p_j^\ast}\^\mu & = \pclosed{E_j, -p\sin\theta^\ast \cos\phi^\ast, -p\sin\theta^\ast \sin\phi^\ast, -p \cos\theta^\ast}, \\
                {k^\ast}\^\mu   & = \frac{\sqrt{s}}{2} \frac{1-z}{\sqrt{z}} \pclosed{1, 0, 0, -1},
            \end{align}
        \end{subequations}
        where \(E_{i,j}(z) = \frac{zs + m^2_{i,j} - m^2_{j,i}}{2 \sqrt{zs}}\) and \(p(z) = \frac{\sqrt{\lambda\pclosed{zs, m^2_i, m^2_j}}}{2 \sqrt{zs}}\).



        \begin{figure}
            \centering
            \begin{subfigure}{.49\textwidth}
                \centering
                \inputtikz[auto, >=latex, scale=1.1]{kinematics_lab_frame}
                \caption{Angular definition in the centre-of-mass frame of the initial particles with momenta \(\vec{k}_{i,j}\).}
            \end{subfigure}
            \hfill
            \begin{subfigure}{.49\textwidth}
                \centering
                \inputtikz[auto, >=latex, scale=1.1]{kinematics_com_frame}
                \caption{Angular definitions in the centre-of-mass frame of the outgoing particles with momenta \(\vec{p}_{i,j}\).}
            \end{subfigure}
            \caption{}
        \end{figure}

        \clearpage


        \begin{equation}
            \lambda(x, y, z) = x^2 + y^2 + z^2 - 2xy - 2xz - 2yz
        \end{equation}

        \begin{align}
            \nonumber
            t_{ii}(z, y, \theta^\ast, \phi^\ast) \equiv (k_i - p_i)^2 = \frac{1}{2z} \bigg\{ (m_i^2 + m_j^2)z - (m_i^2 - m_j^2)y(1-z) + sz(yz - y - z) \\
            2z \sqrt{y(1-y)} \sqrt{\lambda(Q^2,m_i^2,m_j^2)}\sin\theta^\ast \cos\phi^\ast + (z-y-yz)\sqrt{\lambda(Q^2,m_i^2,m_j^2)} \cos\theta^\ast \bigg\}
        \end{align}

        \begin{align}
            \nonumber
            m^2_{gi}(z, \theta^\ast) \equiv (p_i + k)^2 = \frac{1}{2z} \bigg\{ (m_i^2-m_j^2) + (m_i^2+m_j^2)z \\
            + sz(1-z) + \sqrt{\lambda(Q^2, m_i^2, m_j^2)} (1-z) \cos\theta^\ast \bigg\}
        \end{align}

        \begin{align}
            \abs{\M}^2 \supset \frac{\pclosed{t_{ii}(z, y, \theta^\ast, \phi^\ast) - m_i^2} \pclosed{m^2_{gi}(z, \theta^\ast) - m_i^2 - (1-z)s}}{s y (1-z) \pclosed{t_{ii}(z, y, \theta^\ast, \phi^\ast) - m_A^2} \pclosed{t_{ii}(z, y, \theta^\ast, \phi^\ast) - m_B^2}}
        \end{align}


\section{Leading Order Cross-Section}

\section{NLO Corrections}
    \begin{TODO}
        \item Discuss supersymmetry breaking in dimensional regularisation and its effect on the cross-section.
    \end{TODO}
    \subsection{Self-Energy Contributions}

    \subsection{Vertex Corrections}

    \subsection{Box Diagrams}

    \subsection{Real Emission}

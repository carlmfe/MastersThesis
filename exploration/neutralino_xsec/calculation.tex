\documentclass[english,notitlepage]{article}

\usepackage{overhead}
\usepackage[style=verbose]{biblatex}

\addbibresource{calculation.bib}



\title{Leading Order Neutralino Calculation}
\author{Carl Martin Fevang}

\begin{document}
\maketitle


\begin{figure} [ht!]
    \centering
    \inputtikz{s_channel}
    \inputtikz{t_channel}
    \inputtikz{u_channel}
\end{figure}

\begin{subequations}
    \begin{align}
        s         & = \pclosed{p_1 + p_2}^2 = \pclosed{k_1 + k_2}^2 \\
        t         & = \pclosed{p_1-k_1}^2 = \pclosed{p_2-k_2}^2     \\
        u         & = \pclosed{p_1-k_2}^2 = \pclosed{p_2-k_1}^2     \\
        s + t + u & = m_i^2 + m_j^2
    \end{align}
\end{subequations}

\begin{subequations}
    \begin{align}
         & (p_1 \cdot p_2) = \frac{s}{2}         &  & (k_1 \cdot k_2) = \frac{s-m_i^2-m_j^2}{2} \\
         & (p_1 \cdot k_1) = \frac{m_i^2 - t}{2} &  & (p_2 \cdot k_2) = \frac{m_j^2-t}{2}       \\
         & (p_1 \cdot k_2) = \frac{m_j^2 - u}{2} &  & (p_2 \cdot k_1) = \frac{m_i^2-u}{2}
    \end{align}
\end{subequations}
\begin{subequations}
    \begin{align} \nonumber
        \M_s = -\frac{g^2}{2c_W^2} D_Z(s) & \bclosed{ \wbar{u}_i\gamma^\mu\pclosed{O^{''L}_{ij}P_L + O^{''R}_{ij}P_R} v_j }    \\
                                          & \times \bclosed{ \wbar{v}_2 \gamma_\mu \pclosed{C^L_{Zqq}P_L + C^R_{Zqq}P_R} u_1 } \\ \nonumber
        \M_t = -2g^2 D_{\wtilde{q}}(t)    & \bclosed{ \wbar{u}_i \pclosed{C_{i}^{L*}P_L + C_{i}^{R*} P_R} u_1 }                \\
                                          & \times \bclosed{ \wbar{v}_2 \pclosed{C^R_{j}P_L + C^L_{j}P_R} v_j }                \\ \nonumber
        \M_u = -2g^2 D_{\wtilde{q}}(u)    & \bclosed{ \wbar{u}_j \pclosed{C_{j}^{L*}P_L + C_{j}^{R*} P_R} u_1 }                \\
                                          & \times \bclosed{ \wbar{v}_2 \pclosed{C^R_{i}P_L + C^L_{i}P_R} v_i }
    \end{align}
\end{subequations}


\begin{subequations}
    \begin{align} \nonumber
        I_{ss} = \sum_\text{spins} \abs{\M_s}^2 = \frac{g^4}{c_W^2} \abs{D_Z(s)}^2 \pclosed{{(C_Z^L)}^2+{(C_Z^R)}^2} & \bigg\{ \abs{O_{ij}^L}^2 \bclosed{ {(m_i^2-t)}^2 + {(m_j^2-t)}^2 } \\
                                                                                                                     & - 2\Re\cclosed{\pclosed{O_{ij}^L}^2} m_i m_j s \bigg\}
    \end{align}
\end{subequations}


\section{Fierz identities}
    Introducing first the generalised gamma matrices \(\Gamma_I^r\) defined in the following way:
    \begin{subequations}
        \begin{align}
            \Gamma_S^0            & = \mathbb{1},                        \\
            \Gamma_V^{0,\ldots,3} & = \gamma^\mu,                        \\
            \Gamma_T^{0,\ldots,5} & = \sigma^{\mu\nu}, \quad (\mu < \nu) \\
            \Gamma_A^{0,\ldots,3} & = \gamma^\mu \gamma^5,               \\
            \Gamma_P^0            & = \gamma^5,
        \end{align}
    \end{subequations}
    where \(\sigma^{\mu\nu} = \frac{i}{2} \bclosed{\gamma^\mu, \gamma^\nu}\) and \(\gamma^5 = i \gamma^0 \gamma^1 \gamma^2 \gamma^3\). The upper index \(r\) is understood to be summed over if it is repeated in an expression, while the index \(I\) is only summed over when explicitly stated. The complement \(\Gamma_{I,r}\) is found by lowering any Lorentz index in the standard way.
    The generalised Fierz identity then tells us that for spinors \(w_{1,\ldots,4}\) that can either be positive energy spinors \(u\) or negative energy spinors \(v\), we have that
    \begin{align}
        \pclosed{\wbar{w}_1 \Gamma_I^r w_2} \pclosed{\wbar{w}_3 \Gamma_{J}^s w_4} & = \sum_{M,N} \tensor*[^{IJ}_{rs}]{C}{^{MN}_{tu}} \pclosed{\wbar{w}_1 \Gamma_M^t w_4} \pclosed{\wbar{w}_3 \Gamma_N^u w_2},
    \end{align}
    with numerical coefficients \(\tensor*[^{IJ}_{rs}]{C}{^{MN}_{tu}}\). The coefficients are found by
    \begin{align}
        \tensor*[^{IJ}_{rs}]{C}{^{MN}_{tu}} & = \frac{1}{16} \tr\bclosed{\Gamma_{M,t} \Gamma_I^r \Gamma_{N,u} \Gamma_J^{s}}
    \end{align}
    The Fierz transformation matrix \(F\) is given by\footcite{Nieves:2003in}
    \begin{align}
        F = \frac{1}{4}
        \begin{bmatrix}
            1  & 1  & \frac{1}{2} & -1 & 1  \\
            4  & -2 & 0           & -2 & -4 \\
            12 & 0  & -2          & 0  & 12 \\
            -4 & -2 & 0           & -2 & 4  \\
            1  & -1 & \frac{1}{2} & 1  & 1
        \end{bmatrix}
    \end{align}
    in the bilinear product basis
    \begin{subequations}
        \begin{align}
            q_S(1234) & = \pclosed{\wbar{w}_1 w_2} \pclosed{\wbar{w}_3 w_4}                                         \\
            q_V(1234) & = \pclosed{\wbar{w}_1 \gamma^\mu w_2} \pclosed{\wbar{w}_3 \gamma_\mu w_4}                   \\
            q_T(1234) & = \pclosed{\wbar{w}_1 \sigma^{\mu\nu} w_2} \pclosed{\wbar{w}_3 \sigma_{\mu\nu} w_4}         \\
            q_A(1234) & = \pclosed{\wbar{w}_1 \gamma^\mu \gamma^5 w_2} \pclosed{\wbar{w}_3 \gamma_\mu \gamma^5 w_4} \\
            q_P(1234) & = \pclosed{\wbar{w}_1 \gamma^5 w_2} \pclosed{\wbar{w}_3 \gamma^5 w_4}                       \\
        \end{align}
    \end{subequations}
    where a vector \(\vec{q}\) is given by
    \begin{align}
        \vec{q}(abcd) & = \sum_{i=1}^5 n_i \vec{e}_i,
    \end{align}
    for some coefficients \(n_i\) and with the canonical unit vectors \(\cclosed{\vec{e}_i} = \Big\{\begin{pmatrix} 1\\0\\\vdots \end{pmatrix}, \begin{pmatrix} 0\\1\\\vdots \end{pmatrix}, \ldots \Big\}\).
    The Dirac quadrilinear \(q\) represents is found by the inner product
    \begin{align}
        \vec{q} \cdot \sum_{i=1}^5 q_{B_i}(abcd) \vec{e}_i,
    \end{align}
    where \(B_i = S, V, T, A, P\).

    The Fierz transformation swapping the indices \(2 \leftrightarrow 4\) can be found by
    \begin{align}
        \vec{q}(1234) = F \vec{q}(1432)
    \end{align}

\section{Factorisation}

    \begin{align}
        \nonumber
          & \bclosed{\wbar{u}_i \pclosed{C^{L*}_i P_L + C^{R*}_i P_R} u_1} \bclosed{\wbar{v}_2 \pclosed{C^{R}_j P_L + C^{L}_j P_R} v_j}                     \\
        \nonumber
        = & C_{SS}\bclosed{\wbar{u}_i u_1} \bclosed{\wbar{v}_2 v_j} + C_{SP}\bclosed{\wbar{u}_i u_1} \bclosed{\wbar{v}_2 \gamma^5 v_j}                      \\
          & + C_{PS}\bclosed{\wbar{u}_i \gamma^5 u_1} \bclosed{\wbar{v}_2 v_j} + C_{PP}\bclosed{\wbar{u}_i \gamma^5 u_1} \bclosed{\wbar{v}_2 \gamma^5 v_j},
    \end{align}
    where we have
    \begin{subequations}
        \begin{align}
            C_{SS} & = \frac{1}{4} \pclosed{C^{L*}_i + C^{R*}_i}\pclosed{C^{L}_j + C^{R}_j}  \\
            C_{SP} & = \frac{1}{4} \pclosed{C^{L*}_i + C^{R*}_i}\pclosed{C^{L}_j - C^{R}_j}  \\
            C_{PS} & = -\frac{1}{4} \pclosed{C^{L*}_i - C^{R*}_i}\pclosed{C^{L}_j + C^{R}_j} \\
            C_{PP} & = -\frac{1}{4} \pclosed{C^{L*}_i - C^{R*}_i}\pclosed{C^{L}_j - C^{R}_j}
        \end{align}
    \end{subequations}

    \begin{align}
        \nonumber
          & \bclosed{\wbar{u}_i \pclosed{C^{L*}_{\nino[i]\wtilde{q}_A q} P_L + C^{R*}_{\nino[i]\wtilde{q}_A q} P_R} u_1}
        \bclosed{\wbar{v}_2 \pclosed{C^{R}_{\nino[j]\wtilde{q}_A q} P_L + C^{L}_{\nino[j]\wtilde{q}_A q} P_R} v_j}                                                                                                                                    \\
        \nonumber
        = & C^{L*}_{\nino[i]\wtilde{q}_A q}C^{R}_{\nino[j]\wtilde{q}_A q} \bclosed{\frac{1}{2} \pclosed{\wbar{u}_i P_L v_j} \pclosed{\wbar{v}_2 P_L u_1} + 3\pclosed{\wbar{u}_i \sigma^{\mu\nu} v_j} \pclosed{\wbar{v}_2 \sigma_{\mu\nu} P_L u_1} }   \\
        \nonumber
          & + C^{L*}_{\nino[i]\wtilde{q}_A q}C^{L}_{\nino[j]\wtilde{q}_A q} \bclosed{2 \pclosed{\wbar{u}_i \gamma^\mu P_L v_j} \pclosed{\wbar{v}_2 \gamma_\mu P_R u_1}}                                                                               \\
        \nonumber
          & + C^{R*}_{\nino[i]\wtilde{q}_A q}C^{R}_{\nino[j]\wtilde{q}_A q} \bclosed{2 \pclosed{\wbar{u}_i \gamma^\mu P_R v_j} \pclosed{\wbar{v}_2 \gamma_\mu P_L u_1}}                                                                               \\
          & + C^{R*}_{\nino[i]\wtilde{q}_A q}C^{L}_{\nino[j]\wtilde{q}_A q} \bclosed{\frac{1}{2} \pclosed{\wbar{u}_i P_R v_j} \pclosed{\wbar{v}_2 P_R u_1} + 3\pclosed{\wbar{u}_i \sigma^{\mu\nu} v_j} \pclosed{\wbar{v}_2 \sigma_{\mu\nu} P_R u_1} }
    \end{align}

\end{document}

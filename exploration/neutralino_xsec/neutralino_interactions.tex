\documentclass[english, notitlepage]{article}

\usepackage{overhead}
\usepackage[style=verbose]{biblatex}

\addbibresource{neutralino_interactions.bib}

\title{Neutralino Interactions in the MSSM}
\author{Carl Martin Fevang}

\begin{document}
\maketitle

\begin{abstract}
    \noindent
    In this document, I derive how to construct the fermion interaction Lagrangian from kinetic and Yukawa terms of the superlagrangian.
    This is then applied to the MSSM superlagrangian to get the Feynman rules for neutralino interaction with the SM \(Z\)-boson and quark/squark pairs.
\end{abstract}

\section{Superfields}
    Here I list some general expansions of fields over superspace, \textit{superfield}.
    The fields are expanded in the superspace coordinates \(\theta_{A=1,2}, \wbar{\theta}^{\dot{A}=1,2}\) that are four Grassmann coordinates imposed in a spinor structure with one left-handed Weyl spinor and a right-handed Weyl spinor.

    A left-handed scalar superfield $\Phi$ can be written out in terms of component
    fields as\footnote{Parentheses are used to clarify Weyl spinor contraction.}
    \begin{align}
        \Phi = A + i(\theta\sigma^\mu\wbar\theta) \partial_\mu A - \frac{1}{4} (\theta\theta)(\wbar\theta\wbar\theta) \dA A +
        \sqrt{2} (\theta\psi) - \frac{i}{\sqrt{2}} (\theta\theta) (\partial_\mu \psi \sigma^\mu \wbar\theta) + (\theta\theta) F,
    \end{align}
    where $A, F$ are complex scalar fields and $\psi$ is a left-handed Weyl spinor field.
    $\Phi$ has a right-handed scalar superfield compliment found by conjugating it:
    \begin{align}
        \Phi^\dag = A^* - i(\theta\sigma^\mu\wbar\theta) \partial_\mu A^* - \frac{1}{4} (\theta\theta)(\wbar\theta\wbar\theta) \dA A^* +
        \sqrt{2} (\wbar\theta\wbar\psi) + \frac{i}{\sqrt{2}} (\wbar\theta\wbar\theta)
        (\theta \sigma^\mu \partial_\mu \wbar\psi) + (\wbar\theta\wbar\theta) F^*,
    \end{align}
    where $\wbar\psi$ is the right-handed compliment of $\psi$ such that $\wbar\psi^{\dot A} = \delta^{\dot A A} {(\psi_A)}^*$.

    A vector superfield $V$ can be written in Wess-Zumino gauge as
    \begin{align}
        \msub{V}{WZ} = (\theta\sigma^\mu\wbar\theta) \bclosed{V_\mu + i\partial_\mu (A - A^*)} + (\theta\theta) (\wbar\theta\wbar\lambda) +
        (\wbar\theta\wbar\theta)(\theta\lambda) + \frac{1}{2}
        (\theta\theta)(\wbar\theta\wbar\theta) D,
    \end{align}
    where $V_\mu$ is a real vector field, $\lambda$ is a left-handed Weyl spinor field and $D$ is a (auxiliary) complex scalar field.
    The $\partial_\mu (A-A^*)$-term represents the gauge freedom remaining in the choice of
    supergauge after choosing Wess-Zumino gauge, and can be ignored when working
    out the interaction terms.
    I note that this gauge implies that no powers of the vector superfield above 2 are non-zero because of the Grassmann content.
    For the remainder of this document, vector superfields will be assumed to be in Wess-Zumino gauge.

\section{MSSM fields}
    For completeness, I list here the relevant superfields containing the neutralinos and the superfields that couple directly to them.
    \subsection{The superfields}
        The neutralino fields are found in the scalar superfield \(SU(2)_L\) doublets $H_u = \begin{pmatrix} H_u^+ \\ H_u^0 \end{pmatrix}$, $H_d = \begin{pmatrix} H_d^0 \\ H_d^- \end{pmatrix}$, and the vector superfields $B^0$ for the $U(1)_Y$ gauge group, and $W^0$ for the $SU(2)_L$ gauge group.
        The fields that couple directly to them are found in these same superfields, the remaining \(SU(2)_L\) vector superfields \(W^\pm\), the scalar superfield \(SU(2)_L\) doublets $L_i = \begin{pmatrix} l_i \\ \nu_i \end{pmatrix}$ and $Q_i = \begin{pmatrix} u_i \\ d_i \end{pmatrix}$, and the $SU(2)_L$ singlet superfields $E_i$, $U_i$ and $D_i$, where \(i=1,2,3\) enumerates the three generations of leptons/quarks.
        \begin{table}[h!]
            \label[tab]{tab:MSSM-fields}
            \centering
            \begin{tabular}{|l|ccc|}
                \hline
                Superfield  & Boson field         & Fermion field        & Auxiliary field \\
                \hline
                $H_{u/d}^0$ & $H_{u/d}^0$         & $\wtilde{H}_{u/d}^0$ & $F_{H^0_{u/d}}$ \\
                $H_{u}^{+}$ & $H_{u}^{+}$         & $\wtilde{H}_{u}^{+}$ & $F_{H^+_{u}}$   \\
                $H_{d}^{-}$ & $H_{d}^{-}$         & $\wtilde{H}_{d}^{-}$ & $F_{H^-_{d}}$   \\
                \hline
                $l_i$       & $\wtilde{l}_{iL}$   & $l_{iL}$             & $F_{l_{iL}}$    \\
                $E_i$       & $\wtilde{l}_{iR}^*$ & $l_{iR}$             & $F_{l_{iR}}^*$  \\
                $\nu_i$     & $\wtilde{\nu}_{iL}$ & $\nu_{iL}$           & $F_{\nu_{iL}}$  \\
                \hline
                $u_i$       & $\wtilde{u}_{iL}$   & $u_{iL}$             & $F_{u_{iL}}$    \\
                $U_i$       & $\wtilde{u}_{iR}^*$ & $u_{iR}$             & $F_{u_{iR}}^*$  \\
                $d_i$       & $\wtilde{d}_{iL}$   & $d_{iL}$             & $F_{d_{iL}}$    \\
                $D_i$       & $\wtilde{d}_{iR}^*$ & $d_{iR}$             & $F_{d_{iR}}^*$  \\
                \hline
                $B^0$       & $B^0_\mu$           & $\bino$              & $D_{B^0}$       \\
                $W^0$       & $W^0_\mu$           & $\wino$              & $D_{W^0}$       \\
                $W^\pm$     & $W^\pm_\mu$         & $\wtilde{W}^\pm$     & $D_{W^\pm}$     \\
                \hline
            \end{tabular}
            \caption{Table of the MSSM superfields and their component field names.Note that the fermion fields are left-handed Weyl spinors, in spite of any \(L\) or \(R\) in the subscript.
                The conjugate superfields changes these to right-handed Weyl spinors.}
        \end{table}

    \subsection{The neutralino component fields}
        Letting $\Psi_{\nino} = \pclosed{\bino, \wino, \hinod, \hinou}$ denote a vector\footnote{I use row vector notation here for convenience.
            In equations this is understood to be a column vector.}
        of the fermion component fields in the $B^0$ and $W^0$ vector superfields and the $H^0_{u/d}$ scalar superfields --- the neutralino mass eigenstates are given by
        \begin{align}
            \nino = \pclosed{\nino[1], \nino[2], \nino[3], \nino[4]} = N \Psi_{\nino},
        \end{align}
        where $N$ is a unitary matrix diagonalising the neutralino mass matrix $M_{\nino\text{-mass}}$.
        \(N\) can be chosen such that the neutralino masses are real and positive.
        The neutralino interactions terms are then found in terms in the superlagrangian that include the vector superfields $B^0$ and $W^0$, and the scalar superfields $H^0_{u/d}$.
        For later use, I note that translating from the $\Psi_{\nino}$-basis to the $\nino$-basis, we have that
        \begin{align}
            \label[eq]{eq:nino_basis}
            \bino = N_{i1}^* \nino[i],\quad \wino = N_{i2}^* \nino[i],\quad \hinod = N_{i3}^* \nino[i],\quad \hinou = N_{i4}^* \nino[i],
        \end{align}
        where a sum over $i$ is implied.

\section{MSSM Superlagrangian}
    Here, I summarise the relevant MSSM superlagrangian terms in which the interactions are found.
    % Idea here is to possibly expand document to include other interactions (like charginos). Right now, this section is bare.
    \subsection{Neutralino interactions}
        The neutralino interactions appear in the kinetic terms of all the superfields
        that couple to the $U(1)_Y$ and $SU(2)_L$ gauge groups, and the Yukawa terms
        that include the Higgs fields in the superpotential.
        The exception is neutralino interaction with the charginos, which is found in the supersymmetric field strength term.

\section{Interaction Lagrangian}
    In the following, I go through how ordinary Lagrangian interaction terms are found from superfield terms in the superlagrangian.
    \subsection{General superlagrangian terms}
        The ordinary interaction Lagrangian is found from integrating over the
        Grassmann variables of the superlagrangian.
        Only terms containing all four Grassmann variables survive this, so we only need to look for the superlagrangian terms that include all of $(\theta\theta)(\wbar\theta\wbar\theta)$.\footnote{Terms with an insufficient amount of $\theta$s are ignored in the following.}
        Looking first at a general kinetic term of a left-handed scalar superfield \(\Phi\) coupled to a $U(1)$ vector superfield \(V\), it has the form\footcite{Binetruy:2006ad}
        \begin{align}
            \msub{\L}{kin} = \Phi^\dag \exp{2qV} \Phi.
        \end{align}
        Using Weyl identities from \cref{app:sec:weyl}, the interactions that include the $\lambda$-fields of the vector superfield can be found as
        \begin{align}
            \label[eq]{eq:lambda-terms}
            \nonumber
            \msub{\L}{kin} & \stackrel{\lambda, \wbar\lambda}{\supset} 2q\cclosed{ A^* (\wbar\theta\wbar\theta)(\theta\lambda) \sqrt{2} (\theta\psi) + \sqrt{2} (\wbar\theta\wbar\psi) (\theta\theta)(\wbar\theta\wbar\lambda) A } \\
                           & \stackrel{\text{\cref{app:eq:weyl1}}}{=} -\sqrt{2}q (\theta\theta)(\wbar\theta\wbar\theta) \cclosed{ (\lambda\psi)A^* + \cc }.
        \end{align}

        Ignoring ordinary kinetic terms, the interactions that include the
        $\psi$-fields of the scalar superfields include
        \begin{align}
            \label[eq]{eq:psi-terms}
            \nonumber
            \msub{\L}{kin} & \stackrel{\psi, \wbar\psi}{\supset} 2q \cclosed{ A^* (\wbar\theta\wbar\theta) (\theta\lambda) \sqrt{2} (\theta\psi) + \sqrt{2} (\wbar\theta\wbar\psi) (\theta\sigma^\mu\wbar\theta) V_\mu \sqrt{2} (\theta\psi) + \sqrt{2} (\wbar\theta\wbar\psi) (\theta\theta) (\wbar\theta\wbar\lambda) A } \\
                           & \stackrel{\text{\cref{app:eq:weyl1,app:eq:weyl2}}}{=} q(\theta\theta)(\wbar\theta\wbar\theta) \cclosed{ -\sqrt{2} (\lambda\psi) A^* + (\psi\sigma^\mu\wbar\psi)V_\mu - \sqrt{2} (\wbar\psi\wbar\lambda)A },
        \end{align}
        where we notice that the $\lambda\psi$-interactions are the same as the ones we found in \cref{eq:lambda-terms}.
        With a Yukawa superpotential on the form
        \begin{align}
            W = y_{ij} \Phi_i \Phi \Phi_j,
        \end{align}
        the superlagrangian looks like
        \begin{align}
            \msub{\L}{Yukawa} = y_{ij} (\wbar\theta\wbar\theta) \Phi_i \Phi \Phi_j + \cc
        \end{align}
        Extracting the $\psi$ fermion interactions from the $\Phi$ superfield, we have
        \begin{align}
            \label[eq]{psi-yukawa-terms}
            \nonumber
            \msub{\L}{Yukawa} & \stackrel{\psi, \wbar\psi}{\supset} y_{ij} (\wbar\theta\wbar\theta) \sqrt{2} (\theta\psi) \cclosed{ A_i \sqrt{2}(\theta\psi_i) + \sqrt{2}(\theta\psi_j)A_j } + \cc \\
                              & \stackrel{\text{\cref{app:eq:weyl1}}}{=} -y_{ij} (\theta\theta)(\wbar\theta\wbar\theta) \cclosed{ A_i(\psi\psi_j) + (\psi_i\psi)A_j + \cc }
        \end{align}

    \subsection{Neutralino interaction terms}
        Now to relate this to the neutralino fields in the MSSM.
        \subsubsection{Gaugino parts}
            First, I will look at the bino and wino interactions.
            From electroweak unification, we have that the coupling $g$ and $g'$ of the $W^a$ and $B^0$ superfields respectively are related by
            \begin{align}
                g' = g t_W,
            \end{align}
            where $t_W = s_W/c_W \equiv \sin\theta_W/\cos\theta_W$ where $\theta_W$ is the Weinberg mixing angle.
            Rewriting using $\sigma_\pm = \frac{1}{2}\pclosed{\sigma_1 \pm i\sigma_2}$, and defining $W^\pm = W^1 \mp iW^2$ and $W^0 \equiv W^3$, we have
            \begin{align}
                Yg'B^0 + \frac{1}{2}g\sigma^a W^a = g\cclosed{ Yt_W B^0 + \frac{1}{2}\sigma_3 W^0 + \frac{1}{2}\sigma_+ W^+ + \frac{1}{2}\sigma_- W^- }
            \end{align}
            So an MSSM superfield doublet $\Phi = \begin{pmatrix} \Phi^+ \\ \Phi^- \end{pmatrix}$ charged under ${U(1)}_Y$ with charge $Y$ and ${SU(2)}_L$ will have a kinetic term
            \begin{align}
                \L_{\Phi\text{-kin}} = \Phi^\dag \exp{2g\bclosed{ Yt_W B^0 + \frac{1}{2}\sigma_3 W^0 + \frac{1}{2}\sigma_+ W^+ + \frac{1}{2}\sigma_- W^- }} \Phi.
            \end{align}
            We can extract the fermion interactions from the vector superfields $B^0$ and $W^0$ using \cref{eq:lambda-terms} to be
            \begin{align}
                \nonumber
                \L_{\Phi\text{-kin}} \stackrel{\bino, \wino}{\supset} -\sqrt{2}g (\theta\theta)(\wbar\theta\wbar\theta) & \Big\{ Yt_W (\bino \psi^+) {A^+}^* + \frac{1}{2} (\wino \psi^+) {A^+}^*      \\
                \label{eq:gaugino_doublet_interaction}
                                                                                                                        & + Yt_W (\bino\psi^-){A^-}^* -\frac{1}{2} (\wino \psi^-){A^-}^* + \cc \Big\}.
            \end{align}
            In the MSSM, the Dirac fermions are made up from two scalar superfields, supplying the left- and right-handed components separately.
            Both superfields couple to the ${U(1)}_Y$ gauge group with charge $Q-I^3$,\footnote{The field supplying the right-handed part has the opposite sign charge such that $\Phi^\dag_R$ and $\Phi_L$ have the same sign.}
            where $Q$ is the electric charge of the fermion, and $I^3$ is the weak isospin; either $\pm \frac{1}{2}$ for the superfields supplying left-handed fermions or 0 for the superfields supplying the right-handed ones.
            Only the left-handed field couples to the ${SU(2)}_L$ gauge group.
            Thus, using \cref{eq:gaugino_doublet_interaction} and \cref{eq:lambda-terms}, the bino and wino interaction with a pair of MSSM fermions formed from a superfield doublet $F = \begin{pmatrix} f_L^+ \\ f_L^- \end{pmatrix}$ and the superfields $f_R^\pm$ are
            \begin{align}
                \label[eq]{F-nino-terms}
                \nonumber
                \L_{\text{EW-kin}} \stackrel{\bino, \wino}{\supset} -\sqrt{2} g (\theta\theta)(\wbar\theta\wbar\theta) & \Big\{ \pclosed{Q_+-\frac{1}{2}}t_W (\bino f_L^+) \wtilde{f}_L^{+*} + \frac{1}{2} (\wino f_L^+) \wtilde{f}_L^{+*}                     \\
                \nonumber
                                                                                                                       & + \pclosed{Q_-+\frac{1}{2}}t_W (\bino f_L^-)\wtilde{f}_L^{-*} - \frac{1}{2} (\wino f_L^-)\wtilde{f}_L^{-*}                            \\
                                                                                                                       & - Q_+t_W (\wbar{\wtilde{B}}^0\wbar{f}_R^+)\wtilde{f}_R^{+*} - Q_-t_W (\wbar{\wtilde{B}}^0\wbar{f}_R^-)\wtilde{f}_R^{-*} + \cc \Big\}.
            \end{align}

            To get the Lagrangian on a familiar form in terms of Dirac spinors, I will define the following fields in a familiar way.
            For clarity, I suppress the \(\pm\) in the fields, and rather write the final Lagrangian on a form which generalises to both of them using \(I^3_f = \pm \frac{1}{2}\) where appropriate.
            \begin{align}
                f = \begin{pmatrix}
                        f_L \\ \wbar{f}_R
                    \end{pmatrix},
                \quad
                \wtilde{B}^0_D = \begin{pmatrix}
                                     \wtilde{B}^0 \\ \wbar{\wtilde{B}}^0
                                 \end{pmatrix},
                \quad
                \wtilde{W}^0_D = \begin{pmatrix}
                                     \wtilde{W}^0 \\ \wbar{\wtilde{W}}^0
                                 \end{pmatrix},
            \end{align}
            with conjugates
            \begin{align}
                \wbar{f} = \begin{pmatrix}
                               f_R \\ \wbar{f}_L
                           \end{pmatrix}^T,
                \quad
                \wbar{\wtilde{B}}^0_D = \begin{pmatrix}
                                            \wtilde{B}^0 \\ \wbar{\wtilde{B}}^0
                                        \end{pmatrix}^T,
                \quad
                \wbar{\wtilde{W}}^0_D = \begin{pmatrix}
                                            \wtilde{W}^0 \\ \wbar{\wtilde{W}}^0
                                        \end{pmatrix}^T.
            \end{align}
            Using \cref{app:sec:weyl2dirac} and integrating (trivially) over the Grassmann coordinates using \(\int \! \mathrm{d}^4 \theta \,(\theta\theta)(\wbar{\theta}\wbar{\theta}) = 1\), we are left with the ordinary Lagrangian term
            \begin{align}
                \label[eq]{f-bino-wino-terms}
                \nonumber
                \L_{f\bino\wino} = -\sqrt{2} g & \Big\{ \wbar{\wtilde{B}}^0_D \bclosed{ \pclosed{Q_f - I^3_f}t_W \wtilde{f}_L^* P_L - Q_f t_W \wtilde{f}_R^* P_R } f \\
                                               & + \wbar{\wtilde{W}}^0_D \pclosed{I^3_f \wtilde{f}_L^* P_L} f + \cc \Big\}.
            \end{align}
            Changing to the $\nino[i]$-basis we have
            \begin{align}
                \label[eq]{nino-sf-f-terms}
                \L_{\nino[i]\wtilde{f}f} = -\sqrt{2} g   \sum_{i} \wbar{\wtilde{\chi}}^0_i & \Big\{ \big[ \underbrace{\pclosed{Q_f - I^3_f}t_W N_{i1}  + I^3_f N_{i2}}_{\equiv C_{\nino[i] \wtilde{f} f}^{L*}} \big] \wtilde{f}_L^* P_L \underbrace{- Q_f t_W N_{i1}}_{\equiv C_{\nino[i] \wtilde{f} f}^{R*}} \wtilde{f}_R^* P_R \Big\} f + \cc
            \end{align}

            We can generalise this to include squark mixing between the left- and right-handed squarks into mass eigenstates \(\wtilde{f}_{A=1,2}\), where
            \begin{align}
                \begin{pmatrix}
                    \wtilde{f}_1 \\
                    \wtilde{f}_2
                \end{pmatrix}
                =
                \begin{bmatrix}
                    c_{\wtilde{f}} & -s_{\wtilde{f}}^* \\
                    s_{\wtilde{f}} & c_{\wtilde{f}}^*
                \end{bmatrix} \begin{pmatrix}
                                  \wtilde{f}_L \\
                                  \wtilde{f}_R
                              \end{pmatrix}.
            \end{align}
            This leaves us with the Lagrangian terms
            \begin{subequations}
                \begin{align}
                    \label[eq]{nino-sf1-f-terms}
                    \L_{\nino[i] \wtilde{f}_1 f} = -\sqrt{2} g \sum_{i} \wbar{\wtilde{\chi}}^0_i & \Big\{ \underbrace{c_{\wtilde{f}}^* C_{\nino[i] \wtilde{f} f}^{L*}}_{\equiv C_{\nino[i] \wtilde{f}_1 f}^{L*}} P_L \underbrace{- s_{\wtilde{f}} C_{\nino[i] \wtilde{f} f}^{R*}}_{\equiv C_{\nino[i] \wtilde{f}_1 f}^{R*}} P_R \Big\} \wtilde{f}_1^* f + \cc \\
                    \label[eq]{nino-sf2-f-terms}
                    \L_{\nino[i] \wtilde{f}_2 f} = -\sqrt{2} g \sum_{i} \wbar{\wtilde{\chi}}^0_i & \Big\{ \underbrace{s_{\wtilde{f}}^* C_{\nino[i] \wtilde{f} f}^{L*}}_{\equiv C_{\nino[i] \wtilde{f}_2 f}^{L*}} P_L + \underbrace{c_{\wtilde{f}} C_{\nino[i] \wtilde{f} f}^{R*}}_{\equiv C_{\nino[i] \wtilde{f}_2 f}^{R*}} P_R \Big\} \wtilde{f}_2^* f + \cc
                \end{align}
            \end{subequations}

        \subsubsection{Higgsino parts}
            The Higgs superfield doublets $H_u = \begin{pmatrix} H_u^+ \\ H_u^0 \end{pmatrix}$ and $H_d = \begin{pmatrix} H_d^0 \\ H_d^- \end{pmatrix}$ have kinetic terms
            \begin{align}
                \L_{H\text{-kin}} = H_u^\dag \exp{g\bclosed{\frac{1}{2}\sigma^a W^a + \frac{1}{2}t_W B^0}} H_u + H_d^\dag \exp{g\bclosed{\frac{1}{2}\sigma^a W^a - \frac{1}{2}t_WB^0}} H_d.
            \end{align}
            These give rise to multiple neutralino interaction terms from the neutral Higgs superfields
            \begin{align}
                \L_{H^0\text{-kin}} = {H_u^0}^\dag \exp{g\bclosed{-\frac{1}{2}W^0 + \frac{1}{2}t_W B^0}} H_u^0 + {H_d^0}^\dag \exp{g\bclosed{\frac{1}{2}W^0 - \frac{1}{2}t_W B^0}} H_d^0.
            \end{align}
            Using \cref{eq:psi-terms} we have the higgsino interaction terms (upper signs correspond to $u$, and lower signs to $d$)
            \begin{align}
                \nonumber
                \L_{\wtilde{H}^0\mathrm{-int}} = & \mp \frac{g}{2} (\wtilde{H}^0_{u/d} \sigma^\mu \wbar{\wtilde{H}}^0_{u/d})\pclosed{W^0_\mu - t_W B^0_\mu}                                        \\
                \label[eq]{eq:L-higgsino_int}
                                                 & \pm \frac{g}{\sqrt{2}} \bclosed{(\wtilde{W}^0 \wtilde{H}^0_{u/d}) H^{0\,*}_{u/d} - t_W (\wtilde{B}^0 \wtilde{H}^0_{u/d}) H^{0\,*}_{u/d} + \cc}.
            \end{align}
            From electroweak symmetry breaking, we have that the \(Z\)-boson vector field is given by \(Z_\mu = c_W W^0_\mu - s_W B^0_\mu\), meaning we can extract a \(Z\)-boson interaction from \cref{eq:L-higgsino_int} as
            \begin{align}
                \nonumber
                \L_{\wtilde{H}^0Z} & = -\frac{g}{2c_W} Z_\mu \bclosed{(\hinou\sigma^\mu\barhinou) - (u \leftrightarrow d)}                                                \\
                \label[eq]{eq:L-higgsino_Z}.
                                   & \stackrel{\mathrm{\cref{app:eq:weyl3}}}{=} \frac{g}{2c_W} Z_\mu \bclosed{(\barhinou\wbar{\sigma}^\mu\hinou) - (u \leftrightarrow d)} \\
            \end{align}
            Converting the Weyl spinors into Dirac spinors in the usual way (see \cref{app:sec:weyl2dirac}), and changing to the \(\nino\)-basis according to \cref{eq:nino_basis}, we get
            \begin{align}
                \nonumber
                \L_{\wtilde{H}^0Z} & = \frac{g}{2c_W} Z_\mu \sum_{ij} \bclosed{N_{i4}N_{j4}^* \barnino[i]\gamma^\mu P_L \nino[j] - (4 \leftrightarrow 3)}                                                                                                                                                                      \\
                \nonumber
                                   & \stackrel{\mathrm{\cref{app:eq:majorana_chiral_identity}}}{=} \frac{g}{4c_W} Z_\mu \sum_{ij} \bclosed{N_{i4}N_{j4}^* \barnino[i]\gamma^\mu P_L \nino[j] - N_{i4}N_{j4}^* \barnino[j]\gamma^\mu P_R \nino[i] - (4 \leftrightarrow 3)}                                                                                                                  \\
                \nonumber
                                   & = \frac{g}{4c_W} Z_\mu \sum_{ij} \bclosed{N_{i4}N_{j4}^* \barnino[i]\gamma^\mu P_L \nino[j] - N_{i4}^*N_{j4} \barnino[i]\gamma^\mu P_R \nino[j] - (4 \leftrightarrow 3)}                                                                                                                  \\
                                   & = \frac{g}{2c_W} Z_\mu \sum_{ij} \barnino[i] \gamma^\mu \Big[ \underbrace{\frac{1}{2}\pclosed{N_{i4}N_{j4}^* - N_{i3}N_{j3}^*}}_{\equiv O^{\prime\prime L}_{ij}}P_L \underbrace{-\frac{1}{2}\pclosed{N_{i4}^*N_{j4} - N_{i3}^*N_{j3}}}_{\equiv O^{\prime\prime R}_{ij}}P_R \Big] \nino[j],
            \end{align}
            where in the third line I relabelled the indices of the second term.

            % \begin{align}
            %     \nonumber
            %     \L_{\wtilde{H}^0 \wino \bino H^0} & = \frac{g}{\sqrt{2}} \bclosed{ \pclosed{(\wino\hinou) - t_W (\bino\hinou)} {H^0_u}^* + \cc - (u \leftrightarrow d)  }                                    \\
            %     \nonumber
            %                                       & = \frac{g}{2\sqrt{2}} \sum_{ij} \bclosed{ \pclosed{N_{i2} - t_W N_{i1}}N_{j4}^*(\barnino[i] P_L \nino[j]) {H^0_u}^* + \cc - (u,4 \leftrightarrow d,3)  } \\
            %                                       & = \frac{g}{2\sqrt{2}} \sum_{ij} \bclosed{ \pclosed{N_{i2} - t_W N_{i1}}(N_{j4}^* {H^0_u}^* - N_{j3}^* {H^0_d}^* )(\barnino[i] P_L \nino[j]) + \cc  }
            % \end{align}

            \subsubsection{Neutralino Feynman rules}
            The neutralino interactions with a quark/squark pair, and with a \(Z\)-boson, can be summarised in the following Feynman rules.
            Due to the symmetry of Majorana particles, we can match either index \(i, j\) with the external neutralinos, such that the corresponding Feynman rule is multiplied by two.
            In the case of identical neutralino interaction, symmetry dictates that the factor of two be removed.
            The external lines drawn indicate that the Feynman rule is identical either way you choose to read it.\footnote{I go more into detail on how to read Feynman diagrams with neutralinos elsewhere.}
            \begin{subequations}
                \begin{align}
                    % \vcenter{\hbox{
                    %         \inputtikz{qqZ_vertex}
                    %     }}
                    %  & = -\frac{ig}{c_W} \gamma^\mu \bclosed{C_{Zqq}^L P_L + C_{Zqq}^R P_R}                                \\
                    \vcenter{\hbox{
                            \inputtikz{chii-chij-Z_vertex}
                        }}
                     & = \frac{ig}{c_W} \gamma^\mu \bclosed{O_{ij}^{\prime\prime L} P_L + O_{ij}^{\prime\prime R} P_R}     \\
                    \vcenter{\hbox{
                            \inputtikz{q-chi-sq_vertex}
                        }}
                     & = -i\sqrt{2}g \bclosed{ C_{\nino[i] \wtilde{q}_A q}^{L*} P_L + C_{\nino[i] \wtilde{q}_A q}^{R*} P_R }
                \end{align}
            \end{subequations}


            {\renewcommand{\arraystretch}{2}
            \begin{table}
                \centering
                \begin{tabular}{|c|c|}
                    \hline
                    Variable                             & Value                                                                                                                           \\
                    \hline
                    \(C_{\nino[i] \wtilde{f}_A f}^{L*}\) & \(\pclosed{\delta^A_1 c_{\wtilde{f}}^* + \delta^A_2 s_{\wtilde{f}^*}} \bclosed{\pclosed{Q_f-I^3_f} t_W N_{i1} + I^3_f N_{i2}}\) \\
                    \(C_{\nino[i] \wtilde{f}_A f}^{R*}\) & \(\pclosed{\delta^A_1 s_{\wtilde{f}} - \delta^A_2 c_{\wtilde{f}}} Q_f t_W N_{i1}\)                                              \\
                    \hline
                    \(O^{\prime\prime L}_{ij}\)          & \(\frac{1}{2} \pclosed{N_{i4} N_{j4}^* - N_{i3} N_{j3}^*}\)                                                                     \\
                    \(O^{\prime\prime R}_{ij}\)          & \(-\frac{1}{2} \pclosed{N_{i4}^* N_{j4} - N_{i3}^* N_{j3}}\)                                                                    \\
                    \hline
                \end{tabular}
                \caption{A summary of the variables used in the derived Feynman rules and their definitions.}
                \label[tab]{tab:variable_definitions}
            \end{table}
            }


            \appendix
\section{Weyl identities}\label[app]{app:sec:weyl}
    A brief summary of the Weyl identities that have been used in the derivations made in this document.
    Given two arbitrary left-handed Weyl spinors \(\psi, \chi\), with right-handed compliments \(\wbar{\psi}, \wbar{\chi}\), the following identities hold.
    \begin{subequations}
        \begin{align}
            \label[eq]{app:eq:weyl1}
            (\theta\psi)(\theta\chi)                                          & = -\frac{1}{2} (\theta\theta) (\psi\chi)                                        \\
            \label[eq]{app:eq:weyl3}
            (\psi \sigma^\mu \wbar\chi)                                       & = -(\wbar\chi \wbar\sigma^\mu \psi)                                             \\
            \label[eq]{app:eq:weyl2}
            (\wbar\theta\wbar\psi) (\theta\sigma^\mu\wbar\theta) (\theta\psi) & = \frac{1}{4} (\theta\theta) (\wbar\theta\wbar\theta) (\psi\sigma^\mu\wbar\psi)
        \end{align}
    \end{subequations}

\section{Weyl spinors to Dirac spinors}\label[app]{app:sec:weyl2dirac}
    We can build a Dirac spinor $\Psi$ using a left-handed Weyl spinor $\psi_L$ and a right-handed Weyl spinor $\wbar{\psi}_R$ such that
    \begin{align}
        \Psi = \begin{pmatrix} \psi_L \\ \wbar{\psi}_R \end{pmatrix}.
    \end{align}
    I note that the labels \(L/R\), perhaps confusingly, do not label whether the Weyl spinor is left- or right-handed, but rather what they were originally \textit{intended} to be --- \(\wbar{\psi}_L\) would still be a right-handed Weyl spinor.
    The $\gamma$-matrices in the Weyl representation are
    \begin{align}
        \gamma^\mu = \begin{bmatrix} 0 & \sigma^\mu \\ \wbar{\sigma}^\mu & 0 \end{bmatrix},
    \end{align}
    so we define the conjugate Dirac spinor
    \begin{align}
        \wbar{\Psi} \equiv \Psi^\dag \gamma^0 = \begin{pmatrix} \psi_R \\ \wbar{\psi}_L \end{pmatrix}^T.
    \end{align}
    A Majorana fermion is constructed from just one Weyl spinor, such that $\psi_L = \psi_R \equiv \psi$.
    The projection operators $P_{L/R}$ project out the left-handed or right-handed Weyl spinors from the Dirac spinor.
    The following Weyl spinor products can then be rewritten in terms of Dirac spinors:
    \begin{subequations}
        \begin{align}
            (\psi_{R} \phi_{L})                      & = \wbar{\Psi} P_L \Phi            \\
            (\wbar{\psi}_L \wbar{\phi}_R)            & = \wbar{\Psi} P_R \Phi            \\
            (\wbar{\psi}_L \wbar{\sigma}^\mu \phi_L) & = \wbar{\Psi} \gamma^\mu P_L \Phi \\
            (\psi_R \sigma^\mu \wbar{\phi}_R)        & = \wbar{\Psi} \gamma^\mu P_R \Phi
        \end{align}
    \end{subequations}
    Using equation \cref{app:eq:weyl3}, we get the Dirac spinor relation between two Majorana spinors \(\Psi_M, \Phi_M\)
    \begin{align}
        \label[eq]{app:eq:majorana_chiral_identity}
        \wbar{\Psi}_M \gamma^\mu P_{L/R} \Phi_M = -\wbar{\Phi}_M \gamma^\mu P_{R/L} \Psi_M
    \end{align}

    % \section{Electroweak theory}
    %     \begin{subequations}
    %         \begin{align}
    %             B^0_\mu & = c_W A^0_\mu - s_W Z^0_\mu \\
    %             W^0_\mu & = s_W A^0_\mu + c_W Z^0_\mu
    %         \end{align}
    %     \end{subequations}

\end{document}
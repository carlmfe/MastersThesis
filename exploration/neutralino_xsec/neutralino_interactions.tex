\documentclass[english, notitlepage]{article}

\usepackage{overhead}
\usepackage[style=verbose]{biblatex}

\addbibresource{neutralino_interactions.bib}

\title{Neutralino Interactions in the MSSM}
\author{Carl Martin Fevang}

\begin{document}
\maketitle

\section{Superfields}
    A left-handed scalar superfield $\Phi$ can be written out in terms of component
    fields as\footnote{Parentheses are used to clarify Weyl spinor contraction.}
    \begin{align}
        \Phi = A + i(\theta\sigma^\mu\wbar\theta) \partial_\mu A - \frac{1}{4} (\theta\theta)(\wbar\theta\wbar\theta) \dA A +
        \sqrt{2} (\theta\psi) - \frac{i}{\sqrt{2}} (\theta\theta) (\partial_\mu \psi \sigma^\mu \wbar\theta) + (\theta\theta) F,
    \end{align}
    where $A, F$ are complex scalar fields and $\psi$ is a left-handed Weyl spinor. $\Psi$ has a right-handed scalar superfield compliment found by conjugating it:
    \begin{align}
        \Phi^\dag = A^* - i(\theta\sigma^\mu\wbar\theta) \partial_\mu A^* - \frac{1}{4} (\theta\theta)(\wbar\theta\wbar\theta) \dA A^* +
        \sqrt{2} (\wbar\theta\wbar\psi) + \frac{i}{\sqrt{2}} (\wbar\theta\wbar\theta)
        (\theta \sigma^\mu \partial_\mu \wbar\psi) + (\wbar\theta\wbar\theta) F^*,
    \end{align}
    where $\wbar\psi$ is the right-handed compliment of $\psi$ such that $\wbar\psi^{\dot A} = \delta^{\dot A A} {(\psi_A)}^*$.

    A vector superfield $V$ can be written in Wess-Zumino gauge as
    \begin{align}
        \msub{V}{WZ} = (\theta\sigma^\mu\wbar\theta) \bclosed{V_\mu + i\partial_\mu (A - A^*)} + (\theta\theta) (\wbar\theta\wbar\lambda) +
        (\wbar\theta\wbar\theta)(\theta\lambda) + \frac{1}{2}
        (\theta\theta)(\wbar\theta\wbar\theta) D,
    \end{align}
    where $V_\mu$ is a real vector field, $\lambda$ is a left-handed Weyl spinor and $D$ is a (auxiliary) complex scalar field. The $\partial_\mu (A-A^*)$-term represents the gauge freedom remaining in the choice of
    supergauge after choosing Wess-Zumino gauge, and can be ignored when working
    out the interaction terms. For the remainder of this document, vector superfields will be assumed to be in Wess-Zumino gauge.

\section{MSSM Superfields}
    For completeness, I list here the relevant superfields containing the
    neutralinos and fields that couple to them. These include the $SU(2)_L$
    superfield doublets $H_u = \begin{pmatrix} H_u^+ \\ H_u^0 \end{pmatrix}$, $H_d = \begin{pmatrix} H_d^0 \\ H_d^- \end{pmatrix}$, $L_i = \begin{pmatrix} l_i \\ \nu_i \end{pmatrix}$ and $Q_i = \begin{pmatrix} u_i \\ d_i \end{pmatrix}$, and $SU(2)_L$ singlet superfields $E_i$, $U_i$ and $D_i$, where \(i=1,2,3\) enumerates the three generations of leptons/quarks. There are also the vector superfields $B^0$ for the $U(1)_Y$ gauge group, and $W^0$ and $W^\pm$ for the $SU(2)_L$ gauge group.
    \begin{table}[h!]
        \label[tab]{tab:MSSM-fields}
        \centering
        \begin{tabular}{|l|ccc|}
            \hline
            Superfield  & Boson field         & Fermion field        & Auxiliary field \\
            \hline
            $H_{u/d}^0$ & $H_{u/d}^0$         & $\wtilde{H}_{u/d}^0$ & $F_{H^0_{u/d}}$ \\
            $H_{u}^{+}$ & $H_{u}^{+}$         & $\wtilde{H}_{u}^{+}$ & $F_{H^+_{u}}$   \\
            $H_{d}^{-}$ & $H_{d}^{-}$         & $\wtilde{H}_{d}^{-}$ & $F_{H^-_{d}}$   \\
            \hline
            $l_i$       & $\wtilde{l}_{iL}$   & $l_{iL}$             & $F_{l_{iL}}$    \\
            $E_i$       & $\wtilde{l}_{iR}^*$ & $l_{iR}$             & $F_{l_{iR}}^*$  \\
            $\nu_i$     & $\wtilde{\nu}_{iL}$ & $\nu_{iL}$           & $F_{\nu_{iL}}$  \\
            \hline
            $u_i$       & $\wtilde{u}_{iL}$   & $u_{iL}$             & $F_{u_{iL}}$    \\
            $U_i$       & $\wtilde{u}_{iR}^*$ & $u_{iR}$             & $F_{u_{iR}}^*$  \\
            $d_i$       & $\wtilde{d}_{iL}$   & $d_{iL}$             & $F_{d_{iL}}$    \\
            $D_i$       & $\wtilde{d}_{iR}^*$ & $d_{iR}$             & $F_{d_{iR}}^*$  \\
            \hline
            $B^0$       & $B^0_\mu$           & $\bino$              & $D_{B^0}$       \\
            $W^0$       & $W^0_\mu$           & $\wino$              & $D_{W^0}$       \\
            $W^\pm$     & $W^\pm_\mu$         & $\wtilde{W}^\pm$     & $D_{W^\pm}$     \\
            \hline
        \end{tabular}
        \caption{Table of the MSSM superfields and their component field names. Note that the fermion fields are left-handed Weyl spinors, in spite of any \(L\) or \(R\) in the subscript. The conjugate superfields changes these to right-handed Weyl spinors.}
    \end{table}

    \subsection{Neutralino Fields}
        Letting $\Psi_{\nino} = \pclosed{\bino, \wino, \hinou, \hinod}$ denote a
        vector\footnote{I use row vector notation here for convenience. In equations
            this is understood to be a column vector.} of the fermion field superpartners
        in the $B^0$ and $W^0$ vector superfields and the $H^0_{u/d}$ scalar superfields --- the
        neutralino mass eigenstates are given by
        \begin{align}
            \nino = \pclosed{\nino[1], \nino[2], \nino[3], \nino[4]} = N \Psi_{\nino},
        \end{align}
        where $N$ is a unitary matrix diagonalising the neutralino mass matrix $M_{\nino\text{-mass}}$. Thus, the neutralino interactions are found in terms in the superlagrangian that include the vector superfields $B^0$ and $W^0$, and the scalar superfields $H^0_{u/d}$. Translating from the $\Psi_{\nino}$-basis to the $\nino$-basis, we have that
        \begin{align}
            \bino = N_{i1}^* \nino[i],\quad \wino = N_{i2}^* \nino[i],\quad \hinod = N_{i3}^* \nino[i],\quad \hinou = N_{i4}^* \nino[i],
        \end{align}
        where a sum over $i$ is implied.

\section{MSSM Superlagrangian}

    \subsection{Neutralino interactions}
        The neutralino interactions appear in the kinetic terms of all the superfields
        that couple to the $U(1)_Y$ and $SU(2)_L$ gauge groups, and the Yukawa terms
        that include the Higgs fields in the superpotential. The exception is neutralino interaction with the charginos, which is found in the supersymmetric field strength term.

\section{Interaction Lagrangian}
    The ordinary interaction Lagrangian is found from integrating over the
    Grassmann variables of the superlagrangian. Only terms containing all four
    Grassmann variables survive this, so we only need to look for the
    superlagrangian terms that include all of
    $(\theta\theta)(\wbar\theta\wbar\theta)$.\footnote{Terms with an insufficient
        amount of $\theta$s are ignored in the following.} Looking first at a general
    kinetic term of a left-handed scalar superfield \(\Phi\) coupled to a $U(1)$ vector superfield \(V\), it has the form\footcite{Binetruy:2006ad}
    \begin{align}
        \msub{\L}{kin} = \Phi^\dag \exp{2qV} \Phi.
    \end{align}
    The interactions that include the $\lambda$-fields of the vector superfield can be found as
    \begin{align}
        \label[eq]{eq:lambda-terms}
        \nonumber
        \msub{\L}{kin} & \stackrel{\lambda, \wbar\lambda}{\supset} 2q\cclosed{ A^* (\wbar\theta\wbar\theta)(\theta\lambda) \sqrt{2} (\theta\psi) + \sqrt{2} (\wbar\theta\wbar\psi) (\theta\theta)(\wbar\theta\wbar\lambda) A } \\
                       & \stackrel{\text{\cref{app:eq:weyl1}}}{=} -\sqrt{2}q (\theta\theta)(\wbar\theta\wbar\theta) \cclosed{ (\lambda\psi)A^* + \cc }.
    \end{align}

    Ignoring ordinary kinetic terms, the interactions that include the
    $\psi$-fields of the scalar superfields include
    \begin{align}
        \label[eq]{eq:psi-terms}
        \nonumber
        \msub{\L}{kin} & \stackrel{\psi, \wbar\psi}{\supset} 2q \cclosed{ A^* (\wbar\theta\wbar\theta) (\theta\lambda) \sqrt{2} (\theta\psi) + \sqrt{2} (\wbar\theta\wbar\psi) (\theta\sigma^\mu\wbar\theta) V_\mu \sqrt{2} (\theta\psi) + \sqrt{2} (\wbar\theta\wbar\psi) (\theta\theta) (\wbar\theta\wbar\lambda) A } \\
                       & \stackrel{\text{\cref{app:eq:weyl1,app:eq:weyl2}}}{=} q(\theta\theta)(\wbar\theta\wbar\theta) \cclosed{ -\sqrt{2} (\lambda\psi) A^* + (\psi\sigma^\mu\wbar\psi)V_\mu - \sqrt{2} (\wbar\psi\wbar\lambda)A },
    \end{align}
    where we notice that the $\lambda\psi$-interactions are the same as the ones we found in \cref{eq:lambda-terms}.
    With a Yukawa superpotential on the form
    \begin{align}
        W = y_{ij} \Phi_i \Phi \Phi_j,
    \end{align}
    the superlagrangian looks like
    \begin{align}
        \msub{\L}{Yukawa} = y_{ij} (\wbar\theta\wbar\theta) \Phi_i \Phi \Phi_j + \cc
    \end{align}
    Extracting the $\psi$ fermion interactions from the $\Phi$ superfield, we have
    \begin{align}
        \label[eq]{psi-yukawa-terms}
        \nonumber
        \msub{\L}{Yukawa} & \stackrel{\psi, \wbar\psi}{\supset} y_{ij} (\wbar\theta\wbar\theta) \sqrt{2} (\theta\psi) \cclosed{ A_i \sqrt{2}(\theta\psi_i) + \sqrt{2}(\theta\psi_j)A_j } + \cc \\
                          & \stackrel{\text{\cref{app:eq:weyl1}}}{=} -y_{ij} (\theta\theta)(\wbar\theta\wbar\theta) \cclosed{ A_i(\psi\psi_j) + (\psi_i\psi)A_j + \cc }
    \end{align}

    \subsection{Neutralino interaction terms}
        \subsubsection{Gaugino parts}
            First, I will look at the bino and wino interactions. From electroweak
            unification, we have that the coupling $g$ and $g'$ of the $W^a$ and $B^0$
            superfields respectively are related by
            \begin{align}
                g' = g t_W,
            \end{align}
            where $t_W = s_W/c_W \equiv \sin\theta_W/\cos\theta_W$ where $\theta_W$ is the Weinberg mixing angle.
            Rewriting using $\sigma_\pm = \frac{1}{2}\pclosed{\sigma_1 \pm i\sigma_2}$, and defining $W^\pm = W^1 \mp iW^2$ and $W^0 \equiv W^3$, we have
            \begin{align}
                Yg'B^0 + \frac{1}{2}g\sigma^a W^a = g\cclosed{ Yt_W B^0 + \frac{1}{2}\sigma_3 W^0 + \frac{1}{2}\sigma_+ W^+ + \frac{1}{2}\sigma_- W^- }
            \end{align}
            So an MSSM superfield doublet $\Phi = \begin{pmatrix} \Phi^+ \\ \Phi^- \end{pmatrix}$ charged under ${U(1)}_Y$ with charge $Y$ and ${SU(2)}_L$ will have a kinetic term
            \begin{align}
                \L_{\Phi\text{-kin}} = \Phi^\dag \exp{2g\bclosed{ Yt_W B^0 + \frac{1}{2}\sigma_3 W^0 + \frac{1}{2}\sigma_+ W^+ + \frac{1}{2}\sigma_- W^- }} \Phi.
            \end{align}
            We can extract the fermion interactions from the vector superfields $B^0$ and $W^0$ using \cref{eq:lambda-terms} to be
            \begin{align} \nonumber
                \L_{\Phi\text{-kin}} \stackrel{\bino, \wino}{\supset} -\sqrt{2}g (\theta\theta)(\wbar\theta\wbar\theta) & \Big\{ Yt_W (\bino \psi^+) {A^+}^* + I^3_+ (\wino \psi^+) {A^+}^*       \\
                                                                                                                        & + Yt_W (\bino\psi^-){A^-}^* + I^3_- (\wino \psi^-){A^-}^* + \cc \Big\},
            \end{align}
            where we recognise $I^3_\pm = \pm \frac{1}{2}$ as the eigenvalues of $\frac{1}{2}\sigma_3$.

            In the MSSM, the Dirac fermions are made up from two scalar superfields,
            supplying the left- and right-handed components separately. Both superfields
            couple to the ${U(1)}_Y$ gauge group with charge $Q-I^3$,\footnote{The field
                supplying the right-handed part has the opposite sign charge such that
                $\Phi^\dag_R$ and $\Phi_L$ have the same sign.} where $Q$ is the electric
            charge of the fermion, and $I^3$ is the weak isospin; either $\pm \frac{1}{2}$
            for the superfields supplying left-handed fermions and 0 for the superfields
            supplying the right-handed ones. Only the left-handed field couples to the
            ${SU(2)}_L$ gauge group. Thus, the bino and wino interaction with a pair of
            MSSM fermions formed from a superfield doublet $F = \begin{pmatrix} f_L^+ \\ f_L^- \end{pmatrix}$ and the superfields $f_R^\pm$ are
            \begin{align}
                \label[eq]{F-nino-terms}
                \nonumber
                \L_{\text{EW-kin}} \stackrel{\bino, \wino}{\supset} -\sqrt{2} g (\theta\theta)(\wbar\theta\wbar\theta) & \Big\{ \pclosed{Q_+-\frac{1}{2}}t_W (\bino f_L^+) \wtilde{f}_L^{+*} + \frac{1}{2} (\wino f_L^+) \wtilde{f}_L^{+*}                     \\ \nonumber
                                                                                                                       & + \pclosed{Q_-+\frac{1}{2}}t_W (\bino f_L^-)\wtilde{f}_L^{-*} - \frac{1}{2} (\wino f_L^-)\wtilde{f}_L^{-*}                            \\
                                                                                                                       & - Q_+t_W (\wbar{\wtilde{B}}^0\wbar{f}_R^+)\wtilde{f}_R^{+*} - Q_-t_W (\wbar{\wtilde{B}}^0\wbar{f}_R^-)\wtilde{f}_R^{-*} + \cc \Big\}.
            \end{align}

            To get the Lagrangian on a familiar form in terms of Dirac spinors, I will
            define the following fields, built from left- and right-handed Weyl spinors:
            \begin{align}
                f = \begin{pmatrix}
                        f_L \\ \wbar{f}_R
                    \end{pmatrix},\quad \wtilde{B}^0_D = \begin{pmatrix}
                                                             \wtilde{B}^0 \\ \wbar{\wtilde{B}}^0
                                                         \end{pmatrix},\quad \wtilde{W}^0_D = \begin{pmatrix}
                                                                                                  \wtilde{W}^0 \\ \wbar{\wtilde{W}}^0
                                                                                              \end{pmatrix},
            \end{align}
            with conjugates
            \begin{align}
                \wbar{f} = \begin{pmatrix}
                               f_R \\ \wbar{f}_L
                           \end{pmatrix}^T,\quad \wbar{\wtilde{B}}^0_D = \begin{pmatrix}
                                                                             \wtilde{B}^0 \\ \wbar{\wtilde{B}}^0
                                                                         \end{pmatrix}^T,\quad \wbar{\wtilde{W}}^0_D = \begin{pmatrix}
                                                                                                                           \wtilde{W}^0 \\ \wbar{\wtilde{W}}^0
                                                                                                                       \end{pmatrix}^T,
            \end{align}
            Suppressing $\pm$ in the field names and inserting the isospin $I^3$ for the factors of $\pm\frac{1}{2}$, we have the ordinary Lagrangian term
            \begin{align}
                \label[eq]{f-bino-wino-terms}
                \nonumber
                \L_{f\bino\wino} = -\sqrt{2} g & \Big\{ \wbar{\wtilde{B}}^0_D \bclosed{ \pclosed{Q_f - I^3_f}t_W \tilde{f}_L^* P_L - Q_f t_W \tilde{f}_R^* P_R } f \\
                                               & + \wbar{\wtilde{W}}^0_D \pclosed{I^3_f \tilde{f}_L^* P_L} f + \cc \Big\},
            \end{align}
            after integrating over the Grassmann variables.
            Changing to the $\nino[i]$-basis we have
            \begin{align}
                \label[eq]{nino-sf-f-terms}
                \L_{\nino[i]\wtilde{f}f} = -\sqrt{2} g   \sum_{i} \wbar{\wtilde{\chi}}^0_i & \Big\{ \big[ \underbrace{\pclosed{Q_f - I^3_f}t_W N_{i1}  + I^3_f N_{i2}}_{\equiv C_{\nino[i] \wtilde{f} f}^{L*}} \big] \wtilde{f}_L^* P_L \underbrace{- Q_f t_W N_{i1}}_{\equiv C_{\nino[i] \wtilde{f} f}^{R*}} \wtilde{f}_R^* P_R \Big\} f + \cc
            \end{align}

            We can generalise this to include squark mixing between the left- and right-handed squarks into mass eigenstates \(\wtilde{f}_{A=1,2}\), where
            \begin{align}
                \begin{pmatrix}
                    \wtilde{f}_1 \\
                    \wtilde{f}_2
                \end{pmatrix}
                =
                \begin{bmatrix}
                    c_{\wtilde{f}} & -s_{\wtilde{f}}^* \\
                    s_{\wtilde{f}} & c_{\wtilde{f}}^*
                \end{bmatrix} \begin{pmatrix}
                                  \wtilde{f}_L \\
                                  \wtilde{f}_R
                              \end{pmatrix}.
            \end{align}
            This leaves us with the Lagrangian terms
            \begin{subequations}
                \begin{align}
                    \label[eq]{nino-sf1-f-terms}
                    \L_{\nino[i] \wtilde{f}_1 f} = -\sqrt{2} g \sum_{i} \wbar{\wtilde{\chi}}^0_i & \Big\{ \underbrace{c_{\wtilde{f}} C_{\nino[i] \wtilde{f} f}^{L*}}_{\equiv C_{\nino[i] \wtilde{f}_1 f}^{L*}} P_L \underbrace{- s_{\wtilde{f}}^* C_{\nino[i] \wtilde{f} f}^{R*}}_{\equiv C_{\nino[i] \wtilde{f}_1 f}^{R*}} P_R \Big\} \wtilde{f}_1^* f + \cc \\
                    \label[eq]{nino-sf2-f-terms}
                    \L_{\nino[i] \wtilde{f}_2 f} = -\sqrt{2} g \sum_{i} \wbar{\wtilde{\chi}}^0_i & \Big\{ \underbrace{s_{\wtilde{f}} C_{\nino[i] \wtilde{f} f}^{L*}}_{\equiv C_{\nino[i] \wtilde{f}_2 f}^{L*}} P_L + \underbrace{c_{\wtilde{f}}^* C_{\nino[i] \wtilde{f} f}^{R*}}_{\equiv C_{\nino[i] \wtilde{f}_2 f}^{R*}} P_R \Big\} \wtilde{f}_2^* f + \cc
                \end{align}
            \end{subequations}

        \subsubsection{Higgsino parts}
            The Higgs superfield doublets $H_u = \begin{pmatrix} H_u^+ \\ H_u^0 \end{pmatrix}$ and $H_d = \begin{pmatrix} H_d^0 \\ H_d^- \end{pmatrix}$ have kinetic terms
            \begin{align}
                \L_{H\text{-kin}} = H_u^\dag \exp{g\bclosed{\frac{1}{2}\sigma^a W^a + \frac{1}{2}t_W B^0}} H_u + H_d^\dag \exp{g\bclosed{\frac{1}{2}\sigma^a W^a - \frac{1}{2}t_WB^0}} H_d.
            \end{align}
            These give rise to multiple neutralino interaction terms from the neutral Higgs superfields
            \begin{align}
                \L_{H^0\text{-kin}} = {H_u^0}^\dag \exp{g\bclosed{-\frac{1}{2}W^0 + \frac{1}{2}t_W B^0}} H_u^0 + {H_d^0}^\dag \exp{g\bclosed{\frac{1}{2}W^0 - \frac{1}{2}t_W B^0}} H_d^0.
            \end{align}
            Using \cref{eq:psi-terms} we have the higgsino interaction terms (upper signs correspond to $u$, and lower signs to $d$)
            \begin{align} \nonumber
                \L_{\tilde{H}^0\ldots} = & \mp \frac{g}{2} (\wtilde{H}^0_{u/d} \sigma^\mu \wbar{\wtilde{H}}^0_{u/d})\pclosed{W^0_\mu - t_W B^0_\mu}                                        \\
                                         & \pm \frac{g}{\sqrt{2}} \bclosed{(\wtilde{W}^0 \wtilde{H}^0_{u/d}) H^{0\,*}_{u/d} - t_W (\wtilde{B}^0 \wtilde{H}^0_{u/d}) H^{0\,*}_{u/d} + \cc}.
            \end{align}

            \begin{align} \nonumber
                \L_{\wtilde{H}^0Z} & = -\frac{g}{2c_W} Z_\mu \bclosed{(\hinou\sigma^\mu\barhinou) - (u \leftrightarrow d)}                                                                                                                                                                                                     \\ \nonumber
                                   & = \frac{g}{2c_W} Z_\mu \bclosed{(\barhinou\wbar{\sigma}^\mu\hinou) - (u \leftrightarrow d)}                                                                                                                                                                                               \\ \nonumber
                                   & = \frac{g}{2c_W} Z_\mu \sum_{ij} \bclosed{N_{i4}N_{j4}^* \barnino[i]\gamma^\mu P_L \nino[j] - (4 \leftrightarrow 3)}                                                                                                                                                                      \\ \nonumber
                                   & = \frac{g}{4c_W} Z_\mu \sum_{ij} \bclosed{N_{i4}N_{j4}^* \barnino[i]\gamma^\mu P_L \nino[j] - N_{i4}N_{j4}^* \barnino[j]\gamma^\mu P_R \nino[i] - (4 \leftrightarrow 3)}                                                                                                                  \\ \nonumber
                                   & = \frac{g}{4c_W} Z_\mu \sum_{ij} \bclosed{N_{i4}N_{j4}^* \barnino[i]\gamma^\mu P_L \nino[j] - N_{i4}^*N_{j4} \barnino[i]\gamma^\mu P_R \nino[j] - (4 \leftrightarrow 3)}                                                                                                                  \\
                                   & = \frac{g}{2c_W} Z_\mu \sum_{ij} \barnino[i] \gamma^\mu \Big[ \underbrace{\frac{1}{2}\pclosed{N_{i4}N_{j4}^* - N_{i3}N_{j3}^*}}_{\equiv O^{\prime\prime L}_{ij}}P_L \underbrace{-\frac{1}{2}\pclosed{N_{i4}^*N_{j4} - N_{i3}^*N_{j3}}}_{\equiv O^{\prime\prime R}_{ij}}P_R \Big] \nino[j]
            \end{align}

            % \begin{align} \nonumber
            %     \L_{\wtilde{H}^0 \wino \bino H^0} & = \frac{g}{\sqrt{2}} \bclosed{ \pclosed{(\wino\hinou) - t_W (\bino\hinou)} {H^0_u}^* + \cc - (u \leftrightarrow d)  }                                    \\ \nonumber
            %                                       & = \frac{g}{2\sqrt{2}} \sum_{ij} \bclosed{ \pclosed{N_{i2} - t_W N_{i1}}N_{j4}^*(\barnino[i] P_L \nino[j]) {H^0_u}^* + \cc - (u,4 \leftrightarrow d,3)  } \\
            %                                       & = \frac{g}{2\sqrt{2}} \sum_{ij} \bclosed{ \pclosed{N_{i2} - t_W N_{i1}}(N_{j4}^* {H^0_u}^* - N_{j3}^* {H^0_d}^* )(\barnino[i] P_L \nino[j]) + \cc  }
            % \end{align}

            \begin{subequations}
                \begin{align}
                    \vcenter{\hbox{
                            \inputtikz{qqZ_vertex}
                        }}
                     & = -\frac{ig}{c_W} \gamma^\mu \bclosed{C_{Zqq}^L P_L + C_{Zqq}^R P_R}                                \\
                    \vcenter{\hbox{
                            \inputtikz{chii-chij-Z_vertex}
                        }}
                     & = \frac{ig}{c_W} \gamma^\mu \bclosed{O_{ij}^{\prime\prime L} P_L + O_{ij}^{\prime\prime R} P_R}     \\
                    \vcenter{\hbox{
                            \inputtikz{q-chi-sq_vertex}
                        }}
                     & = -i\sqrt{2}g \bclosed{ C_{\nino[i] \tilde{q}_A q}^{L*} P_L + C_{\nino[i] \tilde{q}_A q}^{R*} P_R }
                \end{align}
            \end{subequations}


            {\renewcommand{\arraystretch}{2}
            \begin{table}
                \centering
                \begin{tabular}{|c|c|}
                    \hline
                    Variable                             & Value                                                                                                                       \\
                    \hline
                    \(C_{\nino[i] \wtilde{f}_A f}^{L*}\) & \(\pclosed{\delta^A_1 c_{\wtilde{f}} + \delta^A_2 s_{\wtilde{f}}} \bclosed{\pclosed{Q_f-I^3_f} t_W N_{i1} + I^3_f N_{i2}}\) \\
                    \(C_{\nino[i] \wtilde{f}_A f}^{R*}\) & \(\pclosed{\delta^A_1 s_{\wtilde{f}}^* - \delta^A_2 c_{\wtilde{f}}^*} Q_f t_W N_{i1}\)                                      \\
                    \hline
                    \(O^{\prime\prime L}_{ij}\)          & \(\frac{1}{2} \pclosed{N_{i4} N_{j4}^* - N_{i3} N_{N_3}^*}\)                                                                \\
                    \(O^{\prime\prime R}_{ij}\)          & \(-\frac{1}{2} \pclosed{N_{i4}^* N_{j4} - N_{i3}^* N_{N_3}}\)                                                               \\
                    \hline
                \end{tabular}
            \end{table}
            }


            \appendix
\section{Weyl identities}
    \begin{align}
        \label[eq]{app:eq:weyl1}
        (\theta\psi)(\theta\chi) & = -\frac{1}{2} (\theta\theta) (\psi\chi)
    \end{align}
    \begin{align}
        \label[eq]{app:eq:weyl3}
        (\psi \sigma^\mu \wbar\chi) = -(\wbar\chi \wbar\sigma^\mu \psi)
    \end{align}
    \begin{align}
        \label[eq]{app:eq:weyl2}
        (\wbar\theta\wbar\psi) (\theta\sigma^\mu\wbar\theta) (\theta\psi) = \frac{1}{4} (\theta\theta) (\wbar\theta\wbar\theta) (\psi\sigma^\mu\wbar\psi)
    \end{align}

\section{Weyl spinors to Dirac spinors}
    We can build a Dirac spinor $\Psi$ using a left-handed Weyl spinor $\psi_L$ and a right-handed Weyl spinor $\wbar{\psi}_R$ such that
    \begin{align}
        \Psi = \begin{pmatrix} \psi_L \\ \wbar{\psi}_R \end{pmatrix}.
    \end{align}
    I note that the labels \(L/R\), perhaps confusingly, do not label whether the Weyl spinor is left- or right-handed, but rather what they were originally \textit{intended} to be --- \(\wbar{\psi}_L\) would still be a right-handed Weyl spinor.
    The $\gamma$-matrices in the Weyl representation are
    \begin{align}
        \gamma^\mu = \begin{bmatrix} 0 & \sigma^\mu \\ \wbar{\sigma}^\mu & 0 \end{bmatrix},
    \end{align}
    so we define the conjugate Dirac spinor
    \begin{align}
        \wbar{\Psi} \equiv \Psi^\dag \gamma^0 = \begin{pmatrix} \psi_R \\ \wbar{\psi}_L \end{pmatrix}^T.
    \end{align}
    A Majorana fermion is constructed from just one Weyl spinor, such that $\psi_L = \psi_R \equiv \psi$.
    The projection operators $P_{L/R}$ project out the left-handed or right-handed Weyl spinors from the Dirac spinor. The following Weyl spinor products can then be rewritten in terms of Dirac spinors:
    \begin{subequations}
        \begin{align}
            (\psi_{R} \phi_{L})                      & = \wbar{\Psi} P_L \Phi            \\
            (\wbar{\psi}_L \wbar{\phi}_R)            & = \wbar{\Psi} P_R \Phi            \\
            (\wbar{\psi}_L \wbar{\sigma}^\mu \phi_L) & = \wbar{\Psi} \gamma^\mu P_L \Phi \\
            (\psi_R \sigma^\mu \wbar{\phi}_R)        & = \wbar{\Psi} \gamma^\mu P_R \Phi
        \end{align}
    \end{subequations}
    Using equation \cref{app:eq:weyl3}, we get the Majorana Dirac spinor relation
    \begin{align}
        \wbar{\Psi}_M \gamma^\mu P_{L/R} \Phi_M = -\wbar{\Phi}_M \gamma^\mu P_{R/L} \Psi_M
    \end{align}

\section{Electroweak theory}
    \begin{subequations}
        \begin{align}
            B^0_\mu & = c_W A^0_\mu - s_W Z^0_\mu \\
            W^0_\mu & = s_W A^0_\mu + c_W Z^0_\mu
        \end{align}
    \end{subequations}

\end{document}